The simulation name file contains information about simulation options, simulation timing, models that are present in the simulation, how models exchange information, and how models are solved.

The present version of \mf uses the concept of a solution group.  For most simulations, a solution group will contain one solution and one model within that solution.  The solution group is designed, however, so that multiple solutions can be solved together in a picard iteration loop.  This might be used in the future to iteratively couple models that cannot be tightly coupled at the matrix level within a single numerical solution.  The solution group is flexible so that multiple solution groups can be included in a simulation.  More information on solution groups will be added to this document as new model types and exchanges are added that can take advantage of the concept.

The simulation name file is read from a file in the current working directory with the name ``mfsim.nam''.  Input within the simulation name file is provided through the following input blocks, which must be listed in the order shown below.  The options block itself is optional.  All other blocks are required.

\vspace{5mm}
\subsection{Structure of Blocks}
\lstinputlisting[style=blockdefinition]{./mf6ivar/tex/sim-nam-options.dat}
\lstinputlisting[style=blockdefinition]{./mf6ivar/tex/sim-nam-timing.dat}
\lstinputlisting[style=blockdefinition]{./mf6ivar/tex/sim-nam-models.dat}
\lstinputlisting[style=blockdefinition]{./mf6ivar/tex/sim-nam-exchanges.dat}
\lstinputlisting[style=blockdefinition]{./mf6ivar/tex/sim-nam-solutiongroup.dat}

\vspace{5mm}
\subsection{Explanation of Variables}
\begin{description}
% DO NOT MODIFY THIS FILE DIRECTLY.  IT IS CREATED BY mf6ivar.py 

\item \textbf{Block: OPTIONS}

\begin{description}
\item \texttt{CONTINUE}---keyword flag to indicate that the simulation should continue even if one or more solutions do not converge.

\item \texttt{NOCHECK}---keyword flag to indicate that the model input check routines should not be called prior to each time step. Checks are performed by default.

\item \texttt{memory\_print\_option}---is a flag that controls printing of detailed memory manager usage to the end of the simulation list file.  NONE means do not print detailed information. SUMMARY means print only the total memory for each simulation component. ALL means print information for each variable stored in the memory manager. NONE is default if MEMORY\_PRINT\_OPTION is not specified.

\item \texttt{maxerrors}---maximum number of errors that will be stored and printed.

\end{description}
\item \textbf{Block: TIMING}

\begin{description}
\item \texttt{tdis6}---is the name of the Temporal Discretization (TDIS) Input File.

\end{description}
\item \textbf{Block: MODELS}

\begin{description}
\item \texttt{mtype}---is the type of model to add to simulation.

\item \texttt{mfname}---is the file name of the model name file.

\item \texttt{mname}---is the user-assigned name of the model.  The model name cannot exceed 16 characters and must not have blanks within the name.  The model name is case insensitive; any lowercase letters are converted and stored as upper case letters.

\item \texttt{partition\_number}---is the user-assigned partition number of the model.  The partition number cannot exceed the user-assigned number of processors and at least one model must be assigned to each of the processors or the program will terminate with an error.  Models that have sequential data dependencies must have the same PARTITION_NUMBER (for example, a GWT and GWF model that are in separate SOLUTIONS but are linked by a GWF-GWT exchange) or the program  will terminate with an error.

\end{description}
\item \textbf{Block: EXCHANGES}

\begin{description}
\item \texttt{exgtype}---is the exchange type.

\item \texttt{exgfile}---is the input file for the exchange.

\item \texttt{exgmnamea}---is the name of the first model that is part of this exchange.

\item \texttt{exgmnameb}---is the name of the second model that is part of this exchange.

\end{description}
\item \textbf{Block: SOLUTIONGROUP}

\begin{description}
\item \texttt{group\_num}---is the group number of the solution group.  Solution groups must be numbered sequentially, starting with group number one.

\item \texttt{mxiter}---is the maximum number of outer iterations for this solution group.  The default value is 1.  If there is only one solution in the solution group, then MXITER must be 1.

\item \texttt{slntype}---is the type of solution.  The Integrated Model Solution (IMS6) is the only supported option in this version.

\item \texttt{slnfname}---name of file containing solution input.

\item \texttt{slnmnames}---is the array of model names to add to this solution.  The number of model names is determined by the number of model names the user provides on this line.

\end{description}


\end{description}

\begin{table}[h]
\caption{Model types available in Version \modflowversion}
\small
\begin{center}
\begin{tabular*}{\columnwidth}{l l}
\hline
\hline
Mtype & Type of Model \\
\hline
GWF6 & Groundwater Flow Model \\
GWT6 & Groundwater Transport Model \\
GWE6 & Groundwater Energy Model \\
PRT6 & Particle Tracking Model \\
\hline 
\end{tabular*}
\label{table:mtype}
\end{center}
\normalsize
\end{table}

\begin{table}[h]
\caption{Exchange types available in Version \modflowversion}
\small
\begin{center}
\begin{tabular*}{\columnwidth}{l p{15cm}}
\hline
\hline
Exgtype & Type of Exchange \\
\hline
GWF6-GWF6 & Exchange between two Groundwater Flow Models.  Input for this file is described in a dedicated section in this guide. \\
GWF6-GWT6 & Exchange between a Groundwater Flow Model and a Groundwater Transport Model.  In the present version, a filename is required for this exchange and the file must exist, however, nothing is read from this file.  \\
GWT6-GWT6 & Exchange between two Groundwater Transport Models.  Input for this file is described in a dedicated section in this guide. \\
\hline
\end{tabular*}
\label{table:exgtype}
\end{center}
\normalsize
\end{table}

\vspace{5mm}
\subsection{Example Input File}
\lstinputlisting[style=inputfile]{./mf6ivar/examples/sim-nam-example.dat}

