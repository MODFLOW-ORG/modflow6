Next to the standard \mf executable, a second, extended program executable is made available. This extended executable comes with additional functionality for which it partially relies on third-party libraries. Extended functionality includes the parallel computing capability and the use of NetCDF4 for input and output data. Because the external dependencies increase the complexity of the installation procedure, \mf will remain available in its standard set of functionality.

Extended \mf contains all features available in the standard executable, runs on the same input configuration, and produces the same results. Conversely, when running with the standard executable, some features described in this document (HPC Utility and NetCDF4 input and output, for example) will not be available and their configuration will be ignored or the program will terminate with an error. These features will be labeled accordingly below.

This sections describes input files that are only available with Extended \mf.  For more information on using Extended \mf, refer to the \href{https://github.com/MODFLOW-ORG/modflow6/wiki}{MODFLOW 6 Wiki} on the version-controlled \href{https://github.com/MODFLOW-ORG/modflow6}{MODFLOW 6 repository}.

\newpage
\subsection{NetCDF in Extended \mf}
Extended \mf supports reading and writing NetCDF model gridded data in 2 formats, structured (DIS) and UGRID layered mesh (DIS, DISV). To write a NetCDF file, configure a model name file NetCDF output file option (NETCDF\_STRUCTURED or NETCDF\_MESH2D) and designate packages to be written with the EXPORT\_ARRAY\_NETCDF option. Any number of supported packages for the model can contain this option. Supported packages contain a griddata block and are integrated with the Input Data Processor (see ``Processing of Program Input'' section). Simulation calculations do not need to run in order to generate the model NetCDF file- it may make sense to generate the file in \mf validate mode as a way of converting existing ASCII griddata to NetCDF. The following example shows a model name file that has configured NetCDF export for UGRID layered mesh with two packages, DIS and IC, contributing griddata to that file:
\lstinputlisting[style=inputfile]{./mf6ivar/examples/ext-netcdf-gwf-sto-export.dat}

A generated model NetCDF file can be used to provide input for a simulation by configuring it in a model name file with the NETCDF input file option. Once this step has been competed, package input files can be updated to use the optional NETCDF keyword to designate that the array should be read from the corresponding model NetCDF input file:
\lstinputlisting[style=inputfile]{./mf6ivar/examples/ext-netcdf-gwf-sto-input.dat}

\newpage
\subsection{NetCDF (NCF) Configuration Utility}
\input{utl_ncf.tex}

\newpage
\subsection{High Performance Computing (HPC) Utility}
\input{utl_hpc.tex}

