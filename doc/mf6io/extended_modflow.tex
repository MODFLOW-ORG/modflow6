Next to the standard \mf executable, a second, extended program executable is made available. This extended executable comes with additional functionality for which it partially relies on third-party libraries. Extended functionality includes the parallel computing capability and the use of NetCDF4 for input and output data. Because the external dependencies increase the complexity of the installation procedure, \mf will remain available in its standard set of functionality.

Extended \mf contains all features available in the standard executable, runs on the same input configuration, and produces the same results. Conversely, when running with the standard executable, some features described in this document (HPC Utility and NetCDF4 input and output, for example) will not be available and their configuration will be ignored or the program will terminate with an error. These features will be labeled accordingly below.

This sections describes input files that are only available with Extended \mf.  For more information on using Extended \mf, refer to the \href{https://github.com/MODFLOW-ORG/modflow6/wiki}{MODFLOW 6 Wiki} on the version-controlled \href{https://github.com/MODFLOW-ORG/modflow6}{MODFLOW 6 repository}.

\newpage
\subsection{NetCDF in Extended \mf}
Extended \mf supports reading and writing NetCDF model gridded data in 2 formats, structured (DIS) and UGRID layered mesh (DIS, DISV). To write a NetCDF file, configure a model name file NetCDF output file option (NETCDF\_STRUCTURED or NETCDF\_MESH2D) and designate packages to be written with the EXPORT\_ARRAY\_NETCDF option. Any number of supported packages for the model can contain this option. Supported packages contain a griddata block and are integrated with the Input Data Processor (see ``Processing of Program Input'' section). Simulation calculations do not need to run in order to generate the model NetCDF file- it may make sense to generate the file in \mf validate mode as a way of converting existing ASCII griddata to NetCDF. The following example shows a model name file that has configured NetCDF export for UGRID layered mesh with two packages, DIS and IC, contributing griddata to that file:
\lstinputlisting[style=inputfile]{./mf6ivar/examples/ext-netcdf-gwf-sto-export.dat}

A generated model NetCDF file can be used to provide input for a simulation by configuring it in a model name file with the NETCDF input file option. Once this step has been competed, package input files can be updated to use the optional NETCDF keyword to designate that the array should be read from the corresponding model NetCDF input file:
\lstinputlisting[style=inputfile]{./mf6ivar/examples/ext-netcdf-gwf-sto-input.dat}

\newpage
\subsection{NetCDF (NCF) Configuration Utility}
The NetCDF configuration (NCF) utility can be activated by specifying the NCF6 option in a DIS or DISV input file.

The netcdf configuration utility applies to model NetCDF exports that are activated with the EXPORT\_NETCDF keyword in a model name file. The EXPORT\_NETCDF keyword can be used whether or not this configuration package is defined. When defined, this package provides options related to data variable chunking, compression and grid mapping (projections).

\vspace{5mm}
\subsection{Structure of Blocks}
\lstinputlisting[style=blockdefinition]{./mf6ivar/tex/utl-ncf-options.dat}

\vspace{5mm}
\subsection{Explanation of Variables}
\begin{description}
% DO NOT MODIFY THIS FILE DIRECTLY.  IT IS CREATED BY mf6ivar.py 

\item \textbf{Block: OPTIONS}

\begin{description}
\item \texttt{ogc\_wkt}---is the CRS well-known text (WKT) string.

\item \texttt{deflate}---is the deflate level (0=min, 9=max) for per variable compression in the netcdf file. Defining the parameter activates per variable compression in the export file at the level specified.

\item \texttt{SHUFFLE}---is the keyword used to turn on the netcdf variable shuffle filter when the deflate option is also set. The shuffle filter has the effect of storing the first byte of all of a variable's values in a chunk contiguously, followed by all the second bytes, etc. This can be an optimization for compression with certain types of data.

\item \texttt{CHUNKING\_UGRID}---is a keyword for providing ugrid dimension chunk sizes. Chunking can dramatically impact data access times and is highly dependent on access patterns (timeseries vs spatial, for example). It can also significanlty impact compressibility of the data.

\item \texttt{ugc\_time}---is the keyword used to provide a ugrid time dimension chunk size.

\item \texttt{ugc\_face}---is the keyword used to provide a ugrid face dimension chunk size.

\end{description}


\end{description}

\vspace{5mm}
\subsection{Example Input File}
\lstinputlisting[style=inputfile]{./mf6ivar/examples/utl-ncf-example.dat}



\newpage
\subsection{High Performance Computing (HPC) Utility}
The High Performance Computing (HPC) utility file for the simulation can be activated by specifying the HPC6 option in the simulation name file.  It's main purpose is to assign the models in a parallel simulation to the available CPU cores for cases where the internal distribution algorithm is not satisfactory. If activated, \mf will read HPC input according to the following description.

\vspace{5mm}
\subsubsection{Structure of Blocks}
\lstinputlisting[style=blockdefinition]{./mf6ivar/tex/utl-hpc-options.dat}
\lstinputlisting[style=blockdefinition]{./mf6ivar/tex/utl-hpc-partitions.dat}

\vspace{5mm}
\subsubsection{Explanation of Variables}
\begin{description}
% DO NOT MODIFY THIS FILE DIRECTLY.  IT IS CREATED BY mf6ivar.py 

\item \textbf{Block: OPTIONS}

\begin{description}
\end{description}
\item \textbf{Block: PARTITIONS}

\begin{description}
\item \texttt{mname}---is the unique model name.

\item \texttt{mrank}---is the zero-based partition number (also: MPI rank or processor id) to which the model will be assigned.

\end{description}


\end{description}

\vspace{5mm}
\subsubsection{Example Input File}
Example 1: HPC input file distributing 6 models over 4 available CPU cores.
\lstinputlisting[style=inputfile]{./mf6ivar/examples/utl-hpc-example1.dat}

\vspace{5mm}
Example 2: HPC input file distributing 3 GWF models coupled individually to 3 GWT models over 2 available CPU cores. Note that the GWT models have to be assigned the same partition numbers as their GWF counterparts.
\lstinputlisting[style=inputfile]{./mf6ivar/examples/utl-hpc-example2.dat}

