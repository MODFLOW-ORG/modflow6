Input to the Zero-Depth-Gradient (ZDG) Package is read from the file that has type ``ZDG6'' in the Name File.  Any number of ZDG Packages can be specified for a single surface water flow model.

\vspace{5mm}
\subsubsection{Structure of Blocks}
\vspace{5mm}

\noindent \textit{FOR EACH SIMULATION}
\lstinputlisting[style=blockdefinition]{./mf6ivar/tex/swf-zdg-options.dat}
\lstinputlisting[style=blockdefinition]{./mf6ivar/tex/swf-zdg-dimensions.dat}
\vspace{5mm}
\noindent \textit{FOR ANY STRESS PERIOD}
\lstinputlisting[style=blockdefinition]{./mf6ivar/tex/swf-zdg-period.dat}
\packageperioddescription

\vspace{5mm}
\subsubsection{Explanation of Variables}
\begin{description}
% DO NOT MODIFY THIS FILE DIRECTLY.  IT IS CREATED BY mf6ivar.py 

\item \textbf{Block: OPTIONS}

\begin{description}
\item \texttt{auxiliary}---defines an array of one or more auxiliary variable names.  There is no limit on the number of auxiliary variables that can be provided on this line; however, lists of information provided in subsequent blocks must have a column of data for each auxiliary variable name defined here.   The number of auxiliary variables detected on this line determines the value for naux.  Comments cannot be provided anywhere on this line as they will be interpreted as auxiliary variable names.  Auxiliary variables may not be used by the package, but they will be available for use by other parts of the program.  The program will terminate with an error if auxiliary variables are specified on more than one line in the options block.

\item \texttt{BOUNDNAMES}---keyword to indicate that boundary names may be provided with the list of zero-depth-gradient boundary cells.

\item \texttt{PRINT\_INPUT}---keyword to indicate that the list of zero-depth-gradient boundary information will be written to the listing file immediately after it is read.

\item \texttt{PRINT\_FLOWS}---keyword to indicate that the list of zero-depth-gradient boundary flow rates will be printed to the listing file for every stress period time step in which ``BUDGET PRINT'' is specified in Output Control.  If there is no Output Control option and ``PRINT\_FLOWS'' is specified, then flow rates are printed for the last time step of each stress period.

\item \texttt{SAVE\_FLOWS}---keyword to indicate that zero-depth-gradient boundary flow terms will be written to the file specified with ``BUDGET FILEOUT'' in Output Control.

\item \texttt{TS6}---keyword to specify that record corresponds to a time-series file.

\item \texttt{FILEIN}---keyword to specify that an input filename is expected next.

\item \texttt{ts6\_filename}---defines a time-series file defining time series that can be used to assign time-varying values. See the ``Time-Variable Input'' section for instructions on using the time-series capability.

\item \texttt{OBS6}---keyword to specify that record corresponds to an observations file.

\item \texttt{obs6\_filename}---name of input file to define observations for the Zero-Depth-Gradient Boundary package. See the ``Observation utility'' section for instructions for preparing observation input files. Tables \ref{table:gwf-obstypetable} and \ref{table:gwt-obstypetable} lists observation type(s) supported by the Zero-Depth-Gradient Boundary package.

\end{description}
\item \textbf{Block: DIMENSIONS}

\begin{description}
\item \texttt{maxbound}---integer value specifying the maximum number of zero-depth-gradient boundary cells that will be specified for use during any stress period.

\end{description}
\item \textbf{Block: PERIOD}

\begin{description}
\item \texttt{iper}---integer value specifying the starting stress period number for which the data specified in the PERIOD block apply.  IPER must be less than or equal to NPER in the TDIS Package and greater than zero.  The IPER value assigned to a stress period block must be greater than the IPER value assigned for the previous PERIOD block.  The information specified in the PERIOD block will continue to apply for all subsequent stress periods, unless the program encounters another PERIOD block.

\item \texttt{cellid}---is the cell identifier, and depends on the type of grid that is used for the simulation.  For a structured grid that uses the DIS input file, CELLID is the layer, row, and column.   For a grid that uses the DISV input file, CELLID is the layer and CELL2D number.  If the model uses the unstructured discretization (DISU) input file, CELLID is the node number for the cell.

\item \texttt{idcxs}---is the identifier for the cross section specified in the CXS Package.  A value of zero indicates the zero-depth-gradient calculation will use parameters for a hydraulically wide channel.

\item \textcolor{blue}{\texttt{width}---is the channel width of the zero-depth gradient boundary. If a cross section is associated with this boundary, the width will be scaled by the cross section information.  If the Options block includes a TIMESERIESFILE entry (see the ``Time-Variable Input'' section), values can be obtained from a time series by entering the time-series name in place of a numeric value.}

\item \textcolor{blue}{\texttt{slope}---is the channel slope used to calculate flow to the zero-depth-gradient boundary. If the Options block includes a TIMESERIESFILE entry (see the ``Time-Variable Input'' section), values can be obtained from a time series by entering the time-series name in place of a numeric value.}

\item \textcolor{blue}{\texttt{rough}---is the Manning channel roughness value used to calculate flow to the zero-depth-gradient boundary. If a cross section is associated with this boundary, the roughness value will be multiplied by the roughness fraction for each part of the cross section.  If the Options block includes a TIMESERIESFILE entry (see the ``Time-Variable Input'' section), values can be obtained from a time series by entering the time-series name in place of a numeric value.}

\item \textcolor{blue}{\texttt{aux}---represents the values of the auxiliary variables for each zero-depth-gradient boundary. The values of auxiliary variables must be present for each zero-depth-gradient boundary. The values must be specified in the order of the auxiliary variables specified in the OPTIONS block.  If the package supports time series and the Options block includes a TIMESERIESFILE entry (see the ``Time-Variable Input'' section), values can be obtained from a time series by entering the time-series name in place of a numeric value.}

\item \texttt{boundname}---name of the zero-depth-gradient boundary cell.  BOUNDNAME is an ASCII character variable that can contain as many as 40 characters.  If BOUNDNAME contains spaces in it, then the entire name must be enclosed within single quotes.

\end{description}


\end{description}

\vspace{5mm}
\subsubsection{Example Input File}
\lstinputlisting[style=inputfile]{./mf6ivar/examples/swf-zdg-example.dat}

%\vspace{5mm}
%\subsubsection{Available observation types}
%Well Package observations include the simulated well rates (\texttt{wel}), the well discharge that is available for the MVR package (\texttt{to-mvr}), and the reduction in the specified \texttt{q} when the \texttt{AUTO\_FLOW\_REDUCE} option is enabled. The data required for each WEL Package observation type is defined in table~\ref{table:gwf-welobstype}. The sum of \texttt{wel} and \texttt{to-mvr} is equal to the simulated well discharge rate, which may be less than the specified \texttt{q} if the \texttt{AUTO\_FLOW\_REDUCE} option is enabled. The \texttt{DNODATA} value is returned if the \texttt{wel-reduction} observation is specified but the \texttt{AUTO\_FLOW\_REDUCE} option is not enabled. Negative and positive values for an observation represent a loss from and gain to the GWF model, respectively.

%\begin{longtable}{p{2cm} p{2.75cm} p{2cm} p{1.25cm} p{7cm}}
%\caption{Available WEL Package observation types} \tabularnewline

%\hline
%\hline
%\textbf{Stress Package} & \textbf{Observation type} & \textbf{ID} & \textbf{ID2} & \textbf{Description} \\
%\hline
%\endhead

%\hline
%\endfoot

%WEL & wel & cellid or boundname & -- & Flow between the groundwater system and a well boundary or a group of well boundaries. \\
WEL & to-mvr & cellid or boundname & -- & Well boundary discharge that is available for the MVR package for a well boundary or a group of well boundaries. \\
WEL & wel-reduction & cellid or boundname & -- & Reduction in the specified well boundary discharge calculated when the \texttt{AUTO\_FLOW\_REDUCE} option is specified.
%\label{table:gwf-welobstype}
%\end{longtable}

%\vspace{5mm}
%\subsubsection{Example Observation Input File}
%\lstinputlisting[style=inputfile]{./mf6ivar/examples/gwf-wel-example-obs.dat}
