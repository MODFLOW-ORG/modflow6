The PRT Model calculates three-dimensional, advective particle trajectories in flowing groundwater. The PRT Model is designed to work with the Groundwater Flow (GWF) Model \citep{modflow6gwf} and uses the same spatial discretization, which may be represented using either a structured (DIS) or an unstructured (DISV) grid. The PRT Model replicates much of the functionality of MODPATH 7 \citep{pollock2016modpath7} and offers support for a much broader class of unstructured grids. The PRT Model can be run in the same simulation as the associated GWF Model or in a separate simulation that reads previously calculated flows from a binary budget file. The version of the PRT Model documented here does not support grids of DISU type, tracking of particles through advanced stress package features such as lakes or streams reaches, or exchange of particles between PRT models.

This section describes the data files for a \mf Particle Tracking (PRT) Model.  A PRT Model is added to the simulation by including a PRT entry in the MODELS block of the simulation name file.  There are currently two types of spatial discretization approaches that can be used with the PRT Model: DIS and DISV.  The input instructions for these three packages are not described here in this section on PRT Model input; input instructions for these three packages are described in the section on GWF Model input.  Note that for a PRT Model, the maximum number of vertices for a cell in a DISV grid is limited to 8.

The PRT Model is designed to permit input to be gathered, as it is needed, from many different files.  Likewise, results from the model calculations can be written to a number of output files. Details about the files used by each package are provided in this section on the PRT Model Instructions.

The PRT Model reads a file called the Name File, which specifies most of the files that will be used in a simulation. Several files are always required whereas other files are optional depending on the simulation. The Output Control Package receives instructions from the user to control the amount and frequency of output.  Details about the Name File and the Output Control Package are described in this section.

For the PRT Model, ``flows'' (unless stated otherwise) represent particle mass ``flow'' in mass per time, rather than groundwater flow.  Each particle is currently assigned unit mass, and the numerical value of the flow can be interpreted as particles per time.

\subsection{Units of Length and Time}
The PRT Model formulates the particle trajectory equations without using prescribed length and time units. Any consistent units of length and time can be used when specifying the input data for a simulation. This capability gives a certain amount of freedom to the user, but care must be exercised to avoid mixing units.  The program cannot detect the use of inconsistent units.

\subsection{Time Stepping}
In \mf time step lengths are controlled by the user and specified in the Temporal Discretization (TDIS) input file.  When the flow model and particle-tracking model run in the same simulation, the time step length specified in TDIS is used for both models.  If the PRT Model runs in a separate simulation, the time discretization may differ.  Instructions for specifying time steps are described in the TDIS section of this user guide; additional information on GWF and PRT configurations are in the Flow Model Interface section.  

\subsection{Particle Mass Budget}
A summary of all inflow (sources) and outflow (sinks) of particle mass is called a mass budget.  The particle mass budget is printed to the PRT Model Listing File for selected time steps.  In the current implementation, each particle is assigned unit mass, and the numerical value of the flow can be interpreted as particles per time.

\subsection{Particle Track Output}

The PRT Model supports both binary and CSV particle track output files. A particle track CSV file contains the output data in tabular format. The first line of the CSV file contains column names. Each subsequent line in the file contains a row of data for a single particle track record, with the following fields:

\vspace{5mm}
\noindent Column 0: \texttt{`KPER'} {\color{red} \footnotesize{INTEGER}} \\
\noindent Column 1: \texttt{`KSTP'} {\color{red} \footnotesize{INTEGER}} \\
\noindent Column 2: \texttt{`IMDL'} {\color{red} \footnotesize{INTEGER}} \\
\noindent Column 3: \texttt{`IPRP'} {\color{red} \footnotesize{INTEGER}} \\
\noindent Column 4: \texttt{`IRPT'} {\color{red} \footnotesize{INTEGER}} \\
\noindent Column 5: \texttt{`ILAY'} {\color{red} \footnotesize{INTEGER}} \\
\noindent Column 6: \texttt{`ICELL'} {\color{red} \footnotesize{INTEGER}} \\
\noindent Column 7: \texttt{`IZONE'} {\color{red} \footnotesize{INTEGER}} \\
\noindent Column 8: \texttt{`ISTATUS'} {\color{red} \footnotesize{INTEGER}} \\
\noindent Column 9: \texttt{`IREASON'} {\color{red} \footnotesize{INTEGER}} \\
\noindent Column 10: \texttt{`TRELEASE'} {\color{red} \footnotesize{DOUBLE}} \\
\noindent Column 11: \texttt{`T'} {\color{red} \footnotesize{DOUBLE}} \\
\noindent Column 12: \texttt{`X'} {\color{red} \footnotesize{DOUBLE}} \\
\noindent Column 13: \texttt{`Y'} {\color{red} \footnotesize{DOUBLE}} \\
\noindent Column 14: \texttt{`Z'} {\color{red} \footnotesize{DOUBLE}} \\
\noindent Column 15: \texttt{`NAME'} {\color{red} \footnotesize{CHARACTER(LEN=LENTXT)}} \\

\vspace{2mm}
\noindent where

\begin{description} \itemsep0pt \parskip0pt \parsep0pt
\item \texttt{KPER} is the stress period number
\item \texttt{KSTP} is the time step number
\item \texttt{IMDL} is the number of the model the particle originated in
\item \texttt{IPRP} is the number of the particle release point (PRP) package the particle originated in
\item \texttt{IRPT} is the release point number
\item \texttt{ILAY} is the layer number
\item \texttt{ICELL} is the cell number
\item \texttt{IZONE} is the zone number
\item \texttt{ISTATUS} is the particle status code
\item \texttt{IREASON} is the reporting reason code
\item \texttt{TRELEASE} is the particle release time
\item \texttt{T} is the particle tracking time
\item \texttt{X} is the particle x coordinate
\item \texttt{Y} is the particle y coordinate
\item \texttt{Z} is the particle z coordinate
\item \texttt{NAME} is the name of the particle release point
\end{description}

The ISTATUS field indicates the status of the particle:

\begin{description} \itemsep0pt \parskip0pt \parsep0pt
\item \texttt{0}: particle was released
\item \texttt{1}: particle is being actively tracked
\item \texttt{2}: particle terminated at a boundary face
\item \texttt{3}: particle terminated in a weak sink cell
\item \texttt{4}: \textit{unused}
\item \texttt{5}: particle terminated in a cell with no exit face
\item \texttt{6}: particle terminated in a stop zone
\item \texttt{7}: particle terminated in an inactive cell
\item \texttt{8}: particle terminated immediately upon release into a dry cell
\item \texttt{9}: particle terminated in a subcell with no exit face
\end{description}

The IREASON field indicates the reason the particle track record was saved:

\begin{description} \itemsep0pt \parskip0pt \parsep0pt
\item \texttt{0}: particle was released
\item \texttt{1}: particle exited a cell
\item \texttt{2}: time step ended
\item \texttt{3}: particle terminated
\item \texttt{4}: particle exited a weak sink cell
\item \texttt{5}: user-specified tracking time
\end{description}

By default, the PRT Model reports all particle events. The user can optionally select which events are reported, as explained in the Output Control (OC) subsection. Because multiple tracking events may coincide (e.g. exiting a cell and exiting a weak sink cell), particle track records may be duplicates except for the ISTATUS and/or IREASON codes.

The binary particle track file contains the same particle track data in a record-based binary format explained in the Particle Track File subsection of the Description of Binary Output Files section.


\newpage
\subsection{PRT Model Name File}
The GWT Model Name File specifies the options and packages that are active for a GWT model.  The Name File contains two blocks: OPTIONS  and PACKAGES. The length of each line must be 299 characters or less. The lines in each block can be in any order.  Files listed in the PACKAGES block must exist when the program starts. 

Comment lines are indicated when the first character in a line is one of the valid comment characters.  Commented lines can be located anywhere in the file. Any text characters can follow the comment character. Comment lines have no effect on the simulation; their purpose is to allow users to provide documentation about a particular simulation. 

\vspace{5mm}
\subsubsection{Structure of Blocks}
\lstinputlisting[style=blockdefinition]{./mf6ivar/tex/gwt-nam-options.dat}
\lstinputlisting[style=blockdefinition]{./mf6ivar/tex/gwt-nam-packages.dat}

\vspace{5mm}
\subsubsection{Explanation of Variables}
\begin{description}
% DO NOT MODIFY THIS FILE DIRECTLY.  IT IS CREATED BY mf6ivar.py 

\item \textbf{Block: OPTIONS}

\begin{description}
\item \texttt{list}---is name of the listing file to create for this GWT model.  If not specified, then the name of the list file will be the basename of the GWT model name file and the '.lst' extension.  For example, if the GWT name file is called ``my.model.nam'' then the list file will be called ``my.model.lst''.

\item \texttt{PRINT\_INPUT}---keyword to indicate that the list of all model stress package information will be written to the listing file immediately after it is read.

\item \texttt{PRINT\_FLOWS}---keyword to indicate that the list of all model package flow rates will be printed to the listing file for every stress period time step in which ``BUDGET PRINT'' is specified in Output Control.  If there is no Output Control option and ``PRINT\_FLOWS'' is specified, then flow rates are printed for the last time step of each stress period.

\item \texttt{SAVE\_FLOWS}---keyword to indicate that all model package flow terms will be written to the file specified with ``BUDGET FILEOUT'' in Output Control.

\item \texttt{export\_netcdf}---keyword that specifies timeseries data for the dependent variable should be written to a model output netcdf file.  No value or ``UGRID'' (ugrid based export) values are supported.

\end{description}
\item \textbf{Block: PACKAGES}

\begin{description}
\item \texttt{ftype}---is the file type, which must be one of the following character values shown in table~\ref{table:ftype-gwt}. Ftype may be entered in any combination of uppercase and lowercase.

\item \texttt{fname}---is the name of the file containing the package input.  The path to the file should be included if the file is not located in the folder where the program was run.

\item \texttt{pname}---is the user-defined name for the package. PNAME is restricted to 16 characters.  No spaces are allowed in PNAME.  PNAME character values are read and stored by the program for stress packages only.  These names may be useful for labeling purposes when multiple stress packages of the same type are located within a single GWT Model.  If PNAME is specified for a stress package, then PNAME will be used in the flow budget table in the listing file; it will also be used for the text entry in the cell-by-cell budget file.  PNAME is case insensitive and is stored in all upper case letters.

\end{description}


\end{description}

\begin{table}[H]
\caption{Ftype values described in this report.  The \texttt{Pname} column indicates whether or not a package name can be provided in the name file}
\small
\begin{center}
\begin{tabular*}{\columnwidth}{l l l}
\hline
\hline
Ftype & Input File Description & \texttt{Pname}\\
\hline
DIS6 & Rectilinear Discretization Input File \\
DISV6 & Discretization by Vertices Input File \\
DISU6 & Unstructured Discretization Input File \\
FMI6 & Flow Model Interface Package &  \\ 
IC6 & Initial Conditions Package \\
OC6 & Output Control Option \\
ADV6 & Advection Package \\ 
DSP6 & Dispersion Package \\ 
SSM6 & Source and Sink Mixing Package \\ 
MST6 & Mobile Storage and Transfer Package \\
IST6 & Immobile Storage and Transfer Package & * \\
CNC6 & Constant Concentration Package & * \\ 
SRC6 & Mass Source Loading Package & * \\ 
LKT6 & Lake Transport Package & * (must be same name as corresponding GWF LAK Package) \\ 
SFT6 & Streamflow Transport Package & * (must be same name as corresponding GWF SFR Package) \\ 
MWT6 & Multi-Aquifer Well Transport Package & * (must be same name as corresponding GWF MAW Package) \\ 
UZT6 & Unsaturated Zone Transport Package & * (must be same name as corresponding GWF UZF Package) \\ 
OBS6 & Observations Option \\
\hline 
\end{tabular*}
\label{table:ftype}
\end{center}
\normalsize
\end{table}

\vspace{5mm}
\subsubsection{Example Input File}
\lstinputlisting[style=inputfile]{./mf6ivar/examples/gwt-nam-example.dat}



%\newpage
%\subsection{Structured Discretization (DIS) Input File}
%\input{gwf/dis}

%\newpage
%\subsection{Discretization with Vertices (DISV) Input File}
%\input{gwf/disv}

%\newpage
%\subsection{Unstructured Discretization (DISU) Input File}
%Discretization information for unstructured grids is read from the file that is specified by ``DISU6'' as the file type.  Only one discretization input file (DISU6, DISV6 or DIS6) can be specified for a model.

The shape and position of each cell can be defined using vertices.  This information is optional and is only read if the number of vertices (NVERT) in the DIMENSIONS block is specified and is assigned a value larger than zero.  If the vertices and two-dimensional cell information is provided in this file, then this information is also written to the binary grid file.  Providing this information may be useful for other postprocessing programs that read the binary grid file.

The DISU Package does not support the concept of layers, which is different from the DISU implementation in MODFLOW-USG.  In \mf~all grid input and output for models that use the DISU Package is entered or written as a one-dimensional array of size nodes.

The DISU VERTICES and CELL2D blocks are not required for all simulations.  These blocks are required if the XT3D or the SAVE\_SPECIFIC\_DISCHARGE options are specified in the NPF Package.  In general, it is recommended to include the VERTICES and CELL2D blocks. 

\vspace{5mm}
\subsubsection{Structure of Blocks}
\lstinputlisting[style=blockdefinition]{./mf6ivar/tex/gwf-disu-options.dat}
\lstinputlisting[style=blockdefinition]{./mf6ivar/tex/gwf-disu-dimensions.dat}
\lstinputlisting[style=blockdefinition]{./mf6ivar/tex/gwf-disu-griddata.dat}
\lstinputlisting[style=blockdefinition]{./mf6ivar/tex/gwf-disu-connectiondata.dat}
\lstinputlisting[style=blockdefinition]{./mf6ivar/tex/gwf-disu-vertices.dat}
\lstinputlisting[style=blockdefinition]{./mf6ivar/tex/gwf-disu-cell2d.dat}

\vspace{5mm}
\subsubsection{Explanation of Variables}
\begin{description}
% DO NOT MODIFY THIS FILE DIRECTLY.  IT IS CREATED BY mf6ivar.py 

\item \textbf{Block: OPTIONS}

\begin{description}
\item \texttt{length\_units}---is the length units used for this model.  Values can be ``FEET'', ``METERS'', or ``CENTIMETERS''.  If not specified, the default is ``UNKNOWN''.

\item \texttt{NOGRB}---keyword to deactivate writing of the binary grid file.

\item \texttt{xorigin}---x-position of the origin used for model grid vertices.  This value should be provided in a real-world coordinate system.  A default value of zero is assigned if not specified.  The value for XORIGIN does not affect the model simulation, but it is written to the binary grid file so that postprocessors can locate the grid in space.

\item \texttt{yorigin}---y-position of the origin used for model grid vertices.  This value should be provided in a real-world coordinate system.  If not specified, then a default value equal to zero is used.  The value for YORIGIN does not affect the model simulation, but it is written to the binary grid file so that postprocessors can locate the grid in space.

\item \texttt{angrot}---counter-clockwise rotation angle (in degrees) of the model grid coordinate system relative to a real-world coordinate system.  If not specified, then a default value of 0.0 is assigned.  The value for ANGROT does not affect the model simulation, but it is written to the binary grid file so that postprocessors can locate the grid in space.

\end{description}
\item \textbf{Block: DIMENSIONS}

\begin{description}
\item \texttt{nodes}---is the number of cells in the model grid.

\item \texttt{nja}---is the sum of the number of connections and NODES.  When calculating the total number of connections, the connection between cell n and cell m is considered to be different from the connection between cell m and cell n.  Thus, NJA is equal to the total number of connections, including n to m and m to n, and the total number of cells.

\item \texttt{nvert}---is the total number of (x, y) vertex pairs used to define the plan-view shape of each cell in the model grid.  If NVERT is not specified or is specified as zero, then the VERTICES and CELL2D blocks below are not read.  NVERT and the accompanying VERTICES and CELL2D blocks should be specified for most simulations.  If the XT3D or SAVE\_SPECIFIC\_DISCHARGE options are specified in the NPF Package, then this information is required.

\end{description}
\item \textbf{Block: GRIDDATA}

\begin{description}
\item \texttt{top}---is the top elevation for each cell in the model grid.

\item \texttt{bot}---is the bottom elevation for each cell.

\item \texttt{area}---is the cell surface area (in plan view).

\end{description}
\item \textbf{Block: CONNECTIONDATA}

\begin{description}
\item \texttt{iac}---is the number of connections (plus 1) for each cell.  The sum of all the entries in IAC must be equal to NJA.

\item \texttt{ja}---is a list of cell number (n) followed by its connecting cell numbers (m) for each of the m cells connected to cell n. The number of values to provide for cell n is IAC(n).  This list is sequentially provided for the first to the last cell. The first value in the list must be cell n itself, and the remaining cells must be listed in an increasing order (sorted from lowest number to highest).  Note that the cell and its connections are only supplied for the GWF cells and their connections to the other GWF cells.  Also note that the JA list input may be divided such that every node and its connectivity list can be on a separate line for ease in readability of the file. To further ease readability of the file, the node number of the cell whose connectivity is subsequently listed, may be expressed as a negative number, the sign of which is subsequently converted to positive by the code.

\item \texttt{ihc}---is an index array indicating the direction between node n and all of its m connections.  If IHC = 0 then cell n and cell m are connected in the vertical direction.  Cell n overlies cell m if the cell number for n is less than m; cell m overlies cell n if the cell number for m is less than n.  If IHC = 1 then cell n and cell m are connected in the horizontal direction.  If IHC = 2 then cell n and cell m are connected in the horizontal direction, and the connection is vertically staggered.  A vertically staggered connection is one in which a cell is horizontally connected to more than one cell in a horizontal connection.

\item \texttt{cl12}---is the array containing connection lengths between the center of cell n and the shared face with each adjacent m cell.

\item \texttt{hwva}---is a symmetric array of size NJA.  For horizontal connections, entries in HWVA are the horizontal width perpendicular to flow.  For vertical connections, entries in HWVA are the vertical area for flow.  Thus, values in the HWVA array contain dimensions of both length and area.  Entries in the HWVA array have a one-to-one correspondence with the connections specified in the JA array.  Likewise, there is a one-to-one correspondence between entries in the HWVA array and entries in the IHC array, which specifies the connection type (horizontal or vertical).  Entries in the HWVA array must be symmetric; the program will terminate with an error if the value for HWVA for an n to m connection does not equal the value for HWVA for the corresponding n to m connection.

\item \texttt{angldegx}---is the angle (in degrees) between the horizontal x-axis and the outward normal to the face between a cell and its connecting cells. The angle varies between zero and 360.0 degrees, where zero degrees points in the positive x-axis direction, and 90 degrees points in the positive y-axis direction.  ANGLDEGX is only needed if horizontal anisotropy is specified in the NPF Package, if the XT3D option is used in the NPF Package, or if the SAVE\_SPECIFIC\_DISCHARGE option is specifed in the NPF Package.  ANGLDEGX does not need to be specified if these conditions are not met.  ANGLDEGX is of size NJA; values specified for vertical connections and for the diagonal position are not used.  Note that ANGLDEGX is read in degrees, which is different from MODFLOW-USG, which reads a similar variable (ANGLEX) in radians.

\end{description}
\item \textbf{Block: VERTICES}

\begin{description}
\item \texttt{iv}---is the vertex number.  Records in the VERTICES block must be listed in consecutive order from 1 to NVERT.

\item \texttt{xv}---is the x-coordinate for the vertex.

\item \texttt{yv}---is the y-coordinate for the vertex.

\end{description}
\item \textbf{Block: CELL2D}

\begin{description}
\item \texttt{icell2d}---is the cell2d number.  Records in the CELL2D block must be listed in consecutive order from 1 to NODES.

\item \texttt{xc}---is the x-coordinate for the cell center.

\item \texttt{yc}---is the y-coordinate for the cell center.

\item \texttt{ncvert}---is the number of vertices required to define the cell.  There may be a different number of vertices for each cell.

\item \texttt{icvert}---is an array of integer values containing vertex numbers (in the VERTICES block) used to define the cell.  Vertices must be listed in clockwise order.

\end{description}


\end{description}

\vspace{5mm}
\subsubsection{Example Input File}
\lstinputlisting[style=inputfile]{./mf6ivar/examples/gwf-disu-example.dat}



\newpage
\subsection{Model Input (MIP) Package}
Model Input (MIP) Package information is read from the file that is specified by ``MIP6'' as the file type.  The MIP Package is required, and only one MIP Package can be specified for a PRT model.  The information read by the MIP Package pertains to the entire PRT model.

\vspace{5mm}
\subsubsection{Structure of Blocks}
%%\lstinputlisting[style=blockdefinition]{./mf6ivar/tex/prt-mip-options.dat}
\lstinputlisting[style=blockdefinition]{./mf6ivar/tex/prt-mip-griddata.dat}

\vspace{5mm}
\subsubsection{Explanation of Variables}
\begin{description}
% DO NOT MODIFY THIS FILE DIRECTLY.  IT IS CREATED BY mf6ivar.py 

\item \textbf{Block: OPTIONS}

\begin{description}
\item \texttt{zero\_method}---the root finding algorithm to solve ternary subcells.  0 euler, 1 brent, 2 chandrupatla, 3 test.

\item \texttt{EXPORT\_ARRAY\_ASCII}---keyword that specifies input grid arrays, which already support the layered keyword, should be written to layered ascii output files.

\end{description}
\item \textbf{Block: GRIDDATA}

\begin{description}
\item \texttt{porosity}---is the aquifer porosity.

\item \texttt{retfactor}---is a real value by which velocity is divided within a given cell.  RETFACTOR can be used to account for solute retardation, i.e., the apparent effect of linear sorption on the velocity of particles that track solute advection.  RETFACTOR may be assigned any real value.  A RETFACTOR value greater than 1 represents particle retardation (slowing), and a value of 1 represents no retardation.  The effect of specifying a RETFACTOR value for each cell is the same as the effect of directly multiplying the POROSITY in each cell by the proposed RETFACTOR value for each cell.  RETFACTOR allows conceptual isolation of effects such as retardation from the effect of porosity.  The default value is 1.

\item \texttt{izone}---is an integer zone number assigned to each cell.  IZONE may be positive, negative, or zero.  The current cell's zone number is recorded with each particle track datum.  If the ISTOPZONE option is set to any value other than zero in a PRP Package, particles released by that PRP Package terminate if they enter a cell whose IZONE value matches ISTOPZONE.  If ISTOPZONE is not specified or is set to zero in a PRP Package, IZONE has no effect on the termination of particles released by that PRP Package.

\end{description}


\end{description}

\vspace{5mm}
\subsubsection{Example Input File}
\lstinputlisting[style=inputfile]{./mf6ivar/examples/prt-mip-example.dat}


\newpage
\subsection{Particle Release Point (PRP) Package}
Particle Release Point (PRP) Package information is read from the file that is specified by ``PRP6'' as the file type.  More than one PRP Package can be specified for a PRT model. 

\vspace{5mm}
\subsubsection{Structure of Blocks}
\lstinputlisting[style=blockdefinition]{./mf6ivar/tex/prt-prp-options.dat}
\lstinputlisting[style=blockdefinition]{./mf6ivar/tex/prt-prp-dimensions.dat}
%%\lstinputlisting[style=blockdefinition]{./mf6ivar/tex/prt-prp-griddata.dat}
\lstinputlisting[style=blockdefinition]{./mf6ivar/tex/prt-prp-packagedata.dat}
\vspace{5mm}
\noindent \textit{FOR ANY STRESS PERIOD}
\lstinputlisting[style=blockdefinition]{./mf6ivar/tex/prt-prp-period.dat}
\packageperioddescription \: If no PERIOD block is specified for any period, a single particle is released from each release point at the beginning of the simulation.
%\noindent All of the stress period information in a PERIOD block will apply only to that stress period; the information will not continue to apply for subsequent stress periods.  Note that this behavior is different from the simple stress packages (CHD, WEL, DRN, RIV,
%GHB, RCH and EVT) and the advanced stress packages (MAW, SFR, LAK, and UZF).

\vspace{5mm}
\subsubsection{Explanation of Variables}
\begin{description}
% DO NOT MODIFY THIS FILE DIRECTLY.  IT IS CREATED BY mf6ivar.py 

\item \textbf{Block: OPTIONS}

\begin{description}
\item \texttt{BOUNDNAMES}---keyword to indicate that boundary names may be provided with the list of particle release points.

\item \texttt{PRINT\_INPUT}---keyword to indicate that the list of all model stress package information will be written to the listing file immediately after it is read.

\item \texttt{exit\_solve\_tolerance}---the convergence tolerance for iterative solution of particle exit location and time in the generalized Pollock's method.  A value of 0.00001 works well for many problems, but the value that strikes the best balance between accuracy and runtime is problem-dependent.

\item \texttt{LOCAL\_Z}---indicates that ``zrpt'' defines the local z coordinate of the release point within the cell, with value of 0 at the bottom and 1 at the top of the cell.  If the cell is partially saturated at release time, the top of the cell is considered to be the water table elevation (the head in the cell) rather than the top defined by the user.

\item \texttt{TRACK}---keyword to specify that record corresponds to a binary track output file.  Each PRP Package may have a distinct binary track output file.

\item \texttt{FILEOUT}---keyword to specify that an output filename is expected next.

\item \texttt{trackfile}---name of the binary output file to write tracking information.

\item \texttt{TRACKCSV}---keyword to specify that record corresponds to a CSV track output file.  Each PRP Package may have a distinct CSV track output file.

\item \texttt{trackcsvfile}---name of the comma-separated value (CSV) file to write tracking information.

\item \texttt{stoptime}---real value defining the maximum simulation time to which particles in the package can be tracked.  Particles that have not terminated earlier due to another termination condition will terminate when simulation time STOPTIME is reached.  If the last stress period in the simulation consists of more than one time step, particles will not be tracked past the ending time of the last stress period, regardless of STOPTIME.  If the last stress period in the simulation consists of a single time step, it is assumed to be a steady-state stress period, and its ending time will not limit the simulation time to which particles can be tracked.  If STOPTIME and STOPTRAVELTIME are both provided, particles will be stopped if either is reached.

\item \texttt{stoptraveltime}---real value defining the maximum travel time over which particles in the model can be tracked.  Particles that have not terminated earlier due to another termination condition will terminate when their travel time reaches STOPTRAVELTIME.  If the last stress period in the simulation consists of more than one time step, particles will not be tracked past the ending time of the last stress period, regardless of STOPTRAVELTIME.  If the last stress period in the simulation consists of a single time step, it is assumed to be a steady-state stress period, and its ending time will not limit the travel time over which particles can be tracked.  If STOPTIME and STOPTRAVELTIME are both provided, particles will be stopped if either is reached.

\item \texttt{STOP\_AT\_WEAK\_SINK}---is a text keyword to indicate that a particle is to terminate when it enters a cell that is a weak sink.  By default, particles are allowed to pass though cells that are weak sinks.

\item \texttt{istopzone}---integer value defining the stop zone number.  If cells have been assigned IZONE values in the GRIDDATA block, a particle terminates if it enters a cell whose IZONE value matches ISTOPZONE.  An ISTOPZONE value of zero indicates that there is no stop zone.  The default value is zero.

\item \texttt{DRAPE}---is a text keyword to indicate that if a particle's release point is in a cell that happens to be inactive at release time, the particle is to be moved to the topmost active cell below it, if any. By default, a particle is not released into the simulation if its release point's cell is inactive at release time.

\item \texttt{RELEASE\_TIMES}---keyword indicating release times will follow

\item \texttt{times}---times to release, relative to the beginning of the simulation.  RELEASE\_TIMES and RELEASE\_TIMESFILE are mutually exclusive.  Explicit release times and release settings configured in the period block are additive and can be mixed and matched.

\item \texttt{RELEASE\_TIMESFILE}---keyword indicating release times file name will follow

\item \texttt{timesfile}---name of the release times file.  RELEASE\_TIMES and RELEASE\_TIMESFILE are mutually exclusive.  Explicit release times and release settings configured in the period block are additive and can be mixed and matched.

\end{description}
\item \textbf{Block: DIMENSIONS}

\begin{description}
\item \texttt{nreleasepts}---is the number of particle release points.

\end{description}
\item \textbf{Block: PACKAGEDATA}

\begin{description}
\item \texttt{irptno}---integer value that defines the PRP release point number associated with the specified PACKAGEDATA data on the line. IRPTNO must be greater than zero and less than or equal to NRELEASEPTS.  The program will terminate with an error if information for a PRP release point number is specified more than once.

\item \texttt{cellid}---is the cell identifier, and depends on the type of grid that is used for the simulation.  For a structured grid that uses the DIS input file, CELLID is the layer, row, and column.   For a grid that uses the DISV input file, CELLID is the layer and CELL2D number.  If the model uses the unstructured discretization (DISU) input file, CELLID is the node number for the cell.

\item \texttt{xrpt}---real value that defines the x coordinate of the release point in model coordinates.  The (x, y, z) location specified for the release point must lie within the cell that is identified by the specified cellid.

\item \texttt{yrpt}---real value that defines the y coordinate of the release point in model coordinates.  The (x, y, z) location specified for the release point must lie within the cell that is identified by the specified cellid.

\item \texttt{zrpt}---real value that defines the z coordinate of the release point in model coordinates or, if the LOCAL\_Z option is active, in local cell coordinates.  The (x, y, z) location specified for the release point must lie within the cell that is identified by the specified cellid.

\item \texttt{boundname}---name of the particle release point. BOUNDNAME is an ASCII character variable that can contain as many as 40 characters. If BOUNDNAME contains spaces in it, then the entire name must be enclosed within single quotes.

\end{description}
\item \textbf{Block: PERIOD}

\begin{description}
\item \texttt{iper}---integer value specifying the stress period number for which the data specified in the PERIOD block apply. IPER must be less than or equal to NPER in the TDIS Package and greater than zero. The IPER value assigned to a stress period block must be greater than the IPER value assigned for the previous PERIOD block. The information specified in the PERIOD block applies only to that stress period.

\item \texttt{releasesetting}---specifies when to release particles within the stress period.  Overrides package-level RELEASETIME option, which applies to all stress periods. By default, RELEASESETTING configures particles for release at the beginning of the specified time steps. For time-offset releases, provide a FRACTION value.

\begin{lstlisting}[style=blockdefinition]
ALL
FIRST
FREQUENCY <frequency>
STEPS <steps(<nstp)>
[FRACTION <fraction(<nstp)>]
\end{lstlisting}

\item \texttt{ALL}---keyword to indicate release of particles at the start of all time steps in the period.

\item \texttt{FIRST}---keyword to indicate release of particles at the start of the first time step in the period. This keyword may be used in conjunction with other keywords to release particles at the start of multiple time steps.

\item \texttt{frequency}---release particles at the specified time step frequency. This keyword may be used in conjunction with other keywords to release particles at the start of multiple time steps.

\item \texttt{steps}---release particles at the start of each step specified in STEPS. This keyword may be used in conjunction with other keywords to release particles at the start of multiple time steps.

\item \texttt{fraction}---release particles after the specified fraction of the time step has elapsed. If FRACTION is not set, particles are released at the start of the specified time step(s). FRACTION must be a single value when used with ALL, FIRST, or FREQUENCY. When used with STEPS, FRACTION may be a single value or an array of the same length as STEPS. If a single FRACTION value is provided with STEPS, the fraction applies to all steps.

\end{description}


\end{description}

\vspace{5mm}
\subsubsection{Example Input File}
\lstinputlisting[style=inputfile]{./mf6ivar/examples/prt-prp-example.dat}



\newpage
\subsection{Output Control (OC) Option}
Input to the Output Control Option of the Groundwater Transport Model is read from the file that is specified as type ``OC6'' in the Name File. If no ``OC6'' file is specified, default output control is used. The Output Control Option determines how and when concentrations are printed to the listing file and/or written to a separate binary output file.  Under the default, concentration and overall transport budget are written to the Listing File at the end of every stress period. The default printout format for concentrations is 10G11.4.  The concentrations and overall transport budget are also written to the list file if the simulation terminates prematurely due to failed convergence.

Output Control data must be specified using words.  The numeric codes supported in earlier MODFLOW versions can no longer be used.

For the PRINT and SAVE options of concentration, there is no option to specify individual layers.  Whenever the concentration array is printed or saved, all layers are printed or saved.

\vspace{5mm}
\subsubsection{Structure of Blocks}
\vspace{5mm}

\noindent \textit{FOR EACH SIMULATION}
\lstinputlisting[style=blockdefinition]{./mf6ivar/tex/gwt-oc-options.dat}
\vspace{5mm}
\noindent \textit{FOR ANY STRESS PERIOD}
\lstinputlisting[style=blockdefinition]{./mf6ivar/tex/gwt-oc-period.dat}

\vspace{5mm}
\subsubsection{Explanation of Variables}
\begin{description}
% DO NOT MODIFY THIS FILE DIRECTLY.  IT IS CREATED BY mf6ivar.py 

\item \textbf{Block: OPTIONS}

\begin{description}
\item \texttt{BUDGET}---keyword to specify that record corresponds to the budget.

\item \texttt{FILEOUT}---keyword to specify that an output filename is expected next.

\item \texttt{budgetfile}---name of the output file to write budget information.

\item \texttt{CONCENTRATION}---keyword to specify that record corresponds to concentration.

\item \texttt{concentrationfile}---name of the output file to write conc information.

\item \texttt{PRINT\_FORMAT}---keyword to specify format for printing to the listing file.

\item \texttt{columns}---number of columns for writing data.

\item \texttt{width}---width for writing each number.

\item \texttt{digits}---number of digits to use for writing a number.

\item \texttt{format}---write format can be EXPONENTIAL, FIXED, GENERAL, or SCIENTIFIC.

\end{description}
\item \textbf{Block: PERIOD}

\begin{description}
\item \texttt{iper}---integer value specifying the starting stress period number for which the data specified in the PERIOD block apply.  IPER must be less than or equal to NPER in the TDIS Package and greater than zero.  The IPER value assigned to a stress period block must be greater than the IPER value assigned for the previous PERIOD block.  The information specified in the PERIOD block will continue to apply for all subsequent stress periods, unless the program encounters another PERIOD block.

\item \texttt{SAVE}---keyword to indicate that information will be saved this stress period.

\item \texttt{PRINT}---keyword to indicate that information will be printed this stress period.

\item \texttt{rtype}---type of information to save or print.  Can be BUDGET or CONCENTRATION.

\item \texttt{ocsetting}---specifies the steps for which the data will be saved.

\begin{lstlisting}[style=blockdefinition]
ALL
FIRST
LAST
FREQUENCY <frequency>
STEPS <steps(<nstp)>
\end{lstlisting}

\item \texttt{ALL}---keyword to indicate save for all time steps in period.

\item \texttt{FIRST}---keyword to indicate save for first step in period. This keyword may be used in conjunction with other keywords to print or save results for multiple time steps.

\item \texttt{LAST}---keyword to indicate save for last step in period. This keyword may be used in conjunction with other keywords to print or save results for multiple time steps.

\item \texttt{frequency}---save at the specified time step frequency. This keyword may be used in conjunction with other keywords to print or save results for multiple time steps.

\item \texttt{steps}---save for each step specified in STEPS. This keyword may be used in conjunction with other keywords to print or save results for multiple time steps.

\end{description}


\end{description}

\vspace{5mm}
\subsubsection{Example Input File}
\lstinputlisting[style=inputfile]{./mf6ivar/examples/gwt-oc-example.dat}


\newpage
\subsection{Flow Model Interface (FMI) Package}
Flow Model Interface (FMI) Package information is read from the file that is specified by ``FMI6'' as the file type.  The FMI Package is optional, but if provided, only one FMI Package can be specified for a GWT model.

For most simulations, the GWT Model needs groundwater flows for every cell in the model grid, for all boundary conditions, and for other terms, such as the flow of water in or out of storage.  The FMI Package is the interface between the GWT Model and simulated groundwater flows coming from a corresponding GWF Model that is running concurrently within the simulation or from a binary budget file that was created from a previous GWF model run.  The following are the different FMI simulation cases:

\begin{itemize}

\item Flows are provided by a corresponding GWF Model running in the same simulation---in this case, all groundwater flows are calculated by the corresponding GWF Model and provided through FMI to the transport model.  This is a common use case in which the user wants to run the flow and transport models as part of a single simulation.  The GWF and GWT models must be part of a GWF-GWT Exchange that is listed in mfsim.nam.

\item There is no groundwater flow and the user is interested only in the effects of diffusion, sorption, and decay or production---in this case, FMI should not be provided in the GWT name file and the GWT model should not be listed in any GWF-GWT Exchanges in mfsim.nam.  In this case, all groundwater flows are assumed to be zero and cells are assumed to be fully saturated.  The SSM Package should not be activated in this case, because there can be no sources or sinks of water.  This type of model simulation without an FMI Package is included as an option to represent diffusion, sorption, and decay or growth in the absence of flow groundwater.

\item Flows are provided from a previous GWF model simulation---in this case FMI should be provided in the GWT name file and the head and budget files should be listed in the FMI options block.  In this case, FMI reads the simulated head and flows from these files and makes them available to the transport model.  There are some additional considerations when the heads and flows are provided from binary files.

\begin{itemize}
\item The binary budget file must contain the simulated flows for all of the packages that were included in the GWF model run.  Saving of flows can be activated for all packages by specifying ``SAVE\_FLOWS'' as an option in the GWF name file.  The GWF Output Control Package must also have ``SAVE BUGET ALL'' specified.  The easiest way to ensure that all flows and heads are saved is to use the following simple form of a GWF Output Control file:

\begin{verbatim}
BEGIN OPTIONS
  HEAD FILEOUT mymodel.hds
  BUDGET FILEOUT mymodel.bud
END OPTIONS

BEGIN PERIOD 1
  SAVE HEAD ALL
  SAVE BUDGET ALL
END PERIOD
\end{verbatim}

\item The binary budget file must have the same number of budget terms listed for each time step.  This will always be the case when the binary budget file is created by \mfdot
\item The binary heads file must have heads saved for all layers in the model.  This will always be the case when the binary head file is created by \mfdot  This was not always the case as previous MODFLOW versions allowed different save options for each layer.
\item If the binary budget and head files have more than one time step for a single stress period, then the budget and head information must be contained within the binary file for every time step in the simulation stress period.
\item The binary budget and head files must correspond in terms of information stored for each time step and stress period.
\item If the binary budget and head files have information provided for only the first time step of each stress period, then this information will be used for all time steps in the GWT model run for that stress period.  This makes it possible to provide flows, for example, from a steady state GWF stress period and have those flows used for all steps in the GWT simulation.  Note that this cannot be done when the GWF and GWT models are run in the same simulation, because in that case, both models are solved for each time step in the stress period, as listed in the TDIS Package.
\end{itemize}

\end{itemize}

\vspace{5mm}
\subsubsection{Structure of Blocks}
\lstinputlisting[style=blockdefinition]{./mf6ivar/tex/gwt-fmi-options.dat}

\vspace{5mm}
\subsubsection{Explanation of Variables}
\begin{description}
% DO NOT MODIFY THIS FILE DIRECTLY.  IT IS CREATED BY mf6ivar.py 

\item \textbf{Block: OPTIONS}

\begin{description}
\item \texttt{FLOW\_IMBALANCE\_CORRECTION}---correct for an imbalance in flows by assuming that any residual flow error comes in or leaves at the concentration of the cell.

\item \texttt{GWFBUDGET}---keyword to specify that record corresponds to the gwfbudget input file.

\item \texttt{FILEIN}---keyword to specify that an input filename is expected next.

\item \texttt{gwfbudgetfile}---name of the binary GWF budget file to read as input for the FMI Package

\item \texttt{GWFHEAD}---keyword to specify that record corresponds to the gwfhead input file.

\item \texttt{gwfheadfile}---name of the binary GWF head file to read as input for the FMI Package

\end{description}


\end{description}

\vspace{5mm}
\subsubsection{Example Input File}
\lstinputlisting[style=inputfile]{./mf6ivar/examples/gwt-fmi-example.dat}


