Particle Release Point (PRP) Package information is read from the file that is specified by ``PRP6'' as the file type.  More than one PRP Package can be specified for a PRT model. 

The PRP Package offers multiple ways to specify particle release times.  Particle release times may either be provided explicitly, relative to the simulation start time, or configured relative to the time discretization of stress periods via period block settings.  When multiple ways of specifying release times are used together, the resulting set of release times is the union of the times specified by each method.

\vspace{5mm}
\subsubsection{Structure of Blocks}
\lstinputlisting[style=blockdefinition]{./mf6ivar/tex/prt-prp-options.dat}
\lstinputlisting[style=blockdefinition]{./mf6ivar/tex/prt-prp-dimensions.dat}
%%\lstinputlisting[style=blockdefinition]{./mf6ivar/tex/prt-prp-griddata.dat}
\lstinputlisting[style=blockdefinition]{./mf6ivar/tex/prt-prp-packagedata.dat}
\vspace{5mm}
\noindent \textit{FOR ANY STRESS PERIOD}
\lstinputlisting[style=blockdefinition]{./mf6ivar/tex/prt-prp-period.dat}
\packageperioddescription \: If no PERIOD block is specified for any period, a single particle is released from each release point at the beginning of the simulation.
%\noindent All of the stress period information in a PERIOD block will apply only to that stress period; the information will not continue to apply for subsequent stress periods.  Note that this behavior is different from the simple stress packages (CHD, WEL, DRN, RIV,
%GHB, RCH and EVT) and the advanced stress packages (MAW, SFR, LAK, and UZF).

\vspace{5mm}
\subsubsection{Explanation of Variables}
\begin{description}
% DO NOT MODIFY THIS FILE DIRECTLY.  IT IS CREATED BY mf6ivar.py 

\item \textbf{Block: OPTIONS}

\begin{description}
\item \texttt{BOUNDNAMES}---keyword to indicate that boundary names may be provided with the list of particle release points.

\item \texttt{PRINT\_INPUT}---keyword to indicate that the list of all model stress package information will be written to the listing file immediately after it is read.

\item \texttt{exit\_solve\_tolerance}---the convergence tolerance for iterative solution of particle exit location and time in the generalized Pollock's method.  A value of 0.00001 works well for many problems, but the value that strikes the best balance between accuracy and runtime is problem-dependent.

\item \texttt{LOCAL\_Z}---indicates that ``zrpt'' defines the local z coordinate of the release point within the cell, with value of 0 at the bottom and 1 at the top of the cell.  If the cell is partially saturated at release time, the top of the cell is considered to be the water table elevation (the head in the cell) rather than the top defined by the user.

\item \texttt{TRACK}---keyword to specify that record corresponds to a binary track output file.  Each PRP Package may have a distinct binary track output file.

\item \texttt{FILEOUT}---keyword to specify that an output filename is expected next.

\item \texttt{trackfile}---name of the binary output file to write tracking information.

\item \texttt{TRACKCSV}---keyword to specify that record corresponds to a CSV track output file.  Each PRP Package may have a distinct CSV track output file.

\item \texttt{trackcsvfile}---name of the comma-separated value (CSV) file to write tracking information.

\item \texttt{stoptime}---real value defining the maximum simulation time to which particles in the package can be tracked.  Particles that have not terminated earlier due to another termination condition will terminate when simulation time STOPTIME is reached.  If the last stress period in the simulation consists of more than one time step, particles will not be tracked past the ending time of the last stress period, regardless of STOPTIME.  If the last stress period in the simulation consists of a single time step, it is assumed to be a steady-state stress period, and its ending time will not limit the simulation time to which particles can be tracked.  If STOPTIME and STOPTRAVELTIME are both provided, particles will be stopped if either is reached.

\item \texttt{stoptraveltime}---real value defining the maximum travel time over which particles in the model can be tracked.  Particles that have not terminated earlier due to another termination condition will terminate when their travel time reaches STOPTRAVELTIME.  If the last stress period in the simulation consists of more than one time step, particles will not be tracked past the ending time of the last stress period, regardless of STOPTRAVELTIME.  If the last stress period in the simulation consists of a single time step, it is assumed to be a steady-state stress period, and its ending time will not limit the travel time over which particles can be tracked.  If STOPTIME and STOPTRAVELTIME are both provided, particles will be stopped if either is reached.

\item \texttt{STOP\_AT\_WEAK\_SINK}---is a text keyword to indicate that a particle is to terminate when it enters a cell that is a weak sink.  By default, particles are allowed to pass though cells that are weak sinks.

\item \texttt{istopzone}---integer value defining the stop zone number.  If cells have been assigned IZONE values in the GRIDDATA block, a particle terminates if it enters a cell whose IZONE value matches ISTOPZONE.  An ISTOPZONE value of zero indicates that there is no stop zone.  The default value is zero.

\item \texttt{DRAPE}---is a text keyword to indicate that if a particle's release point is in a cell that happens to be inactive at release time, the particle is to be moved to the topmost active cell below it, if any. By default, a particle is not released into the simulation if its release point's cell is inactive at release time.

\item \texttt{RELEASE\_TIMES}---keyword indicating release times will follow

\item \texttt{times}---times to release, relative to the beginning of the simulation.  RELEASE\_TIMES and RELEASE\_TIMESFILE are mutually exclusive.  Explicit release times and release settings configured in the period block are additive and can be mixed and matched.

\item \texttt{RELEASE\_TIMESFILE}---keyword indicating release times file name will follow

\item \texttt{timesfile}---name of the release times file.  RELEASE\_TIMES and RELEASE\_TIMESFILE are mutually exclusive.  Explicit release times and release settings configured in the period block are additive and can be mixed and matched.

\end{description}
\item \textbf{Block: DIMENSIONS}

\begin{description}
\item \texttt{nreleasepts}---is the number of particle release points.

\end{description}
\item \textbf{Block: PACKAGEDATA}

\begin{description}
\item \texttt{irptno}---integer value that defines the PRP release point number associated with the specified PACKAGEDATA data on the line. IRPTNO must be greater than zero and less than or equal to NRELEASEPTS.  The program will terminate with an error if information for a PRP release point number is specified more than once.

\item \texttt{cellid}---is the cell identifier, and depends on the type of grid that is used for the simulation.  For a structured grid that uses the DIS input file, CELLID is the layer, row, and column.   For a grid that uses the DISV input file, CELLID is the layer and CELL2D number.  If the model uses the unstructured discretization (DISU) input file, CELLID is the node number for the cell.

\item \texttt{xrpt}---real value that defines the x coordinate of the release point in model coordinates.  The (x, y, z) location specified for the release point must lie within the cell that is identified by the specified cellid.

\item \texttt{yrpt}---real value that defines the y coordinate of the release point in model coordinates.  The (x, y, z) location specified for the release point must lie within the cell that is identified by the specified cellid.

\item \texttt{zrpt}---real value that defines the z coordinate of the release point in model coordinates or, if the LOCAL\_Z option is active, in local cell coordinates.  The (x, y, z) location specified for the release point must lie within the cell that is identified by the specified cellid.

\item \texttt{boundname}---name of the particle release point. BOUNDNAME is an ASCII character variable that can contain as many as 40 characters. If BOUNDNAME contains spaces in it, then the entire name must be enclosed within single quotes.

\end{description}
\item \textbf{Block: PERIOD}

\begin{description}
\item \texttt{iper}---integer value specifying the stress period number for which the data specified in the PERIOD block apply. IPER must be less than or equal to NPER in the TDIS Package and greater than zero. The IPER value assigned to a stress period block must be greater than the IPER value assigned for the previous PERIOD block. The information specified in the PERIOD block applies only to that stress period.

\item \texttt{releasesetting}---specifies when to release particles within the stress period.  Overrides package-level RELEASETIME option, which applies to all stress periods. By default, RELEASESETTING configures particles for release at the beginning of the specified time steps. For time-offset releases, provide a FRACTION value.

\begin{lstlisting}[style=blockdefinition]
ALL
FIRST
FREQUENCY <frequency>
STEPS <steps(<nstp)>
[FRACTION <fraction(<nstp)>]
\end{lstlisting}

\item \texttt{ALL}---keyword to indicate release of particles at the start of all time steps in the period.

\item \texttt{FIRST}---keyword to indicate release of particles at the start of the first time step in the period. This keyword may be used in conjunction with other keywords to release particles at the start of multiple time steps.

\item \texttt{frequency}---release particles at the specified time step frequency. This keyword may be used in conjunction with other keywords to release particles at the start of multiple time steps.

\item \texttt{steps}---release particles at the start of each step specified in STEPS. This keyword may be used in conjunction with other keywords to release particles at the start of multiple time steps.

\item \texttt{fraction}---release particles after the specified fraction of the time step has elapsed. If FRACTION is not set, particles are released at the start of the specified time step(s). FRACTION must be a single value when used with ALL, FIRST, or FREQUENCY. When used with STEPS, FRACTION may be a single value or an array of the same length as STEPS. If a single FRACTION value is provided with STEPS, the fraction applies to all steps.

\end{description}


\end{description}

\vspace{5mm}
\subsubsection{Example Input File}
\lstinputlisting[style=inputfile]{./mf6ivar/examples/prt-prp-example.dat}

