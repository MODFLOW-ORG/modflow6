% DO NOT MODIFY THIS FILE DIRECTLY.  IT IS CREATED BY mf6ivar.py 

\item \textbf{Block: OPTIONS}

\begin{description}
\item \texttt{PRINT\_INPUT}---keyword to indicate that the list of n-point cross-section geometry information will be written to the listing file immediately after it is read.

\end{description}
\item \textbf{Block: DIMENSIONS}

\begin{description}
\item \texttt{ngeo}---is the number of n-point cross-section geometry cells.

\end{description}
\item \textbf{Block: PACKAGEDATA}

\begin{description}
\item \texttt{cellid}---is the cell identifier, and depends on the type of grid that is used for the simulation.  For a structured grid that uses the DIS input file, CELLID is the layer, row, and column.   For a grid that uses the DISV input file, CELLID is the layer and CELL2D number.  If the model uses the unstructured discretization (DISU) input file or the linear discretization with vertices (DISL), CELLID is the node number for the cell.

\item \texttt{npoint}---is the number of vertices required to define the cell.  There may be a different number of vertices for each cell.

\item \texttt{xcoords}---is an array of double precision values containing NPOINT x-coordinate data to define the geometry.  XCOORDS data can be specified as absolute or relative values. XCOORDS must be listed in the order that defines the line representing the cross-section. If the first and last XCOORD AND ZCOORD pair are equal, the n-point cross-section is closed.

\item \texttt{zcoords}---is an array of double precision values containing NPOINT z-coordinate data to define the geometry.  ZCOORDS data can be specified as absolute or relative values. ZCOORDS must be listed in the order that defines the line representing the cross-section. If the first and last XCOORD AND ZCOORD pair are equal, the n-point cross-section is closed.

\end{description}

