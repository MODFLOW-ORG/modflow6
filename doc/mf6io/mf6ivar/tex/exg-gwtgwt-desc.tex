% DO NOT MODIFY THIS FILE DIRECTLY.  IT IS CREATED BY mf6ivar.py 

\item \textbf{Block: OPTIONS}

\begin{description}
\item \texttt{auxiliary}---an array of auxiliary variable names.  There is no limit on the number of auxiliary variables that can be provided. Most auxiliary variables will not be used by the GWF-GWF Exchange, but they will be available for use by other parts of the program.  If an auxiliary variable with the name ``ANGLDEGX'' is found, then this information will be used as the angle (provided in degrees) between the connection face normal and the x axis, where a value of zero indicates that a normal vector points directly along the positive x axis.  The connection face normal is a normal vector on the cell face shared between the cell in model 1 and the cell in model 2 pointing away from the model 1 cell.  Additional information on ``ANGLDEGX'' is provided in the description of the DISU Package.  If an auxiliary variable with the name ``CDIST'' is found, then this information will be used as the straight-line connection distance, including the vertical component, between the two cell centers.  Both ANGLDEGX and CDIST are required if specific discharge is calculated for either of the groundwater models.

\item \texttt{advscheme}---scheme used to solve the advection term.  Can be upstream, central, or TVD.  If not specified, upstream weighting is the default weighting scheme.

\item \texttt{XT3D\_OFF}---deactivate the xt3d method and use the faster and less accurate approximation for this exchange.

\item \texttt{XT3D\_RHS}---add xt3d terms to right-hand side, when possible, for this exchange.

\item \texttt{FILEIN}---keyword to specify that an input filename is expected next.

\end{description}
\item \textbf{Block: DIMENSIONS}

\begin{description}
\item \texttt{nexg}---keyword and integer value specifying the number of GWT-GWT exchanges.

\end{description}
\item \textbf{Block: EXCHANGEDATA}

\begin{description}
\item \texttt{cellidm1}---is the cellid of the cell in model 1 as specified in the simulation name file. For a structured grid that uses the DIS input file, CELLIDM1 is the layer, row, and column numbers of the cell.   For a grid that uses the DISV input file, CELLIDM1 is the layer number and CELL2D number for the two cells.  If the model uses the unstructured discretization (DISU) input file, then CELLIDM1 is the node number for the cell.

\item \texttt{cellidm2}---is the cellid of the cell in model 2 as specified in the simulation name file. For a structured grid that uses the DIS input file, CELLIDM2 is the layer, row, and column numbers of the cell.   For a grid that uses the DISV input file, CELLIDM2 is the layer number and CELL2D number for the two cells.  If the model uses the unstructured discretization (DISU) input file, then CELLIDM2 is the node number for the cell.

\item \texttt{ihc}---is an integer flag indicating the direction between node n and all of its m connections. If IHC = 0 then the connection is vertical.  If IHC = 1 then the connection is horizontal. If IHC = 2 then the connection is horizontal for a vertically staggered grid.

\item \texttt{cl1}---is the distance between the center of cell 1 and the its shared face with cell 2.

\item \texttt{cl2}---is the distance between the center of cell 2 and the its shared face with cell 1.

\item \texttt{hwva}---is the horizontal width of the flow connection between cell 1 and cell 2 if IHC $>$ 0, or it is the area perpendicular to flow of the vertical connection between cell 1 and cell 2 if IHC = 0.

\item \texttt{aux}---represents the values of the auxiliary variables for each GWTGWT Exchange. The values of auxiliary variables must be present for each exchange. The values must be specified in the order of the auxiliary variables specified in the OPTIONS block.

\end{description}

