% DO NOT MODIFY THIS FILE DIRECTLY.  IT IS CREATED BY mf6ivar.py 

\item \textbf{Block: OPTIONS}

\begin{description}
\item \texttt{SAVE\_FLOWS}---keyword to indicate that cell-by-cell flow terms will be written to the file specified with ``BUDGET SAVE FILE'' in Output Control.

\item \texttt{EXPORT\_ARRAY\_ASCII}---keyword that specifies input grid arrays, which already support the layered keyword, should be written to layered ascii output files.

\end{description}
\item \textbf{Block: PERIOD}

\begin{description}
\item \texttt{iper}---integer value specifying the starting stress period number for which the data specified in the PERIOD block apply.  IPER must be less than or equal to NPER in the TDIS Package and greater than zero.  The IPER value assigned to a stress period block must be greater than the IPER value assigned for the previous PERIOD block.  The information specified in the PERIOD block will continue to apply for all subsequent stress periods, unless the program encounters another PERIOD block.

\item \texttt{STEADY-STATE}---keyword to indicate that stress period IPER is steady-state. Steady-state conditions will apply until the TRANSIENT keyword is specified in a subsequent BEGIN PERIOD block.

\item \texttt{TRANSIENT}---keyword to indicate that stress period IPER is transient. Transient conditions will apply until the STEADY-STATE keyword is specified in a subsequent BEGIN PERIOD block.

\end{description}

