% DO NOT MODIFY THIS FILE DIRECTLY.  IT IS CREATED BY mf6ivar.py 

\item \textbf{Block: OPTIONS}

\begin{description}
\item \texttt{SAVE\_FLOWS}---keyword to indicate that MST flow terms will be written to the file specified with ``BUDGET FILEOUT'' in Output Control.

\item \texttt{ZERO\_ORDER\_DECAY}---is a text keyword to indicate that zero-order decay will occur.  Use of this keyword requires that DECAY and DECAY\_SORBED (if sorption is active) are specified in the GRIDDATA block.

\item \texttt{LATENT\_HEAT\_VAPORIZATION}---is a text keyword to indicate that cooling associated with evaporation will occur.  Use of this keyword requires that LATHEATVAP are specified in the GRIDDATA block.  While the MST package does not simulate evaporation, multiple other packages in a GWE simulation may.  For example, evaporation may occur from the surface of streams or lakes.  Owing to the energy consumed by the change in phase, the latent heat of vaporization is required.

\end{description}
\item \textbf{Block: GRIDDATA}

\begin{description}
\item \texttt{porosity}---is the mobile domain porosity, defined as the mobile domain pore volume per mobile domain volume.  The GWE model does not support the concept of an immobile domain in the context of heat transport.

\item \texttt{decay}---is the rate coefficient for zero-order decay for the aqueous phase of the mobile domain.  A negative value indicates heat (energy) production.  The dimensions of decay for zero-order decay is energy per length cubed per time.  Zero-order decay will have no effect on simulation results unless zero-order decay is specified in the options block.

\item \texttt{cps}---is the mass-based heat capacity of dry solids (aquifer material). For example, units of J/kg/C may be used (or equivalent).

\item \texttt{rhos}---is a user-specified value of the density of aquifer material not considering the voids. Value will remain fixed for the entire simulation.  For example, if working in SI units, values may be entered as kg/m3.

\end{description}
\item \textbf{Block: PACKAGEDATA}

\begin{description}
\item \texttt{cpw}---is the mass-based heat capacity of the simulated fluid. For example, units of J/kg/C may be used (or equivalent).

\item \texttt{rhow}---is a user-specified value of the density of water. Value will remain fixed for the entire simulation.  For example, if working in SI units, values may be entered as kg/m3.

\item \texttt{latheatvap}---is the user-specified value for the latent heat of vaporization. For example, if working in SI units, values may be entered as kJ/kg.

\end{description}

