% DO NOT MODIFY THIS FILE DIRECTLY.  IT IS CREATED BY mf6ivar.py 

\item \textbf{Block: OPTIONS}

\begin{description}
\item \texttt{HHFORMULATION\_RHS}---use the variable-density hydraulic head formulation and add off-diagonal terms to the right-hand.  This option will prevent the BUY Package from adding asymmetric terms to the flow matrix.

\item \texttt{denseref}---fluid reference density used in the equation of state.  This value is set to 1000. if not specified as an option.

\end{description}
\item \textbf{Block: DIMENSIONS}

\begin{description}
\item \texttt{nrhospecies}---number of species used in density equation of state.  This value must be one or greater.  The value must be one if concentrations are specified using the CONCENTRATION keyword in the PERIOD block below.

\end{description}
\item \textbf{Block: PACKAGEDATA}

\begin{description}
\item \texttt{irhospec}---integer value that defines the species number associated with the specified PACKAGEDATA data on the line. IRHOSPECIES must be greater than zero and less than or equal to NRHOSPECIES. Information must be specified for each of the NRHOSPECIES species or the program will terminate with an error.  The program will also terminate with an error if information for a species is specified more than once.

\item \texttt{drhodc}---real value that defines the slope of the density-concentration line for this species used in the density equation of state.

\item \texttt{crhoref}---real value that defines the reference concentration value used for this species in the density equation of state.

\item \texttt{modelname}---name of GWT model used to simulate a species that will be used in the density equation of state.

\item \texttt{auxspeciesname}---name of the auxiliary variable used by the GWT model to assign species concentrations for boundary packages.  If a density value is needed by the Buoyancy Package then it will use the concentration values in this AUXSPECIESNAME column in the density equation of state.  For advanced stress packages (LAK, SFR, MAW, and UZF) that have an associated advanced transport package (LKT, SFT, MWT, and UZT), the FLOW\_PACKAGE\_AUXILIARY\_NAME option in the advanced transport package can be used to transfer simulated concentrations into the flow package auxiliary variable.  In this manner, the Buoyancy Package can calculate density values for lakes, streams, multi-aquifer wells, and unsaturated zone flow cells using simulated concentrations.

\end{description}
\item \textbf{Block: PERIOD}

\begin{description}
\item \texttt{iper}---integer value specifying the starting stress period number for which the data specified in the PERIOD block apply.  IPER must be less than or equal to NPER in the TDIS Package and greater than zero.  The IPER value assigned to a stress period block must be greater than the IPER value assigned for the previous PERIOD block.  The information specified in the PERIOD block will continue to apply for all subsequent stress periods, unless the program encounters another PERIOD block.

\item \texttt{elevation}---cell center elevation for each cell in the grid.  This may be useful for constructing two-dimensional cross section models using an areal grid.

\item \texttt{concentration}---user-specified concentration array.  If this array is specified, then these concentrations are converted to densities using the equation of state values specified in the PACKAGEDATA block.  If this CONCENTRATION array is specified, then the NRHOSPECIES dimension must be one.

\end{description}

