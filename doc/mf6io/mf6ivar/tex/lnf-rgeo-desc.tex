% DO NOT MODIFY THIS FILE DIRECTLY.  IT IS CREATED BY mf6ivar.py 

\item \textbf{Block: OPTIONS}

\begin{description}
\item \texttt{PRINT\_INPUT}---keyword to indicate that the list of rectangular geometry information will be written to the listing file immediately after it is read.

\end{description}
\item \textbf{Block: DIMENSIONS}

\begin{description}
\item \texttt{ngeo}---is the number of rectangular geometry cells.

\end{description}
\item \textbf{Block: PACKAGEDATA}

\begin{description}
\item \texttt{cellid}---is the cell identifier, and depends on the type of grid that is used for the simulation.  For a structured grid that uses the DIS input file, CELLID is the layer, row, and column.   For a grid that uses the DISV input file, CELLID is the layer and CELL2D number.  If the model uses the unstructured discretization (DISU) input file or the linear discretization with vertices (DISL), CELLID is the node number for the cell.

\item \texttt{width}---horizontal width of the rectangular geometry type.

\item \texttt{height}---vertical height of the rectangular geometry type.

\end{description}

