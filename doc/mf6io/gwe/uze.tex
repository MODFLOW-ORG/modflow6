Unsaturated Zone Energy Transport (UZE) Package information is read from the file that is specified by ``UZE6'' as the file type.  There can be as many UZE Packages as necessary for a GWE model. Each UZE Package is designed to work with flows from a corresponding GWF UZF Package. By default \mf uses the UZE package name to determine which UZF Package corresponds to the UZE Package.  Therefore, the package name of the UZE Package (as specified in the GWE name file) must match with the name of the corresponding UZF Package (as specified in the GWF name file).  Alternatively, the name of the flow package can be specified using the FLOW\_PACKAGE\_NAME keyword in the options block.  The GWE UZE Package cannot be used without a corresponding GWF UZF Package.

The UZE Package does not have a dimensions block; instead, dimensions for the UZE Package are set using the dimensions from the corresponding UZF Package.  For example, the UZF Package requires specification of the number of cells (NUZFCELLS).  UZE sets the number of UZE cells equal to NUZFCELLS.  Therefore, the PACKAGEDATA block below must have NUZFCELLS entries in it.  Furthermore, UZE requires the area of each UZE object to equal to the area of the grid cell hosting the corresponding UZF object.  If the area of a UZF object is different than the host grid cell, the GWE model will exit with an error message indicating which cell is in violation of this condition. This check is unique to UZE, UZT does not require the area of the corresponding UZF object to equal the area of the host grid cell.  If this error condition occurs, users should check whether the AUXMULTNAME option in the OPTIONS block of the corresponding UZF input file is activated.  Finally, users should not create two (or more) UZF objects in the same cell using multiple UZF input packages.

\vspace{5mm}
\subsubsection{Structure of Blocks}
\lstinputlisting[style=blockdefinition]{./mf6ivar/tex/gwe-uze-options.dat}
\lstinputlisting[style=blockdefinition]{./mf6ivar/tex/gwe-uze-packagedata.dat}
\lstinputlisting[style=blockdefinition]{./mf6ivar/tex/gwe-uze-period.dat}

\vspace{5mm}
\subsubsection{Explanation of Variables}
\begin{description}
% DO NOT MODIFY THIS FILE DIRECTLY.  IT IS CREATED BY mf6ivar.py 

\item \textbf{Block: OPTIONS}

\begin{description}
\item \texttt{flow\_package\_name}---keyword to specify the name of the corresponding flow package.  If not specified, then the corresponding flow package must have the same name as this advanced transport package (the name associated with this package in the GWE name file).

\item \texttt{auxiliary}---defines an array of one or more auxiliary variable names.  There is no limit on the number of auxiliary variables that can be provided on this line; however, lists of information provided in subsequent blocks must have a column of data for each auxiliary variable name defined here.   The number of auxiliary variables detected on this line determines the value for naux.  Comments cannot be provided anywhere on this line as they will be interpreted as auxiliary variable names.  Auxiliary variables may not be used by the package, but they will be available for use by other parts of the program.  The program will terminate with an error if auxiliary variables are specified on more than one line in the options block.

\item \texttt{flow\_package\_auxiliary\_name}---keyword to specify the name of an auxiliary variable in the corresponding flow package.  If specified, then the simulated concentrations from this advanced transport package will be copied into the auxiliary variable specified with this name.  Note that the flow package must have an auxiliary variable with this name or the program will terminate with an error.  If the flows for this advanced transport package are read from a file, then this option will have no effect.

\item \texttt{BOUNDNAMES}---keyword to indicate that boundary names may be provided with the list of unsaturated zone flow cells.

\item \texttt{PRINT\_INPUT}---keyword to indicate that the list of unsaturated zone flow information will be written to the listing file immediately after it is read.

\item \texttt{PRINT\_TEMPERATURE}---keyword to indicate that the list of UZF cell temperatures will be printed to the listing file for every stress period in which ``TEMPERATURE PRINT'' is specified in Output Control.  If there is no Output Control option and PRINT\_TEMPERATURE is specified, then temperatures are printed for the last time step of each stress period.

\item \texttt{PRINT\_FLOWS}---keyword to indicate that the list of unsaturated zone flow rates will be printed to the listing file for every stress period time step in which ``BUDGET PRINT'' is specified in Output Control.  If there is no Output Control option and ``PRINT\_FLOWS'' is specified, then flow rates are printed for the last time step of each stress period.

\item \texttt{SAVE\_FLOWS}---keyword to indicate that unsaturated zone flow terms will be written to the file specified with ``BUDGET FILEOUT'' in Output Control.

\item \texttt{TEMPERATURE}---keyword to specify that record corresponds to temperature.

\item \texttt{tempfile}---name of the binary output file to write temperature information.

\item \texttt{BUDGET}---keyword to specify that record corresponds to the budget.

\item \texttt{FILEOUT}---keyword to specify that an output filename is expected next.

\item \texttt{budgetfile}---name of the binary output file to write budget information.

\item \texttt{BUDGETCSV}---keyword to specify that record corresponds to the budget CSV.

\item \texttt{budgetcsvfile}---name of the comma-separated value (CSV) output file to write budget summary information.  A budget summary record will be written to this file for each time step of the simulation.

\item \texttt{TS6}---keyword to specify that record corresponds to a time-series file.

\item \texttt{FILEIN}---keyword to specify that an input filename is expected next.

\item \texttt{ts6\_filename}---defines a time-series file defining time series that can be used to assign time-varying values. See the ``Time-Variable Input'' section for instructions on using the time-series capability.

\item \texttt{OBS6}---keyword to specify that record corresponds to an observations file.

\item \texttt{obs6\_filename}---name of input file to define observations for the UZE package. See the ``Observation utility'' section for instructions for preparing observation input files. Tables \ref{table:gwf-obstypetable} and \ref{table:gwt-obstypetable} lists observation type(s) supported by the UZE package.

\end{description}
\item \textbf{Block: PACKAGEDATA}

\begin{description}
\item \texttt{uzfno}---integer value that defines the UZF cell number associated with the specified PACKAGEDATA data on the line. UZFNO must be greater than zero and less than or equal to NUZFCELLS. Unsaturated zone flow information must be specified for every UZF cell or the program will terminate with an error.  The program also will terminate with an error if information for a UZF cell is specified more than once.

\item \texttt{strt}---real value that defines the starting temperature for the unsaturated zone flow cell.

\item \textcolor{blue}{\texttt{aux}---represents the values of the auxiliary variables for each unsaturated zone flow. The values of auxiliary variables must be present for each unsaturated zone flow. The values must be specified in the order of the auxiliary variables specified in the OPTIONS block.  If the package supports time series and the Options block includes a TIMESERIESFILE entry (see the ``Time-Variable Input'' section), values can be obtained from a time series by entering the time-series name in place of a numeric value.}

\item \texttt{boundname}---name of the unsaturated zone flow cell.  BOUNDNAME is an ASCII character variable that can contain as many as 40 characters.  If BOUNDNAME contains spaces in it, then the entire name must be enclosed within single quotes.

\end{description}
\item \textbf{Block: PERIOD}

\begin{description}
\item \texttt{iper}---integer value specifying the starting stress period number for which the data specified in the PERIOD block apply.  IPER must be less than or equal to NPER in the TDIS Package and greater than zero.  The IPER value assigned to a stress period block must be greater than the IPER value assigned for the previous PERIOD block.  The information specified in the PERIOD block will continue to apply for all subsequent stress periods, unless the program encounters another PERIOD block.

\item \texttt{uzfno}---integer value that defines the UZF cell number associated with the specified PERIOD data on the line. UZFNO must be greater than zero and less than or equal to NUZFCELLS.

\item \texttt{uzesetting}---line of information that is parsed into a keyword and values.  Keyword values that can be used to start the UZESETTING string include: STATUS, TEMPERATURE, INFILTRATION, UZET, and AUXILIARY.  These settings are used to assign the temperature associated with the corresponding flow terms.  Temperatures cannot be specified for all flow terms.

\begin{lstlisting}[style=blockdefinition]
STATUS <status>
TEMPERATURE <@temperature@>
INFILTRATION <@infiltration@>
UZET <@uzet@>
AUXILIARY <auxname> <@auxval@> 
\end{lstlisting}

\item \texttt{status}---keyword option to define UZF cell status.  STATUS can be ACTIVE, INACTIVE, or CONSTANT. By default, STATUS is ACTIVE, which means that temperature will be calculated for the UZF cell.  If a UZF cell is inactive, then there will be no energy fluxes into or out of the UZF cell and the inactive value will be written for the UZF cell temperature.  If a UZF cell is constant, then the temperature for the UZF cell will be fixed at the user specified value.

\item \textcolor{blue}{\texttt{temperature}---real or character value that defines the temperature for the unsaturated zone flow cell. The specified TEMPERATURE is only applied if the unsaturated zone flow cell is a constant temperature cell. If the Options block includes a TIMESERIESFILE entry (see the ``Time-Variable Input'' section), values can be obtained from a time series by entering the time-series name in place of a numeric value.}

\item \textcolor{blue}{\texttt{infiltration}---real or character value that defines the temperature of the infiltration $(^\circ C)$ for the UZF cell. If the Options block includes a TIMESERIESFILE entry (see the ``Time-Variable Input'' section), values can be obtained from a time series by entering the time-series name in place of a numeric value.}

\item \textcolor{blue}{\texttt{uzet}---real or character value that states what fraction of the simulated unsaturated zone evapotranspiration is associated with evaporation.  The evaporative losses from the unsaturated zone moisture content will have an evaporative cooling effect on the GWE cell from which the evaporation originated. If this value is larger than 1, then it will be reset to 1.  If the Options block includes a TIMESERIESFILE entry (see the ``Time-Variable Input'' section), values can be obtained from a time series by entering the time-series name in place of a numeric value.}

\item \texttt{AUXILIARY}---keyword for specifying auxiliary variable.

\item \texttt{auxname}---name for the auxiliary variable to be assigned AUXVAL.  AUXNAME must match one of the auxiliary variable names defined in the OPTIONS block. If AUXNAME does not match one of the auxiliary variable names defined in the OPTIONS block the data are ignored.

\item \textcolor{blue}{\texttt{auxval}---value for the auxiliary variable. If the Options block includes a TIMESERIESFILE entry (see the ``Time-Variable Input'' section), values can be obtained from a time series by entering the time-series name in place of a numeric value.}

\end{description}


\end{description}

\vspace{5mm}
\subsubsection{Example Input File}
\lstinputlisting[style=inputfile]{./mf6ivar/examples/gwe-uze-example.dat}

\vspace{5mm}
\subsubsection{Available observation types}
Unsaturated Zone Energy Transport Package observations include UZF cell temperature and all of the terms that contribute to the continuity equation for each UZF cell. Additional UZE Package observations include energy flow rates for individual UZF cells, or groups of UZF cells. The data required for each UZE Package observation type is defined in table~\ref{table:gwe-uzeobstype}. Negative and positive values for \texttt{uzt} observations represent a loss from and gain to the GWE model, respectively. For all other flow terms, negative and positive values represent a loss from and gain from the UZE package, respectively.

\begin{longtable}{p{2cm} p{2.75cm} p{2cm} p{1.25cm} p{7cm}}
\caption{Available UZE Package observation types} \tabularnewline

\hline
\hline
\textbf{Stress Package} & \textbf{Observation type} & \textbf{ID} & \textbf{ID2} & \textbf{Description} \\
\hline
\endfirsthead

\captionsetup{textformat=simple}
\caption*{\textbf{Table \arabic{table}.}{\quad}Available UZE Package observation types.---Continued} \tabularnewline

\hline
\hline
\textbf{Stress Package} & \textbf{Observation type} & \textbf{ID} & \textbf{ID2} & \textbf{Description} \\
\hline
\endhead


\hline
\endfoot

% general APT observations
UZE & temperature & uzeno or boundname & -- & uze cell temperature. If boundname is specified, boundname must be unique for each uze cell. \\
UZE & flow-ja-face & uzeno or boundname & uzeno or -- & Energy flow between two uze cells.  If a boundname is specified for ID1, then the result is the total energy flow for all uze cells. If a boundname is specified for ID1 then ID2 is not used.\\
UZE & storage & uzeno or boundname & -- & Simulated energy storage flow rate for a uze cell or group of uze cells. \\
UZE & constant & uzeno or boundname & -- & Simulated energy constant-flow rate for a uze cell or a group of uze cells. \\
UZE & from-mvr & uzeno or boundname & -- & Simulated energy inflow into a uze cell or group of uze cells from the MVE package. Energy inflow is calculated as the product of provider temperature and the mover flow rate. \\
UZE & uze & uzeno or boundname & -- & Energy flow rate for a uze cell or group of uze cells and its aquifer connection(s). \\

%observations specific to the uze package
% infiltration rej-inf uzet rej-inf-to-mvr
UZE & infiltration & uzeno or boundname & -- & Infiltration rate applied to a uze cell or group of uze cells multiplied by the infiltration temperature. \\
UZE & rej-inf & uzeno or boundname & -- & Rejected infiltration rate applied to a uze cell or group of uze cells multiplied by the infiltration temperature. \\
UZE & uzet & uzeno or boundname & -- & Unsaturated zone evapotranspiration rate applied to a uze cell or group of uze cells multiplied by the uze cell temperature. \\
UZE & rej-inf-to-mvr & uzeno or boundname & -- & Rejected infiltration rate applied to a uze cell or group of uze cells multiplied by the infiltration temperature that is sent to the mover package. \\

\label{table:gwe-uzeobstype}
\end{longtable}

\vspace{5mm}
\subsubsection{Example Observation Input File}
\lstinputlisting[style=inputfile]{./mf6ivar/examples/gwe-uze-example-obs.dat}


