Like GWT \citep{modflow6gwt}, the GWE Model simulates three-dimensional transport in flowing groundwater.  The primary difference between GWT and GWE is that heat (i.e., temperature), instead of concentration, is the simulated ``species.'' As such, the GWE Model solves the heat transport equation using numerical methods and a generalized control-volume finite-difference approach, which can be used with regular MODFLOW grids (DIS Package) or with unstructured grids (DISV and DISU Packages).  The GWE Model is designed to work with most of the new capabilities released with the GWF Model, including the Newton flow formulation, XT3D \citep{modflow6xt3d}, unstructured grids, advanced packages, the movement of water between packages.  The GWF and GWE (and, if active, GWT) models operate simultaneously during a \mf simulation to represent coupled groundwater flow and heat transport.  The GWE Model can also run separately from a GWF Model by reading the heads and flows saved by a previously run GWF Model.  The GWE model is also capable of working with the flows from another groundwater flow model as long as the cell-by-cell and boundary flows and groundwater heads are written to ``linker'' files in the correct format.  

The purpose of the GWE Model is to calculate changes in groundwater temperature in both space and time.  Groundwater temperature within an aquifer can change in response to different energy transport processes.  These processes include (1) convective (advective) transport of heat with flowing groundwater, (2) the combined hydrodynamic dispersion processes of velocity-dependent mechanical dispersion and conduction (analogous to chemical diffusion), (3) thermal equilibrium with the aquifer matrix, (4) mixing with fluids from groundwater sources and sinks, and (5) direct addition of thermal energy.

For GWE, the energy present in the aquifer is assumed to instantaneously equilibrate between the aqueous and solid phase domains.  For example, a pulse of heat convecting through an aquifer will be retarded through thermal equilibration with the aquifer material.  Conversely, the introduction of cold groundwater into a previously warm region of the aquifer will warmup, at least in part, as energy within the aquifer matrix transfers to the aqueous phase.  Unlike GWT, the GWE Model type does not support an immobile domain.  The energy that is transferred between the aqeous and solid phases of the groundwater system are tracked in the GWE Model budget.

This section describes the data files for a \mf Groundwater Energy Transport (GWE) Model.  A GWE Model is added to the simulation by including a GWE entry in the MODELS block of the simulation name file.  There are three types of spatial discretization approaches that can be used with the GWE Model: DIS, DISV, and DISU.  The input instructions for these three packages are not described here in this section on GWE Model input; input instructions for these three packages are described in the section on GWF Model input.

The GWE Model is designed to permit input to be gathered, as it is needed, from many different files.  Likewise, results from the model calculations can be written to a number of output files. The GWE Model Listing File is a key file to which the GWE model output is written.  As \mf runs, information about the GWE Model is written to the GWE Model Listing File, including much of the input data (as a record of the simulation) and calculated results.  Details about the files used by each package are provided in this section.

The GWE Model reads a file called the Name File, which specifies most of the files that will be used in a groundwater energy transport simulation. Several files are always required whereas other files are optional depending on the question(s) being addressed by the model. The Output Control Package receives instructions from the user to control the amount and frequency of output.  Details about the Name File and the Output Control Package are described in this section.

For the GWE Model, ``flows'' (unless stated otherwise) represent the ``flow'' of energy, often expressed in units of energy (e.g., joules) per time, rather than groundwater flow.  

\subsection{Information for Existing Heat Transport Modelers}
An important goal of the \mf GWE Model is to alleviate the need for ``parameter equivalents'' when simulating heat transport in groundwater systems.  In the past, codes like HST3D \citep{kipp1987} or VS2DH \citep{healy1996} simulated energy transport directly by supporting the use of native heat transport units.  For example, users could directly specify thermal conductivity of the fluid and solid phases, as well as the heat capacity of both phases.  Alternatively, codes like MT3DMS \citep{zheng1999mt3dms}, MT3D-USGS \citep{mt3dusgs}, and MODFLOW-USG \citep{modflowusg} could be used to simulate the movement of heat in groundwater, but required users to leverage existing variables as surrogates for heat transport.  For example, the molecular diffusion parameter may be used as a surrogate for simulating thermal conduction in an aquifer \citep{mazheng2010, hechtmendez}. 

The following list summarizes important aspects of GWE for simulating heat transport with \mf:

\begin{enumerate}

\item The GWE Model uses parameters that are native to heat transport, including thermal conductivity of water, heat capacity of water, thermal conductivity of the aquifer material, heat capacity of of the aquifer material, and latent heat of vaporization. Therefore, users do not need to pre-calculate ``parameter equivalents'' when generating GWE model input; users can instead enter native parameter values that are readily available.

\item Thermal energy transport budgets written to the \mf list file are reported in units of energy (e.g., joules).  Previously, using a program like MT3D-USGS \citep{mt3dusgs} to simulate heat transport, units in the list file budget did not correspond to thermal energy, but were reported in units of $\frac{m^{3 \;\circ}C}{d}$. To convert to thermal energy units, values in the list file had to be post-processed by multiplying each line item by the density of water ($\rho_w$) and the heat capacity of water ($C_p$) \citep{langevin2008seawat}.

\item Thermal equilibrium between the aqueous and solid phases is assumed.  Thus, simulated temperatures are representive of both phases.  As a result, thermal conduction between adjacent cells may still occur even in the absense of convection.  

\item In GWE, dry cells (devoid of groundwater) remain active for simulating thermal conduction. For example, energy (heat) transfer will be simulated between a partially saturated cell (i.e., ``water-table'' cell) and an overlying dry cell. In this way, a more full accounting of various heat transport processes is represented in the subsurface.  Moreover, this approach readily supports heat transport in the unsaturated-zone when the UZE (unsaturated-zone energy transport) Package is active.  

\item Heat transport is supported for all five of the advanced GWF packages using the following packages in GWE: (1) streamflow energy transport, SFE Package; (2) lake energy transport, LKE Package; (3) multi-aquifer well energy transport, MWE Package; (4) unsaturated zone energy transport, UZE Package; and the (5) Water Mover Package, MVE.  Similar to GWT, GWE will simulate heat transfer between an advanced package and the groundwater system via groundwater surface-water exchange; however, GWE also simulates a conductive transfer of heat between an advanced package feature and the aquifer.  To take advantage of this functionality, users must specify the thermal conductivity of the material separating a stream from the aquifer, for example, the thermal conductivity of the streambed (or lakebed), as well as the thickness of the streambed (or lakebed).  As with the advanced GWT packages, GWE simulates thermal convection between package features, such as between two stream reaches for example.  Also, dispersive heat transport among among advanced package features is not represented, similar to GWT.

\item Where the GWF model simulates evaporation from an open body of water, for example from the surface of a stream or lake, the latent heat of vaporization may be used to simulate evaporative cooling.  As water is converted from liquid to gas, the energy required by the phase change is drawn from the remaining body of water and the resulting cool down is calculated.

\end{enumerate}

Many of the same considerations listed for the GWT model should be kept in mind when developing a GWE model. For convenience, many of those considerations are adapted for GWE and repeated here.

\begin{enumerate}

\item A GWE Model can access flows calculated by a GWF Model that is running in the same simulation as the GWE Model.  Alternatively, a GWE Model can read binary head and budget files created from a previous GWF Model simulation (provided these files contain all of the required information for all time steps); there is no specialized flow and transport link file \citep{zheng2001modflow} as there is for MT3D.  Details on these two different use cases are provided in the chapter on the FMI Package.

\item The GWE Model is based on a generalized control-volume finite-difference method, which means that heat transport can be simulated using regular MODFLOW grids consisting of layers, rows, and columns, or heat transport can be simulated using unstructured grids.

\item GWE and GWT use the same advection package source code.  As a result, advection can be simulated using central-in-space weighting, upstream weighting, or an implicit second-order TVD scheme.  Currently, neither the GWE or GWT models can use a Method of Characteristics (particle-based approaches) or an explicit TVD scheme to simulate convective (or advective) transport.  Consequently, the GWE Model may require a higher level of spatial discretization than other transport models that use higher order terms for advection dominated systems.  This can be an important limitation in problems involving sharp heat fronts. 

\item The Viscosity Package may reference a GWE model directly for adjusting the viscosity-affected groundwater flow.   

\item GWE and GWT use the same Source and Sink Mixing (SSM) Package for representing the effects of GWF stress package inflows and outflows on simulated temperatures and concentrations.  In a GWE simulation, there are two ways in which users can assign concentrations to the individual features in these stress package.  The first way is to activate a temperature auxiliary variable in the corresponding GWF stress package.  In the SSM input file, the user provides the name of the auxiliary variable to be used for temperature.  The second way is to create a special SPC file, which contains user-assigned time-varying temperatures for stress package features.

\item The GWE model includes an MST Package, but does not include an IST Package.  Heat transport-related parameters such as thermal conductivities and heat capacities are specified in the MST Package.

\item A GWE-GWE Exchange (introduced in version 6.5.0) can be used to tightly couple multiple heat transport models, as might be done in a nested grid configuration.  

\item There is no option to automatically run the GWE Model to steady state using a single time step.  This is an option available in MT3DMS \citep{zheng2010supplemental}.  Steady state conditions must be determined by running the transport model under transient conditions until temperatures stabilize.

\item As is the case with GWT, the GWE Model has not yet been programmed to work with the Skeletal Storage, Compaction, and Subsidence (CSUB) Package for the GWF Model.  

\item There are many other differences between the \mf GWE Model and other solute transport models that work with MODFLOW, especially with regards to program design and input and output.  Descriptions for the GWE input and output are described here.

\end{enumerate}

\subsection{Units of Length and Time}
The GWF Model formulates the groundwater flow equation without using prescribed length and time units. Any consistent units of length and time can be used when specifying the input data for a simulation. This capability gives a certain amount of freedom to the user, but care must be exercised to avoid mixing units.  The program cannot detect the use of inconsistent units.

\subsection{Thermal Energy Budget}
A summary of all inflow (sources) and outflow (sinks) of thermal energy is referred to as an energy budget.  \mf calculates an energy budget for the overall model as a check on the acceptability of the solution, and to provide a summary of the sources and sinks of energy to the flow system.  The energy budget is printed to the GWE Model Listing File for specified time steps.

\subsection{Time Stepping}

For the present implementation of the GWE Model, all terms in the heat transport equation are solved implicitly.  With the implicit approach applied to the transport equation, it is possible to take relatively large time steps and efficiently obtain a stable solution.  If the time steps are too large, however, accuracy of the model results will suffer, so there is usually some compromise required between the desired level of accuracy and length of the time step.  An assessment of accuracy can be performed by simply running simulations with shorter time steps and comparing results.

In \mf time step lengths are controlled by the user and specified in the Temporal Discretization (TDIS) input file.  When the flow model and heat transport model are included in the same simulation, then the length of the time step specified in TDIS is used for both models.  If the GWE Model runs in a separate simulation from the GWE Model, then the time steps used for the heat transport model can be different, and likely shorter, than the time steps used for the flow solution.  Instructions for specifying time steps are described in the TDIS section of this user guide; additional information on GWF and GWE configurations are in the Flow Model Interface section.  



\newpage
\subsection{GWE Model Name File}
The GWT Model Name File specifies the options and packages that are active for a GWT model.  The Name File contains two blocks: OPTIONS  and PACKAGES. The length of each line must be 299 characters or less. The lines in each block can be in any order.  Files listed in the PACKAGES block must exist when the program starts. 

Comment lines are indicated when the first character in a line is one of the valid comment characters.  Commented lines can be located anywhere in the file. Any text characters can follow the comment character. Comment lines have no effect on the simulation; their purpose is to allow users to provide documentation about a particular simulation. 

\vspace{5mm}
\subsubsection{Structure of Blocks}
\lstinputlisting[style=blockdefinition]{./mf6ivar/tex/gwt-nam-options.dat}
\lstinputlisting[style=blockdefinition]{./mf6ivar/tex/gwt-nam-packages.dat}

\vspace{5mm}
\subsubsection{Explanation of Variables}
\begin{description}
% DO NOT MODIFY THIS FILE DIRECTLY.  IT IS CREATED BY mf6ivar.py 

\item \textbf{Block: OPTIONS}

\begin{description}
\item \texttt{list}---is name of the listing file to create for this GWT model.  If not specified, then the name of the list file will be the basename of the GWT model name file and the '.lst' extension.  For example, if the GWT name file is called ``my.model.nam'' then the list file will be called ``my.model.lst''.

\item \texttt{PRINT\_INPUT}---keyword to indicate that the list of all model stress package information will be written to the listing file immediately after it is read.

\item \texttt{PRINT\_FLOWS}---keyword to indicate that the list of all model package flow rates will be printed to the listing file for every stress period time step in which ``BUDGET PRINT'' is specified in Output Control.  If there is no Output Control option and ``PRINT\_FLOWS'' is specified, then flow rates are printed for the last time step of each stress period.

\item \texttt{SAVE\_FLOWS}---keyword to indicate that all model package flow terms will be written to the file specified with ``BUDGET FILEOUT'' in Output Control.

\item \texttt{export\_netcdf}---keyword that specifies timeseries data for the dependent variable should be written to a model output netcdf file.  No value or ``UGRID'' (ugrid based export) values are supported.

\end{description}
\item \textbf{Block: PACKAGES}

\begin{description}
\item \texttt{ftype}---is the file type, which must be one of the following character values shown in table~\ref{table:ftype-gwt}. Ftype may be entered in any combination of uppercase and lowercase.

\item \texttt{fname}---is the name of the file containing the package input.  The path to the file should be included if the file is not located in the folder where the program was run.

\item \texttt{pname}---is the user-defined name for the package. PNAME is restricted to 16 characters.  No spaces are allowed in PNAME.  PNAME character values are read and stored by the program for stress packages only.  These names may be useful for labeling purposes when multiple stress packages of the same type are located within a single GWT Model.  If PNAME is specified for a stress package, then PNAME will be used in the flow budget table in the listing file; it will also be used for the text entry in the cell-by-cell budget file.  PNAME is case insensitive and is stored in all upper case letters.

\end{description}


\end{description}

\begin{table}[H]
\caption{Ftype values described in this report.  The \texttt{Pname} column indicates whether or not a package name can be provided in the name file}
\small
\begin{center}
\begin{tabular*}{\columnwidth}{l l l}
\hline
\hline
Ftype & Input File Description & \texttt{Pname}\\
\hline
DIS6 & Rectilinear Discretization Input File \\
DISV6 & Discretization by Vertices Input File \\
DISU6 & Unstructured Discretization Input File \\
FMI6 & Flow Model Interface Package &  \\ 
IC6 & Initial Conditions Package \\
OC6 & Output Control Option \\
ADV6 & Advection Package \\ 
DSP6 & Dispersion Package \\ 
SSM6 & Source and Sink Mixing Package \\ 
MST6 & Mobile Storage and Transfer Package \\
IST6 & Immobile Storage and Transfer Package & * \\
CNC6 & Constant Concentration Package & * \\ 
SRC6 & Mass Source Loading Package & * \\ 
LKT6 & Lake Transport Package & * (must be same name as corresponding GWF LAK Package) \\ 
SFT6 & Streamflow Transport Package & * (must be same name as corresponding GWF SFR Package) \\ 
MWT6 & Multi-Aquifer Well Transport Package & * (must be same name as corresponding GWF MAW Package) \\ 
UZT6 & Unsaturated Zone Transport Package & * (must be same name as corresponding GWF UZF Package) \\ 
OBS6 & Observations Option \\
\hline 
\end{tabular*}
\label{table:ftype}
\end{center}
\normalsize
\end{table}

\vspace{5mm}
\subsubsection{Example Input File}
\lstinputlisting[style=inputfile]{./mf6ivar/examples/gwt-nam-example.dat}



%\newpage
%\subsection{Structured Discretization (DIS) Input File}
%\input{gwf/dis}

%\newpage
%\subsection{Discretization with Vertices (DISV) Input File}
%\input{gwf/disv}

%\newpage
%\subsection{Unstructured Discretization (DISU) Input File}
%Discretization information for unstructured grids is read from the file that is specified by ``DISU6'' as the file type.  Only one discretization input file (DISU6, DISV6 or DIS6) can be specified for a model.

The shape and position of each cell can be defined using vertices.  This information is optional and is only read if the number of vertices (NVERT) in the DIMENSIONS block is specified and is assigned a value larger than zero.  If the vertices and two-dimensional cell information is provided in this file, then this information is also written to the binary grid file.  Providing this information may be useful for other postprocessing programs that read the binary grid file.

The DISU Package does not support the concept of layers, which is different from the DISU implementation in MODFLOW-USG.  In \mf~all grid input and output for models that use the DISU Package is entered or written as a one-dimensional array of size nodes.

The DISU VERTICES and CELL2D blocks are not required for all simulations.  These blocks are required if the XT3D or the SAVE\_SPECIFIC\_DISCHARGE options are specified in the NPF Package.  In general, it is recommended to include the VERTICES and CELL2D blocks. 

\vspace{5mm}
\subsubsection{Structure of Blocks}
\lstinputlisting[style=blockdefinition]{./mf6ivar/tex/gwf-disu-options.dat}
\lstinputlisting[style=blockdefinition]{./mf6ivar/tex/gwf-disu-dimensions.dat}
\lstinputlisting[style=blockdefinition]{./mf6ivar/tex/gwf-disu-griddata.dat}
\lstinputlisting[style=blockdefinition]{./mf6ivar/tex/gwf-disu-connectiondata.dat}
\lstinputlisting[style=blockdefinition]{./mf6ivar/tex/gwf-disu-vertices.dat}
\lstinputlisting[style=blockdefinition]{./mf6ivar/tex/gwf-disu-cell2d.dat}

\vspace{5mm}
\subsubsection{Explanation of Variables}
\begin{description}
% DO NOT MODIFY THIS FILE DIRECTLY.  IT IS CREATED BY mf6ivar.py 

\item \textbf{Block: OPTIONS}

\begin{description}
\item \texttt{length\_units}---is the length units used for this model.  Values can be ``FEET'', ``METERS'', or ``CENTIMETERS''.  If not specified, the default is ``UNKNOWN''.

\item \texttt{NOGRB}---keyword to deactivate writing of the binary grid file.

\item \texttt{xorigin}---x-position of the origin used for model grid vertices.  This value should be provided in a real-world coordinate system.  A default value of zero is assigned if not specified.  The value for XORIGIN does not affect the model simulation, but it is written to the binary grid file so that postprocessors can locate the grid in space.

\item \texttt{yorigin}---y-position of the origin used for model grid vertices.  This value should be provided in a real-world coordinate system.  If not specified, then a default value equal to zero is used.  The value for YORIGIN does not affect the model simulation, but it is written to the binary grid file so that postprocessors can locate the grid in space.

\item \texttt{angrot}---counter-clockwise rotation angle (in degrees) of the model grid coordinate system relative to a real-world coordinate system.  If not specified, then a default value of 0.0 is assigned.  The value for ANGROT does not affect the model simulation, but it is written to the binary grid file so that postprocessors can locate the grid in space.

\end{description}
\item \textbf{Block: DIMENSIONS}

\begin{description}
\item \texttt{nodes}---is the number of cells in the model grid.

\item \texttt{nja}---is the sum of the number of connections and NODES.  When calculating the total number of connections, the connection between cell n and cell m is considered to be different from the connection between cell m and cell n.  Thus, NJA is equal to the total number of connections, including n to m and m to n, and the total number of cells.

\item \texttt{nvert}---is the total number of (x, y) vertex pairs used to define the plan-view shape of each cell in the model grid.  If NVERT is not specified or is specified as zero, then the VERTICES and CELL2D blocks below are not read.  NVERT and the accompanying VERTICES and CELL2D blocks should be specified for most simulations.  If the XT3D or SAVE\_SPECIFIC\_DISCHARGE options are specified in the NPF Package, then this information is required.

\end{description}
\item \textbf{Block: GRIDDATA}

\begin{description}
\item \texttt{top}---is the top elevation for each cell in the model grid.

\item \texttt{bot}---is the bottom elevation for each cell.

\item \texttt{area}---is the cell surface area (in plan view).

\end{description}
\item \textbf{Block: CONNECTIONDATA}

\begin{description}
\item \texttt{iac}---is the number of connections (plus 1) for each cell.  The sum of all the entries in IAC must be equal to NJA.

\item \texttt{ja}---is a list of cell number (n) followed by its connecting cell numbers (m) for each of the m cells connected to cell n. The number of values to provide for cell n is IAC(n).  This list is sequentially provided for the first to the last cell. The first value in the list must be cell n itself, and the remaining cells must be listed in an increasing order (sorted from lowest number to highest).  Note that the cell and its connections are only supplied for the GWF cells and their connections to the other GWF cells.  Also note that the JA list input may be divided such that every node and its connectivity list can be on a separate line for ease in readability of the file. To further ease readability of the file, the node number of the cell whose connectivity is subsequently listed, may be expressed as a negative number, the sign of which is subsequently converted to positive by the code.

\item \texttt{ihc}---is an index array indicating the direction between node n and all of its m connections.  If IHC = 0 then cell n and cell m are connected in the vertical direction.  Cell n overlies cell m if the cell number for n is less than m; cell m overlies cell n if the cell number for m is less than n.  If IHC = 1 then cell n and cell m are connected in the horizontal direction.  If IHC = 2 then cell n and cell m are connected in the horizontal direction, and the connection is vertically staggered.  A vertically staggered connection is one in which a cell is horizontally connected to more than one cell in a horizontal connection.

\item \texttt{cl12}---is the array containing connection lengths between the center of cell n and the shared face with each adjacent m cell.

\item \texttt{hwva}---is a symmetric array of size NJA.  For horizontal connections, entries in HWVA are the horizontal width perpendicular to flow.  For vertical connections, entries in HWVA are the vertical area for flow.  Thus, values in the HWVA array contain dimensions of both length and area.  Entries in the HWVA array have a one-to-one correspondence with the connections specified in the JA array.  Likewise, there is a one-to-one correspondence between entries in the HWVA array and entries in the IHC array, which specifies the connection type (horizontal or vertical).  Entries in the HWVA array must be symmetric; the program will terminate with an error if the value for HWVA for an n to m connection does not equal the value for HWVA for the corresponding n to m connection.

\item \texttt{angldegx}---is the angle (in degrees) between the horizontal x-axis and the outward normal to the face between a cell and its connecting cells. The angle varies between zero and 360.0 degrees, where zero degrees points in the positive x-axis direction, and 90 degrees points in the positive y-axis direction.  ANGLDEGX is only needed if horizontal anisotropy is specified in the NPF Package, if the XT3D option is used in the NPF Package, or if the SAVE\_SPECIFIC\_DISCHARGE option is specifed in the NPF Package.  ANGLDEGX does not need to be specified if these conditions are not met.  ANGLDEGX is of size NJA; values specified for vertical connections and for the diagonal position are not used.  Note that ANGLDEGX is read in degrees, which is different from MODFLOW-USG, which reads a similar variable (ANGLEX) in radians.

\end{description}
\item \textbf{Block: VERTICES}

\begin{description}
\item \texttt{iv}---is the vertex number.  Records in the VERTICES block must be listed in consecutive order from 1 to NVERT.

\item \texttt{xv}---is the x-coordinate for the vertex.

\item \texttt{yv}---is the y-coordinate for the vertex.

\end{description}
\item \textbf{Block: CELL2D}

\begin{description}
\item \texttt{icell2d}---is the cell2d number.  Records in the CELL2D block must be listed in consecutive order from 1 to NODES.

\item \texttt{xc}---is the x-coordinate for the cell center.

\item \texttt{yc}---is the y-coordinate for the cell center.

\item \texttt{ncvert}---is the number of vertices required to define the cell.  There may be a different number of vertices for each cell.

\item \texttt{icvert}---is an array of integer values containing vertex numbers (in the VERTICES block) used to define the cell.  Vertices must be listed in clockwise order.

\end{description}


\end{description}

\vspace{5mm}
\subsubsection{Example Input File}
\lstinputlisting[style=inputfile]{./mf6ivar/examples/gwf-disu-example.dat}



\newpage
\subsection{Initial Conditions (IC) Package}
Initial Conditions (IC) Package information is read from the file that is specified by ``IC6'' as the file type.  Only one IC Package can be specified for a LNF model.

\vspace{5mm}
\subsubsection{Structure of Blocks}
%\lstinputlisting[style=blockdefinition]{./mf6ivar/tex/gwf-ic-options.dat}
\lstinputlisting[style=blockdefinition]{./mf6ivar/tex/lnf-ic-griddata.dat}

\vspace{5mm}
\subsubsection{Explanation of Variables}
\begin{description}
% DO NOT MODIFY THIS FILE DIRECTLY.  IT IS CREATED BY mf6ivar.py 

\item \textbf{Block: GRIDDATA}

\begin{description}
\item \texttt{strt}---is the initial (starting) head---that is, head at the beginning of the LNF Model simulation.  STRT must be specified for all simulations, including steady-state simulations. One value is read for every model cell. For simulations in which the first stress period is steady state, the values used for STRT generally do not affect the simulation (exceptions may occur if cells go dry and (or) rewet). The execution time, however, will be less if STRT includes hydraulic heads that are close to the steady-state solution.  A head value lower than the cell bottom can be provided if a cell should start as dry.

\end{description}


\end{description}

\vspace{5mm}
\subsubsection{Example Input File}
\lstinputlisting[style=inputfile]{./mf6ivar/examples/lnf-ic-example.dat}



\newpage
\subsection{Output Control (OC) Option}
Input to the Output Control Option of the Groundwater Transport Model is read from the file that is specified as type ``OC6'' in the Name File. If no ``OC6'' file is specified, default output control is used. The Output Control Option determines how and when concentrations are printed to the listing file and/or written to a separate binary output file.  Under the default, concentration and overall transport budget are written to the Listing File at the end of every stress period. The default printout format for concentrations is 10G11.4.  The concentrations and overall transport budget are also written to the list file if the simulation terminates prematurely due to failed convergence.

Output Control data must be specified using words.  The numeric codes supported in earlier MODFLOW versions can no longer be used.

For the PRINT and SAVE options of concentration, there is no option to specify individual layers.  Whenever the concentration array is printed or saved, all layers are printed or saved.

\vspace{5mm}
\subsubsection{Structure of Blocks}
\vspace{5mm}

\noindent \textit{FOR EACH SIMULATION}
\lstinputlisting[style=blockdefinition]{./mf6ivar/tex/gwt-oc-options.dat}
\vspace{5mm}
\noindent \textit{FOR ANY STRESS PERIOD}
\lstinputlisting[style=blockdefinition]{./mf6ivar/tex/gwt-oc-period.dat}

\vspace{5mm}
\subsubsection{Explanation of Variables}
\begin{description}
% DO NOT MODIFY THIS FILE DIRECTLY.  IT IS CREATED BY mf6ivar.py 

\item \textbf{Block: OPTIONS}

\begin{description}
\item \texttt{BUDGET}---keyword to specify that record corresponds to the budget.

\item \texttt{FILEOUT}---keyword to specify that an output filename is expected next.

\item \texttt{budgetfile}---name of the output file to write budget information.

\item \texttt{CONCENTRATION}---keyword to specify that record corresponds to concentration.

\item \texttt{concentrationfile}---name of the output file to write conc information.

\item \texttt{PRINT\_FORMAT}---keyword to specify format for printing to the listing file.

\item \texttt{columns}---number of columns for writing data.

\item \texttt{width}---width for writing each number.

\item \texttt{digits}---number of digits to use for writing a number.

\item \texttt{format}---write format can be EXPONENTIAL, FIXED, GENERAL, or SCIENTIFIC.

\end{description}
\item \textbf{Block: PERIOD}

\begin{description}
\item \texttt{iper}---integer value specifying the starting stress period number for which the data specified in the PERIOD block apply.  IPER must be less than or equal to NPER in the TDIS Package and greater than zero.  The IPER value assigned to a stress period block must be greater than the IPER value assigned for the previous PERIOD block.  The information specified in the PERIOD block will continue to apply for all subsequent stress periods, unless the program encounters another PERIOD block.

\item \texttt{SAVE}---keyword to indicate that information will be saved this stress period.

\item \texttt{PRINT}---keyword to indicate that information will be printed this stress period.

\item \texttt{rtype}---type of information to save or print.  Can be BUDGET or CONCENTRATION.

\item \texttt{ocsetting}---specifies the steps for which the data will be saved.

\begin{lstlisting}[style=blockdefinition]
ALL
FIRST
LAST
FREQUENCY <frequency>
STEPS <steps(<nstp)>
\end{lstlisting}

\item \texttt{ALL}---keyword to indicate save for all time steps in period.

\item \texttt{FIRST}---keyword to indicate save for first step in period. This keyword may be used in conjunction with other keywords to print or save results for multiple time steps.

\item \texttt{LAST}---keyword to indicate save for last step in period. This keyword may be used in conjunction with other keywords to print or save results for multiple time steps.

\item \texttt{frequency}---save at the specified time step frequency. This keyword may be used in conjunction with other keywords to print or save results for multiple time steps.

\item \texttt{steps}---save for each step specified in STEPS. This keyword may be used in conjunction with other keywords to print or save results for multiple time steps.

\end{description}


\end{description}

\vspace{5mm}
\subsubsection{Example Input File}
\lstinputlisting[style=inputfile]{./mf6ivar/examples/gwt-oc-example.dat}


\newpage
\subsection{Observation (OBS) Utility for a GWE Model}
GWE & temperature & cellid & -- & Temperature at a specified cell. \\
GWE & flow-ja-face & cellid & cellid & Energy flow in dimensions of watts between two adjacent cells.  The energy flow rate includes the contributions from both advection and conduction (including mechanical dispersion) if those packages are active

\newpage
\subsection{Advection (ADV) Package}
Advection (ADV) Package information is read from the file that is specified by ``ADV6'' as the file type.  Only one ADV Package can be specified for a GWT model. 

\vspace{5mm}
\subsubsection{Structure of Blocks}
\lstinputlisting[style=blockdefinition]{./mf6ivar/tex/gwt-adv-options.dat}

\vspace{5mm}
\subsubsection{Explanation of Variables}
\begin{description}
% DO NOT MODIFY THIS FILE DIRECTLY.  IT IS CREATED BY mf6ivar.py 

\item \textbf{Block: OPTIONS}

\begin{description}
\item \texttt{scheme}---scheme used to solve the advection term.  Can be upstream, central, or TVD.

\end{description}


\end{description}

\vspace{5mm}
\subsubsection{Example Input File}
\lstinputlisting[style=inputfile]{./mf6ivar/examples/gwt-adv-example.dat}



\newpage
\subsection{Dispersion (DSP) Package}
Dispersion (DSP) Package information is read from the file that is specified by ``DSP6'' as the file type.  Only one DSP Package can be specified for a GWT model. 

\vspace{5mm}
\subsubsection{Structure of Blocks}
\lstinputlisting[style=blockdefinition]{./mf6ivar/tex/gwt-dsp-options.dat}
\lstinputlisting[style=blockdefinition]{./mf6ivar/tex/gwt-dsp-griddata.dat}

\vspace{5mm}
\subsubsection{Explanation of Variables}
\begin{description}
% DO NOT MODIFY THIS FILE DIRECTLY.  IT IS CREATED BY mf6ivar.py 

\item \textbf{Block: OPTIONS}

\begin{description}
\item \texttt{XT3D\_OFF}---deactivate the xt3d method to and use the faster and less accurate approximation.  This option may provide a fast and accurate solution under some circumstances, such as when flow aligns with the model grid, there is no mechanical dispersion, or when the longitudinal and transverse dispersivities are equal.  This option may also be used to assess the computational demand of the XT3D approach by noting the run time differences with and without this option on.

\item \texttt{XT3D\_RHS}---add xt3d terms to right-hand side, when possible.  This option uses less memory, but may require more iterations.

\end{description}
\item \textbf{Block: GRIDDATA}

\begin{description}
\item \texttt{diffc}---effective molecular diffusion coefficient.

\item \texttt{alh}---longitudinal dispersivity in horizontal direction.  If flow is strictly horizontal, then this is the longitudinal dispersivity that will be used.  If flow is not strictly horizontal or strictly vertical, then the longitudinal dispersivity is a function of both ALH and ALV.  If mechanical dispersion is represented (by specifying any dispersivity values) then this array is required.

\item \texttt{alv}---longitudinal dispersivity in vertical direction.  If flow is strictly vertical, then this is the longitudinal dispsersivity value that will be used.  If flow is not strictly horizontal or strictly vertical, then the longitudinal dispersivity is a function of both ALH and ALV.  If this value is not specified and mechanical dispersion is represented, then this array is set equal to ALH.

\item \texttt{ath1}---transverse dispersivity in horizontal direction.  This is the transverse dispersivity value for the second ellipsoid axis.  If flow is strictly horizontal and directed in the x direction (along a row for a regular grid), then this value controls spreading in the y direction.  If mechanical dispersion is represented (by specifying any dispersivity values) then this array is required.

\item \texttt{ath2}---transverse dispersivity in horizontal direction.  This is the transverse dispersivity value for the third ellipsoid axis.  If flow is strictly horizontal and directed in the x direction (along a row for a regular grid), then this value controls spreading in the z direction.  If this value is not specified and mechanical dispersion is represented, then this array is set equal to ATH1.

\item \texttt{atv}---transverse dispersivity when flow is in vertical direction.  If flow is strictly vertical and directed in the z direction, then this value controls spreading in the x and y directions.  If this value is not specified and mechanical dispersion is represented, then this array is set equal to ATH2.

\end{description}


\end{description}

\vspace{5mm}
\subsubsection{Example Input File}
\lstinputlisting[style=inputfile]{./mf6ivar/examples/gwt-dsp-example.dat}



\newpage
\subsection{Source and Sink Mixing (SSM) Package}
Source and Sink Mixing (SSM) Package information is read from the file that is specified by ``SSM6'' as the file type.  Only one SSM Package can be specified for a GWE model.  The SSM Package is required if the flow model has any stress packages.

The SSM Package is used to add or remove thermal energy from GWE model cells based on inflows and outflows from GWF stress packages.  If a GWF stress package provides flow into a model cell, that flow can be assigned a user-specified temperature.  If a GWF stress package removes water from a model cell, the temperature of that water is the temperature of the cell from which the water is removed.  For flow boundary conditions that include evapotranspiration, the latent heat of vaporization may be used to represent evaporative cooling.  There are several different ways for the user to specify the temperatures.  

\begin{itemize}
\item The default condition is that sources have a temperature of zero and sinks withdraw water at the calculated temperature of the cell.  This default condition is assigned to any GWF stress package that is not included in a SOURCES block or FILEINPUT block.
\item A second option is to assign auxiliary variables in the GWF model and include a temperature for each stress boundary.  In this case, the user provides the name of the package and the name of the auxiliary variable containing temperature values for each boundary.  As described below for srctype, there are multiple options for defining this behavior.
\item A third option is to prepare an SPC6 file for any desired GWF stress package.  This SPC6 file allows users to change temperatures by stress period, or to use the time-series option to interpolate temperatures by time step.  This third option was introduced in MODFLOW version 6.3.0.  Information for this approach is entered in an optional FILEINPUT block below.  The SPC6 input file supports list-based temperature input for most corresponding GWF stress packages, but also supports a READASARRAYS array-based input format if a corresponding GWF recharge or evapotranspiration package uses the READASARRAYS option.
\end{itemize}

\noindent The auxiliary method and the SPC6 file input method can both be used for a GWE model, but only one approach can be assigned per GWF stress package.   If a flow package specified in the SOURCES or FILEINPUT blocks is also represented using an advanced transport package (SFE, LKE, MWE, or UZE), then the advanced transport package will override SSM calculations for that package.

\vspace{5mm}
\subsubsection{Structure of Blocks}
\lstinputlisting[style=blockdefinition]{./mf6ivar/tex/gwe-ssm-options.dat}
\lstinputlisting[style=blockdefinition]{./mf6ivar/tex/gwe-ssm-sources.dat}
\vspace{5mm}
\noindent \textit{FILEINPUT BLOCK IS OPTIONAL}
\lstinputlisting[style=blockdefinition]{./mf6ivar/tex/gwe-ssm-fileinput.dat}

\vspace{5mm}
\subsubsection{Explanation of Variables}
\begin{description}
% DO NOT MODIFY THIS FILE DIRECTLY.  IT IS CREATED BY mf6ivar.py 

\item \textbf{Block: OPTIONS}

\begin{description}
\item \texttt{PRINT\_FLOWS}---keyword to indicate that the list of SSM flow rates will be printed to the listing file for every stress period time step in which ``BUDGET PRINT'' is specified in Output Control.  If there is no Output Control option and ``PRINT\_FLOWS'' is specified, then flow rates are printed for the last time step of each stress period.

\item \texttt{SAVE\_FLOWS}---keyword to indicate that SSM flow terms will be written to the file specified with ``BUDGET FILEOUT'' in Output Control.

\end{description}
\item \textbf{Block: SOURCES}

\begin{description}
\item \texttt{pname}---name of the flow package for which an auxiliary variable contains a source temperature.  If this flow package is represented using an advanced transport package (SFT, LKT, MWT, or UZT), then the advanced transport package will override SSM terms specified here.

\item \texttt{srctype}---keyword indicating how temperature will be assigned for sources and sinks.  Keyword must be specified as either AUX or AUXMIXED.  For both options the user must provide an auxiliary variable in the corresponding flow package.  The auxiliary variable must have the same name as the AUXNAME value that follows.  If the AUX keyword is specified, then the auxiliary variable specified by the user will be assigned as the concenration value for groundwater sources (flows with a positive sign).  For negative flow rates (sinks), groundwater will be withdrawn from the cell at the simulated temperature of the cell.  The AUXMIXED option provides an alternative method for how to determine the temperature of sinks.  If the cell temperature is larger than the user-specified auxiliary temperature, then the temperature of groundwater withdrawn from the cell will be assigned as the user-specified temperature.  Alternatively, if the user-specified auxiliary temperature is larger than the cell temperature, then groundwater will be withdrawn at the cell temperature.  Thus, the AUXMIXED option is designed to work with the Evapotranspiration (EVT) and Recharge (RCH) Packages where water may be withdrawn at a temperature that is less than the cell temperature.

\item \texttt{auxname}---name of the auxiliary variable in the package PNAME.  This auxiliary variable must exist and be specified by the user in that package.  The values in this auxiliary variable will be used to set the temperature associated with the flows for that boundary package.

\end{description}
\item \textbf{Block: FILEINPUT}

\begin{description}
\item \texttt{pname}---name of the flow package for which an SPC6 input file contains a source temperature.  If this flow package is represented using an advanced transport package (SFT, LKT, MWT, or UZT), then the advanced transport package will override SSM terms specified here.

\item \texttt{SPC6}---keyword to specify that record corresponds to a source sink mixing input file.

\item \texttt{FILEIN}---keyword to specify that an input filename is expected next.

\item \texttt{spc6\_filename}---character string that defines the path and filename for the file containing source and sink input data for the flow package. The SPC6\_FILENAME file is a flexible input file that allows temperatures to be specified by stress period and with time series. Instructions for creating the SPC6\_FILENAME input file are provided in the next section on file input for boundary temperatures.

\item \texttt{MIXED}---keyword to specify that these stress package boundaries will have the mixed condition.  The MIXED condition is described in the SOURCES block for AUXMIXED.  The MIXED condition allows for water to be withdrawn at a temperature that is less than the cell temperature.  It is intended primarily for representing evapotranspiration.

\end{description}


\end{description}

\vspace{5mm}
\subsubsection{Example Input File}
\lstinputlisting[style=inputfile]{./mf6ivar/examples/gwe-ssm-example.dat}

% when obs are ready, they should go here

\newpage
\subsection{Stress Package Concentrations (SPC) -- List-Based Input}
As mentioned in the previous section on the SSM Package, temperatures can be specified for GWF stress packages using auxiliary variables, or they can be specified using input files dedicated to this purpose.  The Stress Package Concentrations (SPC) input file can be used to provide concentrations (temperatures) that are assigned for GWF sources and sinks.  An SPC input file can be list based or array based.  List-based input files can be used for list-based GWF stress packages, such as wells, drains, and rivers.  Array-based input files can be used for array-based GWF stress packages, such as recharge and evapotranspiration (provided the READASARRAYS options is used; these packages can also be provided in a list-based format).  Array-based SPC input files are discussed in the next section.  This section describes the list-based input format for the SPC input file.  

An SPC6 file can be prepared to provide user-specified temperatures for a GWF stress package, such a Well or General-Head Boundary Package, for example.  One SPC6 file applies to one GWF stress package.  Names for the SPC6 input files are provided in the FILEINPUT block of the SSM Package.  SPC6 entries cannot be specified in the GWE name file.  Use of the SPC6 input file is an alternative to specifying stress package temperatures as auxiliary variables in the flow model stress package.  

The boundary number in the PERIOD block corresponds to the boundary number in the GWF stress period package.  Assignment of the boundary number is straightforward for the advanced packages (SFR, LAK, MAW, and UZF) because the features in these advanced packages are defined once at the beginning of the simulation and they do not change.  For the other stress packages, however, the order of boundaries may change between stress periods.  Consider the following Well Package input file, for example:

\begin{verbatim}
# This is an example of a GWF Well Package
# in which the order of the wells changes from
# stress period 1 to 2.  This must be explicitly
# handled by the user if using the SPC6 input
# for a GWE model.
BEGIN options
  BOUNDNAMES
END options

BEGIN dimensions
  MAXBOUND  3
END dimensions

BEGIN period  1
  1 77 65   -2200  SHALLOW_WELL
  2 77 65   -24.0  INTERMEDIATE_WELL
  3 77 65   -6.20  DEEP_WELL
END period

BEGIN period  2
  1 77 65   -1100  SHALLOW_WELL
  3 77 65   -3.10  DEEP_WELL
  2 77 65   -12.0  INTERMEDIATE_WELL
END period
\end{verbatim}

\noindent In this Well input file, the order of the wells changed between periods 1 and 2.  This reordering must be explicitly taken into account by the user when creating an SSMI6 file, because the boundary number in the SSMI file corresponds to the boundary number in the Well input file.  In stress period 1, boundary number 2 is the INTERMEDIATE\_WELL, whereas in stress period 2, boundary number 2 is the DEEP\_WELL.  When using this SSMI capability to specify boundary temperatures, it is recommended that users write the corresponding GWF stress packages using the same number, cell locations, and order of boundary conditions for each stress period.   In addition, users can activate the PRINT\_FLOWS option in the SSM input file.  When the SSM Package prints the individual solute flows to the transport list file, it includes a column containing the boundary temperature.  Users can check the boundary temperatures in this output to verify that they are assigned as intended.

\vspace{5mm}
\subsubsection{Structure of Blocks}
\vspace{5mm}

\noindent \textit{FOR EACH SIMULATION}
\lstinputlisting[style=blockdefinition]{./mf6ivar/tex/utl-spc-options.dat}
\lstinputlisting[style=blockdefinition]{./mf6ivar/tex/utl-spc-dimensions.dat}
\vspace{5mm}
\noindent \textit{FOR ANY STRESS PERIOD}
\lstinputlisting[style=blockdefinition]{./mf6ivar/tex/utl-spc-period.dat}

\vspace{5mm}
\subsubsection{Explanation of Variables}
\begin{description}
% DO NOT MODIFY THIS FILE DIRECTLY.  IT IS CREATED BY mf6ivar.py 

\item \textbf{Block: OPTIONS}

\begin{description}
\item \texttt{PRINT\_INPUT}---keyword to indicate that the list of spc information will be written to the listing file immediately after it is read.

\item \texttt{TS6}---keyword to specify that record corresponds to a time-series file.

\item \texttt{FILEIN}---keyword to specify that an input filename is expected next.

\item \texttt{ts6\_filename}---defines a time-series file defining time series that can be used to assign time-varying values. See the ``Time-Variable Input'' section for instructions on using the time-series capability.

\end{description}
\item \textbf{Block: DIMENSIONS}

\begin{description}
\item \texttt{maxbound}---integer value specifying the maximum number of spc cells that will be specified for use during any stress period.

\end{description}
\item \textbf{Block: PERIOD}

\begin{description}
\item \texttt{iper}---integer value specifying the starting stress period number for which the data specified in the PERIOD block apply.  IPER must be less than or equal to NPER in the TDIS Package and greater than zero.  The IPER value assigned to a stress period block must be greater than the IPER value assigned for the previous PERIOD block.  The information specified in the PERIOD block will continue to apply for all subsequent stress periods, unless the program encounters another PERIOD block.

\item \texttt{bndno}---integer value that defines the boundary package feature number associated with the specified PERIOD data on the line. BNDNO must be greater than zero and less than or equal to MAXBOUND.

\item \texttt{spcsetting}---line of information that is parsed into a keyword and values.  Keyword values that can be used to start the MAWSETTING string include: CONCENTRATION.

\begin{lstlisting}[style=blockdefinition]
CONCENTRATION <@concentration@>
\end{lstlisting}

\item \textcolor{blue}{\texttt{concentration}---is the boundary concentration. If the Options block includes a TIMESERIESFILE entry (see the ``Time-Variable Input'' section), values can be obtained from a time series by entering the time-series name in place of a numeric value. By default, the CONCENTRATION for each boundary feature is zero.}

\end{description}


\end{description}

\subsubsection{Example Input File}
\lstinputlisting[style=inputfile]{./mf6ivar/examples/utl-spc-example.dat}

% SPC array based
\newpage
\subsection{Stress Package Concentrations (SPC) -- Array-Based Input}

This section describes array-based input for the SPC input file.  If the READASARRAYS options is specified for either the GWF Recharge (RCH) or Evapotranspiration (EVT) Packages, then concentrations (temperatures) for these packages can be specified using array-based temperature input.  This SPC array-based input is distinguished from the list-based input in the previous section through specification of the READASARRAYS option.  When the READASARRAYS option is specified, then there is no DIMENSIONS block in the SPC input file.  Instead, the shape of the array for temperatures is the number of rows by number of columns (NROW, NCOL), for a regular MODFLOW grid (DIS), and the number of cells in a layer (NCPL) for a discretization by vertices (DISV) grid.

\vspace{5mm}
\subsubsection{Structure of Blocks}
\vspace{5mm}

\noindent \textit{FOR EACH SIMULATION}
\lstinputlisting[style=blockdefinition]{./mf6ivar/tex/utl-spca-options.dat}
\vspace{5mm}
\noindent \textit{FOR ANY STRESS PERIOD}
\lstinputlisting[style=blockdefinition]{./mf6ivar/tex/utl-spca-period.dat}

\vspace{5mm}
\subsubsection{Explanation of Variables}
\begin{description}
% DO NOT MODIFY THIS FILE DIRECTLY.  IT IS CREATED BY mf6ivar.py 

\item \textbf{Block: OPTIONS}

\begin{description}
\item \texttt{READASARRAYS}---indicates that array-based input will be used for the SPC Package.  This keyword must be specified to use array-based input.

\item \texttt{PRINT\_INPUT}---keyword to indicate that the list of spc information will be written to the listing file immediately after it is read.

\item \texttt{TAS6}---keyword to specify that record corresponds to a time-array-series file.

\item \texttt{FILEIN}---keyword to specify that an input filename is expected next.

\item \texttt{tas6\_filename}---defines a time-array-series file defining a time-array series that can be used to assign time-varying values. See the Time-Variable Input section for instructions on using the time-array series capability.

\end{description}
\item \textbf{Block: PERIOD}

\begin{description}
\item \texttt{iper}---integer value specifying the starting stress period number for which the data specified in the PERIOD block apply.  IPER must be less than or equal to NPER in the TDIS Package and greater than zero.  The IPER value assigned to a stress period block must be greater than the IPER value assigned for the previous PERIOD block.  The information specified in the PERIOD block will continue to apply for all subsequent stress periods, unless the program encounters another PERIOD block.

\item \texttt{concentration}---is the concentration of the associated Recharge or Evapotranspiration stress package.  The concentration array may be defined by a time-array series (see the "Using Time-Array Series in a Package" section).

\end{description}


\end{description}

\subsubsection{Example Input File}
\lstinputlisting[style=inputfile]{./mf6ivar/examples/utl-spca-example.dat}



\newpage
\subsection{Mobile Storage and Transfer (MST) Package}
Mobile Storage and Transfer (MST) Package information is read from the file that is specified by ``MST6'' as the file type.  Only one MST Package can be specified for a GWE model. 

\vspace{5mm}
\subsubsection{Structure of Blocks}
\lstinputlisting[style=blockdefinition]{./mf6ivar/tex/gwe-mst-options.dat}
\lstinputlisting[style=blockdefinition]{./mf6ivar/tex/gwe-mst-griddata.dat}

\vspace{5mm}
\subsubsection{Explanation of Variables}
\begin{description}
% DO NOT MODIFY THIS FILE DIRECTLY.  IT IS CREATED BY mf6ivar.py 

\item \textbf{Block: OPTIONS}

\begin{description}
\item \texttt{SAVE\_FLOWS}---keyword to indicate that MST flow terms will be written to the file specified with ``BUDGET FILEOUT'' in Output Control.

\item \texttt{ZERO\_ORDER\_DECAY}---is a text keyword to indicate that zero-order decay will occur.  Use of this keyword requires that DECAY and DECAY\_SORBED (if sorption is active) are specified in the GRIDDATA block.

\item \texttt{LATENT\_HEAT\_VAPORIZATION}---is a text keyword to indicate that cooling associated with evaporation will occur.  Use of this keyword requires that LATHEATVAP are specified in the GRIDDATA block.  While the MST package does not simulate evaporation, multiple other packages in a GWE simulation may.  For example, evaporation may occur from the surface of streams or lakes.  Owing to the energy consumed by the change in phase, the latent heat of vaporization is required.

\end{description}
\item \textbf{Block: GRIDDATA}

\begin{description}
\item \texttt{porosity}---is the mobile domain porosity, defined as the mobile domain pore volume per mobile domain volume.  The GWE model does not support the concept of an immobile domain in the context of heat transport.

\item \texttt{decay}---is the rate coefficient for zero-order decay for the aqueous phase of the mobile domain.  A negative value indicates heat (energy) production.  The dimensions of decay for zero-order decay is energy per length cubed per time.  Zero-order decay will have no effect on simulation results unless zero-order decay is specified in the options block.

\item \texttt{cps}---is the mass-based heat capacity of dry solids (aquifer material). For example, units of J/kg/C may be used (or equivalent).

\item \texttt{rhos}---is a user-specified value of the density of aquifer material not considering the voids. Value will remain fixed for the entire simulation.  For example, if working in SI units, values may be entered as kg/m3.

\end{description}
\item \textbf{Block: PACKAGEDATA}

\begin{description}
\item \texttt{cpw}---is the mass-based heat capacity of the simulated fluid. For example, units of J/kg/C may be used (or equivalent).

\item \texttt{rhow}---is a user-specified value of the density of water. Value will remain fixed for the entire simulation.  For example, if working in SI units, values may be entered as kg/m3.

\item \texttt{latheatvap}---is the user-specified value for the latent heat of vaporization. For example, if working in SI units, values may be entered as kJ/kg.

\end{description}


\end{description}

\vspace{5mm}
\subsubsection{Example Input File}
\lstinputlisting[style=inputfile]{./mf6ivar/examples/gwe-mst-example.dat}



\newpage
\subsection{Constant Temperature (CNT) Package}
Constant Temperature (CNT) Package information is read from the file that is specified by ``CNT6'' as the file type.  Any number of CNT Packages can be specified for a single GWE model, but the same cell cannot be designated as a constant temperature by more than one CNT entry. 

\vspace{5mm}
\subsubsection{Structure of Blocks}
\vspace{5mm}

\noindent \textit{FOR EACH SIMULATION}
\lstinputlisting[style=blockdefinition]{./mf6ivar/tex/gwe-cnt-options.dat}
\lstinputlisting[style=blockdefinition]{./mf6ivar/tex/gwe-cnt-dimensions.dat}
\vspace{5mm}
\noindent \textit{FOR ANY STRESS PERIOD}
\lstinputlisting[style=blockdefinition]{./mf6ivar/tex/gwe-cnt-period.dat}
\packageperioddescription

\vspace{5mm}
\subsubsection{Explanation of Variables}
\begin{description}
% DO NOT MODIFY THIS FILE DIRECTLY.  IT IS CREATED BY mf6ivar.py 

\item \textbf{Block: OPTIONS}

\begin{description}
\item \texttt{auxiliary}---defines an array of one or more auxiliary variable names.  There is no limit on the number of auxiliary variables that can be provided on this line; however, lists of information provided in subsequent blocks must have a column of data for each auxiliary variable name defined here.   The number of auxiliary variables detected on this line determines the value for naux.  Comments cannot be provided anywhere on this line as they will be interpreted as auxiliary variable names.  Auxiliary variables may not be used by the package, but they will be available for use by other parts of the program.  The program will terminate with an error if auxiliary variables are specified on more than one line in the options block.

\item \texttt{auxmultname}---name of auxiliary variable to be used as multiplier of temperature value.

\item \texttt{BOUNDNAMES}---keyword to indicate that boundary names may be provided with the list of constant temperature cells.

\item \texttt{PRINT\_INPUT}---keyword to indicate that the list of constant temperature information will be written to the listing file immediately after it is read.

\item \texttt{PRINT\_FLOWS}---keyword to indicate that the list of constant temperature flow rates will be printed to the listing file for every stress period time step in which ``BUDGET PRINT'' is specified in Output Control.  If there is no Output Control option and ``PRINT\_FLOWS'' is specified, then flow rates are printed for the last time step of each stress period.

\item \texttt{SAVE\_FLOWS}---keyword to indicate that constant temperature flow terms will be written to the file specified with ``BUDGET FILEOUT'' in Output Control.

\item \texttt{TS6}---keyword to specify that record corresponds to a time-series file.

\item \texttt{FILEIN}---keyword to specify that an input filename is expected next.

\item \texttt{ts6\_filename}---defines a time-series file defining time series that can be used to assign time-varying values. See the ``Time-Variable Input'' section for instructions on using the time-series capability.

\item \texttt{OBS6}---keyword to specify that record corresponds to an observations file.

\item \texttt{obs6\_filename}---name of input file to define observations for the Constant Temperature package. See the ``Observation utility'' section for instructions for preparing observation input files. Tables \ref{table:gwf-obstypetable} and \ref{table:gwt-obstypetable} lists observation type(s) supported by the Constant Temperature package.

\end{description}
\item \textbf{Block: DIMENSIONS}

\begin{description}
\item \texttt{maxbound}---integer value specifying the maximum number of constant temperatures cells that will be specified for use during any stress period.

\end{description}
\item \textbf{Block: PERIOD}

\begin{description}
\item \texttt{iper}---integer value specifying the starting stress period number for which the data specified in the PERIOD block apply.  IPER must be less than or equal to NPER in the TDIS Package and greater than zero.  The IPER value assigned to a stress period block must be greater than the IPER value assigned for the previous PERIOD block.  The information specified in the PERIOD block will continue to apply for all subsequent stress periods, unless the program encounters another PERIOD block.

\item \texttt{cellid}---is the cell identifier, and depends on the type of grid that is used for the simulation.  For a structured grid that uses the DIS input file, CELLID is the layer, row, and column.   For a grid that uses the DISV input file, CELLID is the layer and CELL2D number.  If the model uses the unstructured discretization (DISU) input file, CELLID is the node number for the cell.

\item \textcolor{blue}{\texttt{temp}---is the constant temperature value. If the Options block includes a TIMESERIESFILE entry (see the ``Time-Variable Input'' section), values can be obtained from a time series by entering the time-series name in place of a numeric value.}

\item \textcolor{blue}{\texttt{aux}---represents the values of the auxiliary variables for each constant temperature. The values of auxiliary variables must be present for each constant temperature. The values must be specified in the order of the auxiliary variables specified in the OPTIONS block.  If the package supports time series and the Options block includes a TIMESERIESFILE entry (see the ``Time-Variable Input'' section), values can be obtained from a time series by entering the time-series name in place of a numeric value.}

\item \texttt{boundname}---name of the constant temperature cell.  BOUNDNAME is an ASCII character variable that can contain as many as 40 characters.  If BOUNDNAME contains spaces in it, then the entire name must be enclosed within single quotes.

\end{description}


\end{description}

\vspace{5mm}
\subsubsection{Example Input File}
\lstinputlisting[style=inputfile]{./mf6ivar/examples/gwe-cnt-example.dat}

\vspace{5mm}
\subsubsection{Available observation types}
CNT Package observations are limited to the simulated constant temperature energy flow rate (\texttt{cnt}). The data required for the CNT Package observation type is defined in table~\ref{table:gwe-cntobstype}. Negative and positive values for an observation represent a loss from and gain to the GWE model, respectively.

\begin{longtable}{p{2cm} p{2.75cm} p{2cm} p{1.25cm} p{7cm}}
\caption{Available CNT Package observation types} \tabularnewline

\hline
\hline
\textbf{Model} & \textbf{Observation type} & \textbf{ID} & \textbf{ID2} & \textbf{Description} \\
\hline
\endhead

\hline
\endfoot

CNT & cnt & cellid or boundname & -- & Energy flow between the groundwater system and a constant-temperature boundary or a group of cells with constant-temperature boundaries.

\label{table:gwe-cntobstype}
\end{longtable}

\vspace{5mm}
\subsubsection{Example Observation Input File}
\lstinputlisting[style=inputfile]{./mf6ivar/examples/gwe-cnt-example-obs.dat}


\newpage
\subsection{Flow Model Interface (FMI) Package}
Flow Model Interface (FMI) Package information is read from the file that is specified by ``FMI6'' as the file type.  The FMI Package is optional, but if provided, only one FMI Package can be specified for a GWT model.

For most simulations, the GWT Model needs groundwater flows for every cell in the model grid, for all boundary conditions, and for other terms, such as the flow of water in or out of storage.  The FMI Package is the interface between the GWT Model and simulated groundwater flows coming from a corresponding GWF Model that is running concurrently within the simulation or from a binary budget file that was created from a previous GWF model run.  The following are the different FMI simulation cases:

\begin{itemize}

\item Flows are provided by a corresponding GWF Model running in the same simulation---in this case, all groundwater flows are calculated by the corresponding GWF Model and provided through FMI to the transport model.  This is a common use case in which the user wants to run the flow and transport models as part of a single simulation.  The GWF and GWT models must be part of a GWF-GWT Exchange that is listed in mfsim.nam.

\item There is no groundwater flow and the user is interested only in the effects of diffusion, sorption, and decay or production---in this case, FMI should not be provided in the GWT name file and the GWT model should not be listed in any GWF-GWT Exchanges in mfsim.nam.  In this case, all groundwater flows are assumed to be zero and cells are assumed to be fully saturated.  The SSM Package should not be activated in this case, because there can be no sources or sinks of water.  This type of model simulation without an FMI Package is included as an option to represent diffusion, sorption, and decay or growth in the absence of flow groundwater.

\item Flows are provided from a previous GWF model simulation---in this case FMI should be provided in the GWT name file and the head and budget files should be listed in the FMI options block.  In this case, FMI reads the simulated head and flows from these files and makes them available to the transport model.  There are some additional considerations when the heads and flows are provided from binary files.

\begin{itemize}
\item The binary budget file must contain the simulated flows for all of the packages that were included in the GWF model run.  Saving of flows can be activated for all packages by specifying ``SAVE\_FLOWS'' as an option in the GWF name file.  The GWF Output Control Package must also have ``SAVE BUGET ALL'' specified.  The easiest way to ensure that all flows and heads are saved is to use the following simple form of a GWF Output Control file:

\begin{verbatim}
BEGIN OPTIONS
  HEAD FILEOUT mymodel.hds
  BUDGET FILEOUT mymodel.bud
END OPTIONS

BEGIN PERIOD 1
  SAVE HEAD ALL
  SAVE BUDGET ALL
END PERIOD
\end{verbatim}

\item The binary budget file must have the same number of budget terms listed for each time step.  This will always be the case when the binary budget file is created by \mfdot
\item The binary heads file must have heads saved for all layers in the model.  This will always be the case when the binary head file is created by \mfdot  This was not always the case as previous MODFLOW versions allowed different save options for each layer.
\item If the binary budget and head files have more than one time step for a single stress period, then the budget and head information must be contained within the binary file for every time step in the simulation stress period.
\item The binary budget and head files must correspond in terms of information stored for each time step and stress period.
\item If the binary budget and head files have information provided for only the first time step of each stress period, then this information will be used for all time steps in the GWT model run for that stress period.  This makes it possible to provide flows, for example, from a steady state GWF stress period and have those flows used for all steps in the GWT simulation.  Note that this cannot be done when the GWF and GWT models are run in the same simulation, because in that case, both models are solved for each time step in the stress period, as listed in the TDIS Package.
\end{itemize}

\end{itemize}

\vspace{5mm}
\subsubsection{Structure of Blocks}
\lstinputlisting[style=blockdefinition]{./mf6ivar/tex/gwt-fmi-options.dat}

\vspace{5mm}
\subsubsection{Explanation of Variables}
\begin{description}
% DO NOT MODIFY THIS FILE DIRECTLY.  IT IS CREATED BY mf6ivar.py 

\item \textbf{Block: OPTIONS}

\begin{description}
\item \texttt{FLOW\_IMBALANCE\_CORRECTION}---correct for an imbalance in flows by assuming that any residual flow error comes in or leaves at the concentration of the cell.

\item \texttt{GWFBUDGET}---keyword to specify that record corresponds to the gwfbudget input file.

\item \texttt{FILEIN}---keyword to specify that an input filename is expected next.

\item \texttt{gwfbudgetfile}---name of the binary GWF budget file to read as input for the FMI Package

\item \texttt{GWFHEAD}---keyword to specify that record corresponds to the gwfhead input file.

\item \texttt{gwfheadfile}---name of the binary GWF head file to read as input for the FMI Package

\end{description}


\end{description}

\vspace{5mm}
\subsubsection{Example Input File}
\lstinputlisting[style=inputfile]{./mf6ivar/examples/gwt-fmi-example.dat}



\newpage
\subsection{Groundwater Energy Transport (GWE) Exchange}
Input to the Groundwater Energy Transport (GWE-GWE) Exchange is read from the file that has type ``GWE6-GWE6'' in the Simulation Name File.

The list of exchanges entered into the EXCHANGEDATA block must be identical to the list of exchanges entered for the GWF-GWF input file.  One way to ensure that this information is identical is to put this list into an external file and refer to this same list using the OPEN/CLOSE functionality in both this EXCHANGEDATA input block and the EXCHANGEDATA input block in the GWF-GWF input file.

\vspace{5mm}
\subsubsection{Structure of Blocks}
\lstinputlisting[style=blockdefinition]{./mf6ivar/tex/exg-gwegwe-options.dat}
\lstinputlisting[style=blockdefinition]{./mf6ivar/tex/exg-gwegwe-dimensions.dat}
\lstinputlisting[style=blockdefinition]{./mf6ivar/tex/exg-gwegwe-exchangedata.dat}

\vspace{5mm}
\subsubsection{Explanation of Variables}
\begin{description}
% DO NOT MODIFY THIS FILE DIRECTLY.  IT IS CREATED BY mf6ivar.py 

\item \textbf{Block: OPTIONS}

\begin{description}
\item \texttt{gwfmodelname1}---keyword to specify name of first corresponding GWF Model.  In the simulation name file, the GWE6-GWE6 entry contains names for GWE Models (exgmnamea and exgmnameb).  The GWE Model with the name exgmnamea must correspond to the GWF Model with the name gwfmodelname1.

\item \texttt{gwfmodelname2}---keyword to specify name of second corresponding GWF Model.  In the simulation name file, the GWE6-GWE6 entry contains names for GWE Models (exgmnamea and exgmnameb).  The GWE Model with the name exgmnameb must correspond to the GWF Model with the name gwfmodelname2.

\item \texttt{auxiliary}---an array of auxiliary variable names.  There is no limit on the number of auxiliary variables that can be provided. Most auxiliary variables will not be used by the GWF-GWF Exchange, but they will be available for use by other parts of the program.  If an auxiliary variable with the name ``ANGLDEGX'' is found, then this information will be used as the angle (provided in degrees) between the connection face normal and the x axis, where a value of zero indicates that a normal vector points directly along the positive x axis.  The connection face normal is a normal vector on the cell face shared between the cell in model 1 and the cell in model 2 pointing away from the model 1 cell.  Additional information on ``ANGLDEGX'' is provided in the description of the DISU Package.  If an auxiliary variable with the name ``CDIST'' is found, then this information will be used as the straight-line connection distance, including the vertical component, between the two cell centers.  Both ANGLDEGX and CDIST are required if specific discharge is calculated for either of the groundwater models.

\item \texttt{BOUNDNAMES}---keyword to indicate that boundary names may be provided with the list of GWE Exchange cells.

\item \texttt{PRINT\_INPUT}---keyword to indicate that the list of exchange entries will be echoed to the listing file immediately after it is read.

\item \texttt{PRINT\_FLOWS}---keyword to indicate that the list of exchange flow rates will be printed to the listing file for every stress period in which ``SAVE BUDGET'' is specified in Output Control.

\item \texttt{SAVE\_FLOWS}---keyword to indicate that cell-by-cell flow terms will be written to the budget file for each model provided that the Output Control for the models are set up with the ``BUDGET SAVE FILE'' option.

\item \texttt{adv\_scheme}---scheme used to solve the advection term.  Can be upstream, central, or TVD.  If not specified, upstream weighting is the default weighting scheme.

\item \texttt{CND\_XT3D\_OFF}---deactivate the xt3d method for the dispersive flux and use the faster and less accurate approximation for this exchange.

\item \texttt{CND\_XT3D\_RHS}---add xt3d dispersion terms to right-hand side, when possible, for this exchange.

\item \texttt{FILEIN}---keyword to specify that an input filename is expected next.

\item \texttt{MVE6}---keyword to specify that record corresponds to an energy transport mover file.

\item \texttt{mve6\_filename}---is the file name of the transport mover input file to apply to this exchange.  Information for the transport mover are provided in the file provided with these keywords.

\item \texttt{OBS6}---keyword to specify that record corresponds to an observations file.

\item \texttt{obs6\_filename}---is the file name of the observations input file for this exchange. See the ``Observation utility'' section for instructions for preparing observation input files. Table \ref{table:gwe-obstypetable} lists observation type(s) supported by the GWE-GWE package.

\end{description}
\item \textbf{Block: DIMENSIONS}

\begin{description}
\item \texttt{nexg}---keyword and integer value specifying the number of GWE-GWE exchanges.

\end{description}
\item \textbf{Block: EXCHANGEDATA}

\begin{description}
\item \texttt{cellidm1}---is the cellid of the cell in model 1 as specified in the simulation name file. For a structured grid that uses the DIS input file, CELLIDM1 is the layer, row, and column numbers of the cell.   For a grid that uses the DISV input file, CELLIDM1 is the layer number and CELL2D number for the two cells.  If the model uses the unstructured discretization (DISU) input file, then CELLIDM1 is the node number for the cell.

\item \texttt{cellidm2}---is the cellid of the cell in model 2 as specified in the simulation name file. For a structured grid that uses the DIS input file, CELLIDM2 is the layer, row, and column numbers of the cell.   For a grid that uses the DISV input file, CELLIDM2 is the layer number and CELL2D number for the two cells.  If the model uses the unstructured discretization (DISU) input file, then CELLIDM2 is the node number for the cell.

\item \texttt{ihc}---is an integer flag indicating the direction between node n and all of its m connections. If IHC = 0 then the connection is vertical.  If IHC = 1 then the connection is horizontal. If IHC = 2 then the connection is horizontal for a vertically staggered grid.

\item \texttt{cl1}---is the distance between the center of cell 1 and the its shared face with cell 2.

\item \texttt{cl2}---is the distance between the center of cell 2 and the its shared face with cell 1.

\item \texttt{hwva}---is the horizontal width of the flow connection between cell 1 and cell 2 if IHC $>$ 0, or it is the area perpendicular to flow of the vertical connection between cell 1 and cell 2 if IHC = 0.

\item \texttt{aux}---represents the values of the auxiliary variables for each GWEGWE Exchange. The values of auxiliary variables must be present for each exchange. The values must be specified in the order of the auxiliary variables specified in the OPTIONS block.

\item \texttt{boundname}---name of the GWE Exchange cell.  BOUNDNAME is an ASCII character variable that can contain as many as 40 characters.  If BOUNDNAME contains spaces in it, then the entire name must be enclosed within single quotes.

\end{description}


\end{description}

\vspace{5mm}
\subsubsection{Example Input File}
\lstinputlisting[style=inputfile]{./mf6ivar/examples/exg-gwegwe-example.dat}

\vspace{5mm}
\subsubsection{Available observation types}
GWE-GWE Exchange observations include the simulated flow for any exchange (\texttt{flow-ja-face}). The data required for each GWE-GWE Exchange observation type is defined in table~\ref{table:gwe-gweobstype}. For \texttt{flow-ja-face} observation types, negative and positive values represent a loss from and gain to the first model specified for this exchange.

\begin{longtable}{p{2cm} p{2.75cm} p{2cm} p{1.25cm} p{7cm}}
\caption{Available GWE-GWE Exchange observation types} \tabularnewline

\hline
\hline
\textbf{Exchange} & \textbf{Observation type} & \textbf{ID} & \textbf{ID2} & \textbf{Description} \\
\hline
\endhead

\hline
\endfoot

GWE-GWE & flow-ja-face & exchange number or boundname & -- & Mass flow between model 1 and model 2 for a specified exchange (which is the consecutive exchange number listed in the EXCHANGEDATA block), or the sum of these exchange flows by boundname if boundname is specified.
\label{table:gwe-gweobstype}
\end{longtable}


\vspace{5mm}
\subsubsection{Example Observation Input File}
\lstinputlisting[style=inputfile]{./mf6ivar/examples/exg-gwegwe-example-obs.dat}



