Conduction and Dispersion (CND) Package information is read from the file that is specified by ``CND6'' as the file type.  Only one CND Package can be specified for a GWE model.  The CND Package is based on the mathematical formulation presented for the XT3D option of the NPF Package available to represent full three-dimensional anisotropy in groundwater flow.  XT3D can be computationally expensive and can be turned off to use a simplified and approximate form of the dispersion equations that also account for conduction in a heat transport model.  For most problems, however, XT3D will be required to accurately represent conduction and dispersion. 

\vspace{5mm}
\subsubsection{Structure of Blocks}
\lstinputlisting[style=blockdefinition]{./mf6ivar/tex/gwe-cnd-options.dat}
\lstinputlisting[style=blockdefinition]{./mf6ivar/tex/gwe-cnd-griddata.dat}

\vspace{5mm}
\subsubsection{Explanation of Variables}
\begin{description}
% DO NOT MODIFY THIS FILE DIRECTLY.  IT IS CREATED BY mf6ivar.py 

\item \textbf{Block: OPTIONS}

\begin{description}
\item \texttt{XT3D\_OFF}---deactivate the xt3d method and use the faster and less accurate approximation.  This option may provide a fast and accurate solution under some circumstances, such as when flow aligns with the model grid, there is no mechanical dispersion, or when the longitudinal and transverse dispersivities are equal.  This option may also be used to assess the computational demand of the XT3D approach by noting the run time differences with and without this option on.

\item \texttt{XT3D\_RHS}---add xt3d terms to right-hand side, when possible.  This option uses less memory, but may require more iterations.

\end{description}
\item \textbf{Block: GRIDDATA}

\begin{description}
\item \texttt{alh}---longitudinal dispersivity in horizontal direction.  If flow is strictly horizontal, then this is the longitudinal dispersivity that will be used.  If flow is not strictly horizontal or strictly vertical, then the longitudinal dispersivity is a function of both ALH and ALV.  If mechanical dispersion is represented (by specifying any dispersivity values) then this array is required.

\item \texttt{alv}---longitudinal dispersivity in vertical direction.  If flow is strictly vertical, then this is the longitudinal dispsersivity value that will be used.  If flow is not strictly horizontal or strictly vertical, then the longitudinal dispersivity is a function of both ALH and ALV.  If this value is not specified and mechanical dispersion is represented, then this array is set equal to ALH.

\item \texttt{ath1}---transverse dispersivity in horizontal direction.  This is the transverse dispersivity value for the second ellipsoid axis.  If flow is strictly horizontal and directed in the x direction (along a row for a regular grid), then this value controls spreading in the y direction.  If mechanical dispersion is represented (by specifying any dispersivity values) then this array is required.

\item \texttt{ath2}---transverse dispersivity in horizontal direction.  This is the transverse dispersivity value for the third ellipsoid axis.  If flow is strictly horizontal and directed in the x direction (along a row for a regular grid), then this value controls spreading in the z direction.  If this value is not specified and mechanical dispersion is represented, then this array is set equal to ATH1.

\item \texttt{atv}---transverse dispersivity when flow is in vertical direction.  If flow is strictly vertical and directed in the z direction, then this value controls spreading in the x and y directions.  If this value is not specified and mechanical dispersion is represented, then this array is set equal to ATH2.

\item \texttt{ktw}---thermal conductivity of the simulated fluid

\item \texttt{kts}---thermal conductivity of the aquifer material

\end{description}


\end{description}

\vspace{5mm}
\subsubsection{Example Input File}
\lstinputlisting[style=inputfile]{./mf6ivar/examples/gwe-cnd-example.dat}

