Flow Model Interface (FMI) Package information is read from the file that is specified by ``FMI6'' as the file type. The FMI Package file is required only if the GWT Model is running in a separate simulation from a previously run GWF Model. If the GWT Model is coupled to a GWF Model by an exchange, the FMI Package file is not required. Only one FMI Package can be specified for a GWT model.

The GWT Model needs groundwater flows for model grid cells, for boundary conditions, and for other terms, such as the flow of water in or out of storage.  The FMI Package is the interface between the GWT Model and simulated groundwater flows provided by a corresponding GWF Model that is running concurrently within the simulation or from binary budget files that were created from a previous GWF model run.  The following are several different FMI simulation cases:

\begin{itemize}

\item Flows are provided by a corresponding GWF Model running in the same simulation---in this case, all groundwater flows are calculated by the corresponding GWF Model and provided through FMI to the transport model.  This is a common use case in which the user wants to run the flow and transport models as part of a single simulation.  The GWF and GWT models must be part of a GWF-GWT Exchange that is listed in mfsim.nam.  If a GWF-GWT Exchange is specified by the user, then the user does not need to specify an FMI Package input file for the simulation, unless an FMI option is needed.  If a GWF-GWT Exchange is specified and the FMI Package is specified, then the PACKAGEDATA block below is not read or used.

\item There is no groundwater flow and the user is interested only in the effects of diffusion, sorption, and decay or production---in this case, FMI should not be provided in the GWT name file and the GWT model should not be listed in any GWF-GWT Exchanges in mfsim.nam.  In this case, all groundwater flows are assumed to be zero and cells are assumed to be fully saturated.  The SSM Package should not be activated in this case, because there can be no sources or sinks of water.  Likewise, none of the advanced transport packages (LKT, SFT, MWT, and UZT) should be specified in the GWT name file.  This type of model simulation without an FMI Package is included as an option to represent diffusion, sorption, and decay or growth in the absence of any groundwater flow.

\item Flows are provided from a previous GWF model simulation---in this case the FMI Package should be listed in the GWT name file and the head and budget files should be listed in the FMI PACKAGEDATA block.  In this case, FMI reads the simulated head and flows from these files and makes them available to the transport model.  There are some additional considerations when the heads and flows are provided from binary files.

\begin{itemize}
\item The binary budget file must contain the simulated flows for all of the packages that were included in the GWF model run.  Saving of flows can be activated for all packages by specifying ``SAVE\_FLOWS'' as an option in the GWF name file.  The GWF Output Control Package must also have ``SAVE BUDGET ALL'' specified.  The easiest way to ensure that all flows and heads are saved is to use the following simple form of a GWF Output Control file:

\begin{verbatim}
BEGIN OPTIONS
  HEAD FILEOUT mymodel.hds
  BUDGET FILEOUT mymodel.bud
END OPTIONS

BEGIN PERIOD 1
  SAVE HEAD ALL
  SAVE BUDGET ALL
END PERIOD
\end{verbatim}

\item The binary budget file must have the same number of budget terms listed for each time step.  This will always be the case when the binary budget file is created by \mf.
\item The advanced flow packages (LAK, SFR, MAW, and UZF) all have options for saving a detailed budget file the describes all of the flows for each lake, reach, well, or UZF cell.  These budget files can also be used as input to FMI if a corresponding advanced transport package is needed, such as LKT, SFT, MWT, and UZT.  If the Water Mover Package is also specified for the GWF Model, then the the budget file for the Water Mover Package will also need to be specified as input to this FMI Package.
\item The binary heads file must have heads saved for all layers in the model.  This will always be the case when the binary head file is created by \mf.  This was not always the case as previous MODFLOW versions allowed different save options for each layer.
\item If the binary budget and head files have more than one time step for a single stress period, then the budget and head information must be contained within the binary file for every time step in the simulation stress period.
\item The binary budget and head files must correspond in terms of information stored for each time step and stress period.
\item If the binary budget and head files have information provided for only the first time step of a given stress period, this information will be used for all time steps in that stress period in the GWT simulation. If the final stress period (which may be the only stress period) in the binary budget and head files has information provided for only one time step, this information will be used for any subsequent time steps and stress periods in the GWT simulation. This makes it possible to provide flows, for example, from a steady-state GWF stress period and have those flows used for all GWT time steps in that stress period, for all remaining time steps in the GWT simulation, or for all time steps throughout the entire GWT simulation. With this option, it is possible to have smaller time steps in the GWT simulation than the time steps used in the GWF simulation. Note that this cannot be done when the GWF and GWT models are run in the same simulation, because in that case, both models are solved for each time step in the stress period, as listed in the TDIS Package. This option for reading flows from a previous GWF simulation may offer an efficient alternative to running both models in the same simulation, but it comes at the cost of having potentially very large budget files.
\item The binary grid file is optional but recommended, as it allows \mf to verify that the GWF and GWT model grids are identical.
\end{itemize}

\end{itemize}

\noindent Determination of which FMI use case to invoke requires careful consideration of the different advantages and disadvantages of each case.  For example, running GWT and GWF in the same simulation can often be faster because GWF flows are passed through memory to the GWT model instead of being written to files.  The disadvantage of this approach is that the same time step lengths must be used for both GWF and GWT.  Ultimately, it should be relatively straightforward to test different ways in which GWF and GWT interact and select the use case most appropriate for the particular problem. 

\vspace{5mm}
\subsubsection{Structure of Blocks}
\lstinputlisting[style=blockdefinition]{./mf6ivar/tex/gwt-fmi-options.dat}
\lstinputlisting[style=blockdefinition]{./mf6ivar/tex/gwt-fmi-packagedata.dat}

\vspace{5mm}
\subsubsection{Explanation of Variables}
\begin{description}
% DO NOT MODIFY THIS FILE DIRECTLY.  IT IS CREATED BY mf6ivar.py 

\item \textbf{Block: OPTIONS}

\begin{description}
\item \texttt{FLOW\_IMBALANCE\_CORRECTION}---correct for an imbalance in flows by assuming that any residual flow error comes in or leaves at the concentration of the cell.

\item \texttt{GWFBUDGET}---keyword to specify that record corresponds to the gwfbudget input file.

\item \texttt{FILEIN}---keyword to specify that an input filename is expected next.

\item \texttt{gwfbudgetfile}---name of the binary GWF budget file to read as input for the FMI Package

\item \texttt{GWFHEAD}---keyword to specify that record corresponds to the gwfhead input file.

\item \texttt{gwfheadfile}---name of the binary GWF head file to read as input for the FMI Package

\end{description}


\end{description}

\vspace{5mm}
\subsubsection{Example Input File}
\lstinputlisting[style=inputfile]{./mf6ivar/examples/gwt-fmi-example.dat}

