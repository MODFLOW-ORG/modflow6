Source and Sink Mixing (SSM) Package information is read from the file that is specified by ``SSM6'' as the file type.  Only one SSM Package can be specified for a GWT model.  The SSM Package is required if the flow model has any stress packages.

The SSM Package is used to add or remove solute mass from GWT model cells based on inflows and outflows from GWF stress packages.  If a GWF stress package provides flow into a model cell, that flow can be assigned a user-specified concentration.  If a GWT stress package removes water from a model cell, the concentration of that water is typically the concentration of the cell, but a ``MIXED'' option is also included so that the user can specify the concentration of that withdrawn water.  This may be useful for representing evapotranspiration, for example.  There are several different ways for the user to specify the concentrations.  

\begin{itemize}
\item The default condition is that sources have a concentration of zero and sinks withdraw water at the calculated concentration of the cell.  This default condition is assigned to any GWF stress package that is not included in a SOURCES block or FILEINPUT block.
\item A second option is to assign auxiliary variables in the GWF model and include a concentration for each stress boundary.  In this case, the user provides the name of the package and the name of the auxiliary variable containing concentration values for each boundary.  As described below for srctype, there are multiple options for defining this behavior.
\item A third option is to prepare an SPC6 file for any desired GWF stress package.  This SPC6 file allows users to change concentrations by stress period, or to use the time-series option to interpolate concentrations by time step.  This third option was introduced in MODFLOW version 6.3.0.  Information for this approach is entered in an optional FILEINPUT block below.  The SPC6 input file supports list-based concentration input for most corresponding GWF stress packages, but also supports a READASARRAYS array-based input format if a corresponding GWF recharge or evapotranspiration package uses the READASARRAYS option.
\end{itemize}

\noindent The auxiliary method and the SPC6 file input method can both be used for a GWT model, but only one approach can be assigned per GWF stress package.   If a flow package specified in the SOURCES or FILEINPUT blocks is also represented using an advanced transport package (SFT, LKT, MWT, or UZT), then the advanced transport package will override SSM calculations for that package.

\vspace{5mm}
\subsubsection{Structure of Blocks}
\lstinputlisting[style=blockdefinition]{./mf6ivar/tex/gwt-ssm-options.dat}
\lstinputlisting[style=blockdefinition]{./mf6ivar/tex/gwt-ssm-sources.dat}
\vspace{5mm}
\noindent \textit{FILEINPUT BLOCK IS OPTIONAL}
\lstinputlisting[style=blockdefinition]{./mf6ivar/tex/gwt-ssm-fileinput.dat}

\vspace{5mm}
\subsubsection{Explanation of Variables}
\begin{description}
% DO NOT MODIFY THIS FILE DIRECTLY.  IT IS CREATED BY mf6ivar.py 

\item \textbf{Block: OPTIONS}

\begin{description}
\item \texttt{PRINT\_FLOWS}---keyword to indicate that the list of SSM flow rates will be printed to the listing file for every stress period time step in which ``BUDGET PRINT'' is specified in Output Control.  If there is no Output Control option and ``PRINT\_FLOWS'' is specified, then flow rates are printed for the last time step of each stress period.

\item \texttt{SAVE\_FLOWS}---keyword to indicate that SSM flow terms will be written to the file specified with ``BUDGET FILEOUT'' in Output Control.

\end{description}
\item \textbf{Block: SOURCES}

\begin{description}
\item \texttt{pname}---name of the flow package for which an auxiliary variable contains a source concentration.  If this flow package is represented using an advanced transport package (SFT, LKT, MWT, or UZT), then the advanced transport package will override SSM terms specified here.

\item \texttt{srctype}---keyword indicating how concentration will be assigned for sources and sinks.  Keyword must be specified as either AUX or AUXMIXED.  For both options the user must provide an auxiliary variable in the corresponding flow package.  The auxiliary variable must have the same name as the AUXNAME value that follows.  If the AUX keyword is specified, then the auxiliary variable specified by the user will be assigned as the concenration value for groundwater sources (flows with a positive sign).  For negative flow rates (sinks), groundwater will be withdrawn from the cell at the simulated concentration of the cell.  The AUXMIXED option provides an alternative method for how to determine the concentration of sinks.  If the cell concentration is larger than the user-specified auxiliary concentration, then the concentration of groundwater withdrawn from the cell will be assigned as the user-specified concentration.  Alternatively, if the user-specified auxiliary concentration is larger than the cell concentration, then groundwater will be withdrawn at the cell concentration.  Thus, the AUXMIXED option is designed to work with the Evapotranspiration (EVT) and Recharge (RCH) Packages where water may be withdrawn at a concentration that is less than the cell concentration.

\item \texttt{auxname}---name of the auxiliary variable in the package PNAME.  This auxiliary variable must exist and be specified by the user in that package.  The values in this auxiliary variable will be used to set the concentration associated with the flows for that boundary package.

\end{description}
\item \textbf{Block: FILEINPUT}

\begin{description}
\item \texttt{pname}---name of the flow package for which an SSMI6 input file contains a source concentration.  If this flow package is represented using an advanced transport package (SFT, LKT, MWT, or UZT), then the advanced transport package will override SSM terms specified here.

\item \texttt{SSMI6}---keyword to specify that record corresponds to a source sink mixing input file.

\item \texttt{FILEIN}---keyword to specify that an input filename is expected next.

\item \texttt{ssmi6\_filename}---character string that defines the path and filename for the file containing source and sink input data for the flow package. The SSMI6\_FILENAME file is a flexible input file that allows concentrations to be specified by stress period and with time series. Instructions for creating the SSMI6\_FILENAME input file are provided in the next section on file input for boundary concentrations.

\item \texttt{MIXED}---keyword to specify that these stress package boundaries will have the mixed condition.  The MIXED condition is described in the SOURCES block for AUXMIXED.  The MIXED condition allows for water to be withdrawn at a concentration that is less than the cell concentration.  It is intended primarily for representing evapotranspiration.

\end{description}


\end{description}

\vspace{5mm}
\subsubsection{Example Input File}
\lstinputlisting[style=inputfile]{./mf6ivar/examples/gwt-ssm-example.dat}

% when obs are ready, they should go here

\newpage
\subsection{Stress Package Concentrations (SPC) -- List-Based Input}
As mentioned in the previous section on the SSM Package, concentrations can be specified for GWF stress packages using auxiliary variables, or they can be specified using input files dedicated to this purpose.  The Stress Package Concentrations (SPC) input file can be used to provide concentrations that are assigned for GWF sources and sinks.  An SPC input file can be list based or array based.  List-based input files can be used for list-based GWF stress packages, such as wells, drains, and rivers.  Array-based input files can be used for array-based GWF stress packages, such as recharge and evapotranspiration (provided the READASARRAYS options is used; these packages can also be provided in a list-based format).  Array-based SPC input files are discussed in the next section.  This section describes the list-based input format for the SPC input file.  

An SPC6 file can be prepared to provide user-specified concentrations for a GWF stress package, such a Well or General-Head Boundary Package, for example.  One SPC6 file applies to one GWF stress package.  Names for the SPC6 input files are provided in the FILEINPUT block of the SSM Package.  SPC6 entries cannot be specified in the GWT name file.  Use of the SPC6 input file is an alternative to specifying stress package concentrations as auxiliary variables in the flow model stress package.  

The boundary number in the PERIOD block corresponds to the boundary number in the GWF stress period package.  Assignment of the boundary number is straightforward for the advanced packages (SFR, LAK, MAW, and UZF) because the features in these advanced packages are defined once at the beginning of the simulation and they do not change.  For the other stress packages, however, the order of boundaries may change between stress periods.  Consider the following Well Package input file, for example:

\begin{verbatim}
# This is an example of a GWF Well Package
# in which the order of the wells changes from
# stress period 1 to 2.  This must be explicitly
# handled by the user if using the SPC6 input
# for a GWT model.
BEGIN options
END options

BEGIN dimensions
  MAXBOUND  3
  BOUNDNAMES
END dimensions

BEGIN period  1
  1 77 65   -2200  SHALLOW_WELL
  2 77 65   -24.0  INTERMEDIATE_WELL
  3 77 65   -6.20  DEEP_WELL
END period

BEGIN period  2
  1 77 65   -1100  SHALLOW_WELL
  3 77 65   -3.10  DEEP_WELL
  2 77 65   -12.0  INTERMEDIATE_WELL
END period
\end{verbatim}

\noindent In this Well input file, the order of the wells changed between periods 1 and 2.  This reordering must be explicitly taken into account by the user when creating an SSMI6 file, because the boundary number in the SSMI file corresponds to the boundary number in the Well input file.  In stress period 1, boundary number 2 is the INTERMEDIATE\_WELL, whereas in stress period 2, boundary number 2 is the DEEP\_WELL.  When using this SSMI capability to specify boundary concentrations, it is recommended that users write the corresponding GWF stress packages using the same number, cell locations, and order of boundary conditions for each stress period.   In addition, users can activate the PRINT\_FLOWS option in the SSM input file.  When the SSM Package prints the individual solute flows to the transport list file, it includes a column containing the boundary concentration.  Users can check the boundary concentrations in this output to verify that they are assigned as intended.

\vspace{5mm}
\subsubsection{Structure of Blocks}
\vspace{5mm}

\noindent \textit{FOR EACH SIMULATION}
\lstinputlisting[style=blockdefinition]{./mf6ivar/tex/utl-spc-options.dat}
\lstinputlisting[style=blockdefinition]{./mf6ivar/tex/utl-spc-dimensions.dat}
\vspace{5mm}
\noindent \textit{FOR ANY STRESS PERIOD}
\lstinputlisting[style=blockdefinition]{./mf6ivar/tex/utl-spc-period.dat}

\vspace{5mm}
\subsubsection{Explanation of Variables}
\begin{description}
% DO NOT MODIFY THIS FILE DIRECTLY.  IT IS CREATED BY mf6ivar.py 

\item \textbf{Block: OPTIONS}

\begin{description}
\item \texttt{PRINT\_INPUT}---keyword to indicate that the list of spc information will be written to the listing file immediately after it is read.

\item \texttt{TS6}---keyword to specify that record corresponds to a time-series file.

\item \texttt{FILEIN}---keyword to specify that an input filename is expected next.

\item \texttt{ts6\_filename}---defines a time-series file defining time series that can be used to assign time-varying values. See the ``Time-Variable Input'' section for instructions on using the time-series capability.

\end{description}
\item \textbf{Block: DIMENSIONS}

\begin{description}
\item \texttt{maxbound}---integer value specifying the maximum number of spc cells that will be specified for use during any stress period.

\end{description}
\item \textbf{Block: PERIOD}

\begin{description}
\item \texttt{iper}---integer value specifying the starting stress period number for which the data specified in the PERIOD block apply.  IPER must be less than or equal to NPER in the TDIS Package and greater than zero.  The IPER value assigned to a stress period block must be greater than the IPER value assigned for the previous PERIOD block.  The information specified in the PERIOD block will continue to apply for all subsequent stress periods, unless the program encounters another PERIOD block.

\item \texttt{bndno}---integer value that defines the boundary package feature number associated with the specified PERIOD data on the line. BNDNO must be greater than zero and less than or equal to MAXBOUND.

\item \texttt{spcsetting}---line of information that is parsed into a keyword and values.  Keyword values that can be used to start the MAWSETTING string include: CONCENTRATION.

\begin{lstlisting}[style=blockdefinition]
CONCENTRATION <@concentration@>
\end{lstlisting}

\item \textcolor{blue}{\texttt{concentration}---is the boundary concentration. If the Options block includes a TIMESERIESFILE entry (see the ``Time-Variable Input'' section), values can be obtained from a time series by entering the time-series name in place of a numeric value. By default, the CONCENTRATION for each boundary feature is zero.}

\end{description}


\end{description}

\subsubsection{Example Input File}
\lstinputlisting[style=inputfile]{./mf6ivar/examples/utl-spc-example.dat}

% SPC array based
\newpage
\subsection{Stress Package Concentrations (SPC) -- Array-Based Input}

This section describes array-based input for the SPC input file.  If the READASARRAYS options is specified for either the GWF Recharge (RCH) or Evapotranspiration (EVT) Packages, then concentrations for these packages can be specified using array-based concentration input.  This SPC array-based input is distinguished from the list-based input in the previous section through specification of the READASARRAYS option.  When the READASARRAYS option is specified, then there is no DIMENSIONS block in the SPC input file.  Instead, the shape of the array for concentrations is the number of rows by number of columns (NROW, NCOL), for a regular MODFLOW grid (DIS), and the number of cells in a layer (NCPL) for a discretization by vertices (DISV) grid.

\vspace{5mm}
\subsubsection{Structure of Blocks}
\vspace{5mm}

\noindent \textit{FOR EACH SIMULATION}
\lstinputlisting[style=blockdefinition]{./mf6ivar/tex/utl-spca-options.dat}
\vspace{5mm}
\noindent \textit{FOR ANY STRESS PERIOD}
\lstinputlisting[style=blockdefinition]{./mf6ivar/tex/utl-spca-period.dat}

\vspace{5mm}
\subsubsection{Explanation of Variables}
\begin{description}
% DO NOT MODIFY THIS FILE DIRECTLY.  IT IS CREATED BY mf6ivar.py 

\item \textbf{Block: OPTIONS}

\begin{description}
\item \texttt{READASARRAYS}---indicates that array-based input will be used for the SPC Package.  This keyword must be specified to use array-based input.

\item \texttt{PRINT\_INPUT}---keyword to indicate that the list of spc information will be written to the listing file immediately after it is read.

\item \texttt{TAS6}---keyword to specify that record corresponds to a time-array-series file.

\item \texttt{FILEIN}---keyword to specify that an input filename is expected next.

\item \texttt{tas6\_filename}---defines a time-array-series file defining a time-array series that can be used to assign time-varying values. See the Time-Variable Input section for instructions on using the time-array series capability.

\end{description}
\item \textbf{Block: PERIOD}

\begin{description}
\item \texttt{iper}---integer value specifying the starting stress period number for which the data specified in the PERIOD block apply.  IPER must be less than or equal to NPER in the TDIS Package and greater than zero.  The IPER value assigned to a stress period block must be greater than the IPER value assigned for the previous PERIOD block.  The information specified in the PERIOD block will continue to apply for all subsequent stress periods, unless the program encounters another PERIOD block.

\item \texttt{concentration}---is the concentration of the associated Recharge or Evapotranspiration stress package.  The concentration array may be defined by a time-array series (see the "Using Time-Array Series in a Package" section).

\end{description}


\end{description}

\subsubsection{Example Input File}
\lstinputlisting[style=inputfile]{./mf6ivar/examples/utl-spca-example.dat}

