Input to the Viscosity (VSC) Package is read from the file that has type ``VSC6'' in the Name File.  If the VSC Package is active within a groundwater flow model, then the model will account for the dependence of fluid viscosity on solute concentration and the resulting changes in hydraulic conductivity and stress-package conductances, which vary inversely with viscosity.  Viscosity can be calculated as a function of one or more groundwater solute transport (GWT) species using an approach described in the Supplemental Technical Information document distributed with MODFLOW 6 (Chapter 8).  Only one VSC Package can be specified for a GWF model. The VSC Package can be coupled with one or more GWT Models so that the fluid viscosity is updated dynamically with one or more simulated concentration fields.

The VSC Package calculates fluid viscosity using the following equation from \cite{langevin2008seawat}:

\begin{equation}
\label{eqn:visclinear}
\mu = VISCREF + \sum_{i=1}^{NVISCSPECIES} DVISCDC_i \left ( CONCENTRATION_i - CVISCREF_i \right )
\end{equation}

\noindent where $\mu$ is the calculated viscosity, $VISCREF$ is the viscosity of a reference fluid, typically taken to be freshwater at a temperature of 20 degrees Celsius, $NVISCSPECIES$ is the number of chemical species that contribute to the viscosity calculation, $DVISCDC_i$ is the parameter that describes how viscosity changes linearly as a function of concentration for chemical species $i$ (i.e. the slope of a line that relates viscosity to concentration), $CONCENTRATION_i$ is the concentration of species $i$, and $CVISCREF_i$ is the reference concentration for species $i$ corresponding to when the viscosity of the reference fluid is equal to $VISCREF$, which is normally set to a concentration of zero.

In many applications, variations in temperature have a greater effect on fluid viscosity than variations in solute concentration. When a GWT model is formulated such that one of the transported ``species'' is heat (thermal energy), with ``concentration'' used to represent temperature \citep{zheng2010supplemental}, the viscosity can vary linearly with temperature, as it can with any other ``concentration.''  In that case, $CONCENTRATION_i$ and $CVISCREF_i$ represent the simulated and reference temperatures, respectively, and $DVISCDC_i$ represents the rate at which viscosity changes with temperature. In addition, the viscosity formula can optionally include a nonlinear dependence on temperature. In that case, equation 3 becomes

\begin{equation}
\label{eqn:viscnonlinear}
\mu = \mu_T(T) + \sum_{i=1}^{NVISCSPECIES} DVISCDC_i \left ( CONCENTRATION_i - CVISCREF_i \right )
\end{equation}

\noindent where the first term on the right-hand side, $\mu_T(T)$, is a nonlinear function of temperature, and the summation corresponds to the summation in equation \ref{eqn:visclinear}, in which one of the ``species'' is heat. The nonlinear term in equation \ref{eqn:viscnonlinear} is of the form

\begin{equation}
\label{eqn:munonlinear}
\mu_T(T) = CVISCREF_i \cdot A_2^{\left [ \frac {-A_3 \left ( CONCENTRATION_i - CVISCREF_i \right ) } {\left ( CONCENTRATION_i + A_4 \right ) \left ( CVISCREF_i + A_4 \right )} \right ]}
\end{equation}

\noindent where the coefficients $A_2$, $A_3$, and $A_4$ are specified by the user.  Values for $A_2$, $A_3$, and $A_4$ are commonly 10, 248.7, and 133.15, respectively  \citep{langevin2008seawat, Voss1984sutra}.
 
\subsubsection{Stress Packages}

For head-dependent stress packages, the VSC Package can adjust the conductance used to calculate flow between the boundary and the aquifer to account for variations in viscosity. Conductance is assumed to vary inversely with viscosity.

By default, the boundary viscosity is set equal to VISCREF, which, for freshwater, is typically set equal to 1.0. However, there are two additional options for setting the viscosity of a boundary package.  The first is to assign an auxiliary variable with the name ``VISCOSITY''.  If an auxiliary variable named ``VISCOSITY'' is detected, then it will be assigned as the viscosity of the fluid entering from the boundary.  Alternatively, a viscosity value can be calculated for each boundary using the viscosity equation described above and one or more concentrations provided as auxiliary variables.  In this case, the user must assign one auxiliary variable for each AUXSPECIESNAME listed in the PACKAGEDATA block below.  Thus, there must be NVISCSPECIES auxiliary variables, each with the identical name as those specified in PACKAGEDATA.  The VSC Package will calculate the viscosity for each boundary using these concentrations and the values specified for VISCREF, DVISCDC, and CVISCREF.  If the boundary package contains an auxiliary variable named VISCOSITY and also contains AUXSPECIESNAME auxiliary variables, then the boundary viscosity value will be assigned to the one in the VISCOSITY auxiliary variable.

A GWT Model can be used to calculate concentrations for the advanced stress packages (LAK, SFR, MAW, and UZF) if corresponding advanced transport packages are specified (LKT, SFT, MWT, and UZT).  The advanced stress packages have an input option called FLOW\_PACKAGE\_AUXILIARY\_NAME.  When activated, this option will result in the simulated concentration for a lake or other feature being copied from the advanced transport package into the auxiliary variable for the corresponding GWF stress package.  This means that the viscosity for a lake or stream, for example, can be dynamically updated during the simulation using concentrations from advanced transport packages that are fed into auxiliary variables in the advanced stress packages, and ultimately used by the VSC Package to calculate a fluid viscosity.  This concept also applies when multiple GWT Models are used simultaneously to simulate multiple species.  In this case, multiple auxiliary variables are required for an advanced stress package, with each one representing a concentration from a different GWT Model.  


\begin{longtable}{p{3cm} p{12cm}}
\caption{Description of viscosity terms for stress packages}
\tabularnewline
\hline
\hline
\textbf{Stress Package} & \textbf{Note} \\
\hline
\endhead
\hline
\endfoot
GHB & A VISCOSITY auxiliary variable or one or more auxiliary variables for calculating viscosity in the equation of state can be specified \\
RIV & A VISCOSITY auxiliary variable or one or more auxiliary variables for calculating viscosity in the equation of state can be specified \\
DRN & The drain formulation assumes that the drain boundary contains water of the same viscosity as the discharging water; auxiliary variables have no effect on the drain calculation  \\
LAK & A VISCOSITY auxiliary variable or one or more auxiliary variables for calculating viscosity in the equation of state can be specified \\
SFR & A VISCOSITY auxiliary variable or one or more auxiliary variables for calculating viscosity in the equation of state can be specified \\
MAW & A VISCOSITY auxiliary variable or one or more auxiliary variables for calculating viscosity in the equation of state can be specified \\
UZF & Viscosity variations not implemented \\
\end{longtable}

\vspace{5mm}
\subsubsection{Structure of Blocks}

\vspace{5mm}
\noindent \textit{FOR EACH SIMULATION}
\lstinputlisting[style=blockdefinition]{./mf6ivar/tex/gwf-vsc-options.dat}
\lstinputlisting[style=blockdefinition]{./mf6ivar/tex/gwf-vsc-dimensions.dat}
\lstinputlisting[style=blockdefinition]{./mf6ivar/tex/gwf-vsc-packagedata.dat}

\vspace{5mm}
\subsubsection{Explanation of Variables}
\begin{description}
% DO NOT MODIFY THIS FILE DIRECTLY.  IT IS CREATED BY mf6ivar.py 

\item \textbf{Block: OPTIONS}

\begin{description}
\item \texttt{viscref}---fluid reference viscosity used in the equation of state.  This value is set to 1.0 if not specified as an option.

\item \texttt{temperature\_species\_name}---string used to identify the auxspeciesname in PACKAGEDATA that corresponds to the temperature species.  There can be only one occurrence of this temperature species name in the PACKAGEDATA block or the program will terminate with an error.  This value has no effect if viscosity does not depend on temperature.

\item \texttt{thermal\_formulation}---may be used for specifying which viscosity formulation to use for the temperature species. Can be either LINEAR or NONLINEAR. The LINEAR viscosity formulation is the default.

\item \texttt{thermal\_a2}---is an empirical parameter specified by the user for calculating viscosity using a nonlinear formulation.  If A2 is not specified, a default value of 10.0 is assigned (Voss, 1984).

\item \texttt{thermal\_a3}---is an empirical parameter specified by the user for calculating viscosity using a nonlinear formulation.  If A3 is not specified, a default value of 248.37 is assigned (Voss, 1984).

\item \texttt{thermal\_a4}---is an empirical parameter specified by the user for calculating viscosity using a nonlinear formulation.  If A4 is not specified, a default value of 133.15 is assigned (Voss, 1984).

\item \texttt{VISCOSITY}---keyword to specify that record corresponds to viscosity.

\item \texttt{FILEOUT}---keyword to specify that an output filename is expected next.

\item \texttt{viscosityfile}---name of the binary output file to write viscosity information.  The viscosity file has the same format as the head file.  Viscosity values will be written to the viscosity file whenever heads are written to the binary head file.  The settings for controlling head output are contained in the Output Control option.

\end{description}
\item \textbf{Block: DIMENSIONS}

\begin{description}
\item \texttt{nviscspecies}---number of species used in the viscosity equation of state.  If either concentrations or temperature (or both) are used to update viscosity then then nrhospecies needs to be at least one.

\end{description}
\item \textbf{Block: PACKAGEDATA}

\begin{description}
\item \texttt{iviscspec}---integer value that defines the species number associated with the specified PACKAGEDATA data entered on each line. IVISCSPECIES must be greater than zero and less than or equal to NVISCSPECIES. Information must be specified for each of the NVISCSPECIES species or the program will terminate with an error.  The program will also terminate with an error if information for a species is specified more than once.

\item \texttt{dviscdc}---real value that defines the slope of the line defining the linear relationship between viscosity and temperature or between viscosity and concentration, depending on the type of species entered on each line.  If the value of AUXSPECIESNAME entered on a line corresponds to TEMPERATURE\_SPECIES\_NAME (in the OPTIONS block), this value will be used when VISCOSITY\_FUNC is equal to LINEAR (the default) in the OPTIONS block.  When VISCOSITY\_FUNC is set to NONLINEAR, a value for DVISCDC must be specified though it is not used.

\item \texttt{cviscref}---real value that defines the reference temperature or reference concentration value used for this species in the viscosity equation of state.  If AUXSPECIESNAME entered on a line corresponds to TEMPERATURE\_SPECIES\_NAME (in the OPTIONS block), then CVISCREF refers to a reference temperature, otherwise it refers to a reference concentration.

\item \texttt{modelname}---name of a GWT model used to simulate a species that will be used in the viscosity equation of state.  This name will have no effect if the simulation does not include a GWT model that corresponds to this GWF model.

\item \texttt{auxspeciesname}---name of an auxiliary variable in a GWF stress package that will be used for this species to calculate the viscosity values.  If a viscosity value is needed by the Viscosity Package then it will use the temperature or concentration values associated with this AUXSPECIESNAME in the viscosity equation of state.  For advanced stress packages (LAK, SFR, MAW, and UZF) that have an associated advanced transport package (LKT, SFT, MWT, and UZT), the FLOW\_PACKAGE\_AUXILIARY\_NAME option in the advanced transport package can be used to transfer simulated temperature or concentration(s) into the flow package auxiliary variable.  In this manner, the Viscosity Package can calculate viscosity values for lakes, streams, multi-aquifer wells, and unsaturated zone flow cells using simulated concentrations.

\end{description}


\end{description}

\vspace{5mm}
\subsubsection{Example Input File}
\lstinputlisting[style=inputfile]{./mf6ivar/examples/gwf-vsc-example.dat}
