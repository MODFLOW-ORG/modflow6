\mf contains most of the functionality of MODFLOW-2005, MODFLOW-NWT, MODFLOW-USG, and MODFLOW-LGR.  To the existing MODFLOW user, however, \mf will feel different from previous MODFLOW versions.  Some packages have been divided, renamed, or removed, and some capabilities, which previously caused confusion or were implemented due to computer memory limitations, are no longer supported (for example, ``quasi-3d confining units'' are not supported in the GWF Model).  The form of the input files for \mf is different from previous MODFLOW versions in that input files are now divided into blocks, and keywords are used to specify options and input variables.  Extensive testing was used as part of the development process to ensure that \mf simulation results are identical to the results from previous MODFLOW versions.  In some cases, it was not possible to exactly replicate the simulation results from previous MODFLOW versions.  In those cases, the differences could be explained by an option that is no longer supported, or because of slight differences in the underlying formulation.  

The following list, which is repeated from \cite{modflow6gwf}, summarizes the major differences between the GWF Model in \mf and previous versions of MODFLOW.  This list is intended for those with a general understanding of the capabilities in previous versions of MODFLOW.

\begin{enumerate}

\item The GWF Model in \mf supports three alternative input packages for specifying the grid used to discretize the groundwater system.  
\begin{itemize}
\item The Discretization (DIS) Package defines a grid based on layers, rows, and columns.  In this report, this type of grid is referred to as a ``regular MODFLOW grid'' because it corresponds to traditional MODFLOW grids.  An interior cell in a regular MODFLOW grid is connected to four adjacent cells in the same layer, to one overlying cell, and to one underlying cell.
\item The Discretization by Vertices (DISV) Package defines a grid using a list of ($x$, $y$) vertex pairs and the number of layers.  A list of vertices is provided by the user to define a two-dimensional horizontal grid in plan view.  This list of vertices may define a regular MODFLOW grid, or they may define more complex grids, such as grids consisting of triangles, hexagons, or Voronoi polygons, for example.  This same two-dimensional horizontal grid applies to each layer in the model.  Cells defined using the DISV Package are referenced by layer number and by the cell number within the horizontal grid.  Within a layer, a cell may be horizontally connected to any number of surrounding cells in that layer.  In the vertical direction a cell can be connected to only one overlying cell and only one underlying cell.  Grids defined with the DISV Package are considered to be unstructured.
\item The unstructured Discretization (DISU) Package is the most flexible of the three packages and is patterned after the unstructured grid implemented in MODFLOW-USG.  For each cell, the user specifies a list of connected cells and the connection properties.  When the DISU Package is used, cells are referenced only by their cell number; unlike the MODFLOW-USG approach, there is no concept of a layer in the DISU Package in \mfcomma but cells may still overlie or underlie one another.  
\end{itemize}

\item For the layered grid types supported in the GWF Model (DIS and DISV), cells can be permanently excluded from the grid for the simulation.  Input values (such as hydraulic conductivity) are still required for these excluded cells, and the program will write special codes or zero values for output, but the program does not allocate memory or store values for excluded cells during run time.  In this case, the matrix equations are formulated for a reduced system in which only the included cells are numbered.  Users can also mark excluded cells as ``vertical pass-through cells.''  When these vertical pass-through cells are encountered, the program connects the cells overlying and underlying the pass-through cell.  This capability allows ``pinched'' cells to be removed from the solution.  These options to exclude cells or exclude them as pass-through cells are available for the DIS and DISV Packages through specification of the IDOMAIN array; the IDOMAIN capability is not available for the DISU Package.

\item There is no longer a Basic Package input file.  Initial head values are specified using an Initial Conditions (IC) Package, and constant heads are specified using the Time Varying Specified Head (CHD) Package.  Cells that are permanently excluded from the simulation can be eliminated using the IDOMAIN capability entered through the DIS or DISV Packages.  For a cell that may transition from inactive (``dry'') to active (``wet'') during a simulation, the user can start the cell as inactive by assigning an initial head below the cell bottom.

\item The Newton-Raphson formulations and accompanying upstream weighting schemes implemented in MODFLOW-NWT and MODFLOW-USG for handling dry or nearly dry cells have been synthesized into a single formulation.  The Newton-Raphson formulation in the GWF Model for \mf remains an optional alternative to the standard formulation used in most previous MODFLOW versions. Much of this report is focused on systematically explaining standard and Newton-Raphson formulations for the GWF Model and its packages.

\item Information on temporal discretization, such as number of stress periods, period lengths, number of time steps, and time step multipliers, is specified at the simulation level, rather than for an individual model.  This information is provided in the Timing Module, which controls the temporal discretization and applies to all models within a simulation.  The Timing Module is part of the \mf framework and is described separately in \cite{modflow6framework}.

\item Aquifer properties used to calculate hydraulic conductance are specified in the Node Property Flow (NPF) Package.  In \mfcomma the NPF Package calculates intercell conductance values, manages cell wetting and drying, and adds Newton-Raphson terms for intercell flow expressions.  The NPF Package allows individual cells to be designated as confined or convertible; this was not an option in previous MODFLOW versions as the designation was by layer.  The NPF Package also has several options for simulating drainage problems and problems involving perched aquifers where an active cell overlies a partially saturated cell.  The default NPF Package behavior (in which none of these options are set) is the most stable for typical groundwater problems.  The default NPF Package behavior does not correspond to the default behavior for other MODFLOW internal flow packages.  The NPF Package does not support quasi-3D confining units.  The NPF Package replaces the Layer Property Flow (LPF), Block-Centered Flow (BCF), and Upstream Weighting (UPW) Packages from previous MODFLOW versions.  Capabilities of the Hydrogeologic Unit Flow (HUF) Package \citep{anderman2000modflow, anderman2003modflow} are not supported in the GWF Model of \mf.

\item Aquifer storage properties are specified in the Storage (STO) Package.  If the STO Package is excluded for a model, then the model represents steady-state conditions.  If the STO Package is included, users can specify steady-state or transient conditions by stress period as needed.  Compressible storage contributions are no longer approximated as zero for unconfined layers; contributions from pore drainage and compressible storage are separated in the model output.

\item The Horizontal Flow Barrier (HFB) Package \citep{hsieh1993hfb, modflow2005} in \mf allows barrier properties and locations to change by stress period.  The capability to change barriers by stress period was not supported in previous MODFLOW versions.

\item The GWF Model in \mf allows multiple stress packages of the same type to be specified for a single GWF Model.  This capability is also available in MODFLOW-CDSS \citep{banta2011modflow}.  Package entries written to the budget file and budget terms in the listing file are written separately for each package.

\item Input of boundary conditions for simulation in multiple stress periods is entered differently than for previous MODFLOW versions. Boundary conditions are specified for a stress period in a ``PERIOD'' block. These boundary conditions remain active at their specified values until a subsequent ``PERIOD'' block is encountered or the end of the simulation is reached.  Individual entries within the ``PERIOD'' block can be specified as a time-series entry.  Values for these variables, which may correspond to a well pumping rate or a drain conductance, for example, are interpolated from a time-series dataset, for each time step, using several different interpolation options.

\item The Flow and Head Boundary (FHB) Package \citep{leake1997documentation, modflow2005} is not supported in \mf; however, its capabilities can be replicated using the WEL Package, the CHD Package, and the new time-series capability.

\item There is one Evapotranspiration (EVT) Package for \mf. The \mf EVT Package contains the functionality of the MODFLOW-2005 EVT Package, the Segmented Evapotranspiration (ETS) Package \citep{modflowdrtpack}, and the Riparian Evapotranspiration (RIP-ET) Package \citep{modflowripetpack}.

\item A new Multi-Aquifer Well (MAW) Package replaces the Multi-Node Well (MNW1 and MNW2) Packages \citep{halford2002, konikow2009}. The new package does not contain all of the options available in MNW1 and MNW2, but it does contain the most commonly used ones.  It also has new capabilities for simulating flowing wells. The MAW Package is solved as part of the matrix solution and is tightly coupled with the GWF Model. This tight coupling with the GWF Model may substantially improve convergence for simulations of groundwater flow to multi-aquifer wells.

\item Most capabilities of the Stream (STR) and Streamflow Routing (SFR) Packages \citep{prudic1989str, modflowsfr1pack, modflowsfr2pack} are included in \mf as a new SFR Package.  The new SFR Package contains all of the functionality of the SFR Package in MODFLOW-2005 with the following exceptions: (a) the concept of a ``segment'' has been eliminated, (b) only rectangular cross sections are supported for stream reaches, and (c) unsaturated zone flow beneath stream reaches cannot be simulated.

\item A new Lake (LAK) Package replaces the existing MODFLOW Lake Packages \citep{modflowlak3pack}. In addition to being able to represent lakes that are incised into the model grid, the new LAK Package can also represent sub-grid scale lakes that are conceptualized as being on top of the model.  The status of a lake can change during the simulation between \texttt{ACTIVE}, \texttt{INACTIVE}, and \texttt{CONSTANT}.  The new package contains most of the capabilities available in previous LAK Packages, including the ability to apply recharge and evapotranspiration to underlying cells if the lake is dry.  The LAK Package documented here does not represent unsaturated zone flow beneath a lake or support for the coalescing lake option described in \cite{modflowlak3pack}. 

\item A new Unsaturated Zone Flow (UZF) Package, based on the one described by \cite{UZF}, is included in the GWF Model of \mfdot The new UZF Package allows the UZF capabilities to be applied to only selected cells of the GWF model. The new UZF Package also supports a multi-layer option, which allows for vertical heterogeneity in unsaturated zone properties.

\item A new Water Mover (MVR) Package is included in \mfdot  The MVR Package can be used to transfer water from individual ``provider'' features of selected packages (WEL, DRN, RIV, GHB, MAW, SFR, LAK, and UZF) to individual ''receiver'' features of the advanced packages (MAW, SFR, LAK, and UZF).  Simple rules are used to determine how much of the available water is moved from the provider to the receiver, which allows management controls to be represented. 

\item \mf contains a flexible new Observation (OBS) capability, which allows the user to define many different types of continuous-in-time or point-in-time observations.  The new OBS capability replaces the Observation Process \citep{hill2000modflow}, the Gage Package, and the HYDMOD capability \citep{hanson1999documentation} in previous MODFLOW versions.  Flow, head, and drawdown observations can be obtained for the GWF Model.  Flow and other package-specific observations, such as the head in a multi-aquifer well or lake stage, for example, can also be obtained.  These observed values can be used subsequently with a parameter estimation program or they can be used to make time-series plots of a wide range of simulated values.  The new OBS capability does not support specification of field-measured observations, calculation of residuals, or interpolation within a grid, as was supported in previous versions of the MODFLOW OBS Process.

\item The GWF Model described in this report does not support the following list of packages and capabilities.  Support for some of these capabilities may be added in future \mf versions.
  \begin{itemize}
    \item Drain with Return Flow Package \citep{modflowdrtpack}
    \item Reservoir Package \citep{fenske1996documentation},
    \item Seawater Intrusion Package \citep{bakker2013documentation},
    \item Surface-Water Routing Process \citep{hughes2012documentation},
    \item Connected Linear Network Process \citep{modflowusg},
    \item Parameter Value File \citep{modflow2005}, and
    \item Link to the MT3DMS Contaminant Transport Model \citep{zheng2001modflow}.  However, MT3D-USGS can read the head and budget files created by MODFLOW 6, but only if the GWF Model uses the DIS Package.  MT3D-USGS will not work with GWF output if the DISV or DISU Packages are used.
  \end{itemize}

\end{enumerate}

In addition to this list of major differences, there are other differences between \mf and previous MODFLOW versions in terms of the input and output files and the way users interact with the program.  These differences include:

\begin{enumerate}

\item The \mf program begins by reading a simulation name file.  The simulation name file must be named ``mfsim.nam.''

\item All real variables in \mf are declared as double precision floating point numbers.  Real variables written to binary output files are also written in double precision.

\item Unit numbers are no longer specified by the user.  Unit numbers are determined automatically by \mf based upon user-provided file names.

\item The GWF Model name file contains a list of packages that are active for the model.  Names for output files are not specified in the name file.  Names for output files, such as the head and budget files are specified in the OC Package.

\item The EXTERNAL option for reading arrays and lists is no longer supported; however, the OPEN/CLOSE option is still supported.  The SFAC option for lists is no longer supported; however, many packages allow for specification of an auxiliary variable which can serve as a multiplier on a column of values in the list.

\item The CHD Package contains new flexibility.  Cells can transition between constant-head cells and active cells during the simulation.  This was not allowed in previous MODFLOW versions.  Also, the CHD Packages no longer performs linear interpolation between a starting (shead) and ending head (ehead).  Only a single head value is provided for each constant-head cell.  The capability to linearly interpolate a head value for each time step within a stress period is available through the use of time series.

\item There are two different forms of input for the RCH and EVT Packages: array-based input and list-based input.  For models that use DIS Package, the RCH and EVT input can be provided as arrays, which is consistent with previous MODFLOW versions.  To use array input, the user must specify the READASARRAYS keyword in the options block.  The READASARRAYS option can also be used for models that use the DISV Package.  If the READASARRAYS option is not specified, then input to the RCH and EVT Packages is provided in list form.  List-based input is the only option supported when the DISU Package is used.

List-based input offers several advantages over the array-based input for specifying recharge and evapotranspiration.  First, multiple list entries can be specified for a single cell.  This makes it possible to divide a cell into multiple areas, and assign a different recharge or evapotranspiration rate for each area (perhaps based on land use or some other criteria).  In this case, the user would likely specify an auxiliary variable to serve as a multiplier.  This multiplier would be calculated by the user and provided in the input file as the fractional cell are for the individual recharge entries.  Another advantage to using list-based input for specifying recharge is that ``boundnames'' can be specified.  Boundnames work with the Observations capability and can be used to sum recharge or evapotranspiration rates for entries with the same boundname.  A disadvantage of the list-based input is that one cannot easily assign recharge or evapotranspiration rates to the entire model without specifying a list of model cells.  For this reason \mf also supports array-based input.

\item Calculation and reporting of drawdown for the model grid is no longer supported, as this calculation is easily performed as a postprocessing step.  Calculation of drawdown is supported as an observation type by the OBS Package; 
drawdown is calculated as the difference between the starting head specified in the IC Package and the calculated head.

\item There are differences in the output files created by \mfcomma such as:
\begin{itemize}

\item A separate listing file is written for the simulation.  This simulation listing file contains information about the simulation, including solver information.  Separate listing files are written for each GWF Model that is part of the simulation.

\item Unformatted head files written by \mf are consistent with those written by previous MODFLOW versions; however, all real values are written in double precision.

\item The budget file written by the GWF Model is always written in ``compact'' form (as opposed to full three-dimensional arrays) and uses new method codes, which allow model and package names to be written to the file.  Simulated intercell flows are always written in a compressed sparse row format, even for regular MODFLOW grids.

\item Information about the GWF Model grid is written to a separate file, called a ``binary grid file'' each time the model runs.  The binary grid file can be used by postprocessing programs for subsequent analysis.  The format of the binary grid file is described in a section on ``Binary Output Files.''

\end{itemize}


\end{enumerate}
