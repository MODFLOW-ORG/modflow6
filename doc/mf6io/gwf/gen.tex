Input to the General Boundary (GEN) Package is read from the file that has type ``GEN6'' in the Name File.  Any number of GEN Packages can be specified for a single groundwater flow model. Stress period data are not specified in the GEN Package input file but are provided to the package using the \mff Application Programming Interface (API).

\vspace{5mm}
\subsubsection{Structure of Blocks}
\vspace{5mm}

\noindent \textit{FOR EACH SIMULATION}
\lstinputlisting[style=blockdefinition]{./mf6ivar/tex/gwf-gen-options.dat}
\lstinputlisting[style=blockdefinition]{./mf6ivar/tex/gwf-gen-dimensions.dat}

\vspace{5mm}
\subsubsection{Explanation of Variables}
\begin{description}
% DO NOT MODIFY THIS FILE DIRECTLY.  IT IS CREATED BY mf6ivar.py 

\item \textbf{Block: OPTIONS}

\begin{description}
\item \texttt{BOUNDNAMES}---keyword to indicate that boundary names may be provided with the list of general boundary cells.

\item \texttt{PRINT\_INPUT}---keyword to indicate that the list of general boundary information will be written to the listing file immediately after it is read.

\item \texttt{PRINT\_FLOWS}---keyword to indicate that the list of general boundary flow rates will be printed to the listing file for every stress period time step in which ``BUDGET PRINT'' is specified in Output Control.  If there is no Output Control option and ``PRINT\_FLOWS'' is specified, then flow rates are printed for the last time step of each stress period.

\item \texttt{SAVE\_FLOWS}---keyword to indicate that general boundary flow terms will be written to the file specified with ``BUDGET FILEOUT'' in Output Control.

\item \texttt{OBS6}---keyword to specify that record corresponds to an observations file.

\item \texttt{FILEIN}---keyword to specify that an input filename is expected next.

\item \texttt{obs6\_filename}---name of input file to define observations for the General Boundary package. See the ``Observation utility'' section for instructions for preparing observation input files. Tables \ref{table:gwf-obstypetable} and \ref{table:gwt-obstypetable} lists observation type(s) supported by the General Boundary package.

\item \texttt{MOVER}---keyword to indicate that this instance of the General Boundary Package can be used with the Water Mover (MVR) Package.  When the MOVER option is specified, additional memory is allocated within the package to store the available, provided, and received water.

\end{description}
\item \textbf{Block: DIMENSIONS}

\begin{description}
\item \texttt{maxbound}---integer value specifying the maximum number of general boundary cells that will be specified for use during any stress period.

\end{description}


\end{description}

\vspace{5mm}
\subsubsection{Example Input File}
\lstinputlisting[style=inputfile]{./mf6ivar/examples/gwf-gen-example.dat}

\vspace{5mm}
\subsubsection{Available observation types}
General Boundary Package observations include the simulated general boundary flow rates (\texttt{gen}) and the general boundary discharge that is available for the MVR package (\texttt{to-mvr}). The data required for each GEN Package observation type is defined in table~\ref{table:gwf-genobstype}. The sum of \texttt{gen} and \texttt{to-mvr} is equal to the simulated general boundary flow rate. Negative and positive values for an observation represent a loss from and gain to the GWF model, respectively.

\begin{longtable}{p{2cm} p{2.75cm} p{2cm} p{1.25cm} p{7cm}}
\caption{Available GEN Package observation types} \tabularnewline

\hline
\hline
\textbf{Stress Package} & \textbf{Observation type} & \textbf{ID} & \textbf{ID2} & \textbf{Description} \\
\hline
\endhead

\hline
\endfoot

GEN & gen & cellid or boundname & -- & Flow between the groundwater system and a general boundary or group of general boundaries. \\
GEN & to-mvr & cellid or boundname & -- & General boundary discharge that is available for the MVR package from a general boundary or group of general boundaries.

\label{table:gwf-genobstype}
\end{longtable}

\vspace{5mm}
\subsubsection{Example Observation Input File}
\lstinputlisting[style=inputfile]{./mf6ivar/examples/gwf-gen-example-obs.dat}
