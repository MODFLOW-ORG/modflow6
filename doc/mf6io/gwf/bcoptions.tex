\newcommand{\packagename}[2]{\texttt{PACKAGENAME packagename}---keyword and name of this instance of the {#1} Package. \texttt{packagename} must be provided as a string, with no more than 16 characters. If not provided, then this {#1} Package instance will be given the name {#2}\_ibcnum, where ibcnum is the sequential {#1} Package number as determined by the order in the GWF Name File.}

\newcommand{\auxname}[1]{\texttt{AUXILIARY auxname [auxname2 [auxname3] ...]}---defines a list of one or more auxiliary variables, with the names \texttt{auxname}, \texttt{auxname2}, and so forth.  There is no limit on the number of auxiliary variables that can be provided on this line; however, lists of information provided in subsequent blocks must have a column of data for each auxiliary variable defined here.   Comments cannot be provided anywhere on this line as they will be interpreted as auxiliary variable names.  Auxiliary variables may not be used by the {#1} model, but they will be available for use by other parts of the program.  The ``AUX'' keyword can be used as a substitute for ``AUXILIARY''. The program will terminate with an error if auxiliary variables are specified on more than one line in the options block.}  

\newcommand{\auxmultname}[1]{\texttt{AUXMULTNAME auxnamemult}---name of auxiliary variable to be used as multiplier of {#1}.}

\newcommand{\boundnames}[1]{\texttt{BOUNDNAMES}---keyword to indicate that boundary names may be provided with the list of {#1} cells.}

\newcommand{\boundnamesnotlayered}[1]{\texttt{BOUNDNAMES}---keyword to indicate that boundary names may be provided with the list of {#1} cells. Incompatible with the READASARRAYS option.}

%\newcommand{\layered}{\texttt{LAYERED}---keyword to indicate that input will be read as arrays. Invalid for an unstructured grid (DISU).}

\newcommand{\printinput}[1]{\texttt{PRINT\_INPUT}---keyword to indicate that the list of {#1} information will be written to the listing file immediately after it is read.}

\newcommand{\printflows}[1]{\texttt{PRINT\_FLOWS}---keyword to indicate that the list of {#1} flow rates will be printed to the listing file for every stress period in which ``BUDGET PRINT'' is specified in Output Control.  If there is no Output Control option and \texttt{PRINT\_FLOWS} is specified, then flow rates are printed for the last time step of each stress period.}

\newcommand{\saveflows}[1]{\texttt{SAVE\_FLOWS}---keyword to indicate that {#1} flow terms will be written to the file specified with ``BUDGET SAVE FILE'' in Output Control.}

\newcommand{\timeseries}[0]{\texttt{TIMESERIESFILE time-series-filename}---defines a time-series file defining time series that can be used to assign time-varying values. See the ``Time-Variable Input'' section for instructions on using the time-series capability.}

\newcommand{\timearrayseries}{\texttt{TIMEARRAYSERIESFILE time-array-series-filename}---defines a time-array-series file defining a time-array series that can be used to assign time-varying values. See the ��Time-Variable Input�� section for instructions on using the time-array series capability.}

\newcommand{\timearrayserieslayered}{\texttt{TIMEARRAYSERIESFILE time-array-series-filename}---name of a time-array-series file defining a time-array series that can be used to assign time-varying values when input is read in array form. See the ��Time-Variable Input�� section for instructions on using the time-array series capability. Valid only when the READASARRAYS option is used.}

\newcommand{\maxbound}[1]{\texttt{MAXBOUND maxbound}---keyword and integer value specifying the maximum number of {#1} cells that will be specified for use during any stress period.}

\newcommand{\iper}[1]{\texttt{BEGIN PERIOD iper}---keywords and integer value specifying the starting stress period number for which the following list of {#1} cells will apply.  The list of {#1} cells specified in this block will remain active with their specified head values until a new ``BEGIN PERIOD'' block is detected.}

\newcommand{\cellid}[0]{\texttt{cellid}---is the cell identifier, and depends on the type of grid that is used for the simulation.  For a structured grid that uses the DIS input file, \texttt{cellid} is the layer, row, and column.   For a grid that uses the DISV input file, \texttt{cellid} is the layer and cell2d number.  If the model uses the unstructured discretization (DISU) input file, then \texttt{cellid} is the node number for the cell.}

\newcommand{\xyzaux}[1]{\texttt{x,y,z}---represents the values of the auxiliary variables for each {#1}. The values of auxiliary variables must be present for each {#1}. The values must be specified in the order of the auxiliary variables specified in the OPTIONS block.  If the package supports time series and the Options block includes a TIMESERIESFILE entry (see the ``Time-Variable Input'' section), values can be obtained from a time series by entering the time-series name in place of a numeric value.}

\newcommand{\boundname}[1]{\texttt{boundname}---name of the {#1} cell.  \texttt{boundname} is an ASCII character variable that can contain as many as 40 characters.  If \texttt{boundname} contains spaces in it, then the entire name must be enclosed within single quotes.}

\newcommand{\newtonopt}[1]{\texttt{NEWTON}---keyword that activates the Newton-Raphson formulation for the {#1} Package.}

\newcommand{\obsopt}[1]{\texttt{OBS8 observation-input-filename}---keyword and name of input file to define observations for the {#1} package. See the ``Observation utility'' section for instructions for preparing observation input files. Table \ref{table:obstype} lists observation type(s) supported by the {#1} package.}

\newcommand{\mover}[1]{\texttt{MOVER}---keyword to indicate that this instance of the {#1} Package can be used with the Water Mover (MVR) Package.  When the \texttt{MOVER} option is specified, additional memory is allocated within the package to store the available, provided, and received water.}

\newcommand{\packageperioddescription}{All of the stress package information in the PERIOD block will continue to apply for subsequent stress periods until the end of the simulation, or until another PERIOD block is encountered.  When a new PERIOD block is encountered, all of the stresses from the previous block are replaced with the stresses in the new PERIOD block.  Note that this behavior is different from the advanced packages (MAW, SFR, LAK, and UZF).  To turn off all of the stresses for a stress period, a PERIOD block must be specified with no entries.  If a PERIOD block is not specified for the first stress period, then no stresses will be applied until the \texttt{iper} value of the first PERIOD block in the file.}

\newcommand{\packageperioddescriptionarray}[1]{All of the stress package information in the PERIOD block will continue to apply for subsequent stress periods until the end of the simulation, or until another PERIOD block is encountered.  When a new PERIOD block is encountered, the array-based input specified by the user will replace the arrays currently in memory.  If an array is not specified in the period block, then that array will retain its present values in memory.  With the array-based input, the user must specify a {#1} rate of zero in order to turn {#1} off for a stress period.  This behavior is different from list-based input in which an empty PERIOD block results in no stresses being applied.}

\newcommand{\advancedpackageperioddescription}[2]{All of the advanced stress package information in the PERIOD block will continue to apply for subsequent stress periods until the end of the simulation, or until another PERIOD block is encountered.  When a new PERIOD block is encountered only the {#2} specified in the new period block will be changed.  A {#1} not specified in the new period block will continue to behave according to its specification in the previous PERIOD block.  Note that this behavior is different from the simple stress packages (CHD, WEL, DRN, RIV, GHB, RCH and EVT), in which any stress not specified in a new PERIOD block will be removed.  To turn off all of the advanced stresses for a stress period, a PERIOD block must be specified with settings that deactivate the {#2}.  If a PERIOD block is not specified for the first stress period, then no stresses will be applied.}
