Input to the Streamflow Routing (SFR) Package is read from the file that has type ``SFR6'' in the Name File. Any number of SFR Packages can be specified for a single groundwater flow model; however, water cannot be routed between reaches in separate packages except in cases where the MVR Package is used to route water between separate packages.

Reach connectivity must be explicitly specified for this version of the SFR Package, unlike the abbreviated SFR Package segment connectivity specified in previous versions of MODFLOW. Explicit specification of reach connectivity has been adopted to facilitate better validation of stream network connectivity by the program. Explicit reach connectivity means that a reach must be specified as an upstream connection for all downstream connections to the reach. Downstream connections for a reach are denoted with a negative reach number. Flow in a reach is unidirectional, always flowing from the upstream end to the downstream end of a reach. An example of the reach connectivity for a hypothetical stream network is shown in figure~\ref{fig:sfr-connectivity}.

\begin{figure}[ht]
	\centering
	\includegraphics[scale=1.0]{../Figures/sfr-connectivity}
	\caption[Illustration of a simple stream network having seven reaches with a junction having two reaches, a confluence of two reaches, and the resulting reach connectivity]{Simple stream network having seven reaches with a junction having two reaches, a confluence of two reaches, and the resulting reach connectivity. Downstream connections for a reach must include the reach as an upstream connection for all downstream connections to the reach. Downstream connections for a  reach are denoted with a negative reach number}
	\label{fig:sfr-connectivity}
\end{figure}


\vspace{5mm}
\subsubsection{Structure of Blocks}
\vspace{5mm}

\noindent \textit{FOR EACH SIMULATION}
\lstinputlisting[style=blockdefinition]{./mf6ivar/tex/gwf-sfr-options.dat}
\lstinputlisting[style=blockdefinition]{./mf6ivar/tex/gwf-sfr-dimensions.dat}
\lstinputlisting[style=blockdefinition]{./mf6ivar/tex/gwf-sfr-packagedata.dat}
\lstinputlisting[style=blockdefinition]{./mf6ivar/tex/gwf-sfr-connectiondata.dat}
\noindent \textit{IF \texttt{ndv} IS GREATER THAN ZERO FOR ANY REACH}
\lstinputlisting[style=blockdefinition]{./mf6ivar/tex/gwf-sfr-diversions.dat}

\vspace{5mm}
\noindent \textit{FOR ANY STRESS PERIOD}
\lstinputlisting[style=blockdefinition]{./mf6ivar/tex/gwf-sfr-period.dat}
\advancedpackageperioddescription{reach}{reaches}

\vspace{5mm}
\subsubsection{Explanation of Variables}
\begin{description}
% DO NOT MODIFY THIS FILE DIRECTLY.  IT IS CREATED BY mf6ivar.py 

\item \textbf{Block: OPTIONS}

\begin{description}
\item \texttt{STORAGE}---keyword that activates storage contributions to the stream-flow routing package continuity equation.

\item \texttt{auxiliary}---defines an array of one or more auxiliary variable names.  There is no limit on the number of auxiliary variables that can be provided on this line; however, lists of information provided in subsequent blocks must have a column of data for each auxiliary variable name defined here.   The number of auxiliary variables detected on this line determines the value for naux.  Comments cannot be provided anywhere on this line as they will be interpreted as auxiliary variable names.  Auxiliary variables may not be used by the package, but they will be available for use by other parts of the program.  The program will terminate with an error if auxiliary variables are specified on more than one line in the options block.

\item \texttt{BOUNDNAMES}---keyword to indicate that boundary names may be provided with the list of stream reach cells.

\item \texttt{PRINT\_INPUT}---keyword to indicate that the list of stream reach information will be written to the listing file immediately after it is read.

\item \texttt{PRINT\_STAGE}---keyword to indicate that the list of stream reach stages will be printed to the listing file for every stress period in which ``HEAD PRINT'' is specified in Output Control.  If there is no Output Control option and PRINT\_STAGE is specified, then stages are printed for the last time step of each stress period.

\item \texttt{PRINT\_FLOWS}---keyword to indicate that the list of stream reach flow rates will be printed to the listing file for every stress period time step in which ``BUDGET PRINT'' is specified in Output Control.  If there is no Output Control option and ``PRINT\_FLOWS'' is specified, then flow rates are printed for the last time step of each stress period.

\item \texttt{SAVE\_FLOWS}---keyword to indicate that stream reach flow terms will be written to the file specified with ``BUDGET FILEOUT'' in Output Control.

\item \texttt{STAGE}---keyword to specify that record corresponds to stage.

\item \texttt{stagefile}---name of the binary output file to write stage information.

\item \texttt{BUDGET}---keyword to specify that record corresponds to the budget.

\item \texttt{FILEOUT}---keyword to specify that an output filename is expected next.

\item \texttt{budgetfile}---name of the binary output file to write budget information.

\item \texttt{BUDGETCSV}---keyword to specify that record corresponds to the budget CSV.

\item \texttt{budgetcsvfile}---name of the comma-separated value (CSV) output file to write budget summary information.  A budget summary record will be written to this file for each time step of the simulation.

\item \texttt{PACKAGE\_CONVERGENCE}---keyword to specify that record corresponds to the package convergence comma spaced values file.

\item \texttt{package\_convergence\_filename}---name of the comma spaced values output file to write package convergence information.

\item \texttt{TS6}---keyword to specify that record corresponds to a time-series file.

\item \texttt{FILEIN}---keyword to specify that an input filename is expected next.

\item \texttt{ts6\_filename}---defines a time-series file defining time series that can be used to assign time-varying values. See the ``Time-Variable Input'' section for instructions on using the time-series capability.

\item \texttt{OBS6}---keyword to specify that record corresponds to an observations file.

\item \texttt{obs6\_filename}---name of input file to define observations for the SFR package. See the ``Observation utility'' section for instructions for preparing observation input files. Tables \ref{table:gwf-obstypetable} and \ref{table:gwt-obstypetable} lists observation type(s) supported by the SFR package.

\item \texttt{MOVER}---keyword to indicate that this instance of the SFR Package can be used with the Water Mover (MVR) Package.  When the MOVER option is specified, additional memory is allocated within the package to store the available, provided, and received water.

\item \texttt{maximum\_picard\_iterations}---integer value that defines the maximum number of Streamflow Routing picard iterations allowed when solving for reach stages and flows as part of the GWF formulate step. Picard iterations are used to minimize differences in SFR package results between subsequent GWF picard (non-linear) iterations as a result of non-optimal reach numbering. If reaches are numbered in order, from upstream to downstream, MAXIMUM\_PICARD\_ITERATIONS can be set to 1 to reduce model run time. By default, MAXIMUM\_PICARD\_ITERATIONS is equal to 100.

\item \texttt{maximum\_iterations}---integer value that defines the maximum number of Streamflow Routing Newton-Raphson iterations allowed for a reach. By default, MAXIMUM\_ITERATIONS is equal to 100. MAXIMUM\_ITERATIONS would only need to be increased from the default value if one or more reach in a simulation has a large water budget error.

\item \texttt{maximum\_depth\_change}---real value that defines the depth closure tolerance. By default, MAXIMUM\_DEPTH\_CHANGE is equal to $1 \times 10^{-5}$. The MAXIMUM\_STAGE\_CHANGE would only need to be increased or decreased from the default value if the water budget error for one or more reach is too small or too large, respectively.

\item \texttt{length\_conversion}---real value that is used to convert user-specified Manning's roughness coefficients from meters to model length units. LENGTH\_CONVERSION should be set to 3.28081, 1.0, and 100.0 when using length units (LENGTH\_UNITS) of feet, meters, or centimeters in the simulation, respectively. LENGTH\_CONVERSION does not need to be specified if LENGTH\_UNITS are meters.

\item \texttt{time\_conversion}---real value that is used to convert user-specified Manning's roughness coefficients from seconds to model time units. TIME\_CONVERSION should be set to 1.0, 60.0, 3,600.0, 86,400.0, and 31,557,600.0 when using time units (TIME\_UNITS) of seconds, minutes, hours, days, or years in the simulation, respectively. TIME\_CONVERSION does not need to be specified if TIME\_UNITS are seconds.

\end{description}
\item \textbf{Block: DIMENSIONS}

\begin{description}
\item \texttt{nreaches}---integer value specifying the number of stream reaches.  There must be NREACHES entries in the PACKAGEDATA block.

\end{description}
\item \textbf{Block: PACKAGEDATA}

\begin{description}
\item \texttt{ifno}---integer value that defines the feature (reach) number associated with the specified PACKAGEDATA data on the line. IFNO must be greater than zero and less than or equal to NREACHES. Reach information must be specified for every reach or the program will terminate with an error.  The program will also terminate with an error if information for a reach is specified more than once.

\item \texttt{cellid}---is the cell identifier, and depends on the type of grid that is used for the simulation. For a structured grid that uses the DIS input file, CELLID is the layer, row, and column. For a grid that uses the DISV input file, CELLID is the layer and CELL2D number. If the model uses the unstructured discretization (DISU) input file, CELLID is the node number for the cell. For reaches that are not connected to an underlying GWF cell, a zero should be specified for each grid dimension. For example, for a DIS grid a CELLID of 0 0 0 should be specified. Reach-aquifer flow is not calculated for unconnected reaches. The keyword NONE can be still be specified to identify unconnected reaches for backward compatibility with previous versions of MODFLOW 6 but eventually NONE will be deprecated and will cause MODFLOW 6 to terminate with an error.

\item \texttt{rlen}---real value that defines the reach length. RLEN must be greater than zero.

\item \texttt{rwid}---real value that defines the reach width. RWID must be greater than zero.

\item \texttt{rgrd}---real value that defines the stream gradient (slope) across the reach. RGRD must be greater than zero.

\item \texttt{rtp}---real value that defines the bottom elevation of the reach.

\item \texttt{rbth}---real value that defines the thickness of the reach streambed. RBTH can be any value if the reach is not connected to an underlying GWF cell. Otherwise, RBTH must be greater than zero.

\item \textcolor{blue}{\texttt{rhk}---real or character value that defines the hydraulic conductivity of the reach streambed. RHK can be any positive value if the reach is not connected to an underlying GWF cell. Otherwise, RHK must be greater than zero. If the Options block includes a TIMESERIESFILE entry (see the ``Time-Variable Input'' section), values can be obtained from a time series by entering the time-series name in place of a numeric value.}

\item \textcolor{blue}{\texttt{man}---real or character value that defines the Manning's roughness coefficient for the reach. MAN must be greater than zero.  If the Options block includes a TIMESERIESFILE entry (see the ``Time-Variable Input'' section), values can be obtained from a time series by entering the time-series name in place of a numeric value.}

\item \texttt{ncon}---integer value that defines the number of reaches connected to the reach.  If a value of zero is specified for NCON an entry for IFNO is still required in the subsequent CONNECTIONDATA block.

\item \textcolor{blue}{\texttt{ustrf}---real value that defines the fraction of upstream flow from each upstream reach that is applied as upstream inflow to the reach. The sum of all USTRF values for all reaches connected to the same upstream reach must be equal to one and USTRF must be greater than or equal to zero. If the Options block includes a TIMESERIESFILE entry (see the ``Time-Variable Input'' section), values can be obtained from a time series by entering the time-series name in place of a numeric value.}

\item \texttt{ndv}---integer value that defines the number of downstream diversions for the reach.

\item \textcolor{blue}{\texttt{aux}---represents the values of the auxiliary variables for each stream reach. The values of auxiliary variables must be present for each stream reach. The values must be specified in the order of the auxiliary variables specified in the OPTIONS block.  If the package supports time series and the Options block includes a TIMESERIESFILE entry (see the ``Time-Variable Input'' section), values can be obtained from a time series by entering the time-series name in place of a numeric value.}

\item \texttt{boundname}---name of the stream reach cell.  BOUNDNAME is an ASCII character variable that can contain as many as 40 characters.  If BOUNDNAME contains spaces in it, then the entire name must be enclosed within single quotes.

\end{description}
\item \textbf{Block: CROSSSECTIONS}

\begin{description}
\item \texttt{ifno}---integer value that defines the feature (reach) number associated with the specified cross-section table file on the line. IFNO must be greater than zero and less than or equal to NREACHES. The program will also terminate with an error if table information for a reach is specified more than once.

\item \texttt{TAB6}---keyword to specify that record corresponds to a cross-section table file.

\item \texttt{FILEIN}---keyword to specify that an input filename is expected next.

\item \texttt{tab6\_filename}---character string that defines the path and filename for the file containing cross-section table data for the reach. The TAB6\_FILENAME file includes the number of entries in the file and the station elevation data in terms of the fractional width and the reach depth. Instructions for creating the TAB6\_FILENAME input file are provided in SFR Reach Cross-Section Table Input File section.

\end{description}
\item \textbf{Block: CONNECTIONDATA}

\begin{description}
\item \texttt{ifno}---integer value that defines the feature (reach) number associated with the specified CONNECTIONDATA data on the line. IFNO must be greater than zero and less than or equal to NREACHES. Reach connection information must be specified for every reach or the program will terminate with an error.  The program will also terminate with an error if connection information for a reach is specified more than once.

\item \texttt{ic}---integer value that defines the reach number of the reach connected to the current reach and whether it is connected to the upstream or downstream end of the reach. Negative IC numbers indicate connected reaches are connected to the downstream end of the current reach. Positive IC numbers indicate connected reaches are connected to the upstream end of the current reach. The absolute value of IC must be greater than zero and less than or equal to NREACHES. IC should not be specified when NCON is zero but must be specified otherwise.

\end{description}
\item \textbf{Block: DIVERSIONS}

\begin{description}
\item \texttt{ifno}---integer value that defines the feature (reach) number associated with the specified DIVERSIONS data on the line. IFNO must be greater than zero and less than or equal to NREACHES.  Reach diversion information must be specified for every reach with a NDV value greater than 0 or the program will terminate with an error.  The program will also terminate with an error if diversion information for a given reach diversion is specified more than once.

\item \texttt{idv}---integer value that defines the downstream diversion number for the diversion for reach IFNO. IDV must be greater than zero and less than or equal to NDV for reach IFNO.

\item \texttt{iconr}---integer value that defines the downstream reach that will receive the diverted water. IDV must be greater than zero and less than or equal to NREACHES. Furthermore, reach  ICONR must be a downstream connection for reach IFNO.

\item \texttt{cprior}---character string value that defines the the prioritization system for the diversion, such as when insufficient water is available to meet all diversion stipulations, and is used in conjunction with the value of FLOW value specified in the STRESS\_PERIOD\_DATA section. Available diversion options include:  (1) CPRIOR = `FRACTION', then the amount of the diversion is computed as a fraction of the streamflow leaving reach IFNO ($Q_{DS}$); in this case, 0.0 $\le$ DIVFLOW $\le$ 1.0.  (2) CPRIOR = `EXCESS', a diversion is made only if $Q_{DS}$ for reach IFNO exceeds the value of DIVFLOW. If this occurs, then the quantity of water diverted is the excess flow ($Q_{DS} -$ DIVFLOW) and $Q_{DS}$ from reach IFNO is set equal to DIVFLOW. This represents a flood-control type of diversion, as described by Danskin and Hanson (2002). (3) CPRIOR = `THRESHOLD', then if $Q_{DS}$ in reach IFNO is less than the specified diversion flow DIVFLOW, no water is diverted from reach IFNO. If $Q_{DS}$ in reach IFNO is greater than or equal to DIVFLOW, DIVFLOW is diverted and $Q_{DS}$ is set to the remainder ($Q_{DS} -$ DIVFLOW)). This approach assumes that once flow in the stream is sufficiently low, diversions from the stream cease, and is the `priority' algorithm that originally was programmed into the STR1 Package (Prudic, 1989).  (4) CPRIOR = `UPTO' -- if $Q_{DS}$ in reach IFNO is greater than or equal to the specified diversion flow DIVFLOW, $Q_{DS}$ is reduced by DIVFLOW. If $Q_{DS}$ in reach IFNO is less than DIVFLOW, DIVFLOW is set to $Q_{DS}$ and there will be no flow available for reaches connected to downstream end of reach IFNO.

\end{description}
\item \textbf{Block: INITIALSTAGES}

\begin{description}
\item \texttt{ifno}---integer value that defines the feature (reach) number associated with the specified initial stage. The program will terminate with a error if IFNO is less than one or greater than NREACHES. The program will also terminate with an error if initialstages information is specified and STORAGE is not specified in the Options block.

\item \texttt{initialstage}---real value that defines the initial stage for the reach. The program will terminate with an error if INITIALSTAGE is less than the RTP value for reach IFNO defined in the PACKAGEDATA block.

\end{description}
\item \textbf{Block: PERIOD}

\begin{description}
\item \texttt{iper}---integer value specifying the starting stress period number for which the data specified in the PERIOD block apply.  IPER must be less than or equal to NPER in the TDIS Package and greater than zero.  The IPER value assigned to a stress period block must be greater than the IPER value assigned for the previous PERIOD block.  The information specified in the PERIOD block will continue to apply for all subsequent stress periods, unless the program encounters another PERIOD block.

\item \texttt{ifno}---integer value that defines the feature (reach) number associated with the specified PERIOD data on the line. IFNO must be greater than zero and less than or equal to NREACHES.

\item \texttt{sfrsetting}---line of information that is parsed into a keyword and values.  Keyword values that can be used to start the SFRSETTING string include: STATUS, BEDK, MANNING, STAGE, INFLOW, RAINFALL, EVAPORATION, RUNOFF, DIVERSION, UPSTREAM\_FRACTION, and AUXILIARY.

\begin{lstlisting}[style=blockdefinition]
STATUS <status>
BEDK <@bedk@>
MANNING <@manning@>
STAGE <@stage@>
INFLOW <@inflow@>
RAINFALL <@rainfall@>
EVAPORATION <@evaporation@>
RUNOFF <@runoff@>
DIVERSION <idv> <@divflow@> 
UPSTREAM_FRACTION <upstream_fraction>
CROSS_SECTION TAB6 FILEIN <tab6_filename> 
AUXILIARY <auxname> <@auxval@> 
\end{lstlisting}

\item \texttt{status}---keyword option to define stream reach status.  STATUS can be ACTIVE, INACTIVE, or SIMPLE. The SIMPLE STATUS option simulates streamflow using a user-specified stage for a reach or a stage set to the top of the reach (depth = 0). In cases where the simulated leakage calculated using the specified stage exceeds the sum of inflows to the reach, the stage is set to the top of the reach and leakage is set equal to the sum of inflows. Upstream fractions should be changed using the UPSTREAM\_FRACTION SFRSETTING if the status for one or more reaches is changed to ACTIVE or INACTIVE. For example, if one of two downstream connections for a reach is inactivated, the upstream fraction for the active and inactive downstream reach should be changed to 1.0 and 0.0, respectively, to ensure that the active reach receives all of the downstream outflow from the upstream reach. By default, STATUS is ACTIVE.

\item \textcolor{blue}{\texttt{bedk}---real or character value that defines the hydraulic conductivity of the reach streambed. BEDK can be any positive value if the reach is not connected to an underlying GWF cell. Otherwise, BEDK must be greater than zero. If the Options block includes a TIMESERIESFILE entry (see the ``Time-Variable Input'' section), values can be obtained from a time series by entering the time-series name in place of a numeric value.}

\item \textcolor{blue}{\texttt{manning}---real or character value that defines the Manning's roughness coefficient for the reach. MANNING must be greater than zero.  If the Options block includes a TIMESERIESFILE entry (see the ``Time-Variable Input'' section), values can be obtained from a time series by entering the time-series name in place of a numeric value.}

\item \textcolor{blue}{\texttt{stage}---real or character value that defines the stage for the reach. The specified STAGE is only applied if the reach uses the simple routing option. If STAGE is not specified for reaches that use the simple routing option, the specified stage is set to the top of the reach. If the Options block includes a TIMESERIESFILE entry (see the ``Time-Variable Input'' section), values can be obtained from a time series by entering the time-series name in place of a numeric value.}

\item \textcolor{blue}{\texttt{inflow}---real or character value that defines the volumetric inflow rate for the streamflow routing reach. If the Options block includes a TIMESERIESFILE entry (see the ``Time-Variable Input'' section), values can be obtained from a time series by entering the time-series name in place of a numeric value. By default, inflow rates are zero for each reach.}

\item \textcolor{blue}{\texttt{rainfall}---real or character value that defines the  volumetric rate per unit area of water added by precipitation directly on the streamflow routing reach. If the Options block includes a TIMESERIESFILE entry (see the ``Time-Variable Input'' section), values can be obtained from a time series by entering the time-series name in place of a numeric value. By default, rainfall  rates are zero for each reach.}

\item \textcolor{blue}{\texttt{evaporation}---real or character value that defines the volumetric rate per unit area of water subtracted by evaporation from the streamflow routing reach. A positive evaporation rate should be provided. If the Options block includes a TIMESERIESFILE entry (see the ``Time-Variable Input'' section), values can be obtained from a time series by entering the time-series name in place of a numeric value. If the volumetric evaporation rate for a reach exceeds the sources of water to the reach (upstream and specified inflows, rainfall, and runoff but excluding groundwater leakage into the reach) the volumetric evaporation rate is limited to the sources of water to the reach. By default, evaporation rates are zero for each reach.}

\item \textcolor{blue}{\texttt{runoff}---real or character value that defines the volumetric rate of diffuse overland runoff that enters the streamflow routing reach. If the Options block includes a TIMESERIESFILE entry (see the ``Time-Variable Input'' section), values can be obtained from a time series by entering the time-series name in place of a numeric value. If the volumetric runoff rate for a reach is negative and exceeds inflows to the reach (upstream and specified inflows, and rainfall but excluding groundwater leakage into the reach) the volumetric runoff rate is limited to inflows to the reach and the volumetric evaporation rate for the reach is set to zero. By default, runoff rates are zero for each reach.}

\item \texttt{DIVERSION}---keyword to indicate diversion record.

\item \texttt{idv}---an integer value specifying which diversion of reach IFNO that DIVFLOW is being specified for.  Must be less or equal to ndv for the current reach (IFNO).

\item \textcolor{blue}{\texttt{divflow}---real or character value that defines the volumetric diversion (DIVFLOW) rate for the streamflow routing reach. If the Options block includes a TIMESERIESFILE entry (see the ``Time-Variable Input'' section), values can be obtained from a time series by entering the time-series name in place of a numeric value.}

\item \texttt{upstream\_fraction}---real value that defines the fraction of upstream flow (USTRF) from each upstream reach that is applied as upstream inflow to the reach. The sum of all USTRF values for all reaches connected to the same upstream reach must be equal to one.

\item \texttt{CROSS\_SECTION}---keyword to specify that record corresponds to a reach cross-section.

\item \texttt{TAB6}---keyword to specify that record corresponds to a cross-section table file.

\item \texttt{FILEIN}---keyword to specify that an input filename is expected next.

\item \texttt{tab6\_filename}---character string that defines the path and filename for the file containing cross-section table data for the reach. The TAB6\_FILENAME file includes the number of entries in the file and the station elevation data in terms of the fractional width and the reach depth. Instructions for creating the TAB6\_FILENAME input file are provided in SFR Reach Cross-Section Table Input File section.

\item \texttt{AUXILIARY}---keyword for specifying auxiliary variable.

\item \texttt{auxname}---name for the auxiliary variable to be assigned AUXVAL.  AUXNAME must match one of the auxiliary variable names defined in the OPTIONS block. If AUXNAME does not match one of the auxiliary variable names defined in the OPTIONS block the data are ignored.

\item \textcolor{blue}{\texttt{auxval}---value for the auxiliary variable.  If the Options block includes a TIMESERIESFILE entry (see the ``Time-Variable Input'' section), values can be obtained from a time series by entering the time-series name in place of a numeric value.}

\end{description}


\end{description}

\vspace{5mm}
\subsubsection{Example Input File}
\lstinputlisting[style=inputfile]{./mf6ivar/examples/gwf-sfr-example.dat}

\vspace{5mm}
\subsubsection{Available observation types}
Streamflow Routing Package observations include reach stage and all of the terms that contribute to the continuity equation for each stream reach. Additional SFR Package observations include the sum of inflows from upstream reaches and from mover terms (\texttt{upstream-flow}) and downstream outflow from a reach prior to diversions and the mover package (\texttt{downstream-flow}). The data required for each SFR Package observation type is defined in table~\ref{table:gwf-sfrobstype}. Negative and positive values for \texttt{sfr} observations represent a loss from and gain to the GWF model, respectively. For all other flow terms, negative and positive values represent a loss from and gain from the SFR package, respectively.

\FloatBarrier
\begin{longtable}{p{2cm} p{2.75cm} p{2cm} p{1.25cm} p{7cm}}
\caption{Available SFR Package observation types} \tabularnewline

\hline
\hline
\textbf{Stress Package} & \textbf{Observation type} & \textbf{ID} & \textbf{ID2} & \textbf{Description} \\
\hline
\endfirsthead

\captionsetup{textformat=simple}
\caption*{\textbf{Table \arabic{table}.}{\quad}Available SFR Package observation types.---Continued} \\

\hline
\hline
\textbf{Stress Package} & \textbf{Observation type} & \textbf{ID} & \textbf{ID2} & \textbf{Description} \\
\hline
\endhead


\hline
\endfoot

SFR & stage & ifno or boundname & -- & Surface-water stage in a stream-reach boundary. If boundname is specified, boundname must be unique for each reach. \\
SFR & ext-inflow & ifno or boundname & -- & Inflow into a stream-reach from an external boundary for a stream-reach or a group of stream-reaches. \\
SFR & inflow & ifno or boundname & -- & Inflow into a stream-reach from upstream reaches for a stream-reach or a group of stream-reaches. \\
SFR & from-mvr & ifno or boundname & -- & Inflow into a stream-reach from the MVR package for a stream-reach or a group of stream-reaches. \\
SFR & rainfall & ifno or boundname & -- & Rainfall rate applied to a stream-reach or a group of stream-reaches. \\
SFR & runoff & ifno or boundname & -- & Runoff rate applied to a stream-reach or a group of stream-reaches. \\
SFR & sfr & ifno or boundname & -- & Simulated flow rate for a stream-reach and its aquifer connection for a stream-reach or a group of stream-reaches. \\
SFR & evaporation & ifno or boundname & -- & Simulated evaporation rate from a stream-reach or a group of stream-reaches. \\
SFR & outflow & ifno or boundname & -- & Outflow from a stream-reach to downstream reaches for a stream-reach or a group of stream-reaches. \\
SFR & ext-outflow & ifno or boundname & -- & Outflow from a stream-reach to an external boundary for a stream-reach or a group of stream-reaches. \\
SFR & to-mvr & ifno or boundname & -- & Outflow from a stream-reach that is available for the MVR package for a stream-reach or a group of stream-reaches. \\
SFR & upstream-flow & ifno or boundname & -- & Upstream flow for a stream-reach or a group of stream-reaches from upstream reaches and the MVR package. \\
SFR & downstream-flow & ifno or boundname & -- & Downstream flow for a stream-reach or a group of stream-reaches prior to diversions and the MVR package. \\
SFR & depth & ifno or boundname & -- & Surface-water depth in a stream-reach boundary. If boundname is specified, boundname must be unique for each reach. \\
SFR & wet-perimeter & ifno or boundname & -- & Wetted perimeter in a stream-reach boundary. If boundname is specified, boundname must be unique for each reach. \\
SFR & wet-area & ifno or boundname & -- & Wetted cross-section area in a stream-reach boundary. If boundname is specified, boundname must be unique for each reach. \\
SFR & wet-width & ifno or boundname & -- & Wetted top width in a stream-reach boundary. If boundname is specified, boundname must be unique for each reach. \\


\label{table:gwf-sfrobstype}
\end{longtable}
\FloatBarrier

\vspace{5mm}
\subsubsection{Example Observation Input File}
\lstinputlisting[style=inputfile]{./mf6ivar/examples/gwf-sfr-example-obs.dat}

\newpage
\subsection{Streamflow Routing Package Cross-Sections Table Input File}
Cross-sections tables of distance and relative depth can be specified for individual reaches.  Cross-Section tables are specified by including file names in the LAKE\_TABLES block of the SFR Package.  These file names correspond to a lake table input file.  The format of the lake table input file is described here.

\vspace{5mm}
\subsubsection{Structure of Blocks}
\vspace{5mm}

\lstinputlisting[style=blockdefinition]{./mf6ivar/tex/utl-sfr-tab-dimensions.dat}
\lstinputlisting[style=blockdefinition]{./mf6ivar/tex/utl-sfr-tab-table.dat}
\vspace{5mm}

\vspace{5mm}
\subsubsection{Explanation of Variables}
\begin{description}
% DO NOT MODIFY THIS FILE DIRECTLY.  IT IS CREATED BY mf6ivar.py 

\item \textbf{Block: DIMENSIONS}

\begin{description}
\item \texttt{nrow}---integer value specifying the number of rows in the reach cross-section table. There must be NROW rows of data in the TABLE block.

\item \texttt{ncol}---integer value specifying the number of columns in the reach cross-section table. There must be NCOL columns of data in the TABLE block. Currently, NCOL must be equal to 2.

\end{description}
\item \textbf{Block: TABLE}

\begin{description}
\item \texttt{xfraction}---real value that defines the station (x) data for the cross-section as a fraction of the width (RWID) of the reach.

\item \texttt{depth}---real value that defines the elevation (z) data for the cross-section as a depth relative to the top elevation of the reach (RTP) and corresponding to the station data on the same line.

\end{description}


\end{description}

\subsubsection{Example Input File}
%\lstinputlisting[style=inputfile]{./mf6ivar/examples/utl-sfr-tab-example.dat}

