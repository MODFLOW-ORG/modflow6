\newcommand{\sign}{\text{sign}}
\newcommand{\matr}[1]{\mathbf{#1}}

MODFLOW gives us flows across cell-cell interfaces, which we can divide by interface areas to get specific discharge components normal to the interfaces. We refer to specific discharge here as ``velocity'' as a shortened form of Darcy velocity, with dimensions of $L/T$. So MODFLOW is essentially giving us normal-component velocity information at each interface. The idea behind the weighted-averaging scheme described here is to combine the normal-component information from the interfaces into estimates of the  $x$, $y$, and $z$ components of velocity at the cell center (node). The weights take into account both distance from the cell center and alignment with the desired velocity component.  The greatest weight is given to interfaces that are closest to the cell center and whose normals are most closely aligned with the desired velocity component. For example, when we are estimating the $x$ component of velocity at the cell center, an interface whose normal is in the $y$ direction is given a weight of zero because it provides a value for the $y$ component of velocity; it tells us nothing about the $x$ component.

\subsection{Estimating the $z$ Component of Velocity}

At each horizontal interface $k$ (along the top or bottom of the cell), MODFLOW gives us the interfacial flow, which we can divide by the area to get the $z$ component of velocity at the interface, $v_k^z$. Taking a weighted average of the estimates from all of the cell's horizontal interfaces, we get the following cell-center estimate of the $z$ component of velocity:

\begin{equation}
\label{vz}
v^z = \sum_{k=1}^K \phi_k^z v_k^z,
\end{equation}

\noindent where the summations are over the cell's horizontal interfaces (locally numbered 1 through $K$), and $\phi_k^z$ is the weight assigned to the estimate from interface  $k$, $v_k^z$, for the purpose of computing the cell-center value, $v^z$. We'll discuss the weights in more detail later.

\subsection{Estimating the $x$ and $y$ Components of Velocity}

At each vertical interface $i$ (along a ``side'' of the cell), MODFLOW gives us the interfacial flow, which we can divide by the area to get the component of velocity normal to the interface, $v_i^n$. This normal component is, of course, related to the full velocity vector at interface $i$, $v_i$, by

\begin{equation}
\label{nivi}
\matr{n_i} \cdot \matr{v_i} = v_i^n,
\end{equation}

\noindent or

\begin{equation}
\label{nivi2}
n_i^x v_i^x + n_i^y v_i^y + n_i^z v_i^z = v_i^n.
\end{equation}

For concreteness, suppose we're currently interested in estimating the $x$ component of velocity at interface $i$ (so we can ultimately use it, together with similar estimates at the other interfaces, to inform our estimate of the $x$ component of velocity at the cell center).  Recognizing that $n_i^z = 0$  for vertical interfaces, and solving \ref{nivi2} for the $x$ component, we get

\begin{equation}
\label{vix}
v_i^x = \left ( v_i^n - n_i^y v^y \right ) / n_i^x.
\end{equation}

Equation \ref{vix} is an estimate of the $x$ component of velocity based on information from interface $i$ . Taking a weighted average of the estimates from all of the cell's vertical interfaces, we get the following cell-center estimate of the $x$ component of velocity:

\begin{equation}
\label{vx}
v^x = \sum_{i=1}^N \phi_i^x v_i^x = \sum_{i=1}^N \frac{\phi_i^x v_i^n}{n_i^x} - \left( \sum_{i=1}^N \frac{\phi_i^x n_i^y}{n_i^x}  \right ) v^y,
\end{equation}

We now have two equations, \ref{vix} and \ref{vx}, in two unknowns, $v^x$ and $v^y$. Solving this 2x2 system, we get

\begin{equation}
\label{vxAB}
v^x = \frac{1}{1 - A^{xy} A^{yx}} \sum_{i=1}^N  \left( B_i^x - A^{xy} B_i^y \right ) v_i^n
\end{equation}

\begin{equation}
\label{vxAB}
v^y = \frac{1}{1 - A^{xy} A^{yx}} \sum_{i=1}^N  \left( B_i^y - A^{yx} B_i^x \right ) v_i^n,
\end{equation}

\noindent where

\begin{equation}
\label{Axy}
A^{xy} = \sum_{i=1}^N B_i^x n_i^y
\end{equation}

\begin{equation}
\label{Ayx}
A^{yx} = \sum_{i=1}^N B_i^y n_i^x,
\end{equation}

\noindent and

\begin{equation}
\label{Bix}
B_i^x = \frac{\phi_i^x}{n_i^x}
\end{equation}

\begin{equation}
\label{Biy}
B_i^y = \frac{\phi_i^y}{n_i^y}.
\end{equation}

Both equations \ref{Bix} and \ref{Biy} have the potential to blow up when $n_i^x$ and $n_i^y$ go to zero, but this will be taken care of in the formulation of the weights.

\subsection{Weights}

For estimation of the $z$ component of velocity, we can define a set of weights based on the distance (however we care to measure that---any reasonable measure will do) from each interface $k$ to the cell center, $D_k$, as follows:

\begin{equation}
\label{omegaz}
\omega_k^z = 1 - \frac{D_k}{\sum_{m=1}^K D_m}.
\end{equation}

Interfaces that are closest to the cell center receive the greatest weights. However, as written above, the weights don't add up to 1, so we?ll normalize them such that they do:

\begin{equation}
\label{phiz}
\phi_k^z = \frac{\omega_k^z}{K - 1} = \frac{1}{K - 1} \left (1 - \frac{D_k}{\sum_{m=1}^K D_m} \right ).
\end{equation}

As noted earlier, for estimating the $x$ and $y$ components of velocity, we also want to take into account how closely the normal-component information at each vertical interface aligns with the velocity component ($x$ or $y$) we're trying to estimate.  For that purpose, we can define the following set of weights:

\begin{equation}
\label{omegax}
\omega_i^x = \left [ 1 - \frac{D_i \left | n_i^x \right | }{\sum_{l=1}^N{} D_l  \left | n_l^x \right | } \right ] \left | n_i^x \right |,
\end{equation}

and,

\begin{equation}
\label{omegay}
\omega_i^y = \left [ 1 - \frac{D_i \left | n_i^y \right | }{\sum_{l=1}^N{} D_l  \left | n_l^y \right | } \right ] \left | n_i^y \right |.
\end{equation}

These weight equations can also be written as

\begin{equation}
\label{omegax2}
\omega_i^x = \left [ \frac{\sum_{l=1}^N D_l  \left | n_l^x \right |  - D_i \left | n_i^x \right | }{\sum_{l=1}^N D_l  \left | n_l^x \right | } \right ] \left | n_i^x \right |,
\end{equation}

and,

\begin{equation}
\label{omegay2}
\omega_i^y = \left [ \frac{\sum_{l=1}^N{} D_l  \left | n_l^y \right | - D_i \left | n_i^y \right | }{\sum_{l=1}^N{} D_l  \left | n_l^y \right | } \right ] \left | n_i^y \right |.
\end{equation}

These don't add up to 1, so they are normalized so that they do:

\begin{equation}
\label{phix}
\phi_i^x =  \frac{\omega_i^x \left | n_i^x \right | }{\sum_{l=1}^N \omega_l^x  \left | n_l^x \right | } ,
\end{equation}

and,

\begin{equation}
\label{phiy}
\phi_i^y =  \frac{\omega_i^y \left | n_i^y \right | }{\sum_{l=1}^N \omega_l^y  \left | n_l^y \right | } .
\end{equation}

These weights are substituted into the following equations for $B$, which are needed to solve for $v^x$ and $v^y$.

\begin{equation}
\label{bx}
B_i^x =  \frac{\phi_i^x }{ n_i^x } ,
\end{equation}

and,

\begin{equation}
\label{by}
B_i^y =  \frac{\phi_i^y }{ n_i^y }.
\end{equation}

Development of an efficient solution scheme is obtained by substituting equations ~\ref{omegax2} and ~\ref{omegay2} into equations ~\ref{phix} and ~\ref{phiy}.  This results in the following expressions for $\phi_i^x$ and $\phi_i^y$:

\begin{equation}
\label{phix2}
\phi_i^x =  \frac{\left [ \frac{\sum_{l=1}^N D_l  \left | n_l^x \right |  - D_i \left | n_i^x \right | }{\sum_{l=1}^N D_l  \left | n_l^x \right | } \right ] \left | n_i^x \right | \left | n_i^x \right | }{\sum_{j=1}^N \left [ \frac{\sum_{l=1}^N D_l  \left | n_l^x \right |  - D_j \left | n_j^x \right | }{\sum_{l=1}^N D_l  \left | n_l^x \right | }  \right ] \left | n_j^x \right |  \left | n_j^x \right |  } ,
\end{equation}

and,

\begin{equation}
\label{phiy2}
\phi_i^y =  \frac{\left [ \frac{\sum_{l=1}^N D_l  \left | n_l^y \right |  - D_i \left | n_i^y \right | }{\sum_{l=1}^N D_l  \left | n_l^y \right | } \right ] \left | n_i^y \right | \left | n_i^y \right | }{\sum_{j=1}^N \left [ \frac{\sum_{l=1}^N D_l  \left | n_l^y \right |  - D_j \left | n_j^y \right | }{\sum_{l=1}^N D_l  \left | n_l^y \right | } \right ] \left | n_j^y \right |  \left | n_j^y \right |  } .
\end{equation}

Equations ~\ref{phix2} and ~\ref{phiy2} can be simplified as:

\begin{equation}
\label{phix3}
\phi_i^x =  \frac{\left [ \sum_{l=1}^N D_l  \left | n_l^x \right |  - D_i \left | n_i^x \right |  \right ] \left | n_i^x \right | \left | n_i^x \right | }{\sum_{j=1}^N \left [ \sum_{l=1}^N D_l  \left | n_l^x \right |  - D_j \left | n_j^x \right | \right ] \left | n_j^x \right |  \left | n_j^x \right |  } ,
\end{equation}

and

\begin{equation}
\label{phiy3}
\phi_i^y =  \frac{\left [ \sum_{l=1}^N D_l  \left | n_l^y \right |  - D_i \left | n_i^y \right |  \right ] \left | n_i^y \right | \left | n_i^y \right | }{\sum_{j=1}^N \left [ \sum_{l=1}^N D_l  \left | n_l^y \right |  - D_j \left | n_j^y \right | \right ] \left | n_j^y \right |  \left | n_j^y \right |  } .
\end{equation}

Final combined equations for the $B$ terms are derived by substituting ~\ref{phix3} and ~\ref{phiy3} into ~\ref{bx} and ~\ref{by} to give:

\begin{equation}
\label{bx2}
B_i^x =  \frac{\left [ \sum_{l=1}^N D_l  \left | n_l^x \right |  - D_i \left | n_i^x \right |  \right ] \left | n_i^x \right | \sign (n_i^x)  }{\sum_{j=1}^N \left [ \sum_{l=1}^N D_l  \left | n_l^x \right |  - D_j \left | n_j^x \right | \right ] \left | n_j^x \right |  \left | n_j^x \right | } ,
\end{equation}

and

\begin{equation}
\label{bx2}
B_i^y =  \frac{\left [ \sum_{l=1}^N D_l  \left | n_l^y \right |  - D_i \left | n_i^y \right |  \right ] \left | n_i^y \right | \sign (n_i^y)  }{\sum_{j=1}^N \left [ \sum_{l=1}^N D_l  \left | n_l^y \right |  - D_j \left | n_j^y \right | \right ] \left | n_j^y \right |  \left | n_j^y \right | } .
\end{equation}


