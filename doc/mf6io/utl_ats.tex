The Adaptive Time Step (ATS) utility for the TDIS Package can be activated by specifying the ATS6 option in the TDIS input file.  If activated, \mf will read ATS input according to the following description.

The adaptive time step utility is activated for any stress periods that are listed in the PERIODDATA block below.  If a stress period is adaptive, then the \texttt{nstp} and \texttt{tsmult} parameters in the TDIS input file have no affect on time step progression.  Instead the ATS settings specified for the period are used to control the time step progression.

The ATS implementation implemented in \mf is patterned after the approach implemented in MODFLOW-USG.  There are two fundamental parts to the ATS utility.  The first is the capability to handle failure of a solution to converge.  If ATS is active for a stress period in which the solution fails to converge, then the program will continue to try smaller time steps until convergence is achieved or the length of the time step reaches the lower allowable limit (\texttt{dtmin}).  Once this lower limit on the time step is reached, then the program will follow the established logic for non-adaptive time steps.  That is, the program with either stop and write concluding information, or the program will continue to the next time step if the CONTINUE option is specified in the simulation name file.

The second fundamental part of the ATS utility is dynamic adjustment of the time step size according to simulation behavior.  The ATS utility in \mf has been implemented in a generic and modular manner in which any model, exchange, or solution can submit a preferred time length to be used in determining the time step length.  The ATS utility will proceed with the smallest time step submitted by these different simulation components.  In the present implementation, the following can be used to affect ATS behavior.

\begin{itemize}
\item The Iterative Model Solution (IMS) can optionally submit a preferred time step length based on the convergence pattern for the previous time step.  If the numerical solution is relatively easy (as measured by the number of outer iterations), then the length of the next time step will increase by a factor of the \texttt{dtadj} variable.  Conversely, if the solution is difficult to obtain, then the length of the next time step will decrease, by dividing the previous time step length by the \texttt{dtadj} variable.  
\item The Advection (ADV) Package of the GWT Model can optionally submit a preferred time step size that honors a Courant stability constraint.  If activated by the user, the ADV Package will check all active transport model cells and calculate the largest time step for the model grid subject to a user-specified Courant factor.  This Courant factor is number of cells or the fraction of a cell volume that advection is allowed to occur during a single time step.
\end{itemize}

In the present ATS implementation, time series variables are interpolated based on the starting and ending times of the time step.  If solution failure was encountered and a time step is retried with a smaller time step size, time series variables are re-interpolated for the shortened time step.  In most cases, this is the intended behavior, however, if time series contain a much finer level of temporal detail, then this additional detail could exacerbate convergence problems.

A limitation with the present ATS implementation is that there is no way to explicitly specify times within a stress period for saving output.  Output can be obtained at the end of a period, and within a period according the Output Control time step settings.  For example, the Output Control settings allow for printing and saving based on the FIRST, LAST, FREQUENCY, and STEPS options, but these are based on time steps, the lengths of which are adaptive and not necessarily known before the simulation.  Thus, there is no way to request output at specific times within a stress period managed by ATS.  If observations are used for models and packages, observations are written for every time step.  For automated parameter estimation applications, additional post-processing of output files may be required in order to align simulated values with measurements.  

\vspace{5mm}
\subsection{Structure of Blocks}
%\lstinputlisting[style=blockdefinition]{./mf6ivar/tex/sim-ats-options.dat}
\lstinputlisting[style=blockdefinition]{./mf6ivar/tex/utl-ats-dimensions.dat}
\lstinputlisting[style=blockdefinition]{./mf6ivar/tex/utl-ats-perioddata.dat}

\vspace{5mm}
\subsection{Explanation of Variables}
\begin{description}
% DO NOT MODIFY THIS FILE DIRECTLY.  IT IS CREATED BY mf6ivar.py 

\item \textbf{Block: DIMENSIONS}

\begin{description}
\item \texttt{maxats}---is the number of records in the subsequent perioddata block that will be used for adaptive time stepping.

\end{description}
\item \textbf{Block: PERIODDATA}

\begin{description}
\item \texttt{iperats}---is the period number to designate for adaptive time stepping.  The remaining ATS values on this line will apply to period iperats.  iperats must be greater than zero.  A warning is printed if iperats is greater than nper.

\item \texttt{dt0}---is the initial time step length for period iperats.  If dt0 is zero, then the final step from the previous stress period will be used as the initial time step.  The program will terminate with an error message if dt0 is negative.

\item \texttt{dtmin}---is the minimum time step length for this period.  This value must be greater than zero and less than dtmax.  dtmin must be a small value in order to ensure that simulation times end at the end of stress periods and the end of the simulation.  A small value, such as 1.e-5, is recommended.

\item \texttt{dtmax}---is the maximum time step length for this period.  This value must be greater than dtmin.

\item \texttt{dtadj}---is the time step multiplier factor for this period.  If the number of outer solver iterations are less than the product of the maximum number of outer iterations (OUTER\_MAXIMUM) and ATS\_OUTER\_MAXIMUM\_FRACTION (an optional variable in the IMS input file with a default value of 1/3), then the time step length is multiplied by dtadj.  If the number of outer solver iterations are greater than the product of the maximum number of outer iterations and ATS\_OUTER\_MAXIMUM\_FRACTION, then the time step length is divided by dtadj.  dtadj must be zero, one, or greater than one.  If dtadj is zero or one, then it has no effect on the simulation.  A value between 2.0 and 5.0 can be used as an initial estimate.

\item \texttt{dtfailadj}---is the divisor of the time step length when a time step fails to converge.  If there is solver failure, then the time step will be tried again with a shorter time step length calculated as the previous time step length divided by dtfailadj.  dtfailadj must be zero, one, or greater than one.  If dtfailadj is zero or one, then time steps will not be retried with shorter lengths.  In this case, the program will terminate with an error, or it will continue of the CONTINUE option is set in the simulation name file.  Initial tests with this variable should be set to 5.0 or larger to determine if convergence can be achieved.

\end{description}


\end{description}

\vspace{5mm}
\subsection{Example Input File}
\lstinputlisting[style=inputfile]{./mf6ivar/examples/utl-ats-example.dat}

