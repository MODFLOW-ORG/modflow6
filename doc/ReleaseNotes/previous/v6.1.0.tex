
	\subsection{Version mf6.1.0--December 12, 2019}
	
	\underline{NEW FUNCTIONALITY}
	\begin{itemize}
		\item Added the Skeletal Storage, Compaction, and Subsidence (CSUB) Package. The one-dimensional effective-stress based compaction theory implemented in the CSUB Package is documented in \cite{leake2007modflow}. The numerical approach used for delay interbeds in the CSUB package is documented in \cite{hoffmann2003modflow} and uses the same one-dimensional effective-stress based compaction theory as coarse-grained and fine-grained no-delay interbed sediments. A number of example problems that use the CSUB Package are documented in the ``MODFLOW 6 CSUB Package Example Problems'' pdf document included in this and subsequent releases.
	\end{itemize}
	
	\underline{BASIC FUNCTIONALITY}
	\begin{itemize}
		\item Added an error check to the DISU Package that ensures that an underlying cell has a top elevation that is less than or equal to the bottom of an overlying cell.  An underlying cell is one in which the IHC value for the connection is zero and the connecting node number is greater than the cell node number.
		\item Added restricted IDOMAIN support for DISU grids.  Users can specify an optional IDOMAIN in the DISU Package input file.  IDOMAIN values must be zero or one.  Vertical pass-through cells (specified with an IDOMAIN value of -1 in the DIS or DISV Package input files) are not supported for DISU.   
		\item NPF Package will now write a message to the GWF Model list file to indicate when the SAVE\_SPECIFIC\_DISCHARGE option is invoked.
		\item Added two new options to the NPF Package.  The K22OVERK option allows the user to enter the anisotropy ratio for K22.  If activated, the K22 values entered by the user in the NPF input file will be multiplied by the K values entered in the NPF input file.  The K33OVERK option allows the user to enter the anisotropy ratio for K33.  If activated, the K33 values entered by the user in the NPF input file will be multiplied by the K values entered in the NPF input file.  With this K33OVERK option, for example, the user can specify a value of 0.1 for K33 and have all K33 values be one tenth of the values specified for K.  The program will terminate with an error if these options are invoked, but arrays for K22 and/or K33 are not provided in the NPF input file.
		\item Added new MAXERRORS option to mfsim.nam.  If specified, the maximum number of errors stored and printed will be limited to this number.  This can prevent a situation where memory will run out when there are an excessive number of errors.
		\item Refactored many parts of the code to remove unused variables, conform to stricter FORTRAN standard checks, and allow for new development efforts to be included in the code base.
	\end{itemize}
	
	\underline{STRESS PACKAGES}
	\begin{itemize}
		\item There was an error in the calculation of the segmented evapotranspiration rate for the case where the rate did not decrease with depth.  There was another error in which PETM0 was being used as the evapotranspiration rate at the surface instead of the proportion of the evapotranspiration rate at the surface.
	\end{itemize}
	
	\underline{ADVANCED STRESS PACKAGES}
	\begin{itemize}
		\item Corrected the way auxiliary variables are handled for the advanced packages.  In some cases, values for auxiliary variables were not being correctly written to the GWF Model budget file or to the advanced package budget file.  A consistent approach for updating and saving auxiliary variables was implemented for the MAW, SFR, LAK, and UZF Packages.
		\item The user guide was updated to include a missing laksetting that was omitted from the PERIOD block.  The laksetting description now includes an INFLOW option; a description for INFLOW is also now included.
		\item The LAK package was incorrectly making an error check against NOUTLETS instead of NLAKES.
		\item For the advanced stress packages, values assigned to the auxiliary variables were not written correctly to the GWF Model budget file, but the values were correct in the advanced package budget file.  Program was modified so that auxiliary variables are correctly written to the GWF Model budget file.
		\item Corrected several error messages issued by the SFR Package that were not formatted correctly.  
		\item Fixed a bug in which the lake stage stable would sometimes result in touching numbers.  This only occurred for negative lake stages.
		\item The UZF Package was built on the UZFKinematicType, which used an array of structures.  A large array like this, can cause memory problems.  The UZFKinematicType was replaced with a new UzfCellGroupType, which is a structure of arrays and is much more memory efficient.  The underlying UZF algorithm did not change.
	\end{itemize}
	
	\underline{SOLUTION}
	\begin{itemize}
		\item Add ALL and FIRST options to optional NO\_PTC optional keyword in OPTIONS block. If NO\_PTC option is FIRST, PTC is disabled for the first stress period but is applied in all subsequent steady-state stress periods. If NO\_PTC option is ALL, PTC is disabled for all steady-state stress periods. If the NO\_PTC options is not defined, PTC is disabled for all steady-state stress periods (this is consistent with the behaviour of the NO\_PTC option in previous versions).
	\end{itemize}
	
