% Use this template for starting initializing the release notes
% after a release has just been made.
	
	\item \currentmodflowversion

	\underline{NEW FUNCTIONALITY}
	\begin{itemize}
		\item The Input Data Processor (IDP), first released in version 6.4.2, is a general utility for reading user-provided input files.  Package-specific routines for reading input files continue to be replaced by the IDP approach.  For packages that use IDP for input, logging information is written to the simulation list file (mfsim.lst).  Additional information on the IDP and the list of supported packages is contained in the MODFLOW 6 Description of Input and Output (mf6io.pdf) under a section titled ``Processing of Program Input.''
		\item The source code was refactored to support compilation of a parallel version of MODFLOW 6 based on the Message Passing Interface (MPI) and the Portable, Extensible Toolkit for Scientific Computation (PETSc) libraries.  The parallel version of MODFLOW is considered preliminary. Limited testing of the parallel version has been performed on laptops, desktops, and supercomputers, but significant changes are expected in future releases. User support for the parallel version of MODFLOW 6 may be provided in the future.
	%	\item xxx
	\end{itemize}

	\underline{EXAMPLES}
	\begin{itemize}
		\item A new exampled called ex-gwf-rad-disu was added.  This new example uses a DISU grid to represent radial groundwater flow to a pumping well.
		\item A new exampled called ex-gwf-curv-90 was added.  This new example demonstrates use of a DISV grid to represent a curvilinear spatial discretization.  For this example, the curvilinear grid is applied to one quarter of a radial groundwater flow system.
		\item A new exampled called ex-gwf-curvilin was added.  This new example uses a curvilinear grid, represented with the DISV Package, to simulate groundwater flow through a multi-region aquifer with bends in the domain boundaries.
	\end{itemize}

	\textbf{\underline{BUG FIXES AND OTHER CHANGES TO EXISTING FUNCTIONALITY}} \\
	\underline{BASIC FUNCTIONALITY}
	\begin{itemize}
		\item Improve error message if the size of data read from a binary array file is inconsistent with READARRAY control line and variable description keywords.
		\item The area calculation for cells in the DISV package was inaccurate for some cases with very large cell vertex coordinates.  The area calculation was improved by using transformed cell vertex coordinates prior to making the area calculation.
		\item Auxiliary variables in RCH and EVT Array-Based input packages are now reset to zero when otherwise not specified in period input data and the auxiliary parameter is not controlled by a time-series.
	%	\item xxx
	%	\item xxx
	\end{itemize}

	\underline{INTERNAL FLOW PACKAGES}
	\begin{itemize}
		\item The data header in the binary output file written by the viscosity (VSC) package was printing `` VISCOSI'' instead of `` VISCOSITY''. The viscosity package now prints the full `` VISCOSITY'' header in the binary output file.
		\item The CSUB Package did not support output of z-displacement arrays for models using the DISU package.  The CSUB package was updated to support calculation of z-displacement arrays for DISU model grids.
	%	\item xxx
	\end{itemize}

	\underline{STRESS PACKAGES}
	\begin{itemize}
		\item This release contains a fix for a longstanding issue related to the use of AUXMULTNAME and time series.  Previous release notes included the following description of a known issue: \textit{``The AUXMULTNAME option can be used to scale input values, such as riverbed conductance, using values in an auxiliary column.  When this AUXMULTNAME option is used, the multiplier value in the AUXMULTNAME column should not be represented with a time series unless the value to scale is also represented with a time series.''}  With this release, the Input Data Processor (IDP) is now used to read stress package input files, and the limitation with AUXMULTNAME and time series no longer applies.
	%	\item xxx
	%	\item xxx
	%	\item xxx
	\end{itemize}

	\underline{ADVANCED STRESS PACKAGES}
	\begin{itemize}
		\item Added functionality to support zero values for each grid dimension when specifying the CELLID for SFR reaches that are not connected to an underlying groundwater grid cell. For example, for a DIS grid a CELLID of 0 0 0 should be specified for unconnected reaches. Warning messages will be issued if NONE is specified for unconnected reaches. Specifying a CELLID of NONE will eventually be deprecated and will cause MODFLOW 6 to terminate with an error.
		\item Added functionality to support specification of a DNODATA (3.0E+30) BEDLEAK value for LAK package connections to identify lake-GWF connections where conductance is solely a function of aquifer properties in the connected GWF cell and lakebed sediments are assumed to be absent. Warning messages will be issued if NONE is specified for LAK package connections. Specifying a BEDLEAK value equal to NONE will eventually be deprecated and will cause MODFLOW 6 to terminate with an error.
		\item SFR diversion would not be updated if the outflow of its upstream reach is zero. If diversion was not zero in the previous stress period, it would report mass balance error in the SFR budget. This bug was fixed by always updating the diversion.
	%	\item xxx
	\end{itemize}

	%\underline{SOLUTION}
	%\begin{itemize}
	%	\item xxx
	%	\item xxx
	%	\item xxx
	%\end{itemize}

	\underline{EXCHANGES}
	\begin{itemize}
		\item A model budget error would occur when a constant-head cell in one model had a direct connection to an active cell in another model.  For the model budget to be calculated correctly a new term called ``FLOW-JA-FACE-CHD'' was added to the GWF model budget.  This term is only included in the table when the GWF Model is connected to another GWF Model using a GWF-GWF Exchange.  
	%	\item xxx
	%	\item xxx
	\end{itemize}
