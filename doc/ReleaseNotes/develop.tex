% Use this template for starting initializing the release notes
% after a release has just been made.
	
	\item \currentmodflowversion
	
	\underline{NEW FUNCTIONALITY}
	\begin{itemize}
		\item A new Groundwater Energy (GWE) transport model is introduced to the code base for simulating heat transport in the subsurface.  Additional information for activating the GWE model type within a MODFLOW 6 simulation is available within the mf6io.pdf document.  New example problems have been developed for testing and demonstrating GWE capabilities (in addition to other internal tests that help verify the accuracy of GWE); however, additional changes to the code and input may be necessary in response to user needs and further testing.
	%	\item xxx
	%	\item xxx
	\end{itemize}

	%\underline{EXAMPLES}
	%\begin{itemize}
	%	\item xxx
	%	\item xxx
	%	\item xxx
	%\end{itemize}

	\textbf{\underline{BUG FIXES AND OTHER CHANGES TO EXISTING FUNCTIONALITY}} \\
	%\underline{BASIC FUNCTIONALITY}
	%\begin{itemize}
	%	\item xxx
	%	\item xxx
	%	\item xxx
	%\end{itemize}

	%\underline{INTERNAL FLOW PACKAGES}
	%\begin{itemize}
	%	\item xxx
	%	\item xxx
	%	\item xxx
	%\end{itemize}

	\underline{STRESS PACKAGES}
	\begin{itemize}
		\item Floating point overflow errors would occur when negative conductance (COND) or auxiliary multiplier (AUXMULT) values were specified in the Drain, River, and General Head stress packages. This bug was fixed by checking if COND and AUXMULT values are greater than or equal to zero. The program will terminate with and error if negative COND or AUXMULT values are found.
	%	\item xxx
	%	\item xxx
	\end{itemize}

	\underline{ADVANCED STRESS PACKAGES}
	\begin{itemize}
		\item A divide by zero error would occur in the Streamflow Routing package when reaches were deactivated during a simulation. This bug was fixed by checking if the downstream reach is inactive before calculating the flow to the downstream reach.
	%	\item xxx
	%	\item xxx
	\end{itemize}

	%\underline{SOLUTION}
	%\begin{itemize}
	%	\item xxx
	%	\item xxx
	%	\item xxx
	%\end{itemize}

	%\underline{EXCHANGES}
	%\begin{itemize}
	%	\item xxx
	%	\item xxx
	%	\item xxx
	%\end{itemize}

