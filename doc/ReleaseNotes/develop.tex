% Use this template for starting initializing the release notes
% after a release has just been made.
	
	\item \currentmodflowversion
	
	\underline{NEW FUNCTIONALITY}
	\begin{itemize}
		\item A new adaptive time stepping (ATS) capability was added to the Advection (ADV) Package of the Groundwater Transport (GWT) Model.  A new input option, called ATS\_PERCEL, specifies the fractional cell distance that a particle of water can travel within one time step.  When ATS\_PERCEL is specified by the user, and the ATS utility is activated in the TDIS Package, the ADV Package will calculate the largest time step that will meet this fractional cell distance constraint, and will submit this time step to the ATS utility.  This option may improve time stepping for solute transport models and for variable-density flow and transport models by allowing step lengths to be calculated as a function of the flow system rather than being specified as input by the user.
	%	\item xxx
	%	\item xxx
	\end{itemize}

	\underline{EXAMPLES}
	\begin{itemize}
		\item A new Toth example was added to show how the classic groundwater problem consisting of local, intermediate, and regional flow paths can be simulated with MODFLOW.
	%	\item xxx
	%	\item xxx
	\end{itemize}

	\textbf{\underline{BUG FIXES AND OTHER CHANGES TO EXISTING FUNCTIONALITY}} \\
	\underline{BASIC FUNCTIONALITY}
	\begin{itemize}
		\item With the LOCAL\_Z option enabled, the PRT model's PRP package attempted to check that a particle release point's z coordinate fell within the grid's vertical extent before converting the coordinate from local (normalized on the unit interval) to a global (model) coordinate. This bug was fixed by converting coordinates before conducting checks.
		\item The PRT model could hang upon particle termination due to no (sub)cell exit face. This occurred because a flag signaling particle advance was not set in the proper location.
		\item Entangled with the previous issue, the ternary method could erroneously terminate a particle and report no exit face (before now this would hang) due to precision error in the exit time/position calculation. This could happen when two conditions are both met: the particle enters the subcell very close to one of its vertices, and flow very nearly parallels one of the subcell's faces. We have encountered similar situations before, solved by nudging the particle a small distance into the interior of the subcell before applying the tracking method. This particular case is resolved by increasing the padding distance from $\sisetup{input-digits = 0123456789\epsilon} \num{\epsilon e+2}$ to $\sisetup{input-digits = 0123456789\epsilon} \num{\epsilon e+5}$, where $\epsilon$ is machine precision.
		\item For ASCII input files erroneously containing a mix of line endings, MODFLOW would sometimes proceed with unexpected results. The program was corrected to stop with an error message if an input line contained both carriage returns and line feeds.
		\item Previously the PRT model's default behavior was to track particles until termination, as with MODPATH 7 stop time option 2 (extend). Under extended tracking, the program may not halt if particles enter a cycle in the flow system. This PR changes the default to the equivalent of MP7 stop time option 1 (final), terminating at simulation end unless a new Particle Release Point (PRP) package keyword option EXTEND\_TRACKING is provided. This is meant to provide a stronger guarantee that the program halts under default settings.
		\item A refactor of the energy storage and transfer (EST) package associated with the GWE model type results in different input requirements that breaks backward compatibility with what was required in version 6.5.0.  The PACKAGEDATA block was removed from the EST package input.  In its place, the heat capabity of water, the specified density of water, and the latent heat of vaporization are instead given default values that can be overridden by specifying alternative values in the OPTIONS block.
		\item The PRT model's cell face flows were improperly combined with boundary flows; for cell faces with active neighbors, the face flow replaced any boundary flows (likely a rare situation because IFLOWFACE is typically applied to faces on the boundary of the active domain). The face flow calculation has been corrected.
		\item A bad pointer reference has been fixed in the PRT model. Depending on the combination of platform and compiler used to build the program, this could result in crashes (e.g. divide by zero errors) or in silent failures to properly pass particles downward between vertically adjacent cells, leading to early termination. The latter could occur as a consequence of undefined behavior which prevented detection of situations when a particle should exit a cell through its bottom face.
		\item For a UZF setup with multiple UZF objects in a cell, the auxmultname option would cause the program to issue an unnecessary error.  The program was corrected to remove the unnecessary error.  The revised program was tested to ensure that the assignment of multiple UZF objects per cell works when it should and fails when the cumulative area of the UZF objects within a cell exceed the total area for the cell.
		\item An array out of bounds error for Z-displacement output in inactive cells has been fixed in the CSUB package. Depending on the combination of platform and compiler used to build the program, this could result in crashes.
		\item Initialize a few uninitialized variables in the CSUB package. Depending on the combination of platform and compiler used to build the program, this could result in different program behaviour.
		\item Add a warning if saving convergence data for the CSUB package when delay beds are not included in a simulation. Also modified the convergence output data so that it is clear that DVMAX and DSTORAGEMAX data and locations are not calculated in this case (by writing `-\,-' to the output file). 
		\item When using SFT in a model where some of the stream reaches are not connected to an active GWF cell (the cellid parameter is set equal to either 0 or NONE) memory access violations were occurring.  The program was fixed by setting the correct number of reaches connected to GWF cells.  The program was tested using a new example with a DISU grid type and multiple GWF cells deactivated (idomain equals 0) that host SFR reaches.
		\item Support for temperature observations was missing in the Observation (OBS) utility for a GWE model and has been added.
		\item UZF was not writing a message to the GWF listing file when it had finished reading the PACKAGEDATA block.  An appropriate message is now written to the GWF listing file.
		\item Error checking was added to the EST package to ensure that the inputs HEAT\_CAPACITY\_WATER and DENSITY\_WATER are not 0.0.  Values of 0.0 for either parameter result in a divide by zero error.

	%	\item xxx
	\end{itemize}

	%\underline{INTERNAL FLOW PACKAGES}
	%\begin{itemize}
	%	\item xxx
	%	\item xxx
	%	\item xxx
	%\end{itemize}

	%\underline{STRESS PACKAGES}
	%\begin{itemize}
	%	\item xxx
	%	\item xxx
	%	\item xxx
	%\end{itemize}

	%\underline{ADVANCED STRESS PACKAGES}
	%\begin{itemize}
	%	\item xxx
	%	\item xxx
	%	\item xxx
	%\end{itemize}

	%\underline{SOLUTION}
	%\begin{itemize}
	%	\item xxx
	%	\item xxx
	%	\item xxx
	%\end{itemize}

	%\underline{EXCHANGES}
	%\begin{itemize}
	%	\item xxx
	%	\item xxx
	%	\item xxx
	%\end{itemize}

	%\underline{PARALLEL}
	%\begin{itemize}
	%	\item xxx
	%	\item xxx
	%	\item xxx
	%\end{itemize}

