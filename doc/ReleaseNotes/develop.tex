% Use this template for starting initializing the release notes
% after a release has just been made.
	
	\item \currentmodflowversion
	
	\underline{NEW FUNCTIONALITY}
	\begin{itemize}
		\item NetCDF export files can be generated for DIS and DISV based models with the EXPORT\_NETCDF keyword. Currently GWF, GWT and GWE model types are supported. The export files are UGRID compliant and define a layered mesh topology. As such, all gridded array data is layered. By default the model dependent variable is written to the file as timeseries data. Gridded input arrays can be written to the files on a per package (DIS,DISV,NPF,IC,etc) basis by using the EXPORT\_NETCDF\_ARRAY keyword. Furthermore, a grid mapping variable can be added to the file by defining an NCF6 package with an OGC\_WKT option. Using this option allows the file to be displayed in post-processing tools (e.g. QGIS) that can read NetCDF mesh inputs. Additional options related to data variable compression are also available.
	%	\item xxx
	%	\item xxx
	\end{itemize}

	%\underline{EXAMPLES}
	%\begin{itemize}
	%	\item xxx
	%	\item xxx
	%	\item xxx
	%\end{itemize}

	\textbf{\underline{BUG FIXES AND OTHER CHANGES TO EXISTING FUNCTIONALITY}} \\
	\underline{BASIC FUNCTIONALITY}
	\begin{itemize}
		\item With the LOCAL\_Z option enabled, the PRT model's PRP package attempted to check that a particle release point's z coordinate fell within the grid's vertical extent before converting the coordinate from local (normalized on the unit interval) to a global (model) coordinate. This bug was fixed by converting coordinates before conducting checks.
		\item The PRT model could hang upon particle termination due to no (sub)cell exit face. This occurred because a flag signaling particle advance was not set in the proper location.
		\item Entangled with the previous issue, the ternary method could erroneously terminate a particle and report no exit face (before now this would hang) due to precision error in the exit time/position calculation. This could happen when two conditions are both met: the particle enters the subcell very close to one of its vertices, and flow very nearly parallels one of the subcell's faces. We have encountered similar situations before, solved by nudging the particle a small distance into the interior of the subcell before applying the tracking method. This particular case is resolved by increasing the padding distance from $\sisetup{input-digits = 0123456789\epsilon} \num{\epsilon e+2}$ to $\sisetup{input-digits = 0123456789\epsilon} \num{\epsilon e+5}$, where $\epsilon$ is machine precision.
		\item For ASCII input files erroneously containing a mix of line endings, MODFLOW would sometimes proceed with unexpected results. The program was corrected to stop with an error message if an input line contained both carriage returns and line feeds.
		\item Previously the PRT model's default behavior was to track particles until termination, as with MODPATH 7 stop time option 2 (extend). Under extended tracking, the program may not halt if particles enter a cycle in the flow system. This PR changes the default to the equivalent of MP7 stop time option 1 (final), terminating at simulation end unless a new Particle Release Point (PRP) package keyword option EXTEND\_TRACKING is provided. This is meant to provide a stronger guarantee that the program halts under default settings.
		\item A refactor of the energy storage and transfer (EST) package associated with the GWE model type results in different input requirements that breaks backward compatibility with what was required in version 6.5.0.  The PACKAGEDATA block was removed from the EST package input.  In its place, the heat capabity of water, the specified density of water, and the latent heat of vaporization are instead given default values that can be overridden by specifying alternative values in the OPTIONS block.
		\item The PRT model's cell face flows were improperly combined with boundary flows; for cell faces with active neighbors, the face flow replaced any boundary flows (likely a rare situation because IFLOWFACE is typically applied to faces on the boundary of the active domain). The face flow calculation has been corrected.
	%	\item xxx
	\end{itemize}

	%\underline{INTERNAL FLOW PACKAGES}
	%\begin{itemize}
	%	\item xxx
	%	\item xxx
	%	\item xxx
	%\end{itemize}

	%\underline{STRESS PACKAGES}
	%\begin{itemize}
	%	\item xxx
	%	\item xxx
	%	\item xxx
	%\end{itemize}

	%\underline{ADVANCED STRESS PACKAGES}
	%\begin{itemize}
	%	\item xxx
	%	\item xxx
	%	\item xxx
	%\end{itemize}

	%\underline{SOLUTION}
	%\begin{itemize}
	%	\item xxx
	%	\item xxx
	%	\item xxx
	%\end{itemize}

	%\underline{EXCHANGES}
	%\begin{itemize}
	%	\item xxx
	%	\item xxx
	%	\item xxx
	%\end{itemize}

	%\underline{PARALLEL}
	%\begin{itemize}
	%	\item xxx
	%	\item xxx
	%	\item xxx
	%\end{itemize}

