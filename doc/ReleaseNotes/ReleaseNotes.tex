\documentclass[11pt,twoside,twocolumn]{usgsreport}
\usepackage{usgsfonts}
\usepackage{usgsgeo}
\usepackage{usgsidx}
\usepackage[tabletoc]{usgsreporta}

\usepackage{amsmath}
\usepackage{algorithm}
\usepackage{algpseudocode}
\usepackage{bm}
\usepackage{calc}
\usepackage{natbib}
\usepackage{bibentry}
\usepackage{graphicx}
\usepackage{longtable}

\usepackage[T1]{fontenc}

\makeindex
\usepackage{setspace}
% uncomment to make double space 
%\doublespacing
\usepackage{etoolbox}
%\usepackage{verbatim}

\usepackage{titlesec}


\usepackage{hyperref}
\hypersetup{
    pdftitle={MODFLOW~6 Release Notes},
    pdfauthor={MODFLOW~6 Development Team},
    pdfsubject={MODFLOW~6 Release Notes},
    pdfkeywords={MODFLOW, groundwater model, simulation},
    bookmarksnumbered=true,     
    bookmarksopen=true,         
    bookmarksopenlevel=1,       
    colorlinks=true,
    allcolors={blue},          
    pdfstartview=Fit,           
    pdfpagemode=UseOutlines,
    pdfpagelayout=TwoPageRight
}


\graphicspath{{./Figures/}}
\newcommand{\modflowversion}{mf6.4.3.dev0}
\newcommand{\modflowdate}{June 29, 2023}
\newcommand{\currentmodflowversion}{Version \modflowversion---\modflowdate}



\renewcommand{\cooperator}
{the \textusgs\ Water Availability and Use Science Program}
\renewcommand{\reporttitle}
{MODFLOW~6 Release Notes}
\renewcommand{\coverphoto}{coverimage.jpg}
\renewcommand{\GSphotocredit}{Binary computer code illustration.}
\renewcommand{\reportseries}{}
\renewcommand{\reportnumber}{}
\renewcommand{\reportyear}{2017}
\ifdef{\reportversion}{\renewcommand{\reportversion}{\currentmodflowversion}}{}
\renewcommand{\theauthors}{MODFLOW~6 Development Team}
\renewcommand{\thetitlepageauthors}{\theauthors}
%\renewcommand{\theauthorslastfirst}{}
\renewcommand{\reportcitingtheauthors}{MODFLOW~6 Development Team}
\renewcommand{\colophonmoreinfo}{}
\renewcommand{\reportbodypages}{}
\urlstyle{rm}
\renewcommand{\reportwebsiteroot}{https://doi.org/10.5066/}
\renewcommand{\reportwebsiteremainder}{F76Q1VQV}
%\renewcommand{\doisecretary}{RYAN K. ZINKE}
%\renewcommand{\usgsdirector}{William H. Werkheiser}
%\ifdef{\usgsdirectortitle}{\renewcommand{\usgsdirectortitle}{Acting Director}}{}
\ifdef{\usgsissn}{\renewcommand{\usgsissn}{}}{}
\renewcommand{\theconventions}{}
\definecolor{coverbar}{RGB}{32, 18, 88}
\renewcommand{\bannercolor}{\color{coverbar}}
%\renewcommand{\thePSC}{the MODFLOW~6 Development Team}
%\renewcommand{\theeditor}{Christian D. Langevin}
%\renewcommand{\theillustrator}{None}
%\renewcommand{\thefirsttypesetter}{Joseph D. Hughes}
%\renewcommand{\thesecondtypesetter}{Cian Dawson}
\renewcommand{\reportrefname}{References Cited}

\makeatletter
\newcommand{\customlabel}[2]{%
   \protected@write \@auxout {}{\string \newlabel {#1}{{#2}{\thepage}{#2}{#1}{}} }%
   \hypertarget{#1}{}
}

\newcommand{\customcolophon}{
Publishing support provided by the U.S. Geological Survey \\
\theauthors
\newline \newline
For information concerning this publication, please contact:
\newline \newline
Integrated Modeling and Prediction Division \\ U.S. Geological Survey \\ Mail Stop 411 \\ 12201 Sunrise Valley Drive \\ Reston, VA 20192 \\
https://www.usgs.gov/mission-areas/water-resources
}

\renewcommand{\reportrefname}{References Cited}
\newcommand{\inreferences}{%
\renewcommand{\theequation}{R--\arabic{equation}}%
\setcounter{equation}{0}%
\renewcommand{\thefigure}{R--\arabic{figure}}%
\setcounter{figure}{0}%
\renewcommand{\thetable}{R--\arabic{table}}%
\setcounter{table}{0}%
\renewcommand{\thepage}{R--\arabic{page}}%
\setcounter{page}{1}%
}

\newcounter{appendixno}
\setcounter{appendixno}{0}
\newcommand{\inappendix}{%
\addtocounter{appendixno}{1}%
\renewcommand{\theequation}{\Alph{appendixno}--\arabic{equation}}%
\setcounter{equation}{0}%
\renewcommand{\thefigure}{\Alph{appendixno}--\arabic{figure}}%
\setcounter{figure}{0}%
\renewcommand{\thetable}{\Alph{appendixno}--\arabic{table}}%
\setcounter{table}{0}%
\renewcommand{\thepage}{\Alph{appendixno}--\arabic{page}}%
\setcounter{page}{1}%
}

\nobibliography*

\begin{document}

%\makefrontcover
\ifdef{\makefrontcoveralt}{\makefrontcoveralt}{\makefrontcover}

%\makefrontmatter
%\maketoc
\ifdef{\makefrontmatterabv}{\makefrontmatterabv}{\makefrontmatter}

\onecolumn
\pagestyle{body}
\RaggedRight
\hbadness=10000
\pagestyle{body}
\setlength{\parindent}{1.5pc}

% -------------------------------------------------
\section{Introduction}
This document describes MODFLOW~6 Version \modflowversion.  This distribution is packaged for personal computers using modern versions of the Microsoft Windows, Macintosh, and Linux operating systems, although the programs included in the distribution may run on older operating system versions.  The binary executable files in the distribution should run on most personal computers.

Version numbers for MODFLOW~6 follow a major.minor.revision format.  The major number is increased when there are substantial new changes that may break backward compatibility.  The minor number is increased when important, but relatively minor new functionality is added.  The revision number is incremented when errors are corrected in either the program or input files.

MODFLOW~6 is tested with a large number of example problems, and new capabilities are thoroughly tested as they are integrated into the program.  Additional testing can sometimes reveal errors in the program.  These errors are fixed as soon as possible and made available in a subsequent release.  Every effort is made to maintain backward compatibility of the format for MODFLOW~6 input files; however, this goal is not always possible, especially for newer packages and models.  Changes to the program to add new functionality, correct errors, or alter the format for an input file are summarized in this release notes document.


% -------------------------------------------------
\section{Release History}
A list of all MODFLOW 6 releases is shown in table \ref{tab:releases}.  Changes introduced in previous releases are described in Appendix~\ref{app:A}.

\begin{table}[h]
\begin{center}
\caption{List of MODFLOW 6 Releases}
\small 
\begin{tabular*}{12cm}{cl}
\hline
\hline
\textbf{MODFLOW Version} & \textbf{Date}\\
\hline
6.0.0 & August 11, 2017 \\
6.0.1 & September 28, 2017 \\
6.0.2 & February 23, 2018 \\
6.0.3 & August 9, 2018 \\
6.0.4 & March 7, 2019 \\
6.1.0 & December 12, 2019 \\
6.1.1 & June 12, 2020 \\
6.2.0 & October 20, 2020 \\
6.2.1 & February 18, 2021 \\
6.2.2 & July 30, 2021 \\
6.3.0 & March 4, 2022 \\
6.x.x & Month xx, 202x (this release) \\
\hline
\label{tab:releases}
\end{tabular*}
\end{center}
\normalsize
\end{table}


% -------------------------------------------------
\section{Changes Introduced in this Release}
This section describes changes introduced into MODFLOW~6 for the current release.  These changes may substantially affect users.

\begin{itemize}
	
	\item Version mf6.x.x--Month xx, 202x
	
	\underline{NEW FUNCTIONALITY}
	\begin{itemize}
		\item xxx
	%	\item xxx
	%	\item xxx
	\end{itemize}

	%\underline{EXAMPLES}
	%\begin{itemize}
	%	\item New autotest combines DISV with UZF and is named test_gwf_disv_uzf.py. Autotest verifies that the relative behavior of UZET and GWET is as expected.  
	%	\item xxx
	%	\item xxx
	%\end{itemize}

	\textbf{\underline{BUG FIXES AND OTHER CHANGES TO EXISTING FUNCTIONALITY}} \\
	\underline{BASIC FUNCTIONALITY}
	\begin{itemize}
		\item Corrected programming error in XT3D functionality that could occur when running coupled flow models or transport models.  The XT3D code would result in a memory access error when a child model with a much larger level of refinement was coupled to a coarser parent model.  The XT3D code was generalized to handle this situation. 
		\item Corrected a programming error in which the final message would be written twice to the screen and twice to mfsim.lst when the simulation terminated prematurely. 
	%	\item xxx
	\end{itemize}

	%\underline{INTERNAL FLOW PACKAGES}
	%\begin{itemize}
	%	\item xxx
	%	\item xxx
	%	\item xxx
	%\end{itemize}

	\underline{STRESS PACKAGES}
	\begin{itemize}
		\item The Evapotranspiration (EVT) Package was modified to include a new error check if the segmented evapotranspiration capability is active.  If the number of ET segments is greater than 1, then the user must specify values for PXDP (as well as PETM).  For a cell, PXDP is a one-dimensional array of size NSEG - 1.  Values in this array must be greater than zero and less than one.  Furthermore, the values in PXDP must increase monotonically.  The program now checks for these conditions and terminates with an error if these conditions are not met.  The segmented ET capability can used be used for list-based EVT Package input.  Provided that the PXDP conditions are met, this new error check should have no effect on simulation results.
	%	\item xxx
	%	\item xxx
	\end{itemize}

	%\underline{ADVANCED STRESS PACKAGES}
	%\begin{itemize}
	%	\item Add new MAW\_FLOW\_REDUCE\_CSV option for the Multi-Aquifer Well (MAW) Package.  If activated, then this option will result in information being written to a comma-separated value file for each multi-aquifer well and for each time step in which the extraction or injection rate is reduced by the program.  Information is not written for multi-aquifer wells in which the extraction or injection rate is equal to the user-specified extraction or injection rate.
	%	\item xxx
	%	\item xxx
	%\end{itemize}

	%\underline{SOLUTION}
	%\begin{itemize}
	%	\item xxx
	%	\item xxx
	%	\item xxx
	%\end{itemize}

	%\underline{EXCHANGES}
	%\begin{itemize}
	%	\item Fixed a bug that occurred when flow and transport models were run simultaneously and auxiliary variable(s), for example CONCENTRATION, and mover were both active in the OPTIONS block of a boundary package (e.g., WEL). When running the flow and transport problems separately, no issues occurred.  The issue only seemed to be a problem when running flow and transport simultaneously because an indexing variable would get out of sync in SSM that reported an error message to the screen and prematurely exited MF6.  Two new autotests added: the first runs GWF & GWT independently while the second runs them simultaneously and would trigger the condition that this bug report aims to fix.
	%	\item xxx
	%	\item xxx
	%\end{itemize}

	
\end{itemize}

% -------------------------------------------------
\section{Known Issues and Incompatibilities}
This section describes known issues with this release of MODFLOW~6.  

\begin{enumerate}

\item
The AUXMULTNAME option can be used to scale input values, such as riverbed conductance, using values in an auxiliary column.  When this AUXMULTNAME option is used, the multiplier value in the AUXMULTNAME column should not be represented with a time series unless the value to scale is also represented with a time series.  

\item
The capability to use Unsaturated Zone Flow (UZF) routing beneath lakes and streams has not been implemented.

\item
For the Groundwater Transport (GWT) Model, the decay and sorption processes do not apply to the LKT, SFT, MWT and UZT Packages.

\item
The GWT Model does not work with the CSUB Package of the GWF Model.  

\item
The GWT-GWT Exchange requires that both GWF Models are run concurrently in the same simulation.  There is not an option to read GWF-GWF flows from a previous simulation.

\item
The Buoyancy (BUY) Package of the GWF Model cannot be used when the XT3D option is activated for the Node Property Flow (NPF) Package.

\item
If a GWF-GWF Exchange is activated with the XT3D option, then the two connected GWF Models cannot have BUY Packages active.

\end{enumerate}

In addition to the issues shown here, a comprehensive and up-to-date list is available under the issues tab at \url{https://github.com/MODFLOW-USGS/modflow6}.


% -------------------------------------------------
\section{Distribution File}
The following distribution file is for use on personal computers: \texttt{\modflowversion.zip}.  The distribution file is a compressed zip file. The following directory structure is incorporated in the zip file:

% folder structured created by python script
\begin{verbatim}
mf6.1.0/ 
    bin/ 
    doc/ 
    examples/ 
        ex01-twri/ 
        ex02-tidal/ 
        ex03-bcf2ss/ 
        ex04-fhb/ 
        ex05-mfusg1disu/ 
        ex06-mfusg1disv/ 
        ex07-mfusg1lgr/ 
        ex08-mfusg1xt3d/ 
        ex09-bump/ 
        ex10-bumpnr/ 
        ex11-disvmesh/ 
        ex12-hanicol/ 
        ex13-hanirow/ 
        ex14-hanixt3d/ 
        ex15-whirlsxt3d/ 
        ex16-mfnwt2/ 
        ex17-mfnwt3h/ 
        ex18-mfnwt3l/ 
        ex19-zaidel/ 
        ex20-keating/ 
        ex21-sfr1/ 
        ex22-lak2/ 
        ex23-lak4/ 
        ex24-neville/ 
        ex25-flowing-maw/ 
        ex26-Reilly-maw/ 
        ex27-advpakmvr/ 
        ex28-mflgr3/ 
        ex29-vilhelmsen-gc/ 
        ex30-vilhelmsen-gf/ 
        ex31-vilhelmsen-lgr/ 
        ex32-periodicbc/ 
        ex33-csub-jacob/ 
        ex34-csub-sub01/ 
        ex35-csub-holly/ 
        ex36-csub-subwt01/ 
    make/ 
    msvs/ 
    src/ 
        Exchange/ 
        Model/ 
            Geometry/ 
            GroundWaterFlow/ 
            ModelUtilities/ 
        Solution/ 
            SparseMatrixSolver/ 
        Timing/ 
        Utilities/ 
            Memory/ 
            Observation/ 
            OutputControl/ 
            TimeSeries/ 
    utils/ 
        mf5to6/ 
            make/ 
            msvs/ 
            src/ 
                LGR/ 
                MF2005/ 
                NWT/ 
                Preproc/ 
        zonebudget/ 
            make/ 
            msvs/ 
            src/ 
\end{verbatim}


It is recommended that no user files are kept in the \modflowversion~directory structure.  If you do plan to put your own files in the \modflowversion~directory structure, do so only by creating additional subdirectories.

% -------------------------------------------------
\section{Installation and Execution}
There is no installation of MODFLOW~6 other than the requirement that \texttt{\modflowversion.zip} must be unzipped into a location where it can be accessed.  

To make the executable versions of MODFLOW~6 accessible from any directory, the directory containing the executables should be included in the PATH environment variable.  Also, if a prior release of MODFLOW~6 is installed on your system, the directory containing the executables for the prior release should be removed from the PATH environment variable.

As an alternative, the executable file, named ``\texttt{mf6.exe}'' on Windows, in the \modflowversion{}/bin directory can be copied into a directory already included in the PATH environment variable.

To run MODFLOW~6, simply type \texttt{mf6} in a terminal window.  The current working directory must be set to a location where the model input files are located.  Upon execution, MODFLOW~6 will immediately look for file with the name \texttt{mfsim.nam} in the current working directory, and will terminate with an error if it does not find this file.

% -------------------------------------------------
\section{Compiling MODFLOW~6}
MODFLOW~6 has been compiled using Intel Fortran and gfortran on the Windows, Mac, and Linux operating systems.  Because the program uses relatively new Fortran extensions, newer versions of the compilers may be required for successful compilation.  For example, to use gfortran to compile MODFLOW~6, gfortran version 4.9 or newer must be used.  If you have gfortran installed on your computer, you can tell which version it is by entering ``\verb|gfortran --version|'' at a terminal window.

This distribution contains the Microsoft Visual Studio solution and project files for compiling MODFLOW~6 on Windows using the Intel Fortran Compiler.  The files have been used successfully with recent versions of Microsoft Visual Studio Community 2019 and the Intel Fortran Compiler.  To compile MODFLOW~6, open the mf6.sln file in the msvs folder and click Build >  Build Solution.  A separate Visual Studio solution file is also included for building the BMI dynamically linked library version of MODFLOW~6.

This distribution also comes with a makefile for compiling MODFLOW~6 with \texttt{gfortran}.  The makefile is contained in the \texttt{make} folder.

For those familiar with Python, the pymake package can also be used to compile MODFLOW~6.  Additional information on the Python pymake utility can be found at: \url{https://github.com/modflowpy/pymake}.  

% -------------------------------------------------
\section{System Requirements}
MODFLOW~6 is written in Fortran.  It uses features from the 95, 2003, and 2008 language.  The code has been used on UNIX-based computers and personal computers running various forms of the Microsoft Windows operating system.

% -------------------------------------------------
\section{Testing}
The examples distributed with MODFLOW~6 can be run on Windows by navigating to the examples folder and executing the ``\texttt{run.bat}'' batch files within each example folder.  Alternatively, there is a ``\texttt{runall.bat}'' batch file under the examples folder that will run all of the test problems.  For Linux and Mac distributions, equivalent shell scripts (\texttt{run.sh} and \texttt{runall.sh}) are included.

% -------------------------------------------------
\section{MODFLOW~6 Documentation}
Details on the numerical methods and the underlying theory for MODFLOW~6 are described in the following reports and papers:

\begin{itemize}

\item \bibentry{modflow6framework}

\item \bibentry{modflow6gwf}

\item \bibentry{modflow6xt3d}

\item \bibentry{langevin2020hydraulic}

\item \bibentry{morway2021}

\item \bibentry{modflow6api}

\item \bibentry{modflow6gwt}

\item \bibentry{modflow6csub}

\end{itemize}
 
\noindent Description of the MODFLOW~6 input and output is included in this distribution in the ``doc'' folder as mf6io.pdf.

% -------------------------------------------------
% if runtime information exists, then include the run time comparison table
\IfFileExists{./run-time-comparison.tex}{\input{./run-time-comparison.tex}}{}

% -------------------------------------------------
\section{Disclaimer and Notices}

This software has been approved for release by the U.S. Geological Survey (USGS). Although the software has been subjected to rigorous review, the USGS reserves the right to update the software as needed pursuant to further analysis and review. No warranty, expressed or implied, is made by the USGS or the U.S. Government as to the functionality of the software and related material nor shall the fact of release constitute any such warranty. Furthermore, the software is released on condition that neither the USGS nor the U.S. Government shall be held liable for any damages resulting from its authorized or unauthorized use. Also refer to the USGS Water Resources Software User Rights Notice for complete use, copyright, and distribution information.

Notices related to this software are as follows:
\begin{itemize}
\item This software is a product of the U.S. Geological Survey, which is part of the U.S. Government.

\item This software is freely distributed. There is no fee to download and (or) use this software.

\item Users do not need a license or permission from the USGS to use this software. Users can download and install as many copies of the software as they need.

\item As a work of the United States Government, this USGS product is in the public domain within the United States. You can copy, modify, distribute, and perform the work, even for commercial purposes, all without asking permission. Additionally, USGS waives copyright and related rights in the work worldwide through CC0 1.0 Universal Public Domain Dedication (\url{https://creativecommons.org/publicdomain/zero/1.0/}).
\end{itemize}


\newpage
\ifx\usgsdirector\undefined
\addcontentsline{toc}{section}{\hspace{1.5em}\bibname}
\else
\inreferences
\REFSECTION
\fi
\bibliography{../MODFLOW6References}
\bibliographystyle{usgs.bst}



\newpage
\inappendix
\SECTION{Appendix A. Changes Introduced in Previous Versions}
\customlabel{app:A}{A}
\small
\begin{longtable}{p{1.5cm} p{1.5cm} p{3cm} c}
\caption{List of block names organized by component and input file type.  OPEN/CLOSE indicates whether or not the block information can be contained in separate file} \tabularnewline 

\hline
\hline
\textbf{Component} & \textbf{FTYPE} & \textbf{Blockname} & \textbf{OPEN/CLOSE} \\
\hline
\endfirsthead


\captionsetup{textformat=simple}
\caption*{\textbf{Table A--\arabic{table}.}{\quad}List of block names organized by component and input file type.  OPEN/CLOSE indicates whether or not the block information can be contained in separate file.---Continued} \tabularnewline

\hline
\hline
\textbf{Component} & \textbf{FTYPE} & \textbf{Blockname} & \textbf{OPEN/CLOSE} \\
\hline
\endhead

\hline
\endfoot


\hline
SIM & NAM & OPTIONS & yes \\ 
SIM & NAM & TIMING & yes \\ 
SIM & NAM & MODELS & yes \\ 
SIM & NAM & EXCHANGES & yes \\ 
SIM & NAM & SOLUTIONGROUP & yes \\ 
\hline
SIM & TDIS & OPTIONS & yes \\ 
SIM & TDIS & DIMENSIONS & yes \\ 
SIM & TDIS & PERIODDATA & yes \\ 
\hline
EXG & GWFGWF & OPTIONS & yes \\ 
EXG & GWFGWF & DIMENSIONS & yes \\ 
EXG & GWFGWF & EXCHANGEDATA & yes \\ 
\hline
EXG & GWTGWT & OPTIONS & yes \\ 
EXG & GWTGWT & DIMENSIONS & yes \\ 
EXG & GWTGWT & EXCHANGEDATA & yes \\ 
\hline
EXG & GWEGWE & OPTIONS & yes \\ 
EXG & GWEGWE & DIMENSIONS & yes \\ 
EXG & GWEGWE & EXCHANGEDATA & yes \\ 
\hline
SLN & IMS & OPTIONS & yes \\ 
SLN & IMS & NONLINEAR & yes \\ 
SLN & IMS & LINEAR & yes \\ 
\hline
GWF & NAM & OPTIONS & yes \\ 
GWF & NAM & PACKAGES & yes \\ 
\hline
GWF & DIS & OPTIONS & yes \\ 
GWF & DIS & DIMENSIONS & yes \\ 
GWF & DIS & GRIDDATA & no \\ 
\hline
GWF & DISV & OPTIONS & yes \\ 
GWF & DISV & DIMENSIONS & yes \\ 
GWF & DISV & GRIDDATA & no \\ 
GWF & DISV & VERTICES & yes \\ 
GWF & DISV & CELL2D & yes \\ 
\hline
GWF & DISU & OPTIONS & yes \\ 
GWF & DISU & DIMENSIONS & yes \\ 
GWF & DISU & GRIDDATA & no \\ 
GWF & DISU & CONNECTIONDATA & yes \\ 
GWF & DISU & VERTICES & yes \\ 
GWF & DISU & CELL2D & yes \\ 
\hline
GWF & IC & GRIDDATA & no \\ 
\hline
GWF & NPF & OPTIONS & yes \\ 
GWF & NPF & GRIDDATA & no \\ 
\hline
GWF & BUY & OPTIONS & yes \\ 
GWF & BUY & DIMENSIONS & yes \\ 
GWF & BUY & PACKAGEDATA & yes \\ 
\hline
GWF & STO & OPTIONS & yes \\ 
GWF & STO & GRIDDATA & no \\ 
GWF & STO & PERIOD & yes \\ 
\hline
GWF & CSUB & OPTIONS & yes \\ 
GWF & CSUB & DIMENSIONS & yes \\ 
GWF & CSUB & GRIDDATA & no \\ 
GWF & CSUB & PACKAGEDATA & yes \\ 
GWF & CSUB & PERIOD & yes \\ 
\hline
GWF & HFB & OPTIONS & yes \\ 
GWF & HFB & DIMENSIONS & yes \\ 
GWF & HFB & PERIOD & yes \\ 
\hline
GWF & CHD & OPTIONS & yes \\ 
GWF & CHD & DIMENSIONS & yes \\ 
GWF & CHD & PERIOD & yes \\ 
\hline
GWF & WEL & OPTIONS & yes \\ 
GWF & WEL & DIMENSIONS & yes \\ 
GWF & WEL & PERIOD & yes \\ 
\hline
GWF & DRN & OPTIONS & yes \\ 
GWF & DRN & DIMENSIONS & yes \\ 
GWF & DRN & PERIOD & yes \\ 
\hline
GWF & RIV & OPTIONS & yes \\ 
GWF & RIV & DIMENSIONS & yes \\ 
GWF & RIV & PERIOD & yes \\ 
\hline
GWF & GHB & OPTIONS & yes \\ 
GWF & GHB & DIMENSIONS & yes \\ 
GWF & GHB & PERIOD & yes \\ 
\hline
GWF & RCH & OPTIONS & yes \\ 
GWF & RCH & DIMENSIONS & yes \\ 
GWF & RCH & PERIOD & yes \\ 
\hline
GWF & RCHA & OPTIONS & yes \\ 
GWF & RCHA & PERIOD & yes \\ 
\hline
GWF & EVT & OPTIONS & yes \\ 
GWF & EVT & DIMENSIONS & yes \\ 
GWF & EVT & PERIOD & yes \\ 
\hline
GWF & EVTA & OPTIONS & yes \\ 
GWF & EVTA & PERIOD & yes \\ 
\hline
GWF & MAW & OPTIONS & yes \\ 
GWF & MAW & DIMENSIONS & yes \\ 
GWF & MAW & PACKAGEDATA & yes \\ 
GWF & MAW & CONNECTIONDATA & yes \\ 
GWF & MAW & PERIOD & yes \\ 
\hline
GWF & SFR & OPTIONS & yes \\ 
GWF & SFR & DIMENSIONS & yes \\ 
GWF & SFR & PACKAGEDATA & yes \\ 
GWF & SFR & CROSSSECTIONS & yes \\ 
GWF & SFR & CONNECTIONDATA & yes \\ 
GWF & SFR & DIVERSIONS & yes \\ 
GWF & SFR & PERIOD & yes \\ 
\hline
GWF & LAK & OPTIONS & yes \\ 
GWF & LAK & DIMENSIONS & yes \\ 
GWF & LAK & PACKAGEDATA & yes \\ 
GWF & LAK & CONNECTIONDATA & yes \\ 
GWF & LAK & TABLES & yes \\ 
GWF & LAK & OUTLETS & yes \\ 
GWF & LAK & PERIOD & yes \\ 
\hline
GWF & UZF & OPTIONS & yes \\ 
GWF & UZF & DIMENSIONS & yes \\ 
GWF & UZF & PACKAGEDATA & yes \\ 
GWF & UZF & PERIOD & yes \\ 
\hline
GWF & MVR & OPTIONS & yes \\ 
GWF & MVR & DIMENSIONS & yes \\ 
GWF & MVR & PACKAGES & yes \\ 
GWF & MVR & PERIOD & yes \\ 
\hline
GWF & GNC & OPTIONS & yes \\ 
GWF & GNC & DIMENSIONS & yes \\ 
GWF & GNC & GNCDATA & yes \\ 
\hline
GWF & OC & OPTIONS & yes \\ 
GWF & OC & PERIOD & yes \\ 
\hline
GWF & VSC & OPTIONS & yes \\ 
GWF & VSC & DIMENSIONS & yes \\ 
GWF & VSC & PACKAGEDATA & yes \\ 
\hline
GWF & API & OPTIONS & yes \\ 
GWF & API & DIMENSIONS & yes \\ 
\hline
GWT & ADV & OPTIONS & yes \\ 
\hline
GWT & DSP & OPTIONS & yes \\ 
GWT & DSP & GRIDDATA & no \\ 
\hline
GWT & CNC & OPTIONS & yes \\ 
GWT & CNC & DIMENSIONS & yes \\ 
GWT & CNC & PERIOD & yes \\ 
\hline
GWT & DIS & OPTIONS & yes \\ 
GWT & DIS & DIMENSIONS & yes \\ 
GWT & DIS & GRIDDATA & no \\ 
\hline
GWT & DISV & OPTIONS & yes \\ 
GWT & DISV & DIMENSIONS & yes \\ 
GWT & DISV & GRIDDATA & no \\ 
GWT & DISV & VERTICES & yes \\ 
GWT & DISV & CELL2D & yes \\ 
\hline
GWT & DISU & OPTIONS & yes \\ 
GWT & DISU & DIMENSIONS & yes \\ 
GWT & DISU & GRIDDATA & no \\ 
GWT & DISU & CONNECTIONDATA & yes \\ 
GWT & DISU & VERTICES & yes \\ 
GWT & DISU & CELL2D & yes \\ 
\hline
GWT & IC & GRIDDATA & no \\ 
\hline
GWT & NAM & OPTIONS & yes \\ 
GWT & NAM & PACKAGES & yes \\ 
\hline
GWT & OC & OPTIONS & yes \\ 
GWT & OC & PERIOD & yes \\ 
\hline
GWT & SSM & OPTIONS & yes \\ 
GWT & SSM & SOURCES & yes \\ 
GWT & SSM & FILEINPUT & yes \\ 
\hline
GWT & SRC & OPTIONS & yes \\ 
GWT & SRC & DIMENSIONS & yes \\ 
GWT & SRC & PERIOD & yes \\ 
\hline
GWT & MST & OPTIONS & yes \\ 
GWT & MST & GRIDDATA & no \\ 
\hline
GWT & IST & OPTIONS & yes \\ 
GWT & IST & GRIDDATA & no \\ 
\hline
GWT & SFT & OPTIONS & yes \\ 
GWT & SFT & PACKAGEDATA & yes \\ 
GWT & SFT & PERIOD & yes \\ 
\hline
GWT & LKT & OPTIONS & yes \\ 
GWT & LKT & PACKAGEDATA & yes \\ 
GWT & LKT & PERIOD & yes \\ 
\hline
GWT & MWT & OPTIONS & yes \\ 
GWT & MWT & PACKAGEDATA & yes \\ 
GWT & MWT & PERIOD & yes \\ 
\hline
GWT & UZT & OPTIONS & yes \\ 
GWT & UZT & PACKAGEDATA & yes \\ 
GWT & UZT & PERIOD & yes \\ 
\hline
GWT & FMI & OPTIONS & yes \\ 
GWT & FMI & PACKAGEDATA & yes \\ 
\hline
GWT & MVT & OPTIONS & yes \\ 
\hline
GWT & API & OPTIONS & yes \\ 
GWT & API & DIMENSIONS & yes \\ 
\hline
GWE & ADV & OPTIONS & yes \\ 
\hline
GWE & CND & OPTIONS & yes \\ 
GWE & CND & GRIDDATA & no \\ 
\hline
GWE & CTP & OPTIONS & yes \\ 
GWE & CTP & DIMENSIONS & yes \\ 
GWE & CTP & PERIOD & yes \\ 
\hline
GWE & DIS & OPTIONS & yes \\ 
GWE & DIS & DIMENSIONS & yes \\ 
GWE & DIS & GRIDDATA & no \\ 
\hline
GWE & DISV & OPTIONS & yes \\ 
GWE & DISV & DIMENSIONS & yes \\ 
GWE & DISV & GRIDDATA & no \\ 
GWE & DISV & VERTICES & yes \\ 
GWE & DISV & CELL2D & yes \\ 
\hline
GWE & DISU & OPTIONS & yes \\ 
GWE & DISU & DIMENSIONS & yes \\ 
GWE & DISU & GRIDDATA & no \\ 
GWE & DISU & CONNECTIONDATA & yes \\ 
GWE & DISU & VERTICES & yes \\ 
GWE & DISU & CELL2D & yes \\ 
\hline
GWE & ESL & OPTIONS & yes \\ 
GWE & ESL & DIMENSIONS & yes \\ 
GWE & ESL & PERIOD & yes \\ 
\hline
GWE & EST & OPTIONS & yes \\ 
GWE & EST & GRIDDATA & no \\ 
GWE & EST & PACKAGEDATA & yes \\ 
\hline
GWE & IC & GRIDDATA & no \\ 
\hline
GWE & LKE & OPTIONS & yes \\ 
GWE & LKE & PACKAGEDATA & yes \\ 
GWE & LKE & PERIOD & yes \\ 
\hline
GWE & NAM & OPTIONS & yes \\ 
GWE & NAM & PACKAGES & yes \\ 
\hline
GWE & OC & OPTIONS & yes \\ 
GWE & OC & PERIOD & yes \\ 
\hline
GWE & SSM & OPTIONS & yes \\ 
GWE & SSM & SOURCES & yes \\ 
GWE & SSM & FILEINPUT & yes \\ 
\hline
GWE & SFE & OPTIONS & yes \\ 
GWE & SFE & PACKAGEDATA & yes \\ 
GWE & SFE & PERIOD & yes \\ 
\hline
GWE & FMI & OPTIONS & yes \\ 
GWE & FMI & PACKAGEDATA & yes \\ 
\hline
UTL & SPC & OPTIONS & yes \\ 
UTL & SPC & DIMENSIONS & yes \\ 
UTL & SPC & PERIOD & yes \\ 
\hline
UTL & SPCA & OPTIONS & yes \\ 
UTL & SPCA & PERIOD & yes \\ 
\hline
UTL & SPT & OPTIONS & yes \\ 
UTL & SPT & DIMENSIONS & yes \\ 
UTL & SPT & PERIOD & yes \\ 
\hline
UTL & SPTA & OPTIONS & yes \\ 
UTL & SPTA & PERIOD & yes \\ 
\hline
UTL & OBS & OPTIONS & yes \\ 
UTL & OBS & CONTINUOUS & yes \\ 
\hline
UTL & LAKTAB & DIMENSIONS & yes \\ 
UTL & LAKTAB & TABLE & yes \\ 
\hline
UTL & SFRTAB & DIMENSIONS & yes \\ 
UTL & SFRTAB & TABLE & yes \\ 
\hline
UTL & TS & ATTRIBUTES & yes \\ 
UTL & TS & TIMESERIES & yes \\ 
\hline
UTL & TAS & ATTRIBUTES & yes \\ 
UTL & TAS & TIME & no \\ 
\hline
UTL & ATS & DIMENSIONS & yes \\ 
UTL & ATS & PERIODDATA & yes \\ 
\hline
UTL & TVK & OPTIONS & yes \\ 
UTL & TVK & PERIOD & yes \\ 
\hline
UTL & TVS & OPTIONS & yes \\ 
UTL & TVS & PERIOD & yes \\ 


\hline
\end{longtable}
\label{table:blocks}
\normalsize



\justifying
\vspace*{\fill}
\clearpage
\pagestyle{backofreport}
\makebackcover
\end{document}
