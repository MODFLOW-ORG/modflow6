\documentclass[11pt,twoside,twocolumn]{usgsreport}
\usepackage{usgsfonts}
\usepackage{usgsgeo}
\usepackage{usgsidx}
\usepackage[tabletoc]{usgsreporta}

\usepackage{amsmath}
\usepackage{algorithm}
\usepackage{algpseudocode}
\usepackage{bm}
\usepackage{calc}
\usepackage{natbib}
\usepackage{bibentry}
\usepackage{graphicx}
\usepackage{longtable}

\usepackage[T1]{fontenc}

\makeindex
\usepackage{setspace}
% uncomment to make double space 
%\doublespacing
\usepackage{etoolbox}
%\usepackage{verbatim}

\usepackage{titlesec}

\usepackage[hidelinks]{hyperref}
\hypersetup{
    pdftitle={MODFLOW~6 Release Notes},
    pdfauthor={MODFLOW~6 Development Team},
    pdfsubject={MODFLOW~6 Release Notes},
    pdfkeywords={MODFLOW, groundwater model, simulation},
    bookmarksnumbered=true,     
    bookmarksopen=true,         
    bookmarksopenlevel=1,       
    colorlinks=true,
    allcolors={blue},          
    pdfstartview=Fit,           
    pdfpagemode=UseOutlines,
    pdfpagelayout=TwoPageRight
}


\graphicspath{{./Figures/}}
\newcommand{\modflowversion}{mf6.4.3.dev0}
\newcommand{\modflowdate}{June 29, 2023}
\newcommand{\currentmodflowversion}{Version \modflowversion---\modflowdate}



\renewcommand{\cooperator}
{the \textusgs\ Water Availability and Use Science Program}
\renewcommand{\reporttitle}
{MODFLOW~6 Release Notes}
\renewcommand{\coverphoto}{coverimage.jpg}
\renewcommand{\GSphotocredit}{Binary computer code illustration.}
\renewcommand{\reportseries}{}
\renewcommand{\reportnumber}{}
\renewcommand{\reportyear}{2017}
\ifdef{\reportversion}{\renewcommand{\reportversion}{\currentmodflowversion}}{}
\renewcommand{\theauthors}{MODFLOW~6 Development Team}
\renewcommand{\thetitlepageauthors}{\theauthors}
%\renewcommand{\theauthorslastfirst}{}
\renewcommand{\reportcitingtheauthors}{MODFLOW~6 Development Team}
\renewcommand{\colophonmoreinfo}{}
\renewcommand{\theoffice}{Office of Groundwater \\ U.S. Geological Survey \\ Mail Stop 411 \\ 12201 Sunrise Valley Drive \\ Reston, VA 20192 \\ (703) 648-5001}
\renewcommand{\reportbodypages}{}
\urlstyle{rm}
\renewcommand{\reportwebsiteroot}{https://doi.org/10.5066/}
\renewcommand{\reportwebsiteremainder}{F76Q1VQV}
%\renewcommand{\doisecretary}{RYAN K. ZINKE}
%\renewcommand{\usgsdirector}{William H. Werkheiser}
%\ifdef{\usgsdirectortitle}{\renewcommand{\usgsdirectortitle}{Acting Director}}{}
\ifdef{\usgsissn}{\renewcommand{\usgsissn}{}}{}
\renewcommand{\theconventions}{}
\definecolor{coverbar}{RGB}{32, 18, 88}
\renewcommand{\bannercolor}{\color{coverbar}}
%\renewcommand{\thePSC}{the MODFLOW~6 Development Team}
%\renewcommand{\theeditor}{Christian D. Langevin}
%\renewcommand{\theillustrator}{None}
%\renewcommand{\thefirsttypesetter}{Joseph D. Hughes}
%\renewcommand{\thesecondtypesetter}{Cian Dawson}
\renewcommand{\reportrefname}{References Cited}

\makeatletter
\newcommand{\customlabel}[2]{%
   \protected@write \@auxout {}{\string \newlabel {#1}{{#2}{\thepage}{#2}{#1}{}} }%
   \hypertarget{#1}{}
}

\newcommand{\customcolophon}{
Publishing support provided by the U.S. Geological Survey \\
\theauthors
\newline \newline
For information concerning this publication, please contact:
\newline \newline
Office of Groundwater \\ U.S. Geological Survey \\ Mail Stop 411 \\ 12201 Sunrise Valley Drive \\ Reston, VA 20192 \\ (703) 648--5001 \\
https://water.usgs.gov/ogw/
}

\renewcommand{\reportrefname}{References Cited}
\newcommand{\inreferences}{%
\renewcommand{\theequation}{R--\arabic{equation}}%
\setcounter{equation}{0}%
\renewcommand{\thefigure}{R--\arabic{figure}}%
\setcounter{figure}{0}%
\renewcommand{\thetable}{R--\arabic{table}}%
\setcounter{table}{0}%
\renewcommand{\thepage}{R--\arabic{page}}%
\setcounter{page}{1}%
}

\newcounter{appendixno}
\setcounter{appendixno}{0}
\newcommand{\inappendix}{%
\addtocounter{appendixno}{1}%
\renewcommand{\theequation}{\Alph{appendixno}--\arabic{equation}}%
\setcounter{equation}{0}%
\renewcommand{\thefigure}{\Alph{appendixno}--\arabic{figure}}%
\setcounter{figure}{0}%
\renewcommand{\thetable}{\Alph{appendixno}--\arabic{table}}%
\setcounter{table}{0}%
\renewcommand{\thepage}{\Alph{appendixno}--\arabic{page}}%
\setcounter{page}{1}%
}

\nobibliography*

\begin{document}

%\makefrontcover
\ifdef{\makefrontcoveralt}{\makefrontcoveralt}{\makefrontcover}

%\makefrontmatter
%\maketoc
\ifdef{\makefrontmatterabv}{\makefrontmatterabv}{\makefrontmatter}

\onecolumn
\pagestyle{body}
\RaggedRight
\hbadness=10000
\pagestyle{body}
\setlength{\parindent}{1.5pc}

% -------------------------------------------------
\section{Introduction}
This document describes MODFLOW~6 Version \modflowversion.  This distribution is packaged for personal computers using the Microsoft Windows 7 and 10 operating systems, although it may run on other versions of Windows.  The executable file was compiled for 64-bit Windows operating systems and should run on most personal computers.

Version numbers for MODFLOW~6 will follow a major.minor.revision format.  The major number will be increased when there are substantial new changes that may break backward compatibility.  The minor number will be increased when important, but relatively minor new functionality is added.  The revision number will be added when errors are corrected in either the program or input files.

% -------------------------------------------------
\section{History}
This section describes changes introduced into MODFLOW~6 for the current release.  Changes introduced in previous releases are in Appendix~\ref{app:A}. These changes may substantially affect users.

\begin{itemize}
	\item Version mf6.1.1--June 12, 2019

	\underline{NEW FUNCTIONALITY}
	\begin{itemize}
		\item Refactor the source code to support the \href{https://csdms.colorado.edu/wiki/BMI_Description}{Basic Model Interface} (BMI) developed by the \href{https://csdms.colorado.edu/wiki/Main_Page}{Community Surface Dynamics Modeling System} (CSDMS) group. BMI is a set of standard control and query functions that, when added to a model code, make that model both easier to learn and easier to couple with other software elements \citep{PECKHAM20133}. Furthermore, the BMI makes it possible to control MODFLOW~6 execution from scripting languages using bindings for the BMI (for example, python bindings for the BMI available through \href{https://csdms.colorado.edu/wiki/PyMT}{pymt}). The BMI in this version is considered preliminary (alpha release). Limited testing of the BMI has been performed but significant changes are expected in future releases.  User support for the MODFLOW 6 BMI may be provided in the future.
		\item Add silent command line switch (\texttt{-s} or \texttt{\doubledash silent}) that sends all screen output (\texttt{STDOUT}) to a text file (with the name ``mfsim.stdout'').
		\item Add screen output command line switch (\texttt{-l <str>} or \texttt{\doubledash level <str>}) that controls output to the screen (\texttt{STDOUT}). If \texttt{<str>}  is \texttt{summary}, stress period and time step data are not written to \texttt{STDOUT}. If \texttt{<str>} is \texttt{debug}, normal and debug output are written to \texttt{STDOUT}. 
		\item Add simulation mode command line switch (\texttt{-m <str>} or \texttt{\doubledash mode <str>}) that controls the solution mode. If \texttt{<str>}  is \texttt{validate}, model input will be read and checked for errors, but the coefficient matrix or matrices will not be assembled or solved and solution output will not be written.
		\item Add SAVE\_SATURATION option to the Node Property Flow Package.  When invoked, cell saturation is written to the binary budget file as an auxiliary column for a record with the name ``DATA-SAT''.  The cell saturation can be used by post-processors to determine how much of the cell is saturated without having to know the value for ICELLTYPE or the value for head. If a cell is marked as confined (ICELLTYPE=0) then saturation is always one. If ICELLTYPE is one, then saturation ranges between zero and one.
		\item Add option for saving package convergence for the CSUB Package to a comma-separated values (CSV) file. Package convergence is enabled by specifying PACKAGE\_CONVERGENCE FILEOUT $<$package\_convergence\_filename$>$ in the options block for the package.
		\item Add option for saving package convergence for the LAK, SFR, and UZF Packages to comma-separated values (CSV) files. Package convergence for the LAK, SFR, and UZF Packages is enabled by specifying PACKAGE\_CONVERGENCE FILEOUT $<$package\_convergence\_filename$>$ in the options block for the package.
		\item Add CSV\_OUTER\_OUTPUT output option to save outer iteration information to a comma-separated values (CSV) file. The maximum of the model or package dependent-variable change for the outer iteration is written to the CSV file at the end of each outer iteration. 
		\item Add CSV\_INNER\_OUTPUT output option to save inner iteration information to a comma-separated values (CSV) file. The CSV output also the includes maximum dependent-variable change and maximum residual convergence information for the solution and each model (if the solution includes more than one model) and linear acceleration information for each inner iteration. The inner iteration CSV output, which contains a separate line for each inner iteration, is written to the CSV file all at once at the end of each outer iteration.
		\item Add OUTER\_DVCLOSE and INNER\_DVCLOSE variables to replace OUTER\_HCLOSE and INNER\_HCLOSE variables in the IMS Package input file. ``DV'' is used now instead to more generally refer to dependent variable.  Warning messages will be issued if OUTER\_HCLOSE and/or INNER\_HCLOSE variables are specified. OUTER\_HCLOSE and INNER\_HCLOSE variables will eventually be deprecated. 
		\item Add option to scale drain conductance as a function of simulated head over a user-defined range (drainage depth). Linear-conductance scaling is used with the Standard Formulation. Cubic-conductance scaling is used with the Newton-Raphson Formulation. The additional drainage depth variable is specified as an auxiliary variable and AUXDEPTHNAME is used to identify the auxiliary variable defining the drainage depth. The cubic-conductance scaling can be used as a replacement for the groundwater seepage option in the UZF Package. The scaled drainage conductance option can also be used to represent vertical seepage faces and improve model convergence in cells where simulated heads fluctuate around the elevation where the drain begins to discharge groundwater.
		\item Add timeseries support for the reach upstream fraction variable in the SFR package.
		\item Add Picard iterations for the SFR package to minimize differences in SFR package results between subsequent GWF Picard (non-linear) iterations as a result of non-optimal reach numbering. The number of SFR package Picard iterations can be controlled by specifying the maximum number of Picard  iteration to be used in the OPTIONS block (MAXIMUM\_PICARD\_ITERATIONS). If reaches are numbered in order, from upstream to downstream, MAXIMUM\_PICARD\_ITERATIONS can be set to 1 to reduce model run time. Specifying  MAXIMUM\_PICARD\_ITERATIONS to 1 will result in identical SFR package performance to previous versions of MODFLOW~6.
		\item Add flow correction option for the MAW package that corrects the MAW-GWF exchange in cases where the head in a multi-aquifer well is below the bottom of the screen for a connection or the head in a convertible cell connected to a multi-aquifer well is below the cell bottom.  When flow corrections are activated, unit head gradients are used to calculate the flow between a multi-aquifer well and a connected GWF cell. This option is identical to the MODFLOW-USG ``flow-to-dry-cell'' option for flow between a CLN cell and a GWF cell if the cell is convertible \citep{modflowusg}. Flow corrections are enabled by specifying FLOW\_CORRECTION in the OPTIONS block. By default, flow corrections are not made. \emph{Prior to this release (version 6.1.1), flow corrections were made anytime the head in a multi-aquifer well was below the bottom of the screen for a connection--this may result in different results for existing models that can be resolved by using the FLOW\_CORRECTION option.}
		\item Add new document, ``MODFLOW 6 -- Supplemental Technical Information,'' to the doc folder.  This document contains information that was in the mf6io.pdf appendices.  This technical information document may expand with future versions as new features are added.
	\end{itemize}

	\textbf{\underline{BUG FIXES AND OTHER CHANGES TO EXISTING FUNCTIONALITY}} \\
	\underline{BASIC FUNCTIONALITY}
	\begin{itemize}
		\item Correct an error in how the discretization package (for regular MODFLOW grids) calculates the distance between two cells when one or both of the cells are unconfined.  The error in the code would have only affected XT3D simulations with a regular grid, unconfined conditions, and specification of ANGLE2 in the NPF Package.  
		\item Correct an error in the use of the AUXMULTNAME option for boundary packages when time series are used.  A problem remains when time series are used for AUXMULTNAME but not for the column that is scaled by AUXMULTNAME.  This situation should be avoided.
	\end{itemize}

	\underline{STRESS PACKAGES}
	\begin{itemize}
		\item Fix a bug in binary budget file header for CSUB Package budget data written using IMETH=6 (CSUB-ELASTIC and CSUB-INELASTIC) .
		\item Add information on the CSUB Package budget terms and compaction data written the the Input/Output document in the `Description of Groundwater Flow (GWF) Model Binary Output Files' section.
		\item Prior to this release, calculated flows between a standard stress package (WEL, DRN, RIV, GHB, RCH, and EVT) and the connected model cell were based on the RHS and HCOF terms from the previous iteration.  This was not consistent with previous MODFLOW versions.  These packages were modified so that the flows are recalculated using the final converged head solution.  As a result of this change, simulated groundwater flows for these packages may be slightly different (compared to previous releases) if the package HCOF and RHS values depend on the simulated groundwater head.
	\end{itemize}

	\underline{ADVANCED STRESS PACKAGES}
	\begin{itemize}
		\item The code for saving the budget terms for the advanced packages was refactored to use common routines.  These changes should have no affect on simulation results.
		\item In previous releases, the LAK Package would accept negative user-input values for  RAINFALL, EVAPORATION, RUNOFF, INFLOW, and WITHDRAWAL even though the user guide mentioned that negative values are not allowed for these flow terms.  Error checks were added to ensure these values are specified as positive.
		\item Add a storage term to the SFR Package binary output file.  This term is always zero with the present implementation.  An auxiliary variable, called VOLUME, is also written with the storage budget term.  This term contains the calculated water volume in the reach.
		\item Refactor the SFR Package to remove use of RectangularChGeometry objects and added required functionality as private methods in the SFR module.
		\item  Improve error trapping in the MAW Package to catch divide by zero errors when calculating the saturated conductance for wells using the SKIN CONDEQN in connections where the cell  transmissivity (the product of geometric mean of the horizontal hydraulic conductivity and cell thickness) and well transmissivity (the product of HK\_SKIN and screen thickness) is equal to one. Also add error trapping for well connections using the 1) SKIN CONDEQN where the contrast between the cell and well transmissivities are less than one and 2) SKIN and MEAN CONDEQN where the calculated connection saturated conductance is less than zero.
		\item For the Lake Package, the outlet number was written as ID1 and ID2 for the TO-MVR record in the binary budget file.  This has been changed so that the lake number of the connected outlet is written to ID1 and ID2.  This change was implemented so that lake budgets can be calculated using the information in the lake budget file.
		\item The Lake, Streamflow Routing, and Multi-Aquifer Well Packages were modified to save the user-specified stage or head to the binary output file for lakes, reaches, or wells that are specified as being CONSTANT.  Prior to this change, a no-flow value was written to the package binary output files for constant stage lakes and streams and constant head multi-aquifer wells.  The no-flow value is still written for those lakes, streams, or wells that are specified by the user as being inactive.  This change should make it easier to post-process the results from these packages.
	\end{itemize}

	\underline{SOLUTION}
	\begin{itemize}
		\item Fix a bug in the linear solver when using the STRICT RCLOSE\_OPTION that prevented termination of inner iterations when the INNER\_DVCLOSE and INNER\_RCLOSE criteria were met but the inner iteration count was greater than one. The inner iterations are now terminated when the INNER\_DVCLOSE and INNER\_RCLOSE criteria are met but the linear solver is considered non-converged if the inner iteration count is greater than one.
		\item Deprecate the CSV\_OUTPUT output option in the OPTIONS BLOCK because the output to the comma-separated values (CSV) file was based on the PRINT\_OPTION option. If CSV\_OUTPUT is specified, it is used to define the file name for the CSV\_OUTER\_OUTPUT output option.
		\item Modify the outer iteration information written to the simulation listing file when PRINT\_OPTION is not NONE to improve the ability of users to evaluate model convergence. Added Package convergence data, eliminated dependent variable changes adjusted by under-relaxation, and flags to indicate when an outer iterations step is considered converged. Information is also provided if PTC causes non-convergence for a outer iteration (even if the model is converged) and if NEWTON UNDER\_RELAXATION resets outer iteration convergence from FALSE to TRUE. Dependent-variable changes for the under-relaxation step in an outer iteration are no longer reported because under-relaxation is only applied if the model or package outer iteration steps do not converge and by definition reduce dependent-variable changes and are not used to evaluate outer iteration convergence.
		\item Deprecate the OUTER\_RCLOSEBND optional variable in the NONLINEAR BLOCK because OUTER\_DVCLOSE is used for all terms used to evaluate package convergence. An warning will be issued if OUTER\_RCLOSEBND is specified.
		\item Deprecate the CSV\_OUTPUT output option. A warning will be issued if the CSV\_OUTPUT option is specified and outer iteration information will be saved to the specified FILEOUT comma-separated values (CSV) file.	
	\end{itemize}

\end{itemize}


% -------------------------------------------------
\section{Known Issues}
This section describes known issues with this release of MODFLOW~6.

\begin{enumerate}

\item
The AUXMULTNAME option can be used to scale input values, such as riverbed conductance, using values in an auxiliary column.  When this AUXMULTNAME option is used, the multiplier value in the AUXMULTNAME column should not be represented with a time series unless the value to scale is also represented with a time series.  

\item
The capability to use Unsaturated Zone Flow (UZF) routing beneath lakes and streams has not been implemented.

\end{enumerate}


% -------------------------------------------------
\section{Distribution File}
The following distribution file is for use on personal computers: \texttt{\modflowversion.zip}.  The distribution file is a compressed zip file. The following directory structure is incorporated in the zip file:

% folder structured created by python script
\begin{verbatim}
mf6.1.0/ 
    bin/ 
    doc/ 
    examples/ 
        ex01-twri/ 
        ex02-tidal/ 
        ex03-bcf2ss/ 
        ex04-fhb/ 
        ex05-mfusg1disu/ 
        ex06-mfusg1disv/ 
        ex07-mfusg1lgr/ 
        ex08-mfusg1xt3d/ 
        ex09-bump/ 
        ex10-bumpnr/ 
        ex11-disvmesh/ 
        ex12-hanicol/ 
        ex13-hanirow/ 
        ex14-hanixt3d/ 
        ex15-whirlsxt3d/ 
        ex16-mfnwt2/ 
        ex17-mfnwt3h/ 
        ex18-mfnwt3l/ 
        ex19-zaidel/ 
        ex20-keating/ 
        ex21-sfr1/ 
        ex22-lak2/ 
        ex23-lak4/ 
        ex24-neville/ 
        ex25-flowing-maw/ 
        ex26-Reilly-maw/ 
        ex27-advpakmvr/ 
        ex28-mflgr3/ 
        ex29-vilhelmsen-gc/ 
        ex30-vilhelmsen-gf/ 
        ex31-vilhelmsen-lgr/ 
        ex32-periodicbc/ 
        ex33-csub-jacob/ 
        ex34-csub-sub01/ 
        ex35-csub-holly/ 
        ex36-csub-subwt01/ 
    make/ 
    msvs/ 
    src/ 
        Exchange/ 
        Model/ 
            Geometry/ 
            GroundWaterFlow/ 
            ModelUtilities/ 
        Solution/ 
            SparseMatrixSolver/ 
        Timing/ 
        Utilities/ 
            Memory/ 
            Observation/ 
            OutputControl/ 
            TimeSeries/ 
    utils/ 
        mf5to6/ 
            make/ 
            msvs/ 
            src/ 
                LGR/ 
                MF2005/ 
                NWT/ 
                Preproc/ 
        zonebudget/ 
            make/ 
            msvs/ 
            src/ 
\end{verbatim}


It is recommended that no user files are kept in the \modflowversion~directory structure.  If you do plan to put your own files in the \modflowversion~directory structure, do so only by creating additional subdirectories.

% -------------------------------------------------
\section{Installation and Execution}
There is no installation of MODFLOW~6 other than the requirement that \texttt{\modflowversion.zip} must be unzipped into a location where it can be accessed.  

To make the executable versions of MODFLOW~6 accessible from any directory, the directory containing the executables should be included in the PATH environment variable.  Also, if a prior release of MODFLOW~6 is installed on your system, the directory containing the executables for the prior release should be removed from the PATH environment variable.

As an alternative, the executable file, ``\texttt{mf6.exe}'', in the \modflowversion{}/bin directory can be copied into a directory already included in the PATH environment variable.

To run MODFLOW~6, simply type \texttt{mf6} in a terminal window.  The current working directory must be set to a location where the model input files are located.  Upon execution, MODFLOW~6 will immediately look for file with the name \texttt{mfsim.nam} in the current working directory, and will terminate with an error if it does not find this file.

% -------------------------------------------------
\section{Compiling MODFLOW~6}
MODFLOW~6 has been compiled using Intel Visual Fortran and gfortran on the Windows and Mac/OS operating systems.  Because the program uses relatively new Fortran extensions, newer versions of the compilers may be required for successful compilation.  For example, to use gfortran to compile MODFLOW~6, gfortran version 4.9 or newer must be used.  If you have gfortran installed on your computer, you can tell which version it is by entering ``\verb|gfortran --version|'' at a terminal window.

This distribution contains the Microsoft Visual Studio solution and project files for compiling MODFLOW~6 on Windows using the Intel Fortran Compiler.  The files have been used successfully with Microsoft Visual Studio Community 2019 and Intel(R) Visual Fortran Compiler 2020.0.166.  To compile MODFLOW~6, open the mf6.sln file in the msvs folder and click Build >  Build Solution.  A separate Visual Studio solution file is also included for building the BMI dynamically linked library version of MODFLOW~6.

This distribution also comes with a makefile for compiling MODFLOW~6 with \texttt{gfortran}.  The makefile is contained in the \texttt{make} folder.

For those familiar with Python, the pymake package can also be used to compile MODFLOW~6.  Additional information on the Python pymake utility can be found at: \url{https://github.com/modflowpy/pymake}.  

% -------------------------------------------------
\section{System Requirements}
MODFLOW~6 is written in Fortran.  It uses features from the 95, 2003, and 2008 language.  The code has been used on UNIX-based computers and personal computers running various forms of the Microsoft Windows operating system.

% -------------------------------------------------
\section{Testing}
The examples distributed with MODFLOW~6 can be run by navigating to the examples folder and executing the ``\texttt{run.bat}'' batch files within each example folder.  Alternatively, there is a ``\texttt{runall.bat}'' batch file under the examples folder that will run all of the test problems.

% -------------------------------------------------
\section{MODFLOW~6 Documentation}
Details on the numerical methods and the underlying theory for MODFLOW~6 are described in the following reports:

\begin{itemize}

\item \bibentry{modflow6framework}

\item \bibentry{modflow6gwf}

\item \bibentry{modflow6xt3d}

\end{itemize}
 
\noindent Description of the MODFLOW~6 input and output is included in this distribution in the ``doc'' folder as mf6io.pdf.

% -------------------------------------------------
\section{Test Problems}
The following is a list of test problems distributed with MODFLOW~6.  Characteristics of these tests are contained in Table \ref{table:examples}.


% example tex files created by python scripts

\begin{itemize}
\item ex01-twri---This is the TWRI problem described in the MODFLOW-2005 documentation and included with the MODFLOW-2005 examples \citep{modflow2005}

\item ex02-tidal---This problem demonstrates the time series and observation capabilities of MODFLOW 6.  Use of multiple boundary packages for a single simulation is also demonstrated by including three recharge packages.

\item ex03-bcf2ss---This is the BCF2SS problem that is distributed with MODFLOW-2005 \citep{modflow2005}.  This problem demonstrates the wetting and drying capability in MODFLOW 6. The MODFLOW 6 problem is constructed with two layers (like the MODFLOW-2005 model) but the thickness of the confining bed is included in model layer 2 and the horizontal hydraulic conductivity of layer 2 is half that of model layer 2 in the MODFLOW-2005 model in order to calculate the correct horizontal conductance.

\item ex04-fhb---This problem is included with the MODFLOW-2005 examples \citep{modflow2005}. This problem demonstrates how the time-series functionality, combined with the Constant-Head and Well Packages, can be used to replace the Flow and Head Boundary (FHB) Package.

\item ex05-mfusg1disu---This is the first test problem presented in the MODFLOW-USG documentation \citep{modflowusg}.  It is included as an example problem to demonstrate a simple unstructured groundwater flow model.  The model uses ghost nodes to improve the accuracy of the groundwater flow solution.

\item ex06-mfusg1disv---This is the first test problem presented in the MODFLOW-USG documentation \citep{modflowusg}.  It is included as an example problem to demonstrate the DISV Package for a simple groundwater flow model.  The model uses ghost nodes to improve the accuracy of the groundwater flow solution.

\item ex07-mfusg1lgr---This is also the first test problem presented in the MODFLOW-USG manual \citep{modflowusg}; however, it is represented using two separate structured models.  The models are connected using a Groundwater Flow to Groundwater Flow (GWF-GWF) Exchange.  These two models are solved simultaneously in the same matrix equations.  A ghost-node correction is also applied to improve the flow calculation between models.

\item ex08-mfusg1xt3d---This is the first test problem presented in the MODFLOW-USG documentation \citep{modflowusg}.  It is included as an example problem to demonstrate the DISV Package for a simple groundwater flow model.  The model uses the XT3D formulation to improve the accuracy of the groundwater flow solution.

\item ex09-bump---This is a one-layer steady-state problem involving wetting and drying.  There is a rise in the bottom surface of the model, and groundwater flows around the rise.

\item ex10-bumpnr---This is a one-layer steady-state problem designed to test the Newton-Raphson approach.  There is a rise in the bottom surface of the model, and groundwater flows around the rise.

\item ex11-disvmesh---Demonstration of a triangular mesh with the DISV Package to discretize a circular island with a radius of 1500 meters.  The model has 2 layers and uses 2778 vertices (NVERT) to delineate 5240 cells per layer (NCPL).  General-head boundaries are assigned to model layer 1 for cells outside of a 1025 m radius circle.  Recharge is applied to the top of the model.

\item ex12-hanicol---Simple steady state model using a regular MODFLOW grid to simulate the response of an anisotropic confined aquifer to a pumping well. A constant-head boundary condition surrounds the active domain.  K22 is set to 100.0, which causes hydraulic conductivity in column direction to be 100 x more than K, which is in row direction.  Drawdown is more pronounced in column direction.

\item ex13-hanirow---Simple steady state model using a regular MODFLOW grid to simulate the response of an anisotropic confined aquifer to a pumping well. A constant-head boundary condition surrounds the active domain.  K22 is set to 0.01, which causes K in column direction to be 100 x less than K in the row direction.  Drawdown is more pronounced in row direction.

\item ex14-hanixt3d---Simple steady state model using a regular MODFLOW grid to simulate the response of an anisotropic confined aquifer to a pumping well. For this problem, the XT3D formulation is used so that hydraulic conductivity ellipse can be rotated in the x-y plane.  A constant-head boundary condition surrounds the active domain.  K22 is set to 0.01, which causes hydraulic conductivity in the column direction (prior to rotation) to be 100 x less than K in the row direction.  This ellipse is then rotated in the x-y plane by specifying a value for ANGLE1 in the NPF Package.  ANGLE1 is specifed with a constant value of 15 degrees for the entire grid, which means the dominant K component is rotated 15 degrees counter clockwise.  Drawdown is more pronounced along the dominant axis of the hydraulic conductivity ellipse.

\item ex15-whirlsxt3d---This is a 10 layer steady-state problem involving anisotropic groundwater flow.  The XT3D formulation is used to represent variable hydraulic conductivitity ellipsoid orientations.  The resulting flow pattern consists of groundwater whirls, as described in the XT3D documentation report \citep{modflow6xt3d}.

\item ex16-mfnwt2---This is the the second example problem described in the MODFLOW-NWT documentation \cite{modflownwt} and is based on ``problem 2'' in \cite{mcdonaldetal1991wetdry}. A fourth steady-state stress period has been added to the problem for comparison with fig. 8D in \cite{modflownwt}.

\item ex17-mfnwt3h---This is the high recharge case of the third example problem described in the MODFLOW-NWT documentation \citep{modflownwt}.

\item ex18-mfnwt3l---This is the low recharge case of the third example problem described in the MODFLOW-NWT documentation \citep{modflownwt}.

\item ex19-zaidel---This is the stair-step problem described in \cite{zaidel2013discontinuous}.  In this simulation, the Newton-Raphson formulation is used to improve simulation convergence.

\item ex20-keating---This is an example problem described in \cite{keating2009stable}.  The problem involves recharge through the unsaturated zone onto an aquitard.  The Newton-Raphson formulation is used for this problem to obtain a solution.

\item ex21-sfr1---This is the stream-aquifer interaction example problem (test 1) from the Streamflow Routing Package documentation \citep{prudic1989str}.  The specified depth segments in the original problem have been converted to active reaches and the diversion has been converted from UPTO to FRACTION CPRIOR type. This problem is simulated using the Streamflow Routing (SFR) Package in MODFLOW 6.

\item ex22-lak2---This is the lake-stream-aquifer interaction example problem (test simulation 2) from the Lake Package documentation \citep{modflowlak3pack}.  This problem is simulated using the Lake (LAK) and Streamflow Routing (SFR) Packages in MODFLOW 6. The Mover (MVR) Package is also used to exchange water between the SFR and LAK Packages.

\item ex23-lak4---This is the lake-aquifer interaction example problem (test simulation 4) from the Lake Package documentation \citep{modflowlak3pack}.  This problem is simulated using the Lake (LAK) Package in MODFLOW 6.

\item ex24-neville---This is the multi-aquifer well simulation described in \cite{nevilletonkin2004}.  This problem is simulated using the Multi-Aquifer Well (MAW) Package in MODFLOW 6.

\item ex25-flowing-maw---This is a multi-aquifer well simulation that demonstrates how to implement the flowing well option available in Multi-Aquifer Well (MAW) Package in MODFLOW 6. Aquifer properties and initial heads are identical to \cite{nevilletonkin2004}.  The pumping rate for well in the center of the domain is 0.0 cubic meters per day and the flowing well discharge elevation and conductance are specified to be 0.0 meters and 7,500 square meters per day.

\item ex26-Reilly-maw---This is the unstressed multi-aquifer well simulation described in \cite{reilly1989bias}.  This problem is simulated using the Multi-Aquifer Well (MAW) Package in MODFLOW 6.

\item ex27-advpakmvr---This is a variant of the unsaturated zone-stream-aquifer interaction example problem (test simulation 2) from the Unsaturated Zone Flow Package documentation \citep{UZF}.  The problem was modified to include two lakes.  The problems includes a two layer aquifer with the two lakes connected to the stream network.  This problem is simulated using the Unsaturated Zone Flow (UZF), Lake (LAK), and Streamflow Routing (SFR) Packages in MODFLOW 6. The Mover (MVR) Package is also used to exchange water between the UZF, LAK, and SFR Packages. Infiltration rates, ET rates, streamed Ks and lakebed leakances were changed to lower the water table below the interface of layers 1 and 2. This was done to demonstrate unsaturated flow through multiple layers. Aquifer K values were also changed.

\item ex28-mflgr3---The is Example 3 from the MODFLOW-LGR2 documentation \citep{mehl2013modflow}.

\item ex29-vilhelmsen-gc---This is the Globally Coarse (GC) model described in \cite{vilhelmsen2012evaluation}.

\item ex30-vilhelmsen-gf---This is the Globally Fine (GF) model described in \cite{vilhelmsen2012evaluation}.

\item ex31-vilhelmsen-lgr---This is the Local Grid Refinement (LGR) model described in \cite{vilhelmsen2012evaluation}.

\item ex32-periodicbc---Periodic boundary condition problem is based on \cite{laattoe2014spatial}. A MODFLOW 6 GWF-GWF Exchange is used to connect the left column with the right column.

\item ex33-csub-jacob---This is the \cite{jacob1939fluctuations} train problem, which simulates elastic compaction of aquifer materials in response to the loading of an aquifer by a passing train. The problem is described in the MODFLOW 6 CSUB Package Example Problems document released with MODFLOW 6 (version 6.1.0 or higher). This problem demonstrates elastic compaction of coarse-grained aquifer materials and time-varying loading at the surface to simulate the train passing an observation well in MODFLOW 6. The default effective-stress formulation is used to simulate coarse-grained material compaction.

\item ex34-csub-sub01---This is the Problem 1 from the Subsidence Package for MODFLOW-2000 documentation \citep{hoffmann2003modflow}, which is described in the MODFLOW 6 CSUB Package Example Problems document released with MODFLOW 6 (version 6.1.0 or higher).  This simulates the drainage of a thick interbed caused by a step decrease in hydraulic head in the aquifer in MODFLOW 6. The thick interbed is simulated using a delay-bed interbed and the default effective-stress formulation is used to simulate thick interbed compaction.

\item ex35-csub-holly---This is the one-dimensional MODFLOW 6 extensometer model based on the model developed by \cite{sneed2008} to simulate aquitard drainage, compaction and, land subsidence at the Holly site, located at the Edwards Air Force base, in response to effective stress changes caused by groundwater pumpage in the Antelope Valley in southern California. This problem is described in the MODFLOW 6 CSUB Package Example Problems document released with MODFLOW 6 (version 6.1.0 or higher). The model simulates compaction in a combination of no-delay and delay interbeds and the default effective-stress formulation is used to simulate interbed compaction.

\item ex36-csub-subwt01---This is the one-dimensional compaction in a three-dimensional flow field problem that is described in the MODFLOW 6 CSUB Package Example Problems document released with MODFLOW 6 (version 6.1.0 or higher). This problem is based on the problem presented in the SUB-WT package for MODFLOW-2005 report \citep{leake2007modflow} and represents groundwater development in a hypothetical aquifer that includes some features typical of basin-fill aquifers in an arid or semi-arid environment. The problem of \cite{leake2007modflow} was modified to include compaction of coarse-grained aquifer materials and water compressibility. The model simulates compaction in no-delay interbeds and the default effective-stress formulation is used to simulate interbed compaction.

\item ex37-draindepth---This is a modified version of the unsaturated zone-stream-aquifer interaction example problem (test simulation 2) from the Unsaturated Zone Flow Package documentation \cite{modflownwt}.  Originally, this problem is simulated using the Unsaturated Zone Flow (UZF) and Streamflow Routing (SFR) Packages in MODFLOW 6. The drain (DRN) package has been added to simulate groundwater discharge to the surface that was originally simulated using the UZF SIMULATE\_GWSEEP option with a SURFDEP of 1.0 in all UZF cells. Drains were specified in every cell containing a UZF cell and have a elevation 0.5 feet below land surface, a conductance of 25. square feet per day (a value equal to the conductance calculated by the UZF package when the SIMULATE\_GWSEEP option is specified), and a drainage depth of 1.0 foot.  The top of the model has been modified to be equal to the top elevation of all SFR reaches, in cells containing SFR reaches, in order to simulate water-levels that exceed land surface and drain discharge from the model. The Mover (MVR) Package is used to move water from the DRN and UZF Packages to the SFR Package. The Newton-Raphson formulation was specified to allow a direct comparison to a model with the UZF package SIMULATE\_GWSEEP option and no drains.

\end{itemize}
    


\small
\begin{longtable}{p{3cm} p{1cm} p{3cm} p{2.5cm}p{4cm}}
\caption{List of example problems and simulation characteristics}\tabularnewline


\hline
\hline
\textbf{Name} & \textbf{NPER} & \textbf{Namefile(s)} & \textbf{Dimensions (NLAY, NROW, NCOL), (NLAY, NCPL) or (NODES)}  & \textbf{Stress Packages} \\
\hline
\endfirsthead

\hline
\hline
\textbf{Name} & \textbf{NPER} & \textbf{Namefile(s)} & \textbf{Dimensions (NLAY, NROW, NCOL) or (NODES)}  & \textbf{Stress Packages} \\
\hline
\endhead


ex01-twri & 1 & \parbox[t]{3cm}{ twri.nam \\}& \parbox[t]{3cm}{ (3, 15, 15) \\}& \parbox[t]{4cm}{ CHD WEL DRN RCH  \\}\\
\hline
ex02-tidal & 4 & \parbox[t]{3cm}{ AdvGW\_tidal.nam \\}& \parbox[t]{3cm}{ (3, 15, 10) \\}& \parbox[t]{4cm}{ WEL RIV RCH GHB EVT  \\}\\
\hline
ex03-bcf2ss & 2 & \parbox[t]{3cm}{ bcf2ss.nam \\}& \parbox[t]{3cm}{ (2, 10, 15) \\}& \parbox[t]{4cm}{ WEL RIV RCH  \\}\\
\hline
ex04-fhb & 3 & \parbox[t]{3cm}{ fhb2015.nam \\}& \parbox[t]{3cm}{ (1, 3, 10) \\}& \parbox[t]{4cm}{ CHD WEL  \\}\\
\hline
ex05-mfusg1disu & 1 & \parbox[t]{3cm}{ flow.nam \\}& \parbox[t]{3cm}{ (121,) \\}& \parbox[t]{4cm}{ CHD  \\}\\
\hline
ex06-mfusg1disv & 1 & \parbox[t]{3cm}{ flow.nam \\}& \parbox[t]{3cm}{ (1, 121) \\}& \parbox[t]{4cm}{ CHD RCH  \\}\\
\hline
ex07-mfusg1lgr & 1 & \parbox[t]{3cm}{ model1.nam \\ model2.nam \\}& \parbox[t]{3cm}{ (1, 7, 7) \\ (1, 9, 9) \\}& \parbox[t]{4cm}{ CHD  \\ none \\}\\
\hline
ex08-mfusg1xt3d & 1 & \parbox[t]{3cm}{ flow.nam \\}& \parbox[t]{3cm}{ (1, 121) \\}& \parbox[t]{4cm}{ CHD RCH  \\}\\
\hline
ex09-bump & 1 & \parbox[t]{3cm}{ flowdivert.nam \\}& \parbox[t]{3cm}{ (1, 51, 51) \\}& \parbox[t]{4cm}{ CHD  \\}\\
\hline
ex10-bumpnr & 1 & \parbox[t]{3cm}{ flowdivert.nam \\}& \parbox[t]{3cm}{ (1, 51, 51) \\}& \parbox[t]{4cm}{ CHD  \\}\\
\hline
ex11-disvmesh & 1 & \parbox[t]{3cm}{ ci.nam \\}& \parbox[t]{3cm}{ (2, 5240) \\}& \parbox[t]{4cm}{ GHB RCH  \\}\\
\hline
ex12-hanicol & 1 & \parbox[t]{3cm}{ model.nam \\}& \parbox[t]{3cm}{ (1, 51, 51) \\}& \parbox[t]{4cm}{ CHD WEL  \\}\\
\hline
ex13-hanirow & 1 & \parbox[t]{3cm}{ model.nam \\}& \parbox[t]{3cm}{ (1, 51, 51) \\}& \parbox[t]{4cm}{ CHD WEL  \\}\\
\hline
ex14-hanixt3d & 1 & \parbox[t]{3cm}{ model.nam \\}& \parbox[t]{3cm}{ (1, 51, 51) \\}& \parbox[t]{4cm}{ CHD WEL  \\}\\
\hline
ex15-whirlsxt3d & 1 & \parbox[t]{3cm}{ model.nam \\}& \parbox[t]{3cm}{ (10, 10, 51) \\}& \parbox[t]{4cm}{ CHD WEL  \\}\\
\hline
ex16-mfnwt2 & 4 & \parbox[t]{3cm}{ test034\_nwtp2.nam \\}& \parbox[t]{3cm}{ (14, 40, 40) \\}& \parbox[t]{4cm}{ CHD RCH  \\}\\
\hline
ex17-mfnwt3h & 1 & \parbox[t]{3cm}{ nwtp3.nam \\}& \parbox[t]{3cm}{ (1, 80, 80) \\}& \parbox[t]{4cm}{ CHD RCH  \\}\\
\hline
ex18-mfnwt3l & 1 & \parbox[t]{3cm}{ nwtp3.nam \\}& \parbox[t]{3cm}{ (1, 80, 80) \\}& \parbox[t]{4cm}{ CHD RCH  \\}\\
\hline
ex19-zaidel & 1 & \parbox[t]{3cm}{ zaidel5m.nam \\}& \parbox[t]{3cm}{ (1, 1, 200) \\}& \parbox[t]{4cm}{ CHD  \\}\\
\hline
ex20-keating & 1 & \parbox[t]{3cm}{ keating.nam \\}& \parbox[t]{3cm}{ (80, 1, 400) \\}& \parbox[t]{4cm}{ RCH CHD  \\}\\
\hline
ex21-sfr1 & 2 & \parbox[t]{3cm}{ test1tr.nam \\}& \parbox[t]{3cm}{ (1, 15, 10) \\}& \parbox[t]{4cm}{ WEL EVT RCH GHB SFR  \\}\\
\hline
ex22-lak2 & 1 & \parbox[t]{3cm}{ lakeex2a.nam \\}& \parbox[t]{3cm}{ (5, 27, 17) \\}& \parbox[t]{4cm}{ EVT RCH SFR LAK CHD MVR  \\}\\
\hline
ex23-lak4 & 1 & \parbox[t]{3cm}{ lakeex4.nam \\}& \parbox[t]{3cm}{ (8, 36, 23) \\}& \parbox[t]{4cm}{ CHD RCH LAK  \\}\\
\hline
ex24-neville & 1 & \parbox[t]{3cm}{ NT\_Transient.nam \\}& \parbox[t]{3cm}{ (2, 101, 101) \\}& \parbox[t]{4cm}{ MAW  \\}\\
\hline
ex25-flowing-maw & 1 & \parbox[t]{3cm}{ FW\_Transient.nam \\}& \parbox[t]{3cm}{ (2, 101, 101) \\}& \parbox[t]{4cm}{ MAW  \\}\\
\hline
ex26-Reilly-maw & 1 & \parbox[t]{3cm}{ Reilly.nam \\}& \parbox[t]{3cm}{ (41, 16, 27) \\}& \parbox[t]{4cm}{ CHD MAW RCH  \\}\\
\hline
ex27-advpakmvr & 24 & \parbox[t]{3cm}{ uzfp3\_lakmvr\_v2.nam \\}& \parbox[t]{3cm}{ (2, 15, 10) \\}& \parbox[t]{4cm}{ SFR LAK WEL GHB UZF MVR  \\}\\
\hline
ex28-mflgr3 & 1 & \parbox[t]{3cm}{ ex3\_parent.nam \\ ex3\_child.nam \\}& \parbox[t]{3cm}{ (3, 15, 15) \\ (6, 15, 18) \\}& \parbox[t]{4cm}{ RIV CHD  \\ RIV  \\}\\
\hline
ex29-vilhelmsen-gc & 1 & \parbox[t]{3cm}{ parent.nam \\}& \parbox[t]{3cm}{ (9, 61, 49) \\}& \parbox[t]{4cm}{ RIV RCH  \\}\\
\hline
ex30-vilhelmsen-gf & 1 & \parbox[t]{3cm}{ TM9\_global\_gv.nam \\}& \parbox[t]{3cm}{ (25, 183, 147) \\}& \parbox[t]{4cm}{ RIV RCH  \\}\\
\hline
ex31-vilhelmsen-lgr & 1 & \parbox[t]{3cm}{ TM9\_parent\_GN.nam \\ Child\_GN.nam \\}& \parbox[t]{3cm}{ (9, 61, 49) \\ (25, 90, 78) \\}& \parbox[t]{4cm}{ RIV RCH  \\ RIV RCH  \\}\\
\hline
ex32-periodicbc & 1 & \parbox[t]{3cm}{ pbc.nam \\}& \parbox[t]{3cm}{ (190, 1, 100) \\}& \parbox[t]{4cm}{ CHD  \\}\\
\hline
ex33-csub-jacob & 2 & \parbox[t]{3cm}{ fig4\_base.nam \\}& \parbox[t]{3cm}{ (3, 1, 35) \\}& \parbox[t]{4cm}{ none \\}\\
\hline
ex34-csub-sub01 & 1 & \parbox[t]{3cm}{ sub01es.nam \\}& \parbox[t]{3cm}{ (1, 1, 3) \\}& \parbox[t]{4cm}{ CHD  \\}\\
\hline
ex35-csub-holly & 353 & \parbox[t]{3cm}{ holly.nam \\}& \parbox[t]{3cm}{ (14, 1, 1) \\}& \parbox[t]{4cm}{ chd  \\}\\
\hline
ex36-csub-subwt01 & 3 & \parbox[t]{3cm}{ csub\_subwt02b.nam \\}& \parbox[t]{3cm}{ (4, 20, 15) \\}& \parbox[t]{4cm}{ CHD WEL RCH  \\}\\
\hline
\hline
\end{longtable}
\label{table:examples}
\normalsize



% -------------------------------------------------
\section{Disclaimer and Notices}

This software has been approved for release by the U.S. Geological Survey (USGS). Although the software has been subjected to rigorous review, the USGS reserves the right to update the software as needed pursuant to further analysis and review. No warranty, expressed or implied, is made by the USGS or the U.S. Government as to the functionality of the software and related material nor shall the fact of release constitute any such warranty. Furthermore, the software is released on condition that neither the USGS nor the U.S. Government shall be held liable for any damages resulting from its authorized or unauthorized use. Also refer to the USGS Water Resources Software User Rights Notice for complete use, copyright, and distribution information.

Notices related to this software are as follows:
\begin{itemize}
\item This software is a product of the U.S. Geological Survey, which is part of the U.S. Government.

\item This software is freely distributed. There is no fee to download and (or) use this software.

\item Users do not need a license or permission from the USGS to use this software. Users can download and install as many copies of the software as they need.

\item As a work of the United States Government, this USGS product is in the public domain within the United States. You can copy, modify, distribute, and perform the work, even for commercial purposes, all without asking permission. Additionally, USGS waives copyright and related rights in the work worldwide through CC0 1.0 Universal Public Domain Dedication (\url{https://creativecommons.org/publicdomain/zero/1.0/}).
\end{itemize}


\newpage
\ifx\usgsdirector\undefined
\addcontentsline{toc}{section}{\hspace{1.5em}\bibname}
\else
\inreferences
\REFSECTION
\fi
\bibliography{../MODFLOW6References}
\bibliographystyle{usgs.bst}



\newpage
\inappendix
\SECTION{Appendix A. Changes Introduced in Previous Versions}
\customlabel{app:A}{A}
\small
\begin{longtable}{p{1.5cm} p{1.5cm} p{3cm} c}
\caption{List of block names organized by component and input file type.  OPEN/CLOSE indicates whether or not the block information can be contained in separate file} \tabularnewline 

\hline
\hline
\textbf{Component} & \textbf{FTYPE} & \textbf{Blockname} & \textbf{OPEN/CLOSE} \\
\hline
\endfirsthead


\captionsetup{textformat=simple}
\caption*{\textbf{Table A--\arabic{table}.}{\quad}List of block names organized by component and input file type.  OPEN/CLOSE indicates whether or not the block information can be contained in separate file.---Continued} \tabularnewline

\hline
\hline
\textbf{Component} & \textbf{FTYPE} & \textbf{Blockname} & \textbf{OPEN/CLOSE} \\
\hline
\endhead

\hline
\endfoot


\hline
SIM & NAM & OPTIONS & yes \\ 
SIM & NAM & TIMING & yes \\ 
SIM & NAM & MODELS & yes \\ 
SIM & NAM & EXCHANGES & yes \\ 
SIM & NAM & SOLUTIONGROUP & yes \\ 
\hline
SIM & TDIS & OPTIONS & yes \\ 
SIM & TDIS & DIMENSIONS & yes \\ 
SIM & TDIS & PERIODDATA & yes \\ 
\hline
EXG & GWFGWF & OPTIONS & yes \\ 
EXG & GWFGWF & DIMENSIONS & yes \\ 
EXG & GWFGWF & EXCHANGEDATA & yes \\ 
\hline
EXG & GWTGWT & OPTIONS & yes \\ 
EXG & GWTGWT & DIMENSIONS & yes \\ 
EXG & GWTGWT & EXCHANGEDATA & yes \\ 
\hline
EXG & GWEGWE & OPTIONS & yes \\ 
EXG & GWEGWE & DIMENSIONS & yes \\ 
EXG & GWEGWE & EXCHANGEDATA & yes \\ 
\hline
SLN & IMS & OPTIONS & yes \\ 
SLN & IMS & NONLINEAR & yes \\ 
SLN & IMS & LINEAR & yes \\ 
\hline
GWF & NAM & OPTIONS & yes \\ 
GWF & NAM & PACKAGES & yes \\ 
\hline
GWF & DIS & OPTIONS & yes \\ 
GWF & DIS & DIMENSIONS & yes \\ 
GWF & DIS & GRIDDATA & no \\ 
\hline
GWF & DISV & OPTIONS & yes \\ 
GWF & DISV & DIMENSIONS & yes \\ 
GWF & DISV & GRIDDATA & no \\ 
GWF & DISV & VERTICES & yes \\ 
GWF & DISV & CELL2D & yes \\ 
\hline
GWF & DISU & OPTIONS & yes \\ 
GWF & DISU & DIMENSIONS & yes \\ 
GWF & DISU & GRIDDATA & no \\ 
GWF & DISU & CONNECTIONDATA & yes \\ 
GWF & DISU & VERTICES & yes \\ 
GWF & DISU & CELL2D & yes \\ 
\hline
GWF & IC & GRIDDATA & no \\ 
\hline
GWF & NPF & OPTIONS & yes \\ 
GWF & NPF & GRIDDATA & no \\ 
\hline
GWF & BUY & OPTIONS & yes \\ 
GWF & BUY & DIMENSIONS & yes \\ 
GWF & BUY & PACKAGEDATA & yes \\ 
\hline
GWF & STO & OPTIONS & yes \\ 
GWF & STO & GRIDDATA & no \\ 
GWF & STO & PERIOD & yes \\ 
\hline
GWF & CSUB & OPTIONS & yes \\ 
GWF & CSUB & DIMENSIONS & yes \\ 
GWF & CSUB & GRIDDATA & no \\ 
GWF & CSUB & PACKAGEDATA & yes \\ 
GWF & CSUB & PERIOD & yes \\ 
\hline
GWF & HFB & OPTIONS & yes \\ 
GWF & HFB & DIMENSIONS & yes \\ 
GWF & HFB & PERIOD & yes \\ 
\hline
GWF & CHD & OPTIONS & yes \\ 
GWF & CHD & DIMENSIONS & yes \\ 
GWF & CHD & PERIOD & yes \\ 
\hline
GWF & WEL & OPTIONS & yes \\ 
GWF & WEL & DIMENSIONS & yes \\ 
GWF & WEL & PERIOD & yes \\ 
\hline
GWF & DRN & OPTIONS & yes \\ 
GWF & DRN & DIMENSIONS & yes \\ 
GWF & DRN & PERIOD & yes \\ 
\hline
GWF & RIV & OPTIONS & yes \\ 
GWF & RIV & DIMENSIONS & yes \\ 
GWF & RIV & PERIOD & yes \\ 
\hline
GWF & GHB & OPTIONS & yes \\ 
GWF & GHB & DIMENSIONS & yes \\ 
GWF & GHB & PERIOD & yes \\ 
\hline
GWF & RCH & OPTIONS & yes \\ 
GWF & RCH & DIMENSIONS & yes \\ 
GWF & RCH & PERIOD & yes \\ 
\hline
GWF & RCHA & OPTIONS & yes \\ 
GWF & RCHA & PERIOD & yes \\ 
\hline
GWF & EVT & OPTIONS & yes \\ 
GWF & EVT & DIMENSIONS & yes \\ 
GWF & EVT & PERIOD & yes \\ 
\hline
GWF & EVTA & OPTIONS & yes \\ 
GWF & EVTA & PERIOD & yes \\ 
\hline
GWF & MAW & OPTIONS & yes \\ 
GWF & MAW & DIMENSIONS & yes \\ 
GWF & MAW & PACKAGEDATA & yes \\ 
GWF & MAW & CONNECTIONDATA & yes \\ 
GWF & MAW & PERIOD & yes \\ 
\hline
GWF & SFR & OPTIONS & yes \\ 
GWF & SFR & DIMENSIONS & yes \\ 
GWF & SFR & PACKAGEDATA & yes \\ 
GWF & SFR & CROSSSECTIONS & yes \\ 
GWF & SFR & CONNECTIONDATA & yes \\ 
GWF & SFR & DIVERSIONS & yes \\ 
GWF & SFR & PERIOD & yes \\ 
\hline
GWF & LAK & OPTIONS & yes \\ 
GWF & LAK & DIMENSIONS & yes \\ 
GWF & LAK & PACKAGEDATA & yes \\ 
GWF & LAK & CONNECTIONDATA & yes \\ 
GWF & LAK & TABLES & yes \\ 
GWF & LAK & OUTLETS & yes \\ 
GWF & LAK & PERIOD & yes \\ 
\hline
GWF & UZF & OPTIONS & yes \\ 
GWF & UZF & DIMENSIONS & yes \\ 
GWF & UZF & PACKAGEDATA & yes \\ 
GWF & UZF & PERIOD & yes \\ 
\hline
GWF & MVR & OPTIONS & yes \\ 
GWF & MVR & DIMENSIONS & yes \\ 
GWF & MVR & PACKAGES & yes \\ 
GWF & MVR & PERIOD & yes \\ 
\hline
GWF & GNC & OPTIONS & yes \\ 
GWF & GNC & DIMENSIONS & yes \\ 
GWF & GNC & GNCDATA & yes \\ 
\hline
GWF & OC & OPTIONS & yes \\ 
GWF & OC & PERIOD & yes \\ 
\hline
GWF & VSC & OPTIONS & yes \\ 
GWF & VSC & DIMENSIONS & yes \\ 
GWF & VSC & PACKAGEDATA & yes \\ 
\hline
GWF & API & OPTIONS & yes \\ 
GWF & API & DIMENSIONS & yes \\ 
\hline
GWT & ADV & OPTIONS & yes \\ 
\hline
GWT & DSP & OPTIONS & yes \\ 
GWT & DSP & GRIDDATA & no \\ 
\hline
GWT & CNC & OPTIONS & yes \\ 
GWT & CNC & DIMENSIONS & yes \\ 
GWT & CNC & PERIOD & yes \\ 
\hline
GWT & DIS & OPTIONS & yes \\ 
GWT & DIS & DIMENSIONS & yes \\ 
GWT & DIS & GRIDDATA & no \\ 
\hline
GWT & DISV & OPTIONS & yes \\ 
GWT & DISV & DIMENSIONS & yes \\ 
GWT & DISV & GRIDDATA & no \\ 
GWT & DISV & VERTICES & yes \\ 
GWT & DISV & CELL2D & yes \\ 
\hline
GWT & DISU & OPTIONS & yes \\ 
GWT & DISU & DIMENSIONS & yes \\ 
GWT & DISU & GRIDDATA & no \\ 
GWT & DISU & CONNECTIONDATA & yes \\ 
GWT & DISU & VERTICES & yes \\ 
GWT & DISU & CELL2D & yes \\ 
\hline
GWT & IC & GRIDDATA & no \\ 
\hline
GWT & NAM & OPTIONS & yes \\ 
GWT & NAM & PACKAGES & yes \\ 
\hline
GWT & OC & OPTIONS & yes \\ 
GWT & OC & PERIOD & yes \\ 
\hline
GWT & SSM & OPTIONS & yes \\ 
GWT & SSM & SOURCES & yes \\ 
GWT & SSM & FILEINPUT & yes \\ 
\hline
GWT & SRC & OPTIONS & yes \\ 
GWT & SRC & DIMENSIONS & yes \\ 
GWT & SRC & PERIOD & yes \\ 
\hline
GWT & MST & OPTIONS & yes \\ 
GWT & MST & GRIDDATA & no \\ 
\hline
GWT & IST & OPTIONS & yes \\ 
GWT & IST & GRIDDATA & no \\ 
\hline
GWT & SFT & OPTIONS & yes \\ 
GWT & SFT & PACKAGEDATA & yes \\ 
GWT & SFT & PERIOD & yes \\ 
\hline
GWT & LKT & OPTIONS & yes \\ 
GWT & LKT & PACKAGEDATA & yes \\ 
GWT & LKT & PERIOD & yes \\ 
\hline
GWT & MWT & OPTIONS & yes \\ 
GWT & MWT & PACKAGEDATA & yes \\ 
GWT & MWT & PERIOD & yes \\ 
\hline
GWT & UZT & OPTIONS & yes \\ 
GWT & UZT & PACKAGEDATA & yes \\ 
GWT & UZT & PERIOD & yes \\ 
\hline
GWT & FMI & OPTIONS & yes \\ 
GWT & FMI & PACKAGEDATA & yes \\ 
\hline
GWT & MVT & OPTIONS & yes \\ 
\hline
GWT & API & OPTIONS & yes \\ 
GWT & API & DIMENSIONS & yes \\ 
\hline
GWE & ADV & OPTIONS & yes \\ 
\hline
GWE & CND & OPTIONS & yes \\ 
GWE & CND & GRIDDATA & no \\ 
\hline
GWE & CTP & OPTIONS & yes \\ 
GWE & CTP & DIMENSIONS & yes \\ 
GWE & CTP & PERIOD & yes \\ 
\hline
GWE & DIS & OPTIONS & yes \\ 
GWE & DIS & DIMENSIONS & yes \\ 
GWE & DIS & GRIDDATA & no \\ 
\hline
GWE & DISV & OPTIONS & yes \\ 
GWE & DISV & DIMENSIONS & yes \\ 
GWE & DISV & GRIDDATA & no \\ 
GWE & DISV & VERTICES & yes \\ 
GWE & DISV & CELL2D & yes \\ 
\hline
GWE & DISU & OPTIONS & yes \\ 
GWE & DISU & DIMENSIONS & yes \\ 
GWE & DISU & GRIDDATA & no \\ 
GWE & DISU & CONNECTIONDATA & yes \\ 
GWE & DISU & VERTICES & yes \\ 
GWE & DISU & CELL2D & yes \\ 
\hline
GWE & ESL & OPTIONS & yes \\ 
GWE & ESL & DIMENSIONS & yes \\ 
GWE & ESL & PERIOD & yes \\ 
\hline
GWE & EST & OPTIONS & yes \\ 
GWE & EST & GRIDDATA & no \\ 
GWE & EST & PACKAGEDATA & yes \\ 
\hline
GWE & IC & GRIDDATA & no \\ 
\hline
GWE & LKE & OPTIONS & yes \\ 
GWE & LKE & PACKAGEDATA & yes \\ 
GWE & LKE & PERIOD & yes \\ 
\hline
GWE & NAM & OPTIONS & yes \\ 
GWE & NAM & PACKAGES & yes \\ 
\hline
GWE & OC & OPTIONS & yes \\ 
GWE & OC & PERIOD & yes \\ 
\hline
GWE & SSM & OPTIONS & yes \\ 
GWE & SSM & SOURCES & yes \\ 
GWE & SSM & FILEINPUT & yes \\ 
\hline
GWE & SFE & OPTIONS & yes \\ 
GWE & SFE & PACKAGEDATA & yes \\ 
GWE & SFE & PERIOD & yes \\ 
\hline
GWE & FMI & OPTIONS & yes \\ 
GWE & FMI & PACKAGEDATA & yes \\ 
\hline
UTL & SPC & OPTIONS & yes \\ 
UTL & SPC & DIMENSIONS & yes \\ 
UTL & SPC & PERIOD & yes \\ 
\hline
UTL & SPCA & OPTIONS & yes \\ 
UTL & SPCA & PERIOD & yes \\ 
\hline
UTL & SPT & OPTIONS & yes \\ 
UTL & SPT & DIMENSIONS & yes \\ 
UTL & SPT & PERIOD & yes \\ 
\hline
UTL & SPTA & OPTIONS & yes \\ 
UTL & SPTA & PERIOD & yes \\ 
\hline
UTL & OBS & OPTIONS & yes \\ 
UTL & OBS & CONTINUOUS & yes \\ 
\hline
UTL & LAKTAB & DIMENSIONS & yes \\ 
UTL & LAKTAB & TABLE & yes \\ 
\hline
UTL & SFRTAB & DIMENSIONS & yes \\ 
UTL & SFRTAB & TABLE & yes \\ 
\hline
UTL & TS & ATTRIBUTES & yes \\ 
UTL & TS & TIMESERIES & yes \\ 
\hline
UTL & TAS & ATTRIBUTES & yes \\ 
UTL & TAS & TIME & no \\ 
\hline
UTL & ATS & DIMENSIONS & yes \\ 
UTL & ATS & PERIODDATA & yes \\ 
\hline
UTL & TVK & OPTIONS & yes \\ 
UTL & TVK & PERIOD & yes \\ 
\hline
UTL & TVS & OPTIONS & yes \\ 
UTL & TVS & PERIOD & yes \\ 


\hline
\end{longtable}
\label{table:blocks}
\normalsize



\justifying
\vspace*{\fill}
\clearpage
\pagestyle{backofreport}
\makebackcover
\end{document}
