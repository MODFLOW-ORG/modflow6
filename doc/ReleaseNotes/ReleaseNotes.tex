\documentclass[11pt,twoside,twocolumn]{usgsreport}
\usepackage{usgsfonts}
\usepackage{usgsgeo}
\usepackage{usgsidx}
\usepackage[tabletoc]{usgsreporta}

\usepackage{amsmath}
\usepackage{algorithm}
\usepackage{algpseudocode}
\usepackage{bm}
\usepackage{calc}
\usepackage{natbib}
\usepackage{bibentry}
\usepackage{graphicx}
\usepackage{longtable}

\usepackage[T1]{fontenc}

\makeindex
\usepackage{setspace}
% uncomment to make double space 
%\doublespacing
\usepackage{etoolbox}
%\usepackage{verbatim}

\usepackage{titlesec}


\usepackage[hidelinks]{hyperref}
\hypersetup{
    pdftitle={MODFLOW~6 Release Notes},
    pdfauthor={MODFLOW~6 Development Team},
    pdfsubject={MODFLOW~6 Release Notes},
    pdfkeywords={MODFLOW, groundwater model, simulation},
    bookmarksnumbered=true,     
    bookmarksopen=true,         
    bookmarksopenlevel=1,       
    colorlinks=true,
    allcolors={blue},          
    pdfstartview=Fit,           
    pdfpagemode=UseOutlines,
    pdfpagelayout=TwoPageRight
}


\graphicspath{{./Figures/}}
\newcommand{\modflowversion}{mf6.4.3.dev0}
\newcommand{\modflowdate}{June 29, 2023}
\newcommand{\currentmodflowversion}{Version \modflowversion---\modflowdate}



\renewcommand{\cooperator}
{the \textusgs\ Water Availability and Use Science Program}
\renewcommand{\reporttitle}
{MODFLOW~6 Release Notes}
\renewcommand{\coverphoto}{coverimage.jpg}
\renewcommand{\GSphotocredit}{Binary computer code illustration.}
\renewcommand{\reportseries}{}
\renewcommand{\reportnumber}{}
\renewcommand{\reportyear}{2017}
\ifdef{\reportversion}{\renewcommand{\reportversion}{\currentmodflowversion}}{}
\renewcommand{\theauthors}{MODFLOW~6 Development Team}
\renewcommand{\thetitlepageauthors}{\theauthors}
%\renewcommand{\theauthorslastfirst}{}
\renewcommand{\reportcitingtheauthors}{MODFLOW~6 Development Team}
\renewcommand{\colophonmoreinfo}{}
\renewcommand{\theoffice}{Office of Groundwater \\ U.S. Geological Survey \\ Mail Stop 411 \\ 12201 Sunrise Valley Drive \\ Reston, VA 20192 \\ (703) 648-5001}
\renewcommand{\reportbodypages}{}
\urlstyle{rm}
\renewcommand{\reportwebsiteroot}{https://doi.org/10.5066/}
\renewcommand{\reportwebsiteremainder}{F76Q1VQV}
%\renewcommand{\doisecretary}{RYAN K. ZINKE}
%\renewcommand{\usgsdirector}{William H. Werkheiser}
%\ifdef{\usgsdirectortitle}{\renewcommand{\usgsdirectortitle}{Acting Director}}{}
\ifdef{\usgsissn}{\renewcommand{\usgsissn}{}}{}
\renewcommand{\theconventions}{}
\definecolor{coverbar}{RGB}{32, 18, 88}
\renewcommand{\bannercolor}{\color{coverbar}}
%\renewcommand{\thePSC}{the MODFLOW~6 Development Team}
%\renewcommand{\theeditor}{Christian D. Langevin}
%\renewcommand{\theillustrator}{None}
%\renewcommand{\thefirsttypesetter}{Joseph D. Hughes}
%\renewcommand{\thesecondtypesetter}{Cian Dawson}
\renewcommand{\reportrefname}{References Cited}

\makeatletter
\newcommand{\customlabel}[2]{%
   \protected@write \@auxout {}{\string \newlabel {#1}{{#2}{\thepage}{#2}{#1}{}} }%
   \hypertarget{#1}{}
}

\newcommand{\customcolophon}{
Publishing support provided by the U.S. Geological Survey \\
\theauthors
\newline \newline
For information concerning this publication, please contact:
\newline \newline
Office of Groundwater \\ U.S. Geological Survey \\ Mail Stop 411 \\ 12201 Sunrise Valley Drive \\ Reston, VA 20192 \\ (703) 648--5001 \\
https://water.usgs.gov/ogw/
}

\renewcommand{\reportrefname}{References Cited}
\newcommand{\inreferences}{%
\renewcommand{\theequation}{R--\arabic{equation}}%
\setcounter{equation}{0}%
\renewcommand{\thefigure}{R--\arabic{figure}}%
\setcounter{figure}{0}%
\renewcommand{\thetable}{R--\arabic{table}}%
\setcounter{table}{0}%
\renewcommand{\thepage}{R--\arabic{page}}%
\setcounter{page}{1}%
}

\newcounter{appendixno}
\setcounter{appendixno}{0}
\newcommand{\inappendix}{%
\addtocounter{appendixno}{1}%
\renewcommand{\theequation}{\Alph{appendixno}--\arabic{equation}}%
\setcounter{equation}{0}%
\renewcommand{\thefigure}{\Alph{appendixno}--\arabic{figure}}%
\setcounter{figure}{0}%
\renewcommand{\thetable}{\Alph{appendixno}--\arabic{table}}%
\setcounter{table}{0}%
\renewcommand{\thepage}{\Alph{appendixno}--\arabic{page}}%
\setcounter{page}{1}%
}

\nobibliography*

\begin{document}

%\makefrontcover
\ifdef{\makefrontcoveralt}{\makefrontcoveralt}{\makefrontcover}

%\makefrontmatter
%\maketoc
\ifdef{\makefrontmatterabv}{\makefrontmatterabv}{\makefrontmatter}

\onecolumn
\pagestyle{body}
\RaggedRight
\hbadness=10000
\pagestyle{body}
\setlength{\parindent}{1.5pc}

% -------------------------------------------------
\section{Introduction}
This document describes MODFLOW~6 Version \modflowversion.  This distribution is packaged for personal computers using the Microsoft Windows 7 and 10 operating systems, although it may run on other versions of Windows.  The executable file was compiled for 64-bit Windows operating systems and should run on most personal computers.

Version numbers for MODFLOW~6 will follow a major.minor.revision format.  The major number will be increased when there are substantial new changes that may break backward compatibility.  The minor number will be increased when important, but relatively minor new functionality is added.  The revision number will be added when errors are corrected in either the program or input files.

MODFLOW~6 is tested with a large number of example problems, and new capabilities are thoroughly tested as they are integrated into the program.  Additional testing can sometimes reveal errors in the program.  These errors are fixed as soon as possible and made available in a subsequent release.  Every effort is made to maintain backward compatibility of the format for MODFLOW~6 input files; however, this goal is not always possible, especially for newer packages and models.  Changes to the program to add new functionality, correct errors, or alter the format for an input file are summarized in this release notes document.

% -------------------------------------------------
\section{History}
This section describes changes introduced into MODFLOW~6 for the current release.  Changes introduced in previous releases are in Appendix~\ref{app:A}. These changes may substantially affect users.

\begin{itemize}
	\item Version mf6.2.2--July 30, 2021

	\underline{NEW FUNCTIONALITY}
	\begin{itemize}
	        \item A new Adaptive Time Step (ATS) utility was added.  The ATS utility allows any stress period to be overridden with an alternative time stepping approach.  The ATS utility implements two main capabilities (1) the capability to retry failed time steps with a shorter time step repeatedly until convergence is achieved, and (2) the capability to shorten and lengthen time steps based on simulation behavior.  These capabilities are described in the user input and output guide in a new section on the ATS utility.
	        \item A new option for printing water contents to a dedicated output file has been added to UZF.  To activate, the keyword WATER\_CONTENT is added to the OPTIONS block of UZF, followed by FILEOUT, followed by the user-specified output file name, for example ``water-content.uzf.bin''.  The approach is analogous to the STAGE option within the SFR options block.  Contents of the new file will be written in binary and can be read using flopy's binaryfile utility.  
	        \item The residual balance error for groundwater flow and solute transport is now written to the diagonal position of the flowja array, which is marked with the text description ``FLOW-JA-FACE''.  The flowja array is optionally written to the binary model budget file according to user settings in the output control file and other package input files.
	        \item A new option for simulating specific storage changes only when a cell is fully saturated has been added to the storage (STO) package. To activate, the SS\_CONFINED\_ONLY keyword is added to the OPTIONS block in the STO Package. This option is identical to the approach used to calculate storage changes under confined conditions in MODFLOW-2005.
	\end{itemize}
	
	\underline{EXAMPLES}
	\begin{itemize}
	        \item Added the following new examples: 
	        \begin{itemize}
	          \item ex-gwt-hecht-mendez
	          \item ex-gwf-capture (This example is described in mf6examples.pdf to demonstrate functionality of the Application Programming Interface; it is not included in the examples folder of this distribution as it requires python and several python packages)
	        \end{itemize}
	        \item Added new citation to this document.  The \cite{morway2021} paper describes the use of the Water Mover Package in MODFLOW~6 to represent natural and managed hydrologic connections. 
	\end{itemize}

	\textbf{\underline{BUG FIXES AND OTHER CHANGES TO EXISTING FUNCTIONALITY}} \\
	\underline{BASIC FUNCTIONALITY}
	\begin{itemize}
		\item The specific storage formulation in the storage (STO) package has been modified to eliminate the dependency of the original formulation on the vertical datum. The original specific storage formulation also overestimated storage changes for cells that resaturated or desaturated in successive time steps. Furthermore, the sign of the specific storage change was incorrect in cells with negative heads and resaturated or desaturated in successive time steps. The revised specific storage formulation resolves all of the deficiencies of the original formulation and accurately simulates specific storage changes under water table conditions but will change the results for existing models. Testing indicates that the differences between models run with the original and revised specific storage formulation are generally small but tend to increase in models with large specific storage values or have cells that repeatedly resaturated or desaturated in successive time steps.
		\item The convergence failure message message written to GWF and GWT listing files (FAILED TO MEET SOLVER CONVERGENCE CRITERIA) is now written after the budget summary tables.  In previous releases this convergence failure message was written prior to printing heads and concentrations, which often resulted in this message being unnoticed by users.
	        \item The order of output written to the GWF and GWT listing files for a time step was reorganized in a consistent manner with model and package flows coming first, followed by dependent variables, and then concluding with budget summary tables.
	        \item The DISU Package checks to make sure that the top of a cell is not higher than the bottom of an overlying cell.  A new option was added to the DISU Package to allow the user to specify the vertical offset tolerance used in this check.
	        \item Add DISU Package check to ensure that JA(IA(n)) is equal to n and that no values in JA are less than zero or greater than nodes.
	        \item When IDOMAIN is used with the DISU Package and any IDOMAIN value is zero, then the program was expecting all JA values to be positive. The program is supposed to allow a negative JA value to be specified for the corresponding cell (in the diagonal position), but this was not working.  A fix was implemented to allow a negative cell number to be specified in the diagonal position of the JA array when the IDOMAIN capability is active.
	        \item A new check was added to the Horizontal Flow Barrier (HFB) Package to ensure that barriers are between cells that are horizontally connected.  The program would previously continue running if a barrier was between vertically connected cells.
	        \item There was no check to prevent the zero-order decay functionality of the Mobile Storage and Transfer (MST) and Immobile Storage and Transfer (IST) Packages in the GWT Model from producing negative concentrations.  The program now reduces the zero-order decay rate for the aqueous and sorbed phases (for the mobile and immobile domains) to ensure that decay does not consume more mass than is available.  These changes do not affect zero-order growth.
	        \item If a binary budget file from a GWF Model was larger than about 2 Gigabytes, then it could not be used as input for a subsequent GWT Model.  The program was modified to use a long integer to store the byte position.
	        \item The program was terminating with a non-zero return code if the simulation did not converge.  This is the intended behavior, unless the CONTINUE option is specified in the simulation name file.  The program now terminates with a return code of zero if the simulation does not converge, but the CONTINUE option is set and the program reaches the end of the simulation.
	\end{itemize}

%	\underline{STRESS PACKAGES}
%	\begin{itemize}
%	        \item xxx  
%	        \item xxx  
%	        \item xxx  
%	\end{itemize}

	\underline{ADVANCED STRESS PACKAGES}
	\begin{itemize}
	        \item The UZF water-content observation by depth was giving an error, because a check was using the wrong index to retrieve the cell top and bottom elevations for the requested observation.  The program was modified to use the correct index, and the output is now as expected.  Note that this bug is not related to the new WC keyword in the OPTIONS block, but rather is related to OBS6 output option.
	        \item Amend surfdep error check with landflag.  Deep cells (non-land surface cells) should not require surfdep > 0
	        \item In the LAK observation package, users can specify ``lak'' to get a summary of lake-groundwater exchange.  Users could specify a lake number without specifying a specific connection number (variable ``iconn'').  Code will now stop if lake number is provided without a matching connection number.  Code will still provide a summary of total lake-groundwater exchange when BOUNDNAME is entered for the variable ID.  This also will fix a similar issue for the observation types ``wetted-area'' and ``conductance'', since both require ID2 when ID is an integer corresponding to a lake number.
	        \item In the MAW observation package, users can specify ``maw'' to get a summary of well-groundwater exchange.  The code was allowing users to specify a well number without requiring specification of a connection number (variable ``icon'').  Code will now stop if well number is provided without a matching connection number.  Code will still provide a summary of total well-groundwater exchange when BOUNDNAME is entered for the variable ID.  This also will fix a similar issue for the observation type ``conductance'', since both require ID2 when ID is an integer corresponding to a well number.
	\end{itemize}

%	\underline{SOLUTION}
%	\begin{itemize}
%	        \item xxx  
%	        \item xxx  
%	        \item xxx  
%	\end{itemize}

\end{itemize}


% -------------------------------------------------
\section{Known Issues}
This section describes known issues with this release of MODFLOW~6.  

\begin{enumerate}

\item
The AUXMULTNAME option can be used to scale input values, such as riverbed conductance, using values in an auxiliary column.  When this AUXMULTNAME option is used, the multiplier value in the AUXMULTNAME column should not be represented with a time series unless the value to scale is also represented with a time series.  

\item
The capability to use Unsaturated Zone Flow (UZF) routing beneath lakes and streams has not been implemented.

\item
Multi-point cross-section geometries have not been implemented in the Streamflow Routing (SFR) package.

\item
For the Groundwater Transport (GWT) Model, the decay and sorption processes do not apply to the LKT, SFT, MWT and UZT Packages.

\item
The GWT Model does not yet work with the CSUB Package of the GWF Model.  

\end{enumerate}

In addition to the issues shown here, a comprehensive and up-to-date list is available under the issues tab at \url{https://github.com/MODFLOW-USGS/modflow6}.


% -------------------------------------------------
\section{Distribution File}
The following distribution file is for use on personal computers: \texttt{\modflowversion.zip}.  The distribution file is a compressed zip file. The following directory structure is incorporated in the zip file:

% folder structured created by python script
\begin{verbatim}
mf6.1.0/ 
    bin/ 
    doc/ 
    examples/ 
        ex01-twri/ 
        ex02-tidal/ 
        ex03-bcf2ss/ 
        ex04-fhb/ 
        ex05-mfusg1disu/ 
        ex06-mfusg1disv/ 
        ex07-mfusg1lgr/ 
        ex08-mfusg1xt3d/ 
        ex09-bump/ 
        ex10-bumpnr/ 
        ex11-disvmesh/ 
        ex12-hanicol/ 
        ex13-hanirow/ 
        ex14-hanixt3d/ 
        ex15-whirlsxt3d/ 
        ex16-mfnwt2/ 
        ex17-mfnwt3h/ 
        ex18-mfnwt3l/ 
        ex19-zaidel/ 
        ex20-keating/ 
        ex21-sfr1/ 
        ex22-lak2/ 
        ex23-lak4/ 
        ex24-neville/ 
        ex25-flowing-maw/ 
        ex26-Reilly-maw/ 
        ex27-advpakmvr/ 
        ex28-mflgr3/ 
        ex29-vilhelmsen-gc/ 
        ex30-vilhelmsen-gf/ 
        ex31-vilhelmsen-lgr/ 
        ex32-periodicbc/ 
        ex33-csub-jacob/ 
        ex34-csub-sub01/ 
        ex35-csub-holly/ 
        ex36-csub-subwt01/ 
    make/ 
    msvs/ 
    src/ 
        Exchange/ 
        Model/ 
            Geometry/ 
            GroundWaterFlow/ 
            ModelUtilities/ 
        Solution/ 
            SparseMatrixSolver/ 
        Timing/ 
        Utilities/ 
            Memory/ 
            Observation/ 
            OutputControl/ 
            TimeSeries/ 
    utils/ 
        mf5to6/ 
            make/ 
            msvs/ 
            src/ 
                LGR/ 
                MF2005/ 
                NWT/ 
                Preproc/ 
        zonebudget/ 
            make/ 
            msvs/ 
            src/ 
\end{verbatim}


It is recommended that no user files are kept in the \modflowversion~directory structure.  If you do plan to put your own files in the \modflowversion~directory structure, do so only by creating additional subdirectories.

% -------------------------------------------------
\section{Installation and Execution}
There is no installation of MODFLOW~6 other than the requirement that \texttt{\modflowversion.zip} must be unzipped into a location where it can be accessed.  

To make the executable versions of MODFLOW~6 accessible from any directory, the directory containing the executables should be included in the PATH environment variable.  Also, if a prior release of MODFLOW~6 is installed on your system, the directory containing the executables for the prior release should be removed from the PATH environment variable.

As an alternative, the executable file, ``\texttt{mf6.exe}'', in the \modflowversion{}/bin directory can be copied into a directory already included in the PATH environment variable.

To run MODFLOW~6, simply type \texttt{mf6} in a terminal window.  The current working directory must be set to a location where the model input files are located.  Upon execution, MODFLOW~6 will immediately look for file with the name \texttt{mfsim.nam} in the current working directory, and will terminate with an error if it does not find this file.

% -------------------------------------------------
\section{Compiling MODFLOW~6}
MODFLOW~6 has been compiled using Intel Visual Fortran and gfortran on the Windows and Mac/OS operating systems.  Because the program uses relatively new Fortran extensions, newer versions of the compilers may be required for successful compilation.  For example, to use gfortran to compile MODFLOW~6, gfortran version 4.9 or newer must be used.  If you have gfortran installed on your computer, you can tell which version it is by entering ``\verb|gfortran --version|'' at a terminal window.

This distribution contains the Microsoft Visual Studio solution and project files for compiling MODFLOW~6 on Windows using the Intel Fortran Compiler.  The files have been used successfully with Microsoft Visual Studio Community 2019 and Intel(R) Visual Fortran Compiler 2020.0.166.  To compile MODFLOW~6, open the mf6.sln file in the msvs folder and click Build >  Build Solution.  A separate Visual Studio solution file is also included for building the BMI dynamically linked library version of MODFLOW~6.

This distribution also comes with a makefile for compiling MODFLOW~6 with \texttt{gfortran}.  The makefile is contained in the \texttt{make} folder.

For those familiar with Python, the pymake package can also be used to compile MODFLOW~6.  Additional information on the Python pymake utility can be found at: \url{https://github.com/modflowpy/pymake}.  

% -------------------------------------------------
\section{System Requirements}
MODFLOW~6 is written in Fortran.  It uses features from the 95, 2003, and 2008 language.  The code has been used on UNIX-based computers and personal computers running various forms of the Microsoft Windows operating system.

% -------------------------------------------------
\section{Testing}
The examples distributed with MODFLOW~6 can be run by navigating to the examples folder and executing the ``\texttt{run.bat}'' batch files within each example folder.  Alternatively, there is a ``\texttt{runall.bat}'' batch file under the examples folder that will run all of the test problems.

% -------------------------------------------------
\section{MODFLOW~6 Documentation}
Details on the numerical methods and the underlying theory for MODFLOW~6 are described in the following reports and papers:

\begin{itemize}

\item \bibentry{modflow6framework}

\item \bibentry{modflow6gwf}

\item \bibentry{modflow6xt3d}

\item \bibentry{langevin2020hydraulic}

\item \bibentry{morway2021}

\end{itemize}
 
\noindent Description of the MODFLOW~6 input and output is included in this distribution in the ``doc'' folder as mf6io.pdf.

% -------------------------------------------------
% if runtime information exists, then include the run time comparison table
\IfFileExists{./run-time-comparison.tex}{\input{./run-time-comparison.tex}}{}

% -------------------------------------------------
\section{Disclaimer and Notices}

This software has been approved for release by the U.S. Geological Survey (USGS). Although the software has been subjected to rigorous review, the USGS reserves the right to update the software as needed pursuant to further analysis and review. No warranty, expressed or implied, is made by the USGS or the U.S. Government as to the functionality of the software and related material nor shall the fact of release constitute any such warranty. Furthermore, the software is released on condition that neither the USGS nor the U.S. Government shall be held liable for any damages resulting from its authorized or unauthorized use. Also refer to the USGS Water Resources Software User Rights Notice for complete use, copyright, and distribution information.

Notices related to this software are as follows:
\begin{itemize}
\item This software is a product of the U.S. Geological Survey, which is part of the U.S. Government.

\item This software is freely distributed. There is no fee to download and (or) use this software.

\item Users do not need a license or permission from the USGS to use this software. Users can download and install as many copies of the software as they need.

\item As a work of the United States Government, this USGS product is in the public domain within the United States. You can copy, modify, distribute, and perform the work, even for commercial purposes, all without asking permission. Additionally, USGS waives copyright and related rights in the work worldwide through CC0 1.0 Universal Public Domain Dedication (\url{https://creativecommons.org/publicdomain/zero/1.0/}).
\end{itemize}


\newpage
\ifx\usgsdirector\undefined
\addcontentsline{toc}{section}{\hspace{1.5em}\bibname}
\else
\inreferences
\REFSECTION
\fi
\bibliography{../MODFLOW6References}
\bibliographystyle{usgs.bst}



\newpage
\inappendix
\SECTION{Appendix A. Changes Introduced in Previous Versions}
\customlabel{app:A}{A}
\small
\begin{longtable}{p{1.5cm} p{1.5cm} p{3cm} c}
\caption{List of block names organized by component and input file type.  OPEN/CLOSE indicates whether or not the block information can be contained in separate file} \tabularnewline 

\hline
\hline
\textbf{Component} & \textbf{FTYPE} & \textbf{Blockname} & \textbf{OPEN/CLOSE} \\
\hline
\endfirsthead


\captionsetup{textformat=simple}
\caption*{\textbf{Table A--\arabic{table}.}{\quad}List of block names organized by component and input file type.  OPEN/CLOSE indicates whether or not the block information can be contained in separate file.---Continued} \tabularnewline

\hline
\hline
\textbf{Component} & \textbf{FTYPE} & \textbf{Blockname} & \textbf{OPEN/CLOSE} \\
\hline
\endhead

\hline
\endfoot


\hline
SIM & NAM & OPTIONS & yes \\ 
SIM & NAM & TIMING & yes \\ 
SIM & NAM & MODELS & yes \\ 
SIM & NAM & EXCHANGES & yes \\ 
SIM & NAM & SOLUTIONGROUP & yes \\ 
\hline
SIM & TDIS & OPTIONS & yes \\ 
SIM & TDIS & DIMENSIONS & yes \\ 
SIM & TDIS & PERIODDATA & yes \\ 
\hline
EXG & GWFGWF & OPTIONS & yes \\ 
EXG & GWFGWF & DIMENSIONS & yes \\ 
EXG & GWFGWF & EXCHANGEDATA & yes \\ 
\hline
EXG & GWTGWT & OPTIONS & yes \\ 
EXG & GWTGWT & DIMENSIONS & yes \\ 
EXG & GWTGWT & EXCHANGEDATA & yes \\ 
\hline
EXG & GWEGWE & OPTIONS & yes \\ 
EXG & GWEGWE & DIMENSIONS & yes \\ 
EXG & GWEGWE & EXCHANGEDATA & yes \\ 
\hline
SLN & IMS & OPTIONS & yes \\ 
SLN & IMS & NONLINEAR & yes \\ 
SLN & IMS & LINEAR & yes \\ 
\hline
GWF & NAM & OPTIONS & yes \\ 
GWF & NAM & PACKAGES & yes \\ 
\hline
GWF & DIS & OPTIONS & yes \\ 
GWF & DIS & DIMENSIONS & yes \\ 
GWF & DIS & GRIDDATA & no \\ 
\hline
GWF & DISV & OPTIONS & yes \\ 
GWF & DISV & DIMENSIONS & yes \\ 
GWF & DISV & GRIDDATA & no \\ 
GWF & DISV & VERTICES & yes \\ 
GWF & DISV & CELL2D & yes \\ 
\hline
GWF & DISU & OPTIONS & yes \\ 
GWF & DISU & DIMENSIONS & yes \\ 
GWF & DISU & GRIDDATA & no \\ 
GWF & DISU & CONNECTIONDATA & yes \\ 
GWF & DISU & VERTICES & yes \\ 
GWF & DISU & CELL2D & yes \\ 
\hline
GWF & IC & GRIDDATA & no \\ 
\hline
GWF & NPF & OPTIONS & yes \\ 
GWF & NPF & GRIDDATA & no \\ 
\hline
GWF & BUY & OPTIONS & yes \\ 
GWF & BUY & DIMENSIONS & yes \\ 
GWF & BUY & PACKAGEDATA & yes \\ 
\hline
GWF & STO & OPTIONS & yes \\ 
GWF & STO & GRIDDATA & no \\ 
GWF & STO & PERIOD & yes \\ 
\hline
GWF & CSUB & OPTIONS & yes \\ 
GWF & CSUB & DIMENSIONS & yes \\ 
GWF & CSUB & GRIDDATA & no \\ 
GWF & CSUB & PACKAGEDATA & yes \\ 
GWF & CSUB & PERIOD & yes \\ 
\hline
GWF & HFB & OPTIONS & yes \\ 
GWF & HFB & DIMENSIONS & yes \\ 
GWF & HFB & PERIOD & yes \\ 
\hline
GWF & CHD & OPTIONS & yes \\ 
GWF & CHD & DIMENSIONS & yes \\ 
GWF & CHD & PERIOD & yes \\ 
\hline
GWF & WEL & OPTIONS & yes \\ 
GWF & WEL & DIMENSIONS & yes \\ 
GWF & WEL & PERIOD & yes \\ 
\hline
GWF & DRN & OPTIONS & yes \\ 
GWF & DRN & DIMENSIONS & yes \\ 
GWF & DRN & PERIOD & yes \\ 
\hline
GWF & RIV & OPTIONS & yes \\ 
GWF & RIV & DIMENSIONS & yes \\ 
GWF & RIV & PERIOD & yes \\ 
\hline
GWF & GHB & OPTIONS & yes \\ 
GWF & GHB & DIMENSIONS & yes \\ 
GWF & GHB & PERIOD & yes \\ 
\hline
GWF & RCH & OPTIONS & yes \\ 
GWF & RCH & DIMENSIONS & yes \\ 
GWF & RCH & PERIOD & yes \\ 
\hline
GWF & RCHA & OPTIONS & yes \\ 
GWF & RCHA & PERIOD & yes \\ 
\hline
GWF & EVT & OPTIONS & yes \\ 
GWF & EVT & DIMENSIONS & yes \\ 
GWF & EVT & PERIOD & yes \\ 
\hline
GWF & EVTA & OPTIONS & yes \\ 
GWF & EVTA & PERIOD & yes \\ 
\hline
GWF & MAW & OPTIONS & yes \\ 
GWF & MAW & DIMENSIONS & yes \\ 
GWF & MAW & PACKAGEDATA & yes \\ 
GWF & MAW & CONNECTIONDATA & yes \\ 
GWF & MAW & PERIOD & yes \\ 
\hline
GWF & SFR & OPTIONS & yes \\ 
GWF & SFR & DIMENSIONS & yes \\ 
GWF & SFR & PACKAGEDATA & yes \\ 
GWF & SFR & CROSSSECTIONS & yes \\ 
GWF & SFR & CONNECTIONDATA & yes \\ 
GWF & SFR & DIVERSIONS & yes \\ 
GWF & SFR & PERIOD & yes \\ 
\hline
GWF & LAK & OPTIONS & yes \\ 
GWF & LAK & DIMENSIONS & yes \\ 
GWF & LAK & PACKAGEDATA & yes \\ 
GWF & LAK & CONNECTIONDATA & yes \\ 
GWF & LAK & TABLES & yes \\ 
GWF & LAK & OUTLETS & yes \\ 
GWF & LAK & PERIOD & yes \\ 
\hline
GWF & UZF & OPTIONS & yes \\ 
GWF & UZF & DIMENSIONS & yes \\ 
GWF & UZF & PACKAGEDATA & yes \\ 
GWF & UZF & PERIOD & yes \\ 
\hline
GWF & MVR & OPTIONS & yes \\ 
GWF & MVR & DIMENSIONS & yes \\ 
GWF & MVR & PACKAGES & yes \\ 
GWF & MVR & PERIOD & yes \\ 
\hline
GWF & GNC & OPTIONS & yes \\ 
GWF & GNC & DIMENSIONS & yes \\ 
GWF & GNC & GNCDATA & yes \\ 
\hline
GWF & OC & OPTIONS & yes \\ 
GWF & OC & PERIOD & yes \\ 
\hline
GWF & VSC & OPTIONS & yes \\ 
GWF & VSC & DIMENSIONS & yes \\ 
GWF & VSC & PACKAGEDATA & yes \\ 
\hline
GWF & API & OPTIONS & yes \\ 
GWF & API & DIMENSIONS & yes \\ 
\hline
GWT & ADV & OPTIONS & yes \\ 
\hline
GWT & DSP & OPTIONS & yes \\ 
GWT & DSP & GRIDDATA & no \\ 
\hline
GWT & CNC & OPTIONS & yes \\ 
GWT & CNC & DIMENSIONS & yes \\ 
GWT & CNC & PERIOD & yes \\ 
\hline
GWT & DIS & OPTIONS & yes \\ 
GWT & DIS & DIMENSIONS & yes \\ 
GWT & DIS & GRIDDATA & no \\ 
\hline
GWT & DISV & OPTIONS & yes \\ 
GWT & DISV & DIMENSIONS & yes \\ 
GWT & DISV & GRIDDATA & no \\ 
GWT & DISV & VERTICES & yes \\ 
GWT & DISV & CELL2D & yes \\ 
\hline
GWT & DISU & OPTIONS & yes \\ 
GWT & DISU & DIMENSIONS & yes \\ 
GWT & DISU & GRIDDATA & no \\ 
GWT & DISU & CONNECTIONDATA & yes \\ 
GWT & DISU & VERTICES & yes \\ 
GWT & DISU & CELL2D & yes \\ 
\hline
GWT & IC & GRIDDATA & no \\ 
\hline
GWT & NAM & OPTIONS & yes \\ 
GWT & NAM & PACKAGES & yes \\ 
\hline
GWT & OC & OPTIONS & yes \\ 
GWT & OC & PERIOD & yes \\ 
\hline
GWT & SSM & OPTIONS & yes \\ 
GWT & SSM & SOURCES & yes \\ 
GWT & SSM & FILEINPUT & yes \\ 
\hline
GWT & SRC & OPTIONS & yes \\ 
GWT & SRC & DIMENSIONS & yes \\ 
GWT & SRC & PERIOD & yes \\ 
\hline
GWT & MST & OPTIONS & yes \\ 
GWT & MST & GRIDDATA & no \\ 
\hline
GWT & IST & OPTIONS & yes \\ 
GWT & IST & GRIDDATA & no \\ 
\hline
GWT & SFT & OPTIONS & yes \\ 
GWT & SFT & PACKAGEDATA & yes \\ 
GWT & SFT & PERIOD & yes \\ 
\hline
GWT & LKT & OPTIONS & yes \\ 
GWT & LKT & PACKAGEDATA & yes \\ 
GWT & LKT & PERIOD & yes \\ 
\hline
GWT & MWT & OPTIONS & yes \\ 
GWT & MWT & PACKAGEDATA & yes \\ 
GWT & MWT & PERIOD & yes \\ 
\hline
GWT & UZT & OPTIONS & yes \\ 
GWT & UZT & PACKAGEDATA & yes \\ 
GWT & UZT & PERIOD & yes \\ 
\hline
GWT & FMI & OPTIONS & yes \\ 
GWT & FMI & PACKAGEDATA & yes \\ 
\hline
GWT & MVT & OPTIONS & yes \\ 
\hline
GWT & API & OPTIONS & yes \\ 
GWT & API & DIMENSIONS & yes \\ 
\hline
GWE & ADV & OPTIONS & yes \\ 
\hline
GWE & CND & OPTIONS & yes \\ 
GWE & CND & GRIDDATA & no \\ 
\hline
GWE & CTP & OPTIONS & yes \\ 
GWE & CTP & DIMENSIONS & yes \\ 
GWE & CTP & PERIOD & yes \\ 
\hline
GWE & DIS & OPTIONS & yes \\ 
GWE & DIS & DIMENSIONS & yes \\ 
GWE & DIS & GRIDDATA & no \\ 
\hline
GWE & DISV & OPTIONS & yes \\ 
GWE & DISV & DIMENSIONS & yes \\ 
GWE & DISV & GRIDDATA & no \\ 
GWE & DISV & VERTICES & yes \\ 
GWE & DISV & CELL2D & yes \\ 
\hline
GWE & DISU & OPTIONS & yes \\ 
GWE & DISU & DIMENSIONS & yes \\ 
GWE & DISU & GRIDDATA & no \\ 
GWE & DISU & CONNECTIONDATA & yes \\ 
GWE & DISU & VERTICES & yes \\ 
GWE & DISU & CELL2D & yes \\ 
\hline
GWE & ESL & OPTIONS & yes \\ 
GWE & ESL & DIMENSIONS & yes \\ 
GWE & ESL & PERIOD & yes \\ 
\hline
GWE & EST & OPTIONS & yes \\ 
GWE & EST & GRIDDATA & no \\ 
GWE & EST & PACKAGEDATA & yes \\ 
\hline
GWE & IC & GRIDDATA & no \\ 
\hline
GWE & LKE & OPTIONS & yes \\ 
GWE & LKE & PACKAGEDATA & yes \\ 
GWE & LKE & PERIOD & yes \\ 
\hline
GWE & NAM & OPTIONS & yes \\ 
GWE & NAM & PACKAGES & yes \\ 
\hline
GWE & OC & OPTIONS & yes \\ 
GWE & OC & PERIOD & yes \\ 
\hline
GWE & SSM & OPTIONS & yes \\ 
GWE & SSM & SOURCES & yes \\ 
GWE & SSM & FILEINPUT & yes \\ 
\hline
GWE & SFE & OPTIONS & yes \\ 
GWE & SFE & PACKAGEDATA & yes \\ 
GWE & SFE & PERIOD & yes \\ 
\hline
GWE & FMI & OPTIONS & yes \\ 
GWE & FMI & PACKAGEDATA & yes \\ 
\hline
UTL & SPC & OPTIONS & yes \\ 
UTL & SPC & DIMENSIONS & yes \\ 
UTL & SPC & PERIOD & yes \\ 
\hline
UTL & SPCA & OPTIONS & yes \\ 
UTL & SPCA & PERIOD & yes \\ 
\hline
UTL & SPT & OPTIONS & yes \\ 
UTL & SPT & DIMENSIONS & yes \\ 
UTL & SPT & PERIOD & yes \\ 
\hline
UTL & SPTA & OPTIONS & yes \\ 
UTL & SPTA & PERIOD & yes \\ 
\hline
UTL & OBS & OPTIONS & yes \\ 
UTL & OBS & CONTINUOUS & yes \\ 
\hline
UTL & LAKTAB & DIMENSIONS & yes \\ 
UTL & LAKTAB & TABLE & yes \\ 
\hline
UTL & SFRTAB & DIMENSIONS & yes \\ 
UTL & SFRTAB & TABLE & yes \\ 
\hline
UTL & TS & ATTRIBUTES & yes \\ 
UTL & TS & TIMESERIES & yes \\ 
\hline
UTL & TAS & ATTRIBUTES & yes \\ 
UTL & TAS & TIME & no \\ 
\hline
UTL & ATS & DIMENSIONS & yes \\ 
UTL & ATS & PERIODDATA & yes \\ 
\hline
UTL & TVK & OPTIONS & yes \\ 
UTL & TVK & PERIOD & yes \\ 
\hline
UTL & TVS & OPTIONS & yes \\ 
UTL & TVS & PERIOD & yes \\ 


\hline
\end{longtable}
\label{table:blocks}
\normalsize



\justifying
\vspace*{\fill}
\clearpage
\pagestyle{backofreport}
\makebackcover
\end{document}
