
The Groundwater Transport (GWT) Model \citep{modflow6gwt} in \mf can simulate a variety of solute-transport processes in aquifer material that includes ``mobile" and ``immobile" domains. The mobile and immobile domains are conceptualized as coexisting within a model cell and can exchange solute with each other. The mobile domain is always simulated by the GWT Model.  Transport in an immobile domain, which may be optionally defined by the user, is simulated by the Immobile Storage and Transfer (IST) Package \citep{modflow6gwt} for the GWT Model. Multiple instances of the IST Package may be invoked to represent multiple immobile domains.

In \mf version 6.4.2, the parameterization of the equations that govern transport in the mobile and immobile domains has been revised, and corresponding changes have been made to the input requirements for porosity, domain fraction, and bulk density. The revised parameterization is expected to be more intuitive for users in many mobile-immobile transport applications. It will also allow users to model a wider variety of solute transport scenarios involving immobile-domain sorption than the original parameterization used in versions of \mf up to and including version 6.4.1.

Input for existing \mf models that include one or more immobile domains can be converted from the original parameterization to the revised parameterization such that the simulated transport behavior remains the same. No changes to the model input are needed for existing \mf models that do not include an immobile domain.

The remainder of this chapter begins with a review of the original parameterization of transport described in \citep{modflow6gwt}. The revised parameterization is then presented, and guidance is provided for converting existing model input from the original parameterization to the revised parameterization.

\subsection{Review of the Original Parameterization} \label{sec:origparamreview}

With the original parameterization used in versions of \mf up to and including version 6.4.1, the partial differential equation that represents the conservation of solute mass at points in the mobile domain is equation 2--1 of \cite{modflow6gwt}, which is reproduced here as

\begin{equation}
\label{eqn:gwtpdeorig}
\begin{aligned}
\frac {\partial \left ( S_w \theta_m C \right )}{\partial t} = 
- \nabla \cdot \left ( \matr{q} C  \right ) 
+ \nabla \cdot \left ( S_w \theta_m \matr{D} \nabla C \right ) 
+ q'_s C_s + M_s  
- \lambda_1 \theta_m S_w C - \gamma_1 \theta_m S_w \\
- f_m \rho_b \frac {\partial \left ( S_w \overline{C} \right ) }{\partial t} 
- \lambda_2 f_m \rho_b S_w \overline{C} - \gamma_2 f_m \rho_b S_w 
- \sum \limits_{im=1}^{nim}  \zeta_{im} S_w \left ( C - C_{im} \right )
\end{aligned}
\end{equation}

\noindent where $S_w$ is the water saturation defined as the volume of water per volume of voids ($L^3/L^3$), $C$ is the mobile-domain volumetric concentration of solute expressed as mass of dissolved solute per unit volume of mobile-domain fluid ($M/L^3$), $t$ is time ($T$), $\matr{q}$ is the vector of specific discharge ($L/T$), $\matr{D}$ is the second-order tensor of hydrodynamic dispersion coefficients ($L^2/T$), $q'_s$ is the volumetric flow rate per unit volume of aquifer (defined as positive for flow into the aquifer) for mass sources and sinks ($1/T$), $C_s$ is the volumetric solute concentration of the source or sink fluid ($M/L^3$), $M_s$ is rate of solute mass loading per unit volume of aquifer ($M/L^3T$), $\lambda_1$ is the first-order decay rate coefficient for dissolved solute in the mobile domain ($1/T$), $\gamma_1$ is the zero-order decay rate coefficient for dissolved solute in the mobile domain ($M/L^3T$), $\overline{C}$ is the mass-fraction concentration of sorbate (sorbed solute) expressed as mass of sorbate per unit mass of solid aquifer material in the mobile domain ($M/M$), $\lambda_2$ is the first-order decay rate coefficient ($1/T$) for sorbate in the mobile domain, $\gamma_2$ is the zero-order decay rate coefficient ($M/MT$) for sorbate in the mobile domain, $nim$ is the number of immobile domains, $\zeta_{im}$ is the rate coefficient for the transfer of mass between the mobile domain and immobile domain $im$ ($1/T$), $C_{im}$ is an immobile-domain volumetric concentration of solute expressed as mass of dissolved solute per unit volume of fluid in immobile domain $im$ ($M/L^3$), $\rho_{b}$ is the bulk density of the aquifer material defined as the mass of solid aquifer material per unit volume of aquifer ($M/L^3$), $\theta_m$ is the mobile-domain effective porosity defined as defined as volume of voids participating in mobile-domain transport per unit volume of aquifer ($L^3/L^3$), and $f_m$ is the fraction of the mass of aquifer solid material that is in the mobile domain ($M/M$). To avoid potential confusion with the total porosity, in equation~\ref{eqn:gwtpdeorig} the mobile porosity is intentionally given the symbol $\theta_m$, which is different than the symbol $\theta$ used by \cite{modflow6gwt}. Also, note that \cite{modflow6gwt} define $f_m$ as ``the fraction of aquifer solid material available for sorptive exchange with the mobile phase under fully saturated conditions," which is correct only if all the aquifer solid material in the mobile domain is available for sorptive exchange and the fraction is understood to be a mass fraction.

The Immobile Storage and Transfer (IST) Package for the GWT Model allows users to designate a fraction of a model cell as immobile. With the original parameterization used in versions of \mf up to and including version 6.4.1, the partial differential equation that represents the conservation of solute mass at points in an immobile domain is equation 7--2 of \cite{modflow6gwt}, which is reproduced here as 

\begin{equation}
\label{eqn:gwtistpdeorig}
\begin{split}
\theta_{im} \frac{\partial C_{im} }{\partial t} + f_{im} \rho_b \frac{\partial \overline{C}_{im}}{\partial t} = 
- \lambda_{1,im} \theta_{im} C_{im} - \lambda_{2,im}  f_{im} \rho_b \overline{C}_{im} \\
- \gamma_{1,im} \theta_{im} - \gamma_{2,im} f_{im}  \rho_b 
+ \zeta_{im} S_w \left ( C - C_{im} \right ),
\end{split}
\end{equation}

\noindent where $\overline{C}_{im}$ is the mass-fraction concentration of sorbate (sorbed solute) expressed as mass of sorbate per unit mass of solid aquifer material in the immobile domain ($M/M$), $\lambda_{1,im}$ is the first-order reaction rate coefficient for dissolved solute in the immobile domain ($1/T$), $\lambda_{2,im}$ is the first-order reaction rate coefficient for sorbate in the immobile domain ($1/T$), $\gamma_{1,im}$ is the zero-order reaction rate coefficient for dissolved solute in the immobile domain ($ML^{-3}T^{-1}$), $\gamma_{2,im}$ is the zero-order reaction rate coefficient for sorbate in the immobile domain ($M M^{-1}T^{-1}$), $\theta_{im}$ is the effective porosity in the immobile domain defined as defined as volume of voids participating in immobile-domain transport per unit volume of immobile domain $im$ ($L^3/L^3$), and $f_{im}$ is the fraction of the mass of aquifer solid material that is in immobile domain $im$ ($M/M$). Note that \cite{modflow6gwt} define $f_{im}$ as ``the fraction of aquifer solid material available for sorptive exchange with the immobile domain under fully saturated conditions," which is correct only if all the aquifer solid material in the immobile domain is available for sorptive exchange and the fraction is understood to be a mass fraction.

The original model parameters that are affected by the revised parameterization are listed in table~\ref{table:origparam}. Note that in the original parameterization, division of the aquifer into mobile and immobile domains is conceptualized in terms solid mass fractions, and porosities and bulk densities are defined on a per-aquifer-volume basis. Note also that the user is required (in versions of \mf up to and including version 6.4.1) to provide the value of the overall bulk density, $\rho_b$, in the input for each package that represents a domain in which sorption is active (the MST Package for the mobile domain, and the IST Package for immobile domains).

\begin{table}[!ht]
  \small
  \centering
  \caption{Symbols, descriptions, and definitions of original mobile and immobile domain model parameters (used in versions of \mf up to and including version 6.4.1) that are affected by the revised parameterization. In the original parameterization, division of the aquifer into domains is conceptualized in terms of solid mass fractions, and domain properties are defined on a per-aquifer-volume basis} \tabularnewline 

  \begin{tabular}{z{1.50cm}
                  z{3.50cm}
                  z{5.00cm}
                  z{3.50cm}
                  }
    % header
    \hline
    \rowcolor{Gray}
    \multicolumn{1}{ z{1.50cm} }{\textbf{Symbol}} & 
    \multicolumn{1}{ z{3.50cm} }{\textbf{Description}} & 
    \multicolumn{1}{ z{5.00cm} }{\textbf{Definition}} &
    \multicolumn{1}{ z{3.50cm} }{\textbf{Notes}} \\
    \hline

    $\theta_m$ &  porosity (mobile domain) &  $\frac{mobile \; domain \; pore \; volume}{aquifer \; volume}$ & sorption-related input parameter \\
    
\rowcolor{Gray}
    $\theta_{im}$ &  porosity (immobile domain $im$) &  $\frac{immobile \; domain \; pore \; volume}{aquifer \; volume}$ & sorption-related input parameter \\

    $\rho_{b}$ & overall bulk density (mobile and immobile domains combined) &  $\frac{aquifer \; solid \; mass}{aquifer \; volume}$ & sorption-related input parameter \\

\rowcolor{Gray}
    $f_{m}$ &  solid mass fraction (mobile domain) &  $\frac{mobile \; domain \; solid \; mass}{aquifer \; solid \; mass}$ & sorption-related parameter (not input; calculated by \mf from immobile-domain solid mass fractions) \\

    $f_{im}$ &  solid mass fraction (immobile domain $im$) &  $\frac{immobile \; domain \; solid \; mass}{aquifer \; solid \; mass}$ & sorption-related parameter (not input; calculated by \mf from user-specified domain porosities) \\
    
    \hline
  \end{tabular}
  \label{table:origparam}
\end{table}


Immobile-domain solid mass fractions, $f_{im}$, are not input by the user; rather, they are calculated by \mf (in versions up to and including 6.4.1) from user-input porosities using

\begin{equation}
\label{eqn:fim1}
f_{im} = \frac{\theta_{im}}{\theta_t}.
\end{equation}

\noindent where $\theta_t$ is the total porosity [which is erroneously represented by the symbol $\theta$ in the related text on p. 7--2 of \cite{modflow6gwt}]. The mobile-domain solid mass fraction is then calculated by \mf (in versions up to and including 6.4.1) using

\begin{equation}
\label{eqn:fm0}
f_m = 1 - \sum_{im}f_{im},
\end{equation}

\noindent where the summation is over all immobile domains specified by the user. Equations~\ref{eqn:fm0} and~\ref{eqn:fim1}, together with the definition of total porosity,

\begin{equation}
\label{eqn:thetat1}
\theta_t = \theta_m + \sum_{im}{\theta_{im}},
\end{equation}

\noindent imply that the default value of the mobile domain solid mass fraction is given by

\begin{equation}
\label{eqn:fm1}
f_m = 1 - \sum_{im}f_{im} = 1 - \frac{\sum_{im}\theta_{im}}{\theta_t} = \frac{\theta_m}{\theta_t}.
\end{equation}

\noindent If there are no immobile domains, the total porosity is the same as the mobile porosity and $f_m$ is 1.

\subsection{Revised Parameterization} \label{sec:revisedparam}

The revised parameterization described in this chapter differs from the original parameterization, which is described in \cite{modflow6gwt} and summarized above, in the following ways:

\begin{itemize}
\item Two sets of parameter substitutions are implemented in governing equations~\ref{eqn:gwtpdeorig} and~\ref{eqn:gwtistpdeorig}. These substitutions recast the model input in terms of parameters that are expected be more intuitive for users in many applications.
  \begin{itemize}
  \item Domain porosities $\theta_{m}$ and $\theta_{im}$, which are defined on a per-aquifer-volume basis, are replaced by the mathematically equivalent expressions $\hat{f}_{m} \phi_{m}$ and $\hat{f}_{im} \phi_{im}$, respectively, where $\hat{f}_{m}$ and $\hat{f}_{im}$ are domain volume fractions (instead of mass fractions) and $\phi_{m}$ and $\phi_{im}$ are domain porosities defined on a per-domain-volume basis.
  \item The parameter products $f_{m} \rho_{b}$ and $f_{im} \rho_{b}$ are replaced by the mathematically equivalent products $\hat{f}_{m} \rho_{b, m}$ and $\hat{f}_{im} \rho_{b, im}$, respectively, where $\rho_{b, m}$ and $\rho_{b, im}$ are domain bulk densities defined on a per-domain-volume basis.
  \end{itemize}
\item Unlike the domain mass fractions $f_{m}$ and $f_{im}$, which are set automatically by \mf (in versions up to and including 6.4.1) based on domain porosities in the original parameterization, the domain volume fractions $\hat{f}_{m}$ and $\hat{f}_{im}$ are specified directly by the user in the revised parameterization, offering the flexibility to accurately characterize sorption in a wider variety of mobile-immobile systems.
\end{itemize}

With the changes to the parameterization summarized above, the partial differential equations that represent the conservation of solute mass at points in the mobile and immobile domains can be written as

\begin{equation}
\label{eqn:gwtpde}
\begin{aligned}
\frac {\partial \left ( S_w \theta_m C \right )}{\partial t} = 
- \nabla \cdot \left ( \matr{q} C  \right ) 
+ \nabla \cdot \left ( S_w \theta_m \matr{D} \nabla C \right ) 
+ q'_s C_s + M_s  
- \lambda_1 \theta_m S_w C - \gamma_1 \theta_m S_w \\
- \hat{f}_m \rho_{b,m} \frac {\partial \left ( S_w \overline{C} \right ) }{\partial t} 
- \lambda_2 \hat{f}_m \rho_{b,m} S_w \overline{C} - \gamma_2 \hat{f}_m \rho_{b,m} S_w 
- \sum \limits_{im=1}^{nim}  \zeta_{im} S_w \left ( C - C_{im} \right )
\end{aligned}
\end{equation}

\noindent where

\begin{equation}
\label{eqn:thetam_from_revparams}
\theta_{m} = \hat{f}_{m} \phi_{m} ,
\end{equation}

\noindent and

\begin{equation}
\label{eqn:gwtistpde}
\begin{split}
\theta_{im} \frac{\partial C_{im} }{\partial t} + \hat{f}_{im} \rho_{b,im} \frac{\partial \overline{C}_{im}}{\partial t} = 
- \lambda_{1,im} \theta_{im} C_{im} - \lambda_{2,im}  \hat{f}_{im} \rho_{b,im} \overline{C}_{im} \\
- \gamma_{1,im} \theta_{im} - \gamma_{2,im} \hat{f}_{im} \rho_{b,im} 
+ \zeta_{im} S_w \left ( C - C_{im} \right ),
\end{split}
\end{equation}

\noindent where

\begin{equation}
\label{eqn:thetaim_from_revparams}
\theta_{im} = \hat{f}_{im} \phi_{im} .
\end{equation}

\noindent respectively. Model parameters and variables in equations~\ref{eqn:gwtpde} and ~\ref{eqn:gwtistpde} are defined as in equations~\ref{eqn:gwtpdeorig} and ~\ref{eqn:gwtistpdeorig}, except for the new parameters introduced in the revised parameterization: $\hat{f}_m$ is the volume fraction of the mobile domain defined as the volume of mobile domain per volume of aquifer ($L^3/L^3$), $\hat{f}_{im}$ is the volume fraction of the immobile domain defined as the volume of mobile domain $im$ per volume of aquifer ($L^3/L^3$), $\rho_{b,m}$ is the bulk density of aquifer material in the mobile domain defined as mass of solid aquifer material per unit volume of mobile domain ($M/L^3$), $\rho_{b,im}$ is the bulk density of aquifer material in the immobile domain defined as mass of solid aquifer material per unit volume of immobile domain $im$ ($M/L^3$), $\phi_m$ is the mobile-domain effective porosity defined as defined as volume of voids participating in mobile-domain transport per unit volume of mobile domain ($L^3/L^3$), and $\phi_{im}$ is the effective porosity in the immobile domain defined as defined as volume of voids participating in immobile-domain transport per unit volume of mobile domain $im$ ($L^3/L^3$). In the \mf code, porosities $\theta_{m}$ and $\theta_{im}$ are calculated from the new input parameters using equations~\ref{eqn:thetam_from_revparams} and ~\ref{eqn:thetaim_from_revparams} before being incorporated into discretized representations of equations~\ref{eqn:gwtpde} and~\ref{eqn:gwtistpde}. The new parameters are discussed in more detail below.

\subsubsection{Revised Parameters}

The new parameters are summarized in table~\ref{table:revparam}. Note that in  the revised parameterization, division of the aquifer into mobile and immobile domains is conceptualized in terms of volume fractions, and porosities and bulk densities are defined on a per-domain-volume basis. When an aquifer can be divided into multiple domains that occupy distinct, well-defined volumes, as when the mobile and immobile domains represent different lithologies, it may be intuitive to think in terms of the domain volume fractions and local domain properties (properties defined on a per-domain-volume basis).

\begin{table}[!ht]
  \small
  \centering
  \caption{Symbols, descriptions, and definitions of revised mobile and immobile domain model parameters introduced in \mf version 6.5.0. In the revised parameterization, division of the aquifer into domains is conceptualized in terms of volume fractions, and domain properties are defined on a per-domain-volume basis} \tabularnewline 

  \begin{tabular}{z{1.50cm}
                  z{3.50cm}
                  z{5.00cm}
                  z{3.50cm}
                  }
    % header
    \hline
    \rowcolor{Gray}
    \multicolumn{1}{ z{1.50cm} }{\textbf{Symbol}} & 
    \multicolumn{1}{ z{3.50cm} }{\textbf{Description}} & 
    \multicolumn{1}{ z{5.00cm} }{\textbf{Definition}} &
    \multicolumn{1}{ z{3.50cm} }{\textbf{Notes}} \\
    \hline

    $\phi_m$ &  porosity (mobile domain) &  $\frac{mobile \; domain \; pore \; volume}{mobile \; domain \; volume}$ & transport input parameter (not directly related to sorption) \\
    
\rowcolor{Gray}
    $\phi_{im}$ &  porosity (immobile domain $im$) &  $\frac{immobile \; domain \; pore \; volume}{immobile \; domain \; volume}$ & transport input parameter (not directly related to sorption) \\

    $\rho_{b,m}$ & bulk density (mobile domain) &  $\frac{mobile \; domain \; solid \; mass}{mobile \; domain \; volume}$ & sorption-related input parameter \\
    
\rowcolor{Gray}
    $\rho_{b,im}$ & bulk density (immobile domain $im$) &  $\frac{immobile \; domain \; solid \; mass}{immobile \; domain \; volume}$ & sorption-related input parameter \\

    $\hat{f}_{m}$ &  volume fraction (mobile domain) &  $\frac{mobile \; domain \; volume}{aquifer \; volume}$ & transport and sorption-related parameter (not input; calculated by \mf from immobile-domain volume fractions) \\

\rowcolor{Gray}
    $\hat{f}_{im}$ &  volume fraction (immobile domain $im$) &  $\frac{immobile \; domain \; volume}{aquifer \; volume}$ & transport and sorption-related input parameter \\

    \hline
  \end{tabular}
  \label{table:revparam}
\end{table}


\noindent All parameters in table~\ref{table:revparam} are model input parameters except the mobile-domain volume fraction, $\hat{f}_m$, which is calculated by \mf from the user-specified immobile-domain volume fractions using

\begin{equation}
\label{eqn:fm5}
\hat{f}_m = 1 -  \sum_{im}{\hat{f}_{im}},
\end{equation}

\noindent where the summation is over all immobile domains specified by the user. If there are no immobile domains,  $\hat{f}_m$ is set to 1.

Porosities and bulk densities in table~\ref{table:revparam} are defined on a per-domain-volume basis, i.e., as an amount per unit volume of domain. The total porosity, $\theta_t$, can be calculated as the volume-weighted average of the domain porosities, $\phi_m$ and $\phi_{im}$:

\begin{equation}
\label{eqn:thetat}
\theta_{t} = \theta_{m} + \sum_{im}{\theta_{im}} = \hat{f}_{m} \phi_{m} + \sum_{im}{\hat{f}_{im} \phi_{im}} ,
\end{equation}

\noindent Similarly, the overall bulk density, $\rho_b$, can be calculated as the volume-weighted average of the domain bulk densities, $\rho_{b, m}$ and $\rho_{b, im}$:

\begin{equation}
\label{eqn:rhob2}
\rho_{b} = \hat{f}_m \rho_{b, m} + \sum_{im}{\hat{f}_{im} \rho_{b, im}}.
\end{equation}

\subsection{Conversion of Existing Model Input to the Revised Parameterization} \label{sec:inputconversion}

Conversion of existing \mf model input to the revised parameterization such that the simulation gives the same numerical results requires that the revised mobile-domain solute conservation equation, equation~\ref{eqn:gwtpde}, be numerically equivalent to the original mobile-domain solute conservation equation, equation~\ref{eqn:gwtpdeorig}. This is achieved when $\hat{f}_{m}$ and $\phi_{m}$ satisfy

\begin{equation}
\label{eqn:equiv_porm}
\hat{f}_{m} \phi_{m} = \theta_{m}
\end{equation}

\noindent and, if sorption is active in the mobile domain, $\hat{f}_{m}$ and $\rho_{b, m}$ satisfy

\begin{equation}
\label{eqn:equiv_rhobm}
\hat{f}_{m} \rho_{b, m} = f_{m} \rho_{b} .
\end{equation}

\noindent For a model that includes at least one immobile domain, the revised immobile-domain solute conservation equation, equation~\ref{eqn:gwtistpde}, must also be numerically equivalent to the original immobile-domain solute conservation equation, equation~\ref{eqn:gwtistpdeorig}. This is achieved when $\hat{f}_{im}$ and $\phi_{im}$ satisfy

\begin{equation}
\label{eqn:equiv_porim}
\hat{f}_{im} \phi_{im} = \theta_{im}
\end{equation}

\noindent and, if sorption is active in immobile domain $im$, $\hat{f}_{im}$ and $\rho_{b, im}$ satisfy

\begin{equation}
\label{eqn:equiv_rhobim}
\hat{f}_{im} \rho_{b, im} = f_{im} \rho_{b} .
\end{equation}

\subsubsection{Simulation Without Immobile Domains}

For an existing simulation without immobile domains, which is the most common use case, no conversion of input parameters is needed to obtain the same simulation results. In this case, \mf sets $f_{m} = 1$ in the original parameterization and $\hat{f}_{m} = 1$ in the revised parameterization. Thus, equations~\ref{eqn:equiv_porm} and~\ref{eqn:equiv_rhobm} reduce to $\phi_{m} = \theta_{m}$ and $\rho_{b, m} = \rho_{b}$, respectively, and are satisfied by default by the original input values.

\subsubsection{Simulation With At Least One Immobile Domain}

For an existing simulation that includes at least one immobile domain, the values of the revised input parameters $\phi_{m}$, $\phi_{im}$, and $\hat{f}_{im}$ must be set such that they satisfy equations~\ref{eqn:equiv_porm} and~\ref{eqn:equiv_porim} in order to obtain the same simulation results. The value of $\hat{f}_{m}$ is calculated by \mf using equation~\ref{eqn:fm5}. Note that the immobile-domain fractions $\hat{f}_{im}$ must sum to less than one so that the resulting mobile-domain fraction, $\hat{f}_{m}$, is greater than zero.

If the existing simulation also includes sorption, the value of the domain bulk density ($\rho_{b, m}$ or $\rho_{b, im}$) must be specified for each domain in which sorption is active. For a mobile domain with sorption, $\rho_{b, m}$ must satisfy equation~\ref{eqn:equiv_rhobm}. For an immobile domain with sorption, $\rho_{b, im}$ must satisfy equation~\ref{eqn:equiv_rhobim}.

Because the revised parameterization involves more input parameters than the original parameterization, the choice of new parameter values that produce the same simulation results is not unique. One approach, for example, is to specify the values of the immobile domain fractions $\hat{f}_{im}$. The required values of $\hat{f}_{m}$, $\phi_{m}$, $\rho_{b, m}$, $\phi_{im}$, and $\rho_{b, im}$ are then determined by equations~\ref{eqn:fm5} and~\ref{eqn:equiv_porm}~--~\ref{eqn:equiv_rhobim}, respectively. If the only goal is to obtain the same simulation results, without regard for whether the revised parameter values are realistic or representative of the domains being modeled, one may, for example, set each immobile domain fraction to $\hat{f}_{im} = 1 / N_{dom}$, where $N_{dom}$ is the total number of domains including the mobile domain.

\subsubsection{Unrealistic Parameter Values}

It is possible that in some cases the process of converting parameters will produce revised parameter values that are unrealistic or inconsistent with independently measured or estimated values. For example, selection of apparently reasonable values of $\hat{f}_{im}$ may require setting $\hat{f}_{m}$, $\phi_{m}$, $\rho_{b, m}$, $\phi_{im}$, and/or $\rho_{b, im}$ to unreasonable values. Assuming the conversion equations have been applied correctly and the independently measured or estimated values are reasonably accurate, other possible sources of inconsistency are errors in setting the original parameter values; selection of unrealistic values for revised parameters ($\hat{f}_{im}$ values, for example) as a starting point for the conversion; and the setting of the original immobile domain fractions $f_{im}$ according to equation~\ref{eqn:fim1}, which may be a poor approximation for some of the domains being modeled.
